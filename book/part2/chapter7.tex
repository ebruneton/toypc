% This work is licensed under the Creative Commons Attribution NonCommercial
% ShareAlike 4.0 International License. To view a copy of the license, visit
% https://creativecommons.org/licenses/by-nc-sa/4.0/

\renewcommand{\rustfile}{chapter7}
\setcounter{rustid}{0}

\rust{
  context.write_backup("website/backups", "memory_editor.txt")?;
}

\chapter{Memory Editor}\label{chapter:memory-editor}

Our toy computer is now fully assembled, and we have drivers to input and
output data with its own keyboard and screen. The last remaining step to make
it completely autonomous is to provide a way to enter programs and to run them,
with the keyboard. The most basic way to do this is to write a program similar
to the boot assistant, but using the keyboard instead of an external computer.
This chapter implements such a program, hereafter called the ``memory editor''.
We test it at the end to read the memory, control a LED with the keyboard, and
run other programs.

\section{User interface}

A possible user interface for our memory editor would be to use the same
interface as the boot assistant. We would thus type commands such as ``{\tt
wC0000,\#}'', ``{\tt WC0000,1234\#}'', or ``{\tt GC0000\#}'', to read words,
write words, or call functions. And the screen would display the commands and
their results. This could lead to a very small program, but would not be very
practical. Instead, we propose to make better use of the screen, by displaying
a ``memory page'' at a time (256 bytes). In this way, we can read the memory
like a book, instead of one word at a time. We also get more context when
editing it, or before calling a function. More precisely, we propose the
following user interface:

\begin{flushleft}
{\tt 00000003 00000002 00000001 00000000 {\color{gray} 20000100}}\\
{\tt 00000007 00000006 {\color{green0} 12345678} 00000004 {\color{gray}
20000110}}\\
$\ldots$ {\em 13 other rows} $\ldots$\\
{\tt 000000FF 000000FE 000000FD 000000FC {\color{gray} 200001F0}}\\
~\\
{\tt ~~~~~~~~ ~~~~~~~~ ~~~~~~~~ 2345678\underline{A} {\color{gray} 20000114}}\\
\end{flushleft}

\begin{itemize}
  \item A {\em page view} shows 16 rows of 4 words each, followed by the row's
  start address. As we did so far in this book, addresses increase {\em from
  right to left} and from top to bottom (because this is more adapted to
  little-endian order than left to right).

  \item Two {\em fields} below the page view show a value $V$ and an address
  $A$, that the user can edit. When editing $V$, the current value at address
  $A$ is highlighted in the page view. This value is not changed until Enter is
  pressed. When this happens $V$ is written at address $A$ and $A$ is
  incremented by 4 to edit the next word.

  \item When editing $A$ nothing else changes until Enter is pressed. When this
  happens, $V$ is updated to show the value at this address. The page view is
  updated as well to show the page containing this address.
\end{itemize}

To simplify the implementation, editing a field can only be done by typing an
hexadecimal digit ([0-9] or [a-f] in lowercase). Doing so shifts the current
value to the left by one hex digit, and inserts the new hex digit on the right.
For instance, if the current value is {\tt 12345678}, typing {\tt A} gives {\tt
2345678A}.

Only one field can be edited at a time. The currently ``selected'' field is
indicated with a blinking cursor under its least significant digit. Typing
``w'' selects the address field. Typing Enter when editing the address selects
the value field.

Pressing ``r'' calls the function at the {\em highlighted address}. This is the
address which is highlighted in the page view. It might differ from the value
in the address field (if this value has been edited and Enter has not been
pressed yet).

Finally, for convenience -- this is not strictly necessary -- the arrow keys
can be used to go to the next or previous word or row. The left (resp. right)
arrow increases (resp. decreases) the highlighted address by 4. The up (resp.
down) arrow decreases (resp. increases) the highlighted address by 16.

\section{State variables}\label{subsection:med-state-variables}

The state of the above user interface can be described with only 3 variables:

\begin{itemize}
  \item $address$: the address of the highlighted word in the page view. This
  variable is sufficient to determine which page contains the highlighted word,
  and thus to draw the page view (see the next section).

  \item $mode$: which field is currently being edited. We use $mode=0$ if it is
  the value field, and $mode=1$ if it is the address field.

  \item $input$: the current value input with the keyboard. This value is shown
  either in the value field (if $mode=0$), or in the address field (if
  $mode=1$). When Enter is pressed, it is either written in memory at $address$
  (if $mode=0$), or replaces $address$ (if $mode=1$).
\end{itemize}

\rust{
  let driver_address = context.memory_region("keyboard_driver").end();
  let mut b = BytecodeAssembler::new(RegionKind::Default, driver_address);
  b.import_labels(context.memory_region("clock_driver"));
  b.import_labels(context.memory_region("foundations"));
  b.import_labels(context.memory_region("graphics_card_driver"));
  b.import_labels(context.memory_region("keyboard_driver"));

  const MED_ADDRESS: u32 = 0x400E1A9C;
  const MED_MODE: u32 = 0x400E1AA0;
  const MED_VALUE: u32 = 0x400E1AA4;
}

To simplify the implementation we use 3 more of the 8 General Purpose Backup
Registers in the ``Controllers'' memory region to store these variables (we
already used 3 of them for the keyboard driver, see
\cref{subsection:keyboard-driver-impl}). More precisely we use the registers at
addresses \rs{hex(MED_ADDRESS)}, \rs{hex(MED_MODE)}, and \rs{hex(MED_VALUE)},
respectively (see \cref{fig:boot-memory-map} and in Chapter 17 in
\cite{SAM3X8E}). For convenience, we provide the following functions to get and
set their values, called {\em getters} and {\em setters}. We store them after
the keyboard driver, at address \rs{hex(driver_address)}:

\begin{TwoColumns}
\rs{b.func("med_get_address", &[], "address", &["private"])}\\
\bytecode{
  b.cst(MED_ADDRESS);
  b.load();
  b.retv();
}
\\
\rs{b.func("med_set_address", &["address"], "", &["private"])}\\
\bytecode{
  b.cst(MED_ADDRESS);
  b.get("address");
  b.store();
  b.ret();
}
\\
\rs{b.func("med_get_mode", &[], "mode", &["private"])}\\
\bytecode{
  b.cst(MED_MODE);
  b.load();
  b.retv();
}
\\
\rs{b.func("med_set_mode", &["mode"], "", &["private"])}\\
\bytecode{
  b.cst(MED_MODE);
  b.get("mode");
  b.store();
  b.ret();
}
\\
\rs{b.func("med_get_input", &[], "input", &["private"])}\\
\bytecode{
  b.cst(MED_VALUE);
  b.load();
  b.retv();
}
\\
\rs{b.func("med_set_input", &["input"], "", &["private"])}\\
\bytecode{
  b.cst(MED_VALUE);
  b.get("input");
  b.store();
  b.ret();
}
\\
~
\end{TwoColumns}

\section{Drawing functions}

\subsection{Page view}

To draw the page view we first need a function to draw its most basic element,
an hexadecimal digit. The following function draws the hexadecimal digit $d$
corresponding to the 4 least significant bits of its argument $x$ ($d=x \wedge
F_{16}$). If $d \le 10$ it should draw a character between ``0'' and ``9'',
which have contiguous ASCII codes in [\hexa{30},\hexa{39}]. It thus draws the
character $d+\hexa{30}$. Otherwise, if $d \ge 10$, it should draw a character
between ``A'' and ``F'', which have contiguous ASCII codes in
[\hexa{41},\hexa{46}]. It thus draws the character $d-10+\hexa{41}=y+\hexa{37}$:

\begin{Paragraph}
\begin{paracol}{2}
\rs{b.func("gpu_draw_hex_digit", &["x"], "", &[])}

Compute $d=x \wedge \hexa{F}$. The result is in the $5^{th}$ stack frame slot,
after the function argument and 4 saved registers.

\bytecode[switchcolumn]{
  b.get("x");
  b.cst8(0xF);
  b.and();
  b.def("d");
}

If $d \ge 10$, skip the next two instructions.

\bytecode[switchcolumn]{
  b.get("d");
  b.cst8(10);
  b.ifge("greater_than_or_equal_10");
}

Otherwise, push \hexa{30} on the stack and skip the next instruction.

\bytecode[switchcolumn]{
  b.cst8(0x30);
  b.goto("draw_hex_digit");
}

Push \hexa{37} on the stack.

\bytecode[switchcolumn]{
  b.label("greater_than_or_equal_10");
  b.cst8(0x37);
  b.label("draw_hex_digit");
}

Add $d$ to the last pushed value (either \hexa{30} or \hexa{37}) and draw the
resulting character with \verb!gpu_draw_char!.

\bytecode[switchcolumn]{
  b.get("d");
  b.add();
  b.call("gpu_draw_char");
  b.ret();
}
\end{paracol}
\end{Paragraph}

With this we can write a function to draw the two digits of a byte $b$ ($b \gg
4$ and $b$):

\begin{TwoColumns}
\rs{b.func("gpu_draw_hex_byte", &["b"], "", &[])}\\
\bytecode{
  b.get("b");
  b.cst8(4);
  b.lsr();
  b.call("gpu_draw_hex_digit");
  b.get("b");
  b.call("gpu_draw_hex_digit");
  b.ret();
}
\end{TwoColumns}

\noindent which can itself be used to write a function drawing the 4 bytes
of a word $w$ ($w \gg 24$, $w \gg 16$, $w \gg 8$, and $w$):

\begin{TwoColumns}
\rs{b.func("gpu_draw_hex_word", &["w"], "", &[])}\\
\bytecode{
  b.get("w");
  b.cst8(24);
  b.lsr();
  b.call("gpu_draw_hex_byte");
  b.get("w");
  b.cst8(16);
  b.lsr();
  b.call("gpu_draw_hex_byte");
  b.get("w");
  b.cst8(8);
  b.lsr();
  b.call("gpu_draw_hex_byte");
  b.get("w");
  b.call("gpu_draw_hex_byte");
  b.ret();
}
\end{TwoColumns}

We continue with a function to draw the word at a given address, followed by a
space (whose ASCII code is \hexa{20}, rendered with \verb!gpu_draw_char!):

\begin{TwoColumns}
\rs{b.func("med_draw_page_word_at", &["address"], "", &["private"])}\\
\bytecode{
  b.get("address");
  b.load();
  b.call("gpu_draw_hex_word");
  b.cst8(0x20);
  b.call("gpu_draw_char");
  b.ret();
}
\end{TwoColumns}

In turn, this can be used to write a function drawing a line of the page view,
given its start $address$. This function sets the color to green $(0,7,0)$,
draws the 4 words at addresses $address+12$, $address+8$, $address+4$, and
$address$, sets the color to white $(7,7,3)$, and finally draws the start
$address$ of the row:

\begin{TwoColumns}
\rs{b.func("med_draw_page_row", &["address"], "", &["private"])}\\
\bytecode{
  b.cst_0();
  b.cst8(7);
  b.cst_0();
  b.call("gpu_set_color");
  b.get("address");
  b.cst8(12);
  b.add();
  b.call("med_draw_page_word_at");
  b.get("address");
  b.cst8(8);
  b.add();
  b.call("med_draw_page_word_at");
  b.get("address");
  b.cst8(4);
  b.add();
  b.call("med_draw_page_word_at");
  b.get("address");
  b.call("med_draw_page_word_at");
  b.cst8(7);
  b.cst8(7);
  b.cst8(3);
  b.call("gpu_set_color");
  b.get("address");
  b.call("gpu_draw_hex_word");
  b.ret();
}
\end{TwoColumns}

We can finally implement a function to draw the page view. This function takes
an $address$ and draws the page containing it, with the word at $address$
highlighted in yellow. The $i^{th}$ page corresponds to addresses in
$[256i,256(i+1)[$. The page containing $address$ is thus such that $i=\lfloor
address/256 \rfloor=address \gg 8$, and starts at $256i=i\ll 8=address\ \wedge$
\hexa{FFFFFF00}. In fact, to simplify the highlighting, and to make sure that
the word at $address$ is fully contained in the page, we actually use $256i +
(address\ \mathrm{mod}\ 4)$ as the page's $base$ address. This gives
$base=256i+(address \wedge 3)=address\ \wedge$ \hexa{FFFFFF03}. Note that since
each row contains 16 bytes, the $j^{th}$ row in a page starts
at address $base+16j$.

The word to highlight is $h$ words after $base$, with $h=(address-base)/4$.
This corresponds to the $k^{th}$ word in the $l^{th}$ row (starting from the
right), with $k = h\ \mathrm{mod}\ 4$ and $l=\lfloor h/4 \rfloor$. This, in
turn, corresponds to column $c_h=9(3 - k)$ and row $r_h=l$, since each word
uses 9 characters (counting the space). Substituting $k$ and $l$ with their
values finally gives $(c_h,r_h) = (27 - 9(h \wedge 3), h \gg 2)$.

The \verb!med_draw_page! function follows from the above computations. It first
computes $base$, then uses a loop to draw the 16 rows, and finally draws the
highlighted word on top, at the above coordinates:\bigskip

\begin{paracol}{2}
\rs{b.func("med_draw_page", &["address"], "", &["private"])}

Compute $base$. The result is in the $5^{th}$ stack frame slot.

\bytecode[switchcolumn]{
  b.get("address");
  b.cst(0xFFFFFF03);
  b.and();
  b.def("base");
}

Initialize $j$ to 0 (in the $6^{th}$ slot).

\bytecode[switchcolumn]{
  b.cst_0();
  b.def("j");
}

Set the cursor to $(0, j)$, the top-left corner of the next row to draw.

\bytecode[switchcolumn]{
  b.label("draw_row_loop");
  b.cst_0();
  b.get("j");
  b.call("gpu_set_cursor");
}

Draw the next row, starting at $base + 16j = base + (j \ll 4)$.

\bytecode[switchcolumn]{
  b.get("base");
  b.get("j");
  b.cst8(4);
  b.lsl();
  b.add();
  b.call("med_draw_page_row");
}

Increment $j$ by 1 to prepare drawing the next row.

\bytecode[switchcolumn]{
  b.cst_1();
  b.add();
}

If $j < 16$, go back above to draw the next row.

\bytecode[switchcolumn]{
  b.get("j");
  b.cst8(16);
  b.iflt("draw_row_loop");
}

Compute $h=(address-base)/4$. The result is in the $7^{th}$ stack frame slot.

\bytecode[switchcolumn]{
  b.get("address");
  b.get("base");
  b.sub();
  b.cst8(2);
  b.lsr();
  b.def("h");
}

Compute $c_h=27 - 9(h \wedge 3)$.

\bytecode[switchcolumn]{
  b.cst8(27);
  b.cst8(9);
  b.get("h");
  b.cst8(3);
  b.and();
  b.mul();
  b.sub();
}

Compute $r_h=\lfloor h/4 \rfloor=h \gg 2$.

\bytecode[switchcolumn]{
  b.get("h");
  b.cst8(2);
  b.lsr();
}

Set the cursor to these coordinates.

\bytecode[switchcolumn]{
  b.call("gpu_set_cursor");
}

Set the color to yellow $(7, 7, 0)$.

\bytecode[switchcolumn]{
  b.cst8(7);
  b.cst8(7);
  b.cst_0();
  b.call("gpu_set_color");
}

Draw the word at $address$ and return.

\bytecode[switchcolumn]{
  b.get("address");
  b.call("med_draw_page_word_at");
  b.ret();
}
\end{paracol}

\subsection{Fields}

To draw the fields we first provide two functions to draw arbitrary values in
each field. These functions just draw their argument $x$ at the correct column
and row coordinates -- namely $(27,18)$ and $(36,18)$ -- and with the correct
color (green and white):

\begin{TwoColumns}
\rs{b.func("med_draw_value", &["x"], "", &["private"])}\\
\bytecode{
  b.cst8(27);
  b.cst8(18);
  b.call("gpu_set_cursor");
  b.cst_0();
  b.cst8(7);
  b.cst_0();
  b.call("gpu_set_color");
  b.get("x");
  b.call("gpu_draw_hex_word");
  b.ret();
}
\\
\rs{b.func("med_draw_address", &["x"], "", &["private"])}\\
\bytecode{
  b.cst8(36);
  b.cst8(18);
  b.call("gpu_set_cursor");
  b.cst8(7);
  b.cst8(7);
  b.cst8(3);
  b.call("gpu_set_color");
  b.get("x");
  b.call("gpu_draw_hex_word");
  b.ret();
}
\end{TwoColumns}

We use them in the following function to draw the two fields, depending on the
current mode. If $mode=0$, the first half of the function draws $address$ in
the address field, and $input$ in the value field. It then sets the cursor
under the least significant digit of the value field. Otherwise, if $mode \ne
0$, the second half draws the value stored in memory at $address$ in the value
field, and draws $input$ in the address field. It finally sets the cursor under
the least significant digit of the address field:

\begin{TwoColumns}
\rs{b.func("med_draw_fields", &[], "", &["private"])}\\
\bytecode{
  b.call("med_get_mode");
  b.cst_0();
  b.ifne("not_edit_value");
  b.call("med_get_address");
  b.call("med_draw_address");
  b.call("med_get_input");
  b.call("med_draw_value");
  b.cst8(34);
  b.cst8(18);
  b.call("gpu_set_cursor");
  b.ret();
  b.label("not_edit_value");
  b.call("med_get_address");
  b.load();
  b.call("med_draw_value");
  b.call("med_get_input");
  b.call("med_draw_address");
  b.cst8(43);
  b.cst8(18);
  b.call("gpu_set_cursor");
  b.ret();
}
\end{TwoColumns}

\section{Editing functions}

We can now implement functions to react to keyboard inputs. These functions
update the state variables and redraw the page view and/or the fields,
depending on the key typed. We start with a function to enter a new hexadecimal
digit $x$. As specified above, this should shift $input$ to the left by one hex
digit, and insert $x$ on the right. This can be done with $input \leftarrow
(input \ll 4) + x$. After that the fields must be redrawn (but not the page
view). Characters have an opaque black background, so we don't need to erase
the previous values before drawing new ones:

\begin{TwoColumns}
\rs{b.func("med_new_digit", &["x"], "", &["private"])}\\
\bytecode{
  b.call("med_get_input");
  b.cst8(4);
  b.lsl();
  b.get("x");
  b.add();
  b.call("med_set_input");
  b.call("med_draw_fields");
  b.ret();
}
\end{TwoColumns}

A function to enter a new highlighted address $x$ is also useful, since several
keys can change this address (the Enter key and the arrow keys). The following
function sets $address$ to $x$, sets $input$ to the current value at this
address, and finally redraws the page view and the fields:

\begin{TwoColumns}
\rs{b.func("med_new_address", &["x"], "", &["private"])}\\
\bytecode{
  b.get("x");
  b.call("med_set_address");
  b.get("x");
  b.load();
  b.call("med_set_input");
  b.get("x");
  b.call("med_draw_page");
  b.call("med_draw_fields");
  b.ret();
}
\end{TwoColumns}

With this we can write a function to handle the Enter key. Pressing this key
has two different effects, depending on the current $mode$. If $mode=0$ the
value field is being edited. In this case we want to store $input$ in memory at
$address$, and set $address+4$ as the new highlighted address. This is what the
first part of the function does, after a test of the $mode$ value. The second
part, after the first \insn{ret}, handles the case $mode \ne 0$: it sets
$mode$ to $0$ and the new address to $input$:

\begin{TwoColumns}
\rs{b.func("med_handle_enter", &[], "", &["private"])}\\
\bytecode{
  b.call("med_get_mode");
  b.cst_0();
  b.ifne("not_enter_value");
  b.call("med_get_address");
  b.call("med_get_input");
  b.store();
  b.call("med_get_address");
  b.cst8(4);
  b.add();
  b.call("med_new_address");
  b.ret();
  b.label("not_enter_value");
  b.cst_0();
  b.call("med_set_mode");
  b.call("med_get_input");
  b.call("med_new_address");
  b.ret();
}
\end{TwoColumns}

We finally provide a function to handle any character $c$ typed on the
keyboard. This function is quite long because there are ``many'' characters to
support. But its overall structure is regular. It is made of several sequences
of instructions $S_i$, one for each supported character (\eg, ``w'' or ``r'')
or range of characters (\eg, [``0''-``9''] or [``a''-``f'']). Each sequence
$S_i$ starts with one or two conditional jumps to go to the next sequence
$S_{i+1}$, if $c$ is not a character handled by $S_i$. The sequence continues
with instructions to handle this character, and ends with a \insn{ret}:\bigskip

\begin{paracol}{2}
\rs{b.func("med_handle_char", &["c"], "", &["private"])}

$S_0$: decimal digits [0-9]. If $c$ < ``0'' (ASCII code \hexa{30}), go to $S_1$.

\bytecode[switchcolumn]{
  b.get("c");
  b.cst8(0x30);
  b.iflt("not_decimal");
}

If $c$ > ``9'' (ASCII code \hexa{39}), go to $S_1$.

\bytecode[switchcolumn]{
  b.get("c");
  b.cst8(0x39);
  b.ifgt("not_decimal");
}

Otherwise, enter the new hex digit $d=c-\hexa{30}$ and return.

\bytecode[switchcolumn]{
  b.get("c");
  b.cst8(0x30);
  b.sub();
  b.call("med_new_digit");
  b.ret();
  b.label("not_decimal");
}

$S_1$: hexadecimal digits [a-f]. If $c$ < ``a'' (ASCII code \hexa{61}), go to
$S_2$.

\bytecode[switchcolumn]{
  b.get("c");
  b.cst8(0x61);
  b.iflt("not_hexadecimal");
}

If $c$ > ``f'' (ASCII code \hexa{66}), go to $S_2$.

\bytecode[switchcolumn]{
  b.get("c");
  b.cst8(0x66);
  b.ifgt("not_hexadecimal");
}

Otherwise, enter the new hex digit $d=c-\hexa{61}+10=c-\hexa{57}$ and return.

\bytecode[switchcolumn]{
  b.get("c");
  b.cst8(0x57);
  b.sub();
  b.call("med_new_digit");
  b.ret();
  b.label("not_hexadecimal");
}

$S_2$: Enter key. If $c$ is not Enter (ASCII code \hexa{0A}), go to $S_3$.

\bytecode[switchcolumn]{
  b.get("c");
  b.cst8(10);
  b.ifne("not_enter");
}

Otherwise, handle this key and return.

\bytecode[switchcolumn]{
  b.call("med_handle_enter");
  b.ret();
  b.label("not_enter");
}

$S_3$: ``w'' character. If $c$ is not ``w'' (ASCII code \hexa{77}), go to $S_4$.

\bytecode[switchcolumn]{
  b.get("c");
  b.cst8(0x77);
  b.ifne("not_w");
}

Otherwise, handle this key: set $address$ as the new $input$, set $mode$ to 1,
redraw the fields, and return.

\bytecode[switchcolumn]{
  b.call("med_get_address");
  b.call("med_set_input");
  b.cst_1();
  b.call("med_set_mode");
  b.call("med_draw_fields");
  b.ret();
  b.label("not_w");
}

$S_4$: ``r'' character. If $c$ is not ``r'' (ASCII code \hexa{72}), go to $S_5$.

\bytecode[switchcolumn]{
  b.get("c");
  b.cst8(0x72);
  b.ifne("not_r");
}

Otherwise, call the function at $address$, clear and redraw the screen (the
called function might have changed it), and return.

\bytecode[switchcolumn]{
  b.call("med_get_address");
  b.calld();
  b.call("gpu_clear_screen");
  b.call("med_get_address");
  b.call("med_new_address");
  b.ret();
  b.label("not_r");
}

$S_5$: ArrowLeft character. If $c$ is not ArrowLeft (character
\hexa{6B}+128=\hexa{EB}), go to $S_6$.

\bytecode[switchcolumn]{
  b.get("c");
  b.cst8(0xEB);
  b.ifne("not_arrow_left");
}

Otherwise, set $address + 4$ as the new address and return.

\bytecode[switchcolumn]{
  b.call("med_get_address");
  b.cst8(4);
  b.add();
  b.call("med_new_address");
  b.ret();
  b.label("not_arrow_left");
}

$S_5$: ArrowRight character. If $c$ is not ArrowRight (character
\hexa{74}+128=\hexa{F4}), go to $S_6$.

\bytecode[switchcolumn]{
  b.get("c");
  b.cst8(0xF4);
  b.ifne("not_arrow_right");
}

Otherwise, set $address - 4$ as the new address and return.

\bytecode[switchcolumn]{
  b.call("med_get_address");
  b.cst8(4);
  b.sub();
  b.call("med_new_address");
  b.ret();
  b.label("not_arrow_right");
}

$S_6$: ArrowUp character. If $c$ is not ArrowUp (character
\hexa{75}+128=\hexa{F5}), go to $S_7$.

\bytecode[switchcolumn]{
  b.get("c");
  b.cst8(0xF5);
  b.ifne("not_arrow_up");
}

Otherwise, set $address - 16$ as the new address and return.

\bytecode[switchcolumn]{
  b.call("med_get_address");
  b.cst8(16);
  b.sub();
  b.call("med_new_address");
  b.ret();
  b.label("not_arrow_up");
}

$S_7$: ArrowDown character. If $c$ is not ArrowDown (character
\hexa{72}+128=\hexa{F2}), go to $S_8$.

\bytecode[switchcolumn]{
  b.get("c");
  b.cst8(0xF2);
  b.ifne("not_arrow_down");
}

Otherwise, set $address + 16$ as the new address and return.

\bytecode[switchcolumn]{
  b.call("med_get_address");
  b.cst8(16);
  b.add();
  b.call("med_new_address");
  b.label("not_arrow_down");
}

$S_8$: any other character. Return.

\bytecode[switchcolumn]{
  b.ret();
}
\end{paracol}

\section{Main function}

We can finally implement the last function of the memory editor, and of our
basic input output system! This function initializes the drivers, sets the
initial $mode$ and $address$ to 0, and finally calls the previous function for
each new character typed on the keyboard, in an endless loop:

\begin{TwoColumns}
\rs{b.func("memory_editor", &[], "", &[])}\\
\bytecode{
  // b.call("boot_mode_select_rom");
  b.call("clock_init");
  b.call("gpu_init");
  b.call("keyboard_init");
  b.cst_0();
  b.call("med_set_mode");
  b.cst_0();
  b.call("med_new_address");
  b.label("med_loop");
  b.call("keyboard_wait_char");
  b.call("med_handle_char");
  b.goto("med_loop");
}
\end{TwoColumns}

Note that this function does {\em not} change the boot mode selection to boot
from ROM at the next reset. Indeed, with the memory editor, our toy computer is
now completely autonomous, and we no longer need the boot assistant nor an
external computer to use it. By putting together the code of all the functions
defined in this chapter we get the memory editor's full code:

\rs{b.get_bytecode_listing(0..b.get_instruction_count() as usize, false)}

Lets store it in flash memory. The boot assistant commands to do this are
provided in \verb!part2/memory_editor.txt!. Run them with:

\rust{
  let mut commands = Vec::new();
  commands.extend(b.boot_assistant_commands());
  commands.push(String::from("flash#"));
  context.add_memory_region("memory_editor", b.memory_region());
  write_lines("website/part2", "memory_editor.txt", &commands)?;
  let mut flash_helper = FlashHelper::from_file(
    context.micro_controller(), "website/", "part2/memory_editor.txt")?;
}
\rs{host_log(&flash_helper.read())}

\rust{
  let reset_handler = &context.memory_region("reset_handler");
  let bytecode_handler =
      reset_handler.label_address("bytecode handler program");
}

Note that the \verb!memory_editor! function is not stored at the \hexa{C2000}
address used so far for main functions (see \cref{fig:bios-memory-map}). To run
it on reset we need to change the \hexa{C2000} value, at address
\rs{hex(bytecode_handler)} in the Reset handler (see
\cref{subsection:bios-vector-table}), to the \verb!memory_editor! function
address, namely \rs{hex(b.label_address("memory_editor"))}. Do this as follows:

\rust{
  let mut flash_helper = FlashHelper::new(context.micro_controller());
  flash_helper.write(&format!("W{:X},{:X}#",
      bytecode_handler, b.label_address("memory_editor")));
  flash_helper.write("flash#");
}
\rs{host_log(&flash_helper.read())}

\rust{
  let mut native_size = 0;
  let mut native_insns = 0;
  let mut native_data = 0;
  let native = vec![
    context.memory_region("reset_handler"),
    context.memory_region("hard_fault_handler"),
    context.memory_region("usart0_handler"),
    context.memory_region("interpreter"),
  ];
  for region in &native {
    native_size += region.instruction_bytes + region.data_bytes;
    native_insns += region.instruction_count;
    native_data += region.data_bytes;
  }

  let bytecode = vec![
    context.memory_region("foundations"),
    context.memory_region("clock_driver"),
    context.memory_region("graphics_card_driver"),
    context.memory_region("keyboard_driver"),
    context.memory_region("memory_editor"),
  ];
  let mut bytecode_size = 0;
  let mut bytecode_insns = 0;
  let mut bytecode_data = 0;
  for region in &bytecode {
    bytecode_size += region.instruction_bytes + region.data_bytes;
    bytecode_insns += region.instruction_count;
    bytecode_data += region.data_bytes;
  }
}

The final layout of our basic input output system is shown in
\cref{fig:final-bios-memory-map}, and its most important functions are listed
in \cref{table:bios_functions}. In total, this system consists of
\rs{dec(bytecode_insns)} bytecode instructions, plus \rs{dec(bytecode_data)}
bytes of data, for a total size of \rs{dec(bytecode_size)} bytes. To which we
must add \rs{dec(native_insns)} Cortex M3 instructions for the virtual machine
interpreter and the Reset, Hard Fault and USART handlers, plus
\rs{dec(native_data)} bytes of data, for a total of \rs{dec(native_size)} bytes.

\begin{Figure}
  \rs{define("mmapa", &hex(context.memory_region("interpreter").start))}
  \rs{define("mmapb", &hex(context.memory_region("reset_handler").start))}
  \rs{define("mmapc", &hex(context.memory_region("hard_fault_handler").start))}
  \rs{define("mmapd", &hex(context.memory_region("usart0_handler").start))}
  \rs{define("mmape", &hex(context.memory_region("clock_driver").start))}
  \input{figures/chapter7/bios-memory-map.tex}

  \caption{The final layout of our basic input output system in flash memory.
  Red, blue and gray areas represent machine code, bytecode and unused
  memory, respectively (not to scale).}\label{fig:final-bios-memory-map}
\end{Figure}

\section{Experiments}

\begin{Table}[t]
  \begin{tabular}{|l|l|} \hline
    \makecell{\thead{Function}} & \thead{Address} \\ \hline
    \rs{MemoryRegion::labels_table_rows(bytecode)} \\ \hline
  \end{tabular}

  \caption{The most important functions of the basic input output
  system.}\label{table:bios_functions}
\end{Table}

You can now set the Arduino to boot from flash and restart it:

\rust{
  flash_helper.write("reset#");
}
\rs{host_log(&flash_helper.read())}

\noindent If all goes well, you should see the memory editor's user interface
on the screen, showing the first memory page, starting at address 0. Since we
removed the function call to change the boot mode selection, this memory page
is mapped to flash memory. More precisely, it is mapped to the page starting
\hexa{80000}, which contains the Vector Table. You should recognize, on the
first line, the Vector Table entries given at the end of
\cref{subsection:bios-vector-table}:

\rust{
  let display = Rc::new(RefCell::new(TextDisplay::default()));
  context.set_display(display.clone());

  context.run_until_get_char();
  let init_screen = display.borrow().get_text();

  let vm_address = context.memory_region("interpreter").start;
}

\begin{flushleft}
\rs{med_page_row(init_screen.lines().next().unwrap())}
\end{flushleft}

\noindent They are followed by empty entries (\hexa{FFFFFF}), except the one
for the USART handler. Lets try another memory location. On the Arduino's
keyboard, type ``w\rs{hex_word_low(vm_address)}'' followed by Enter. You should
now see a new memory page, starting with

\rust{
  context.type_ascii(&format!("W{:08X}\n", vm_address));
  let vm_screen = display.borrow().get_text();

  let row0 = vm_screen.lines().next().unwrap();
  let vm_words = &context.memory_region("interpreter").words;
  assert!(row0.starts_with(&format!(
    "{:08X} {:08X} {:08X} {:08X}",
    vm_words[3], vm_words[2], vm_words[1], vm_words[0])));

  let clock_driver_address = context.memory_region("clock_driver").start;
}
\begin{flushleft}
\rs{med_page_row(vm_screen.lines().next().unwrap())}
\end{flushleft}

\noindent This corresponds to the beginning of our virtual machine interpreter,
which is indeed stored at this address (see
\cref{subsection:interpreter-code}). Similarly, you can type
``w\rs{hex_word_low(clock_driver_address)}'' followed by Enter to look at the
clock driver code. You should see a new memory page starting with the code
shown at the end of \cref{section:delay-function}. In the same way, you can
display the graphics card and keyboard driver code, and even the memory editor
code.

\rust{
  context.type_ascii("W20070000\n");
  context.type_ascii("12345678\n");
  let ram_display = display.borrow().get_text();
  let ram_row = ram_display.lines().next().unwrap();
  assert!(ram_row.ends_with("12345678 20070000"));
}

Lets now try to store some values in memory. Type ``w20070000''+Enter to go the
RAM region. Then type ``12345678''+Enter to store this value at this address.
You should see the new value in the page view. We can use this method to
control the Arduino's LED, as we did in \cref{subsection:pio-experiments}, but
with our memory editor instead of the boot assistant. Type ``w400e1000''+Enter
to show the PIO B registers. You should note that all registers are 0, except
the PIO Status Register (\rs{hex(PIOB_PSR)}), as well as the Status, Output
Status, Output Data Status, and Pull-up Status Registers. The bit 12 of these
registers is 1, indicating that the PB12 pin is an output pin, controlled by
the microprocessor, currently set to 1, and with its pull-up resistor disabled.
Indeed, this is the pin used to reset the graphics card, configured in
\verb!gpu_reset! (see \cref{section:gpu-driver}). We can now redo the
experiments of \cref{subsection:pio-experiments}: press the ArrowDown key to
select the \rs{hex(PIOB_OER)} address and then type ``08000000''+Enter: you
should see the LED turning off. Using the arrow keys, select the
\rs{hex(PIOB_SODR)} address and type ``08000000''+Enter: you should see the LED
turning on again.

We can also try to display a reserved memory region. Reading the memory in
these regions causes an exception which should be handled by our Hard Fault
handler, which blinks the LED very fast. To test this type ``w00200000''+Enter
(see \cref{fig:boot-memory-map}). You should see the LED blinking. Moreover,
typing any key should have no effect, since the memory editor effectively
crashed. We could restart it by pressing the RESET button. Instead, to check
that our toy computer is completely autonomous, unplug it from your computer
and plug it to a power outlet with a phone charger. The memory editor should be
running again.

\rust{
  let foundations = context.memory_region("foundations");
  let boot_mode_select_address =
      foundations.label_address("boot_mode_select_rom");
  let boot_mode_select_address_str = format!("``w{:08x}''",
      boot_mode_select_address);
}

As a last experiment, lets try to run a function. We can use
\verb!boot_mode_select_rom! for this. Type
\rs{boot_mode_select_address_str}+Enter to go to this function (see
\cref{table:bios_functions}). Then type ``r''. Nothing changes on the screen,
but the Arduino is now configured to boot from ROM, \ie, to run the boot
assistant, at the next reset. To verify this, unplug the Arduino from the power
outlet and plug it again to your computer. The screen should stay off. We can
verify that the boot assistant is running as follows:

\rust{
  context.type_ascii(&format!("W{:08X}\n", boot_mode_select_address));
  context.type_ascii("R");
  context.micro_controller().borrow_mut().reset();
  let mut flash_helper = FlashHelper::new(context.micro_controller());
  flash_helper.write("V#");
}
\rs{host_log(&flash_helper.read())}

We can then change again the boot mode selection to run from flash, and reset
the Arduino, with:

\rust{
  flash_helper.write("reset#");
}
\rs{host_log(&flash_helper.read())}

At this point the memory editor should be running again. The above method can
be used to temporarily connect our toy computer to an external computer. This
can be useful, for instance, to do a backup of its flash memory. Such a backup
can be done by reading all the flash memory with boot assistant ``w'' commands,
and storing the result in a file on the host computer. This file can then be
used to restore the flash memory, with boot assistant ``W'' commands.
