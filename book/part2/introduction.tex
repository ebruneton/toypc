% This work is licensed under the Creative Commons Attribution NonCommercial
% ShareAlike 4.0 International License. To view a copy of the license, visit
% https://creativecommons.org/licenses/by-nc-sa/4.0/

\chapter*{Introduction}
\addcontentsline{toc}{chapter}{Introduction}

This part has two goals. The first one is to assemble the hardware components
of our toy computer, mainly a keyboard, a screen, and an {\em Arduino Due}
board. This board provides in a single chip a microprocessor, various memories,
and integrated circuits to control external devices (such as light emitting
diodes, motors, sensors, but also keyboards or displays). We present how these
components work, how to connect them together, and how programs can use them.
The second goal of this part is to install on this assembled computer a very
basic input and output system, using the keyboard and the screen, in order to
make it completely autonomous.

Normally the Arduino Due is {\em not} autonomous: it requires an external
computer to be used. The usual process is the following. Users write their
application (\eg, a mobile robot controller) with a text editor in some
programming language. They transform it with a program called a compiler into
machine code that the Arduino can execute. Finally, they send this machine code
with a third program to the Arduino, as a series of 0s and 1s (via a USB
cable). All this happens on an existing computer, with already existing
programs (operating system, text editor, compiler, etc).

Since our goal in this book is to program a toy computer {\em from scratch}, we
should in theory avoid using any existing computer, already programmed.
Otherwise we would need to show how this existing computer was programmed in
the first place (the answer is from yet another already programmed computer --
and so on). To solve this chicken and egg problem, we should instead send 0s
and 1s ``manually'' to the Arduino. Doing this completely by hand, with a
switch, is not possible because the Arduino expects to receive these 0s and 1s
by groups of 8 bits, each group being transmitted at 115,000 bits per second
(\ie, in about 70$\mu s$). Of course, no one can operate a switch at this
speed. Instead, we could build a small digital circuit, connected to a
keyboard, which would send a specific group of 8 bits at 115,200 bits per
second each time a key is pressed (for instance the group \bina{01000000} for
'A', \bina{01000001} for 'B', etc). This circuit would not need to be
programmed, which would avoid the above chicken and egg problem. However, doing
this would be very impractical and error-prone (typos would be hard to detect
without a visual feedback, \ie, some kind of display).

In this part, we therefore use an external computer to program the Arduino.
However, we try to use it in a minimal way to show convincingly that avoiding
its use altogether would be possible. In particular, instead of using a text
editor and a compiler to produce the 0s and 1s from a program written in some
programming language, we compute these bits manually. The only program we use
on the external computer is the one used to send these bits to the Arduino, at
the expected speed. Moreover, we use this method only to install a small
initial program on the Arduino, namely a very basic input and output system.
Its goal is to read other programs (still in binary form at first) input on a
keyboard connected to the Arduino, to output them on a screen also connected to
the Arduino, and to execute them. In other words, its goal is to make our toy
computer completely autonomous, \ie, to avoid any further need of an external
computer (including in \cref{part:compiler,part:operating-system}).

In order to do this, our basic input output system has the following components:
\begin{itemize}
	\item a keyboard driver. Connecting a keyboard to the Arduino does not ``just
	work''. A small program is needed on the Arduino to decode the signals sent
	by the keyboard and to interpret them as characters. This small program is
	called a keyboard driver.

	\item a ``graphics card'' driver. Similarly, connecting a display to the
	Arduino does not ``just work''. For the display we want to use, things are
	even worse: we can't even connect the display directly to the Arduino because
	it does not have the necessary connector. To solve this we use an
	intermediate board, which we call the graphics card. It can connect to the
	display on one end, and to the Arduino on the other. But here again, a small
	program is needed on the Arduino to send the correct signals to the graphics
	card, in order to display the desired characters on the screen. This small
	program is the graphics card driver.

	\item a memory editor. This extremely basic editor uses the above drivers to
	display the memory's content on the screen (in hexadecimal format), and to
	allow the user to edit it, with the keyboard. It can also execute a program
	at a given location in memory. These features allow users to store programs
	in memory and to run them.

	\item a virtual machine. Even if the above programs are small, they would
	still require hundreds of machine code instructions, which use a complicated
	binary format. Writing all these instructions manually is possible but
	painful and error prone. In order to simplify this a bit, we use a tiny {\em
	virtual machine}. This program simulates a virtual microprocessor using very
	simple instructions called {\em bytecode instructions}. This small program
	must be written in Arduino Due's machine code but, once this is done, all
	other programs can be written in simpler bytecode instructions (simulated by
	the virtual machine). This is what we do for the above drivers and the memory
	editor.
\end{itemize}

The rest of this part presents the hardware components of our toy computer,
shows how they are assembled, and explains how our input output system is built
and installed on the Arduino. It is organized as follows:
\begin{itemize}
  \item \cref{chapter:arduino-due} gives an overview of the Arduino Due board
  and of its main chip. We use this at the end to control a LED with commands
  sent ``manually'' to the Arduino.

  \item \cref{chapter:microprocessor} gives an overview of the microprocessor
  inside the Arduino's main chip. It presents a subset of its registers and
  instructions. We use this at the end to blink a LED, with a first program
  made of a few machine code instructions.

  \item \cref{chapter:virtual-machine} builds on this to implement our toy
  virtual machine, written in Arduino's machine code, and to store it in flash
  memory.

  \item \cref{chapter:clock} presents the clock used by the Arduino, and
  explains how to change its frequency. We then use this knowledge to implement
  a small program, using bytecode instructions, to set the clock to its maximum
  frequency.

  \item \cref{chapter:screen} presents how Liquid
  Crystal Displays work in general, how our specific Thin Film Transistor
  display works, what the graphics card is doing, how to communicate with it
  from the Arduino, and how to use it. It then provides the graphics card
  driver implementation. We test it at the end to display the traditional
  ``Hello, World!'' message.

  \item \cref{chapter:keyboard} presents how keyboards work in general, and
  PS/2 keyboards in particular. It then provides a small keyboard driver. We
  test it at the end with a small ``echo'' program, which simply displays on
  screen each key typed on the keyboard.

  \item \cref{chapter:memory-editor} uses all the previous components to
  implement a basic memory editor, finally making our toy computer completely
  autonomous.
\end{itemize}

