% This work is licensed under the Creative Commons Attribution NonCommercial
% ShareAlike 4.0 International License. To view a copy of the license, visit
% https://creativecommons.org/licenses/by-nc-sa/4.0/

\chapter*{Conclusion}
\addcontentsline{toc}{chapter}{Conclusion}

A microcontroller contains in a single chip a microprocessor, various memories,
and specialized circuits to communicate and interact with the external world.
In this part we connected an Arduino Due, based on an Atmel microcontroller, to
a Liquid Crystal Display (LCD), and to a keyboard. We used its boot program in
read-only memory, via an external computer, to store in its flash memory a
basic memory editor program. Like the Atmel's boot program, this editor allows
us to input programs and to run them. However, it does this by using the
keyboard and the LCD connected to the Arduino, \ie, without needing any
external computer. Our toy computer is thus fully autonomous. The rest of this
book illustrates this by progressively improving it, to make it more and more
usable, without using any external computer.

\subsubsection{Further readings}

This part barely scratched the surface of what microcontrollers can do. And it
only presented a very low level method to program them (because the goal of
this book it to program one from scratch). To know more about what
microcontrollers can do, how they work, and how to use them in more convenient
ways, you can read the following books:
\begin{itemize}
  \item ``Arduino Workshop, 2nd Edition: A Hands-on Introduction with 65
  Projects'' \cite{ArduinoWorkshop}. This book gives a very practical
  introduction to the Arduino microcontrollers. It explains how they can be
  programmed with the Arduino Integrated Development Environment, and shows how
  they can be used with many external components (including 7-segments
  displays, micro-SD cards, keypads, touchpads, motors, infrared sensors,
  GPS modules, external memories, etc).

  \item ``Embedded System Design with ARM Cortex-M Microcontrollers:
  Applications with C, C++ and MicroPython'' \cite{EmbeddedSystemDesign}. This
  book explains how microcontrollers work in general, and gives practical
  examples illustrating each aspect. For instance, it presents digital and
  analog signals, conversions between digital and analog signals, interrupts,
  clocks and timers, communication protocols, etc.

  \item ``Embedded Systems Fundamentals with Arm Cortex-M Based
  Microcontrollers: A Practical Approach'' \cite{EmbeddedSystemsFundamentals}.
  This book also gives ``theoretical'' and practical information about
  microcontrollers, with an emphasis on how to use them ``properly'' (\ie, to
  get efficient and responsive programs with low power requirements).
\end{itemize}

These books use programs written in textual form, which is much more
practical than hexadecimal numbers representing machine code or bytecode
instructions. Some of the above books introduce the {\em programming language}
that they use for this, at the same time as they present microcontrollers. But
none of them explains in detail how a program written in textual form can run
on a microprocessor, which can only execute machine code instructions. This
process is introduced in the next part, based on a toy programming language for
our toy computer.