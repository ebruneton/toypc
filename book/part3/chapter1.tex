% This work is licensed under the Creative Commons Attribution NonCommercial
% ShareAlike 4.0 International License. To view a copy of the license, visit
% https://creativecommons.org/licenses/by-nc-sa/4.0/

\renewcommand{\rustfile}{chapter1}
\setcounter{rustid}{0}

\rust{
  context.write_backup("website/backups", "flash_memory_driver.txt")?;
}

\chapter{Flash Memory Driver}\label{chapter:flash-driver}

Before writing our toy compiler, as described in introduction, we need a way to
save it in flash memory. In theory, our memory editor provides everything we
need to do this. Indeed, as we have seen in the previous part, we can save a 64
words page in flash memory by writing these words at their final address, and
then by writing an appropriate value in the Enhanced Embedded Flash Controller
(EEFC) Command Register (which has a well defined address in memory). All these
steps can be done manually with the memory editor, but doing so would not be
very practical. To make this is easier we provide in this chapter a few helper
functions, called the flash memory driver. We use them at the end to store
themselves in flash memory.

\section{Overview}

Due to the flash memory usage constraints, it is not practical to directly edit
data here. Instead, what we can do is edit data in RAM, and then save it in
flash memory. The latter step can be done automatically (see
\cref{fig:flash-driver-overview}):
\begin{itemize}
\item copy the first 64 words of data from RAM to flash memory (we assume in
this chapter that data is always saved at the beginning of a flash memory page),
\item save them by writing the appropriate command in the EEFC Command Register,
\item ...
\item copy the last $n$ words of data from RAM to flash memory, and the $64-n$
remaining words from flash memory to itself (recall that we cannot save a page
without writing all its 64 words first),
\item save them by writing the appropriate command in the EEFC Command Register.
\end{itemize}

Conversely, to edit data which is already in flash memory, we can copy it in
RAM, edit it here, and then save it back. Note that these algorithms use two
kinds of steps: copying memory from one address to another, and saving a page
of flash memory. The rest of this section presents them in more detail.

\begin{Figure}
  \input{figures/chapter1/overview.tex}

  \caption{Saving data (in dark blue) from RAM into flash memory must be done
  page by page. Each 64 words page must be copied (dashed arrows) and then
  saved. The unused words of the last page (light blue) must be copied in place
  so that all the page words are written before it is saved (as required by the
  EEFC component).}\label{fig:flash-driver-overview}
\end{Figure}

\subsection{Memory copy}

As explained above, our flash memory driver needs a function to copy some data
from one address to another. Our text editor, presented in the next chapter,
also needs such a function. The former only needs to copy words, between two
distinct memory regions. But the latter needs to copy bytes, between two
regions which can overlap. To support both use cases, we present here a general
memory copy algorithm.

The basic algorithm to copy $n$ bytes starting at address $src$ to address
$dst$ is very simple. We just need to store the byte loaded from address $src +
i$ to address $dst + i$, for all $i \in [0,n[$. However, the order in which
these operations are done is important if the source and destination regions
overlap. Consider for instance the task of copying $n=10$ bytes from $src=4$ to
$dst=7$ (see \cref{fig:mem-copy}). Starting by copying the byte at $src+0$ to
$dst+0$ would override the byte at $src+3$, leading to an incorrect result. The
solution is to copy the bytes in decreasing order, from $i=n-1$ to $i=0$.
Conversely, copying $n=10$ bytes from $src=7$ to $dst=4$ must be done in
increasing order (starting by copying the byte at $src+9$ to $dst+9$ would
override the byte $src+6$). In summary, bytes must be copied in decreasing
order if $dst \ge src$, and in increasing order otherwise. Note also that we
can start by copying words, and use byte copies only for the last 1 to 3
remaining bytes, leading to \cref{alg:mem-copy}.

\begin{Algorithm}
\caption{Copying $n$ bytes from $src$ to $dst$.}\label{alg:mem-copy}
\begin{algorithmic}[1]
\Begin if $dst < src$
  \State initialize $i$ to 0
  \State while $i+4 \le n$, copy the word at $src+i$ to $dst+i$ and then
  increment $i$ by 4
  \State while $i < n$, copy the byte at $src+i$ to $dst+i$ and then increment
  $i$ by 1
\Continue otherwise
  \State initialize $i$ to n
  \State while $i \ge 4$, decrement $i$ by 4 and then copy the word at $src+i$
  to $dst+i$
  \State while $i > 0$, decrement $i$ by 1 and then copy the byte at $src+i$ to
  $dst+i$
\End
\end{algorithmic}
\end{Algorithm}

\begin{Figure}
  \input{figures/chapter1/mem-copy.tex}

  \caption{Copying 10 bytes (dark blue) from address 4 to address 7 in
  increasing order (\ie, from byte 4 to byte 13) leads to incorrect results (in
  red). Copying them in decreasing order, from byte 13 to byte 4, solves the
  problem.}\label{fig:mem-copy}
\end{Figure}

\subsection{Page flash}\label{subsection:flash-subroutine}

Once the 64 words of a page have been copied from RAM to flash memory, they can
be saved by writing the appropriate value in the EEFC Command Register. One
must then wait until the EEFC Status Register value is 1, indicating that the
operation is done. During this time, the flash memory bank must not be used
(see \cref{section:flash-controller}). Unfortunately, our virtual machine {\em
is} in flash memory. Therefore, we can't use a bytecode function to read the
EEFC Status Register (running it would require the microprocessor to read the
ARM instructions of the virtual machine, \ie, would use the flash memory). A
solution is to use a small subroutine made of ARM instructions, stored in RAM,
to save a page in flash memory without using it. This subroutine can be
implemented as follows:

\rust{
  const R0 : u32 = 0;
  const R1 : u32 = 1;
  let mut a = Assembler::new(0);
  a.push_list(&[R0, R1], true);
  a.ldr_rt_pc_imm8(R0, "command_register");
  a.ldr_rt_pc_imm8(R1, "command_value");
  a.str_rt_rn_imm5(R1, R0, 0); // stores R1 in mem[R0 + 0]
  a.label("wait_ready");
  a.ldr_rt_rn_imm5(R1, R0, 4); // loads mem[R0 + 4] into R1
  a.cmp_rn_imm8(R1, 1);
  a.if_then(Condition::NE, &[]);
  a.b_imm11("wait_ready");
  a.pop_list(&[R0, R1], true);
  a.u16_data(0, "padding, unused");
  a.label("command_register");
  a.u32_data(EEFC1_FCR.address, "EEFC1 Command Register");
  a.label("command_value");
  a.u32_data(0x5A000003, "EEFC1 Command"); // Command value (must add page << 8)
}
\rs{a.get_listing(0..a.get_instruction_count() as usize)}

It starts with a \arm{PUSH} to save the $\mathrm{R0}$ and $\mathrm{R1}$
registers, as well as the Link Register (LR). It then loads the address of the
EEFC Command Register in $\mathrm{R0}$, and the value to write into it in
$\mathrm{R1}$, with two \arm{LDR} instructions (this data is stored just after
the function itself). The next instruction actually stores this value in the
Command Register, thereby starting the flashing process. The next \arm{LDR}
instruction loads the value of the EEFC Status Register (whose address is 4
bytes after the Command Register address, \ie, $\mathrm{R0}+4$). The following
\arm{CMP} instruction compares this value with 1. If it is not equal to 1, the
\arm{B} instruction jumps back to the \arm{LDR} instruction to read the Status
Register again. Otherwise this instruction is skipped and the final \arm{POP}
restores the $\mathrm{R0}$ and $\mathrm{R1}$ registers, and returns to the
caller by moving the saved LR into the Program Counter (PC). The first data
word after that contains the address of the EEFC1 Command Register (we want to
store our compiler in the same flash memory bank as our basic input output
system). The final word is the command to write at this address in order to
flash the 0$^{th}$ page. To flash the $p^{th}$ page instead, $p$ should be put
in bytes 1 and 2 of this word (\eg, to flash the $4^{th}$ page, use
\hexa{5A000403} -- see \cref{section:flash-controller}). In summary, the
complete code of this subroutine is:

\rs{a.get_machine_code_listing(0..a.get_instruction_count() as usize)}

\subsection{Data buffers}\label{subsection:data-buffer}

In order to copy or save data we must know their address, but we must also know
their size in bytes. To avoid having to manually keep track of this size, we
can store it in memory too. One way to do this is to store it in a word located
just before the data themselves (see \cref{fig:data-buffer}). This word is
called a {\em header} or a {\em metadata} (because it is some data about other
data). In the following we call this header and its associated data a {\em data
buffer}. And we provide functions to copy and save data buffers.

\begin{Figure}
  \input{figures/chapter1/data-buffer.tex}

  \caption{A data buffer containing 6 bytes of data (light blue), starting at
  address 10, begins with a 4 bytes header (dark blue) containing the size of
  the following data.}\label{fig:data-buffer}
\end{Figure}

\section{Implementation}\label{section:flash-memory-driver-impl}

\rust{
  let driver_address = next_page_address(
      context.memory_region("memory_editor").end());
}

We can now implement the above algorithms. We do this in a data buffer, so that
our flash memory driver can save itself. This buffer must be saved at the start
of a page, as we assumed at the beginning. Lets use the next page after our
memory editor, page \rs{dec(page_number(driver_address))}, starting at
\rs{hex(driver_address)}. The code will thus start at
\rs{hex(driver_address + 4)}, after the header.

For \cref{alg:mem-copy} we need functions to load and store a single byte. We
already have a \verb!load_byte! function (see \cref{table:bios_functions}), but
we don't have a \verb!store_byte! one. We thus provide one, as follows:

\rust{
  let mut b = BytecodeAssembler::new(RegionKind::DataBuffer, driver_address);
  b.import_labels(context.memory_region("graphics_card_driver"));
}

\begin{TwoColumns}
\rs{b.func("store_byte", &["ptr", "value"], "", &[])}\\
\bytecode{
  // *ptr = (*ptr) & 0xFFFFFF00 | value
  b.get("ptr");
  b.get("ptr");
  b.load();
  b.cst(0xFFFFFF00);
  b.and();
  b.get("value");
  b.or();
  b.store();
  b.ret();
}
\end{TwoColumns}

This function loads the word at $ptr$, discards its 8 least significant bits
(while keeping the others unchanged) by computing the bitwise AND of this word
with \hexa{FFFFFF00}, replaces them with $value$ (supposed to be strictly less
than 256) with a bitwise OR, and finally stores the result back at $ptr$. We
then implement \cref{alg:mem-copy} in the following \verb!mem_copy!
function:

\begin{Paragraph}
\begin{paracol}{2}
\rs{b.func("mem_copy", &["src", "dst", "n"], "dst+n", &[])}

Step 1. If $dst \ge src$, go the second half of this function (see below).

\bytecode[switchcolumn]{
  // if dst < src
  b.get("dst");
  b.get("src");
  b.ifge("mem_copy_forward");
}

Step 2. Initialize $i$ to 0.

\bytecode[switchcolumn]{
  b.cst_0();
  b.def("i");
}

Step 3. If $i+4>n$, go to step 4 ($i$ is stored in the $7^{th}$ stack frame
slot).

\bytecode[switchcolumn]{
  b.label("mem_copy_while1");
  // while i + 4 <= n
  b.get("i");
  b.cst8(4);
  b.add();
  b.get("n");
  b.ifgt("mem_copy_while2");
}

Otherwise, load the word at $src+i$ and store it at $dst+i$, $\ldots$

\bytecode[switchcolumn]{
  // *(dst + i) = *(src + i);
  b.get("dst");
  b.get("i");
  b.add();
  b.get("src");
  b.get("i");
  b.add();
  b.load();
  b.store();
}

$\ldots$ increment $i$ by 4, and go back above to check again if $i+4<n$.

\bytecode[switchcolumn]{
  // i = i + 4;
  b.cst8(4);
  b.add();
  // end while
  b.goto("mem_copy_while1");
}

Step 4. If $i \ge n$, go the end of the function (see below).

\bytecode[switchcolumn]{
  b.label("mem_copy_while2");
  // while i < n
  b.get("i");
  b.get("n");
  b.ifge("mem_copy_end");
}

Otherwise, load the byte at $src+i$ and store it at $dst+i$, $\ldots$

\bytecode[switchcolumn]{
  // store_byte(dst + i, load_byte(src + i));
  b.get("dst");
  b.get("i");
  b.add();
  b.get("src");
  b.get("i");
  b.add();
  b.call("load_byte");
  b.call("store_byte");
}

$\ldots$ increment $i$ by 1, and go back above to check again if $i<n$.

\bytecode[switchcolumn]{
  // i = i + 1;
  b.cst_1();
  b.add();
  b.goto("mem_copy_while2");
}
\end{paracol}
\end{Paragraph}

The second half of the function is similar, and handles the case $dst \ge src$
by copying data in decreasing order, as described in \cref{alg:mem-copy}:

\begin{TwoColumns}
\bytecode{
  b.label("mem_copy_forward");
  // let i = n;
  b.get("n");
  b.label("mem_copy_while3");
  // while i >= 4
  b.get("i");
  b.cst8(4);
  b.iflt("mem_copy_while4");
  // i = i - 4;
  b.cst8(4);
  b.sub();
  // *(dst + i) = *(src + i);
  b.get("dst");
  b.get("i");
  b.add();
  b.get("src");
  b.get("i");
  b.add();
  b.load();
  b.store();
  // end while
  b.goto("mem_copy_while3");
  b.label("mem_copy_while4");
  // while i > 0
  b.get("i");
  b.cst_0();
  b.ifle("mem_copy_end");
  // i = i - 1;
  b.cst_1();
  b.sub();
  // store_byte(dst + i, load_byte(src + i));
  b.get("dst");
  b.get("i");
  b.add();
  b.get("src");
  b.get("i");
  b.add();
  b.call("load_byte");
  b.call("store_byte");
  // end while
  b.goto("mem_copy_while4");
  b.label("mem_copy_end");
}
\end{TwoColumns}

Both parts jump to the following final instructions when the copy is done.
These instructions simply return $dst+n$:

\begin{TwoColumns}
\bytecode{
  // return dst + n;
  b.get("dst");
  b.get("n");
  b.add();
  b.retv();
}
\end{TwoColumns}

Using this memory copy function, it is easy to write a function to copy a data
buffer from $src$ to $dst$. Indeed, we simply need to call \verb!mem_copy! with
$src$, $dst$, and $n=\mathrm{mem32}[src]+4$, the total size of the data buffer
(recall that $\mathrm{mem32}[x]$ means ``the 32 bit value at address $x$''):

\begin{TwoColumns}
\rs{b.func("buffer_copy", &["src", "dst"], "", &[])}\\
\bytecode{
  // mem_copy(src, dst, *src + 4);
  b.get("src");
  b.get("dst");
  b.get("src");
  b.load();
  b.cst8(4);
  b.add();
  b.call("mem_copy");
  b.ret();
}
\end{TwoColumns}

We can now implement a function to copy and save a single page in flash memory.
As described above, to save $n \le 256$ bytes, we must first copy them, then
copy the remaining $256-n$ bytes of the page in place, and finally save the
page by calling the subroutine defined in \cref{subsection:flash-subroutine}.
For this, the subroutine must be stored somewhere in RAM first. The easiest
solution is to store it on the stack. The following function uses this method
to save $n$ bytes starting from $src$ in a page of the Flash1 memory bank
specified by its $page$ index:

\begin{Paragraph}
\begin{paracol}{2}
\rs{b.func("page_flash", &["src", "page", "n"], "", &[])}

If $n=0$ there is nothing to do, return right away. Otherwise execute the
following instructions.

\bytecode[switchcolumn]{
  // if n == 0  return;
  b.get("n");
  b.cst_0();
  b.ifne("page_flash_non_empty");
  b.ret();
  b.label("page_flash_non_empty");
}

Copy $n$ bytes from $src$ to $\hexa{C0000}+256.page$, the address of the
$page^{th}$ page of the Flash1 memory bank. The \verb!mem_copy! call returns
$dst=\hexa{C0000}+256.page+n$, in the $7^{th}$ stack frame slot.

\bytecode[switchcolumn]{
  // let dst = mem_copy(src, (0xC0000 as *u32) + (page << 8), n);
  b.get("src");
  b.cst(0xC0000);
  b.get("page");
  b.cst8(8);
  b.lsl();
  b.add();
  b.get("n");
  b.callr("mem_copy");
  b.def("dst");
}

Copy the remaining $256-n$ bytes of the page in place, from $dst$ to $dst$, and
discard the result returned by \verb!mem_copy!.

\bytecode[switchcolumn]{
  // mem_copy(dst, dst, 256 - n);
  b.get("dst");
  b.get("dst");
  b.cst(0x100);
  b.get("n");
  b.sub();
  b.callr("mem_copy");
  b.pop();
}

Disable the USART interrupts with the Nested Vector Interrupt Controller (see
\cref{section:nvic} and below).

\bytecode[switchcolumn]{
  // *(NVIC_ICER0 as *u32) = 0x20000;
  b.cst(NVIC_ICER0);
  b.cst(0x20000);
  b.store();
}

Push the value to store in the EEFC1 Command Register in order to save the
$page^{th}$ page: $\hexa{5A000003}\ |\ (page \ll 8)$.

\bytecode[switchcolumn]{
  // let command = 0x5A000003 | page << 8;
  let asm_words = a.machine_code();
  b.cst(asm_words[asm_words.len() - 1]);
  b.get("page");
  b.cst8(8);
  b.lsl();
  b.or();
  b.def("command");
  b.comment("");
}

Push the remaining words of the subroutine to save this page. These words must
be pushed in reverse order, because each word is pushed 4 bytes {\em before}
the previous one.

\bytecode[switchcolumn]{
  for i in (0..asm_words.len() - 1).rev() {
    b.cst(asm_words[i]);
    b.def(&if i == 0 { String::new() } else { format!("word{i}") });
    b.comment("");
  }
}

Call the subroutine, which starts in the
\rs{dec(7 + asm_words.len() as u32)}$^{th}$ stack frame slot. Its {\em
interworking address} is the address of this slot (given by the \insn{ptr}
instruction), plus 1.

\bytecode[switchcolumn]{
  b.ptr("");
  b.cst_1();
  b.add(); // add one for interworking address
  b.blx();
}

Re-enable the USART interrupts and return.

\bytecode[switchcolumn]{
  // *(NVIC_ISER0 as *u32) = 0x20000;
  b.cst(NVIC_ISER0);
  b.cst(0x20000);
  b.store();
  b.ret();
}
\end{paracol}
\end{Paragraph}

\noindent A few things should be noted:
\begin{itemize}
\item USART interrupts are temporarily disabled while the page is being saved.
Without this, a key press or release during this time would run the
\verb!keyboard_handler!, which would make use of the flash memory. In turn,
this would cause a Hard Fault because flash memory must not be used while a
page is being saved. Unfortunately, flashing a page takes a few milliseconds,
during which several interrupts could occur. In this case they are lost, except
the last one, which can confuse the keyboard driver. For instance, releasing
the ``r'' key causes two interrupts, for the \hexa{F0} and \hexa{2D} scancodes.
If the first is lost, this appears as a key press (see
\cref{appendix:scancodes}). This problem disappears in the next part.

\item we use \insn{callr} instead of \insn{call} instructions to call
\verb!mem_copy!. The next section explains why (these instructions use an
offset from their own address, instead of an offset from \hexa{C0000} -- see
\cref{subsection:fn-instructions}).

\item $n$ must be a multiple of 4, so that \verb!mem_copy! does not call
\verb!store_byte!. Indeed, \verb!store_byte! does not work in flash memory,
because loads do not ``see'' the effect of stores until the page is saved (see
\cref{subsection:page-write}). If it was called several times to store the
bytes of a word, only the last call would have any effect.
\end{itemize}

We can finally implement the last function of our flash memory driver, which
copies a data buffer starting at $src$ and saves it in the Flash1 memory bank,
starting at the $page^{th}$ page. This function simply calls \verb!page_flash!
for each page.

\begin{Paragraph}
\begin{paracol}{2}
\rs{b.func("buffer_flash", &["src", "page"], "", &[])}

Compute the number of bytes $n$ to copy. This is $\mathrm{mem32}[src]+4$,
rounded upwards to a multiple of 4 (as required by \verb!page_flash!), \ie,
$(\mathrm{mem32}[src]+7)\ \wedge$ \hexa{FFFFFFFC}.

\bytecode[switchcolumn]{
  // let n = (*src + 7) & 0xFFFFFFFC;
  b.get("src");
  b.load();
  b.cst8(7);
  b.add();
  b.cst(0xFFFFFFFC);
  b.and();
  b.def("n");
}

If $n$, in the $6^{th}$ stack frame slot, is greater than 255, jump to the next
instructions. Otherwise, call \verb!page_flash! to copy and save $n$ bytes from
$src$ into $page$, and return.

\bytecode[switchcolumn]{
  b.label("buffer_flash_loop");
  // if n <= 255
  b.get("n");
  b.cst8(255);
  b.ifgt("buffer_flash_full_page");
  // page_flash(src, page, n); return;
  b.get("src");
  b.get("page");
  b.get("n");
  b.callr("page_flash");
  b.ret();
  b.label("buffer_flash_full_page");
}

Call \verb!page_flash! to copy and save $256$ bytes from $src$ into $page$.

\bytecode[switchcolumn]{
  // page_flash(src, page, 256);
  b.get("src");
  b.get("page");
  b.cst(0x100);
  b.callr("page_flash");
}

Increment $src$ by 256.

\bytecode[switchcolumn]{
  // src = src + 256;
  b.get("src");
  b.cst(0x100);
  b.add();
  b.set("src");
}

Increment $page$ by 1.

\bytecode[switchcolumn]{
  // page = page + 1;
  b.get("page");
  b.cst_1();
  b.add();
  b.set("page");
}

Decrement $n$ by 256 and go back above to copy the rest of the data buffer.

\bytecode[switchcolumn,bigskip]{
  // n = n - 256;
  b.cst(0x100); // 55
  b.sub();
  // end loop
  b.goto("buffer_flash_loop");
}
\end{paracol}
\end{Paragraph}

In summary, the main functions of our flash memory driver are those listed in
\cref{table:flash_driver_functions}, and its full code is the following:

\rs{b.get_bytecode_listing(0..b.get_instruction_count() as usize, false)}

\rust{
  let mut commands = Vec::new();
  commands.extend(b.boot_assistant_commands());
  commands.push(String::from("flash#"));
  commands.push(String::from("reset#"));
  write_lines("website/part3", "flash_memory_driver.txt", &commands)?;
}

\begin{Table}
  \begin{tabular}{|l|l|} \hline
    \makecell{\thead{Function}} & \thead{Address} \\ \hline
    \rs{MemoryRegion::labels_table_rows(vec![&b.memory_region()])} \\ \hline
  \end{tabular}

  \caption{The most important functions of the flash memory
    driver.}\label{table:flash_driver_functions}
\end{Table}

\section{Storage}\label{section:flash-driver-storage}

\rust{
  const RAM_ADDRESS: u32 = 0x20070000;
  let mut context1 = context.clone();
}

Lets store our driver in flash memory. For this we must first enter it in RAM,
say at address \rs{hex(RAM_ADDRESS)}, and then save it by calling the
\verb!buffer_flash! function. In the memory editor, type
``w\rs{hex_word_low(RAM_ADDRESS)}''+Enter, and then store the size of our
driver at this address by typing
``\rs{hex_word_low(b.bytecode_size())}''+Enter. Continue by entering each word
of the driver code, listed above, by typing its value followed by Enter.

\rust{
  let display = Rc::new(RefCell::new(TextDisplay::default()));
  context.set_display(display.clone());

  context.add_memory_region("flash_driver", b.memory_region());
  context.micro_controller().borrow_mut().reset();
  context.run_until_get_char();

  context.type_ascii(&b.memory_editor_commands(RAM_ADDRESS));
}

Our driver is now in RAM. To save it in flash memory we must
call \verb!buffer_flash!, at address
$\rs{hex(b.label_address("buffer_flash"))} - \rs{hex(driver_address)} +
\rs{hex(RAM_ADDRESS)}$, with $src=\rs{hex(RAM_ADDRESS)}$ and
$page=\rs{dec(page_number(driver_address))}$. This can be done with the
following function:

\rust{
  let mut c = BytecodeAssembler::default();
}
\begin{TwoColumns}
\rs{c.func("save_driver", &[], "", &["nolink"])}\\
\bytecode{
  c.cst(RAM_ADDRESS);
  c.cst8(page_number(driver_address).try_into().unwrap());
  c.cst(b.label_address("buffer_flash") - driver_address + RAM_ADDRESS);
  c.calld();
  c.ret();
}
\end{TwoColumns}

Note that the call to \verb!buffer_flash! causes indirect calls to
\verb!page_flash! and \verb!mem_copy! which, for now, are in RAM. Hence, the
instructions calling these functions cannot use their final address in flash
memory, since they are not stored there yet! This is why we used \insn{callr}
instructions instead of \insn{call} instructions in the above code. Indeed, by
specifying the callee with an offset from the caller, the code works wherever
it is stored, in RAM or in flash memory. Such code is called {\em position
independent code}. The full code of the above function is the following:

\rs{c.get_bytecode_listing(0..c.get_instruction_count() as usize, false)}

\rust{
  const COMMAND_ADDRESS: u32 = 0x20080000;
}

\noindent With the memory editor, enter these values in an unused RAM region,
for instance starting at address \rs{hex(COMMAND_ADDRESS)}. Then type
``w\rs{hex_word_low(COMMAND_ADDRESS)}''+Enter, followed by ``r'', to run this
function. The driver should now be saved in flash memory. To check this, type
``w\rs{hex_word_low(driver_address)}''+Enter. You should see the following
screen, displaying the same words as those listed above, after the data buffer
header:

\rust{
  context.type_ascii(&c.memory_editor_commands(COMMAND_ADDRESS));
  context.type_ascii(&format!("W{:08X}\n", COMMAND_ADDRESS));
  context.type_ascii("R");
  context.type_ascii(&format!("W{:08X}\n", driver_address));
}

\rust{
  let driver_display = display.borrow().get_text();
  let first_line = driver_display.lines().next().unwrap();
  let second_line = driver_display.lines().nth(1).unwrap();
  assert!(first_line.ends_with(&format!("{:08X} {:08X} {:08X}",
      context.memory_region("flash_driver").words[0],
      context.memory_region("flash_driver").len - 4,
      driver_address)));
}

\begin{Paragraph}
\rs{med_page_row(first_line)}\\
\rs{med_page_row(second_line)}\\
{\tt ...}
\end{Paragraph}

\rust{
  let display1 = Rc::new(RefCell::new(TextDisplay::new()));
  context1.set_display(display1.clone());
  context1.micro_controller().borrow_mut().reset();
  context1.run_until_get_char();

  let boot_mode_address = context.memory_region("foundations")
      .label_address("boot_mode_select_rom");
}

Alternatively, if something went wrong or if you don't want to enter all the
driver code with the keyboard, you can ``cheat'' by saving it via an external
computer, as follows. First run the \verb!boot_mode_select_rom! function by
typing ``w\rs{hex_word_low(boot_mode_address)}''+Enter, followed by ``r''. Then
reset the Arduino and, on the host computer, run the following commands to
flash the driver code and reset the Arduino again:

\rust{
  context1.type_ascii(&format!("W{:08X}\n", boot_mode_address));
  context1.type_ascii("R");
  context1.micro_controller().borrow_mut().reset();
  let mut flash_helper = FlashHelper::from_file(
    context1.micro_controller(), "website/", "part3/flash_memory_driver.txt")?;
}
\rs{host_log(&flash_helper.read())}

Finally, on the Arduino, type ``w\rs{hex_word_low(driver_address)}''+Enter to
check that the driver is indeed in flash memory: you should see the same screen
as above.

\rust{
  context1.run_until_get_char();
  context1.type_ascii(&format!("W{:08X}\n", driver_address));
  let driver_display1 = display1.borrow().get_text();
  let first_line1 = driver_display1.lines().next().unwrap();
  assert_eq!(first_line1, first_line);
}
