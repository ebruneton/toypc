% This work is licensed under the Creative Commons Attribution NonCommercial
% ShareAlike 4.0 International License. To view a copy of the license, visit
% https://creativecommons.org/licenses/by-nc-sa/4.0/

\renewcommand{\rustfile}{introduction}
\setcounter{rustid}{0}

\chapter*{Introduction}
\addcontentsline{toc}{chapter}{Introduction}

Our toy computer is now fully assembled and autonomous. However, it is still
very hard to use. Indeed, even if our virtual machine instructions are much
simpler than ARM instructions, they remain difficult to use. The goal of this
part is thus to provide an easier way to program our toy computer.

To illustrate how this can be done, consider the task of writing a program to
compute the factorial of a number $n$, defined by $factorial(n)=1*2*\ldots*n$
if $n > 0$, and 1 otherwise. One way to write such a program is to use the
property that, for $n>0$, $factorial(n)=factorial(n-1) * n$. This leads to the
following bytecode instructions:

\rust{
  let address = 0xC1000;
  let mut b = BytecodeAssembler::new(RegionKind::Default, address);
}

\begin{TwoColumns}
\rs{b.func("factorial", &["n"], "", &[])}\\
\bytecode{
  b.get("n");
  b.cst_0();
  b.ifne("not_zero");
  b.cst_1();
  b.retv();
  b.label("not_zero");
  b.get("n");
  b.cst_1();
  b.sub();
  b.call("factorial");
  b.get("n");
  b.mul();
  b.retv();
}
\end{TwoColumns}

The first part, in the left column, compares $n$, the $0^{th}$ word in the
function's stack frame, with 0. If it is not equal to 0, it jumps to the second
part, in the right column. Otherwise it returns 1. The second part, in the
right column, subtracts 1 from $n$, calls the factorial function with this
argument (we assume here that it is stored at \rs{hex(address)}), and returns
the result multiplied by $n$. In order to write this program we need to type on
the keyboard the following {\em numbers}:

\rs{b.get_bytecode_listing(0..b.get_instruction_count() as usize, false)}

Typing numbers is error prone, and understanding their meaning requires a lot
of effort. It would be much easier if we could type the following {\em text}
instead:

\begin{Code}
fn 1
  get 0 cst_0 ifne 10 cst_1 retv
  get 0 cst_1 sub call 4096 get 0 mul retv
\end{Code}

In fact this text still contains some numbers, such as 0, 10=\hexa{A} and
4096=\hexa{1000}. The last two are quite tedious to compute. For instance, one
must sum the size of all the instructions up to the first \insn{retv}
(included) to get the value 10. Function addresses such as 4096 also require to
compute the bytecode size of each function to keep track of their addresses. To
avoid having to do this, it would be simpler if we could use {\em labels} to
designate instructions, and {\em identifiers} to designate functions:

\begin{Code}
fn \textbf{factorial} 1
  get 0 cst_0 ifne \textbf{not_zero} cst_1 retv
  \textbf{:not_zero} get 0 cst_1 sub call \textbf{factorial} get 0 mul retv
\end{Code}

Similarly, to get rid of the last numbers, it would be useful to be able to
give names to the function parameters. We could then replace 1, the number of
parameters of the factorial function, with a list of parameter names. And we
could replace 0 with $n$:

\begin{Code}
fn factorial\textbf{(n)}
  get \textbf{n} cst_0 ifne not_zero cst_1 retv
  :not_zero get \textbf{n} cst_1 sub call factorial get \textbf{n} mul retv
\end{Code}

This text would already be much easier to type and to understand than the above
numbers. However, it still contains long sequences of bytecode instructions
which are more complex than the mathematical {\em expressions} they compute.
For instance, \verb!get n cst_1 sub call factorial get n mul! computes
$factorial(n-1)*n$. It would be much easier if we could use these expressions
directly (this example also adds some ``punctuation'' signs, namely curly
braces, commas, and semi-colons):

\begin{Code}
fn factorial(n) \textbf{\{}
  \textbf{n, 0} ifne not_zero; \textbf{1} retv;
  :not_zero \textbf{factorial(n - 1) * n} retv;
\textbf{\}}
\end{Code}

It would also be more natural if we could write the last remaining bytecode
instructions in a different order, closer to the order of words in English. And
instead of writing ``if $n$ is not 0, jump to :not\_zero to not return 1'', it
would be simpler to write ``if $n$ is 0, return 1''. We could even get rid of
the label by putting the instructions to execute when $n$ is 0 inside curly
braces:

\begin{Code}
fn factorial(n) \{
  \textbf{if} n \textbf{==} 0 \textbf{\{ return} 1; \textbf{\}}
  \textbf{return} factorial(n - 1) * n;
\}
\end{Code}

Finally, to make it clear that this function returns a value (some do not), and
takes a number as parameter, it would be practical to have some {\em type
declarations}, such as the following (where \verb!u32! means an ``unsigned 32
bit'' value):

\begin{Code}
fn factorial(n: \textbf{u32}) \textbf{-> u32} \{
  if n == 0 \{ return 1; \}
  return factorial(n - 1) * n;
\}
\end{Code}

In fact the goal of this part is to be able to type programs in this form
(inspired from Rust~\cite{RustProgrammingLanguage}), and to {\em automatically}
get the corresponding bytecode instructions, in numerical form (also called
{\em binary} form). For this we write a program, called a {\em compiler}, which
transforms the program text, called the {\em source code}, into {\em compiled
code}, \ie, bytecode instructions in binary form. This compiler is a large
program, and we can't write it in source code because we don't have a compiler
yet! On the other hand, writing it in binary form would be very hard. To solve
this problem we write the compiler in several steps:

\begin{enumerate}
\item Write a small compiler to compile textual bytecode instructions (\eg,
{\tt fn 1 get 0 $\ldots$}). Write this {\em opcodes compiler} in binary form.

\item Write a compiler for programs using textual bytecode instructions with
function names and instruction labels. Write this {\em labels compiler}, also
called an {\em assembler}, with pure textual bytecode instructions. Compile it
with the opcodes compiler.

\item Rewrite the labels compiler with function names and labels, and improve
it so that it can compile programs using expressions such as {\tt factorial(n -
1) * n}. Compile this {\em expressions compiler} with the labels compiler.

\item Rewrite the expressions compiler with expressions, and improve it in
order to support programs using {\em statements} such as {\tt if} and {\tt
return}. Compile this {\em statements compiler} with the expressions compiler.

\item Rewrite the statements compiler with statements, and improve it to accept
programs with type declarations such as {\tt factorial(n: u32) -> u32 \{ ...
\}}. Compile this {\em types compiler} with the statements compiler.
\end{enumerate}

With this method, only the first compiler needs to be written in binary form.
And this compiler is small because its task is quite simple. Moreover, each
step is easier to do than the previous one because it can use simplifying
features introduced in the previous steps. However, in order to implement this
method, we need a way to type text, \ie, a {\em text editor}. Indeed, all we
have for now is the memory editor, and entering text in memory with it would
require typing the ASCII code {\em numbers} corresponding to each character of
the text! Obviously, this would be even worse than entering programs directly
in binary form, since a program in text form is usually longer than in binary
form. Unfortunately, we can't write a text editor program in any textual form,
since we don't have a text editor yet! We thus start by writing a simple text
editor, directly in binary form.

Before all this, however, we also need a way to store programs in source or
binary form in flash memory. Otherwise, we would loose everything we typed if
we ever do a mistake causing a crash. For this we implement one more driver,
called the flash memory driver (we didn't need this in the previous part
because flashing programs was done from the external computer). This driver
also needs to be implemented in binary form. The rest of this part presents the
above steps in detail. It is organized as follows:

\begin{itemize}
\item \cref{chapter:flash-driver} presents the flash memory driver, used to
read and write data in flash memory. We use it at the end to save itself in
flash memory.

\item \cref{chapter:text-editor} explains how our text editor works, and
presents its implementation in binary form.

\item \cref{chapter:opcodes-compiler} explains how the opcodes compiler works,
and presents its implementation in binary form.

\item \cref{chapter:labels-compiler} explains how the labels compiler works, and
presents its implementation in textual form.

\item \cref{chapter:expressions-compiler} explains what expressions are, how
they can be compiled, and presents the ones supported by our toy compiler. It
then gives the corresponding implementation.

\item \cref{chapter:statements-compiler} does the same with statements.

\item \cref{chapter:types-compiler} does the same with type declarations.

\item \cref{chapter:native-compiler} finally provides a new version of our toy
compiler which produces ARM instructions instead of bytecode instructions. We
use it in the next part to eventually get rid of our virtual machine.
\end{itemize}
