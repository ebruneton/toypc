% This work is licensed under the Creative Commons Attribution NonCommercial
% ShareAlike 4.0 International License. To view a copy of the license, visit
% https://creativecommons.org/licenses/by-nc-sa/4.0/

\chapter*{Conclusion}
\addcontentsline{toc}{chapter}{Conclusion}

A compiler transforms a program written in a well defined programming language
into instructions for a real or virtual machine. For this it decomposes the
program into a sequence of tokens, checks that these tokens follow the
language's grammar, and finally generates instructions based on the program's
grammatical structure. In this part we built a very basic compiler, for a very
simple language, via successive improvements. Starting from a simple ``text to
binary converter'' written in binary form\footnote{We used this starting point
because the goal was to program a compiler from scratch. This is more or less
how the very first assembler was written. Nowadays, a new compiler for a new
language is written in an existing language, and compiled with an existing
compiler (at least for its first version; it can then be rewritten in this new
language and compiled with itself).}, we progressively added support for
labels, expressions, statements, and types.

The resulting language is much easier to use than machine code instructions,
but can still be improved in many ways. For instance, we could introduce new
{\tt u8} and {\tt u16} types to represent 8 and 16 bit numbers. Then the
compiler could automatically use \arm{LDRB}, \arm{STRB}, \arm{LDRH}, and
\arm{STRH} instructions to load and store values of these types in memory. This
would be more efficient, and more practical for the user, than the current {\tt
load8}, {\tt store8}, {\tt load16}, and {\tt store16} functions. As an another
example, we could introduce a new type to represent a sequence of characters,
called a {\em string}. This would be more practical than using a pointer to the
first character, plus a variable containing the total number of characters.

Even without improving the language, the compiler itself can be improved in
many ways, to produce smaller and more efficient code. For instance, an
expression such as {\tt (1 {<}{<} 12) - 1} is currently compiled into
instructions to load $1$ and $12$ into registers, to shift the former with the
latter, to load $1$ again in a register, and to subtract it from the previous
result. Instead, the compiler could compute all this during compilation, and
produce a single instruction to load the result ($4095$) in a register. This
would give both smaller and more efficient code. As another example, already
mentioned in the previous chapter, the compiler could make better use of the
registers, to avoid many instructions copying values between registers and the
memory.

\subsubsection{Further readings}

To learn how these improvements can be implemented, as well as many others, you
can read one of the following books:
\begin{itemize}
  \item ``Compilers: Principles, Techniques, and Tools (2nd Edition)''
  \cite{CompilersPrinciples}. This is a classic textbook about compilers for
  students in computer science. It covers all parts of a compiler (scanner,
  parser, type checking, code generation, code optimization, etc) and presents
  for each part the relevant theory and algorithms.

  \item ``Engineering a Compiler'' \cite{EngineeringCompiler}. This book is
  similar but more recent. Some methods, such as the Static Single Assignment
  form, are presented in more details in this book than
  in~\cite{CompilersPrinciples}. Others, on the hand, are presented more
  extensively in~\cite{CompilersPrinciples} (such as interprocedural analysis).
\end{itemize}

In addition to these books, it is useful to have some basic knowledge about
some generic algorithms and data structures, and their computational complexity
(\ie, how much time and space they need to run). For this ``Introduction to
Algorithms, Third Edition'' \cite{IntroductionToAlgorithms} is a great book.