% This work is licensed under the Creative Commons Attribution NonCommercial
% ShareAlike 4.0 International License. To view a copy of the license, visit
% https://creativecommons.org/licenses/by-nc-sa/4.0/

\appendix
\addtocontents{toc}{\protect\appendixintoc}

\chapter{Bill of Materials}\label{appendix:bom}

The table below lists all the necessary components to assemble our toy
computer, with their price as of July 2024. It is important to use the exact
Arduino, display, driver board and keyboard models listed here. Otherwise they
might not work with the programs presented in this book.

Assembling these components requires some soldering tasks. The tools
necessary for this are not included here (you can avoid buying them if you have
access to a makerspace).

\begin{Table}[ht]
  \begin{tabular}{|l|l|}\hline
    \makecell{\thead{Part}} & \makecell{\thead{Net price}} \\ \hline

    \makecell{Arduino Due \\
      Cortex M3 84MHz, 512~KB Flash, 96~KB RAM, 3.3V \\
      \url{https://store-usa.arduino.cc/products/arduino-due}}
    & \$48 \\ \hline

    \makecell{7$^{\prime\prime}$ TFT Display  \\
      800x480 pixels with Touchscreen \\
      \url{https://www.adafruit.com/product/2354}}
    & \$35 \\ \hline

    \makecell{RA8875 Driver Board \\
      for 40-pin TFT Touch Displays - 800x480 max \\
      \url{https://www.adafruit.com/product/1590}}
    & \$40 \\ \hline

    \makecell{Miniature Keyboard \\
      PS/2 and USB interface\\
      \url{https://www.adafruit.com/product/857}}
    & \$30 \\ \hline

    \makecell{4 channel Logic Level Converter \\
      \url{https://www.sparkfun.com/products/12009}}
    & \$4 \\ \hline

    \makecell{Break Away Headers - Straight \\
      \url{https://www.sparkfun.com/products/116}}
    & \$2 \\ \hline

    \makecell{Half Sized Breadboard - 400 Tie Points \\
      \url{https://www.adafruit.com/product/64}}
    & \$5 \\ \hline

    \makecell{Male/Male Jumper Wires - 20 x 3$^{\prime\prime}$ (75mm) \\
      \url{https://www.adafruit.com/product/1956}}
    & \$2 \\ \hline

    \makecell{Female/Male Jumper Wires - 20 x 3$^{\prime\prime}$ (75mm) \\
      \url{https://www.adafruit.com/product/1953}}
    & \$2 \\ \hline

    \makecell{\bfseries Total} & \makecell{\bfseries \$168} \\ \hline
  \end{tabular}
  \caption{The Bill of Materials of our toy computer.}\label{table:bom}
\end{Table}

\chapter{ASCII codes}\label{appendix:ascii}

The table below lists the character codes defined by the ``American Standard
Code for Information Interchange'' used in this book. The full list can be
found in \cite{ASCII}.

\begin{flushleft}
  \input{generated/ascii_table.tex}
\end{flushleft}

\chapter{IBM PC Set 2 scancodes}\label{appendix:scancodes}

The table below lists the scancodes which are emitted by each key of the
MCSaite keyboard (see \cref{table:bom}) when it is pressed or released. This is
a subset of the IBM PC Set 2 scancodes
(\url{https://wiki.osdev.org/PS/2_Keyboard#Scan_Code_Set_2}).

\begin{flushleft}
  \input{generated/scancode_table.tex}
\end{flushleft}

\chapter{Compiler error codes}\label{appendix:compilercodes}

The table below gives the meaning of the error codes which can be output by the
compiler implemented in \cref{part:compiler}.

\begin{flushleft}
  \input{generated/error_codes_table.tex}
\end{flushleft}

\chapter{Boot Assistant scripts}\label{appendix:python-scripts}

The following sections list the source code of the scripts used in
\cref{part:computer} on the host computer.

\section{boot\_helper.py}

\bigskip
\input{generated/boot_helper.tex}

\section{flash\_helper.py}

\bigskip
\input{generated/flash_helper.tex}
