% This work is licensed under the Creative Commons Attribution NonCommercial
% ShareAlike 4.0 International License. To view a copy of the license, visit
% https://creativecommons.org/licenses/by-nc-sa/4.0/

\renewcommand{\rustfile}{chapter8}
\setcounter{rustid}{0}

\rust{
  context.write_backup("website/backups", "snake_game.txt")?;
}

\chapter{Snake Game}\label{chapter:snake-game}

The Toys operating system is now complete. It can of course be improved in
numerous ways, but this book is already quite long. Thus, instead of doing
this, and since toys are made to play, we use it in this last chapter to
implement a small ``snake'' game.

\section{Requirements}

In this game the player uses the arrow keys to control a ``snake'' moving at
constant speed across the screen. The goal is to ``eat'' as many ``apples'' as
possible. Each time the snake eats one, its get longer and a new apple appears
at random in a free spot.

The snake is represented with a contiguous series of green square cells, from
head to tail. The apple is represented with a red cell (there is only one apple
at any given time). At each time step the head moves in the same direction as
in the previous step or, if an arrow key was pressed, in the arrow direction.
The body follows it, as illustrated in the following example:

\begin{center}
  \input{figures/chapter8/snake-move.tex}
\end{center}

If the head hits a ``wall'' (the screen boundary) or the snake's own body, the
game is over. If the head moves in the apple cell the tail does not move, which
increases the snake's length:

\begin{center}
  \input{figures/chapter8/snake-eat.tex}
\end{center}

Then a new apple appears at random {\em in a free cell}. If the player is
extremely good there might be no free cell left, in which case the game is
over. The score is the number of apples eaten.

\section{Data structures}

In order to implement the above requirements we need to store in memory the
current state of the game. For this we start by dividing the screen in a grid
of cells. Each cell is uniquely identified by its $(x,y)$ coordinates, with $x$
increasing from left to right and $y$ from top to bottom (see
\cref{fig:snake-data-structures}). We can then store the current apple position
with its grid coordinates. Storing the current snake state is more complex. One
possibility is to store a list of $(x,y)$ coordinates, representing the cells
occupied by the snake, from head to tail. We can then move the snake by adding
a new element at the beginning of the list, and by removing the last element
(unless the snake eats the apple). However, this method has two drawbacks:
\begin{enumerate}
  \item to determine if the head hits the body we need to iterate over each
  element of the list, and check whether it is equal to the head.

  \item there is no easy way to find a free cell to place a new apple. The best
  we can do is try a cell $c$ at random, and then iterate over each snake cell
  to check if $c$ is free or not. If not, we need to repeat this process until
  a free cell is found.
\end{enumerate}

\begin{Figure}
  \input{figures/chapter8/data-structures.tex}

  \caption{The game state on the right is represented with the struct on the
  left, containing pointers to the cell states and the free cells, and the
  $(x,y)$ coordinates of the head, tail and apple. The state of a free cell is
  its index in the free cells list. This list contains the $(x,y)$ coordinates
  of each free cell. The state of a used cell is the direction towards the next
  used cell.}\label{fig:snake-data-structures}
\end{Figure}

To mitigate this we can use an additional grid data structure, storing the
current state of each cell: used or free. This state is easy to update when
moving the snake, and avoids a loop over the snake's list of cells in 1) and
2). But this is not sufficient to avoid trying cells at random until a free one
is found.

To solve this issue we introduce a third data structure, namely a list of all
the free cells, identified by their $(x,y)$ coordinates. Finding a free cell is
then very easy: just pick any element in this list at random. But we then need
to remove this cell from the list of free cells. We also need to update this
list when moving the snake. For instance, we need to remove the new head cell
from the list of free cells. This requires finding a cell in this list, given
its coordinates. To do this without iterating over this list (we are trying to
avoid any loop), we can store more than a ``used or free'' state for each cell.
More precisely, we can store for each free cell its index in the list of free
cells. This gives two data structures referencing each other (see
\cref{fig:snake-data-structures}):
\begin{itemize}
  \item a grid data structure storing the state of each cell, \ie, whether it
  is used or free and, if so, its index in the list of free cells.

  \item a list of all the free cells, specified by their $(x,y)$ coordinates in
  the grid data structure.
\end{itemize}

Finally, to avoid a third data structure for the list of snake cells, we encode
this list in the grid structure. More precisely, we store for each used cell
the direction of the next used cell (towards the head). Then the only other
additional state which is needed are the coordinates of the head and tail cells
(see \cref{fig:snake-data-structures}).

\subsection{Operations}\label{subsection:snake-operations}

In order to move the snake the new head cell must be changed from a free cell
to a used cell. And the old tail cell must be changed from a used cell to a
free cell. With the above data structures, these operations can be done as
follows:

\begin{itemize}
  \item to change a cell $(x,y)$ to a free cell (see
  \cref{fig:snake-algorithms}, left):

  \begin{enumerate}
    \item add the $(x,y)$ element at the end of the free cells list. Note $n$
    the new size of this list.

    \item change the state of the $(x,y)$ cell in the grid to ``free, stored at
    index $n-1$'' in the free cells list.
  \end{enumerate}

  \item to change a free cell $(x, y)$ to a used cell, replace this cell in the
  free list with the last free cell, in order to avoid a ``hole'' in this list
  (see \cref{fig:snake-algorithms}, right):

  \begin{enumerate}
    \item get the free cell index stored in the $(x,y)$ cell of the grid, noted
    $i$.

    \item set the state of the $(x,y)$ cell of the grid to ``used''.

    \item get the $(x',y')$ coordinates in the last element of the free cells
    list.

    \item change the free cell index stored in the $(x',y')$ cell of the grid
    to  $i$.

    \item change the coordinates in the $i^{th}$ element of the free cells list
    to $(x',y')$.

    \item remove the last element of the free cells list.
  \end{enumerate}
\end{itemize}

\begin{Figure}
  \input{figures/chapter8/algorithms.tex}

  \caption{Left: changing the $(4,2)$ cell to a free cell. Right: changing the
    free cell $(0,1)$ to a used cell is done by replacing it with the last cell
    in the free cells list.}\label{fig:snake-algorithms}
\end{Figure}

To place a new apple we need to choose a random cell in the free cells list.
For this it suffice to choose a number at random between 0 and $n$ (excluded),
where $n$ is the size of this list. But the microprocessor has no instruction
for this. The solution is to use values which {\em look} random. For instance,
0, 3, 2, 13, 4, 7, 6, 1, 8, 11, 10, 5, 12, 15, 14, 9 seems to contain the
numbers between 0 and 16 (excluded) in random order. In fact each value
$v_{t+1}$ is computed from the previous one $v_t$ with $v_{t+1}=5v_t+3 \mod
16$. More generally, using $v_{t+1}=a.v_t+b \mod m$, can give values which
look random {\em if $a$, $b$, and $m$ are well chosen} (for instance, $a=b=1$
is a bad choice). The Knuth \& Lewis parameters, $a=1664525$, $b=1013904223$,
and $m=32$, are frequently used. They give all the values between 0 and
$2^{32}$ (excluded) in ``random'' order, without repetitions.

\subsection{Encoding}\label{subsection:snake-encodings}

We store the elements of the free cells list one after the other in a buffer,
using one byte for each coordinate. The $x$ (resp. $y$) coordinate of the
$i^{th}$ free cell (counting from 0) is thus at offset $2i$ (resp $2i+1$) from
the beginning of this buffer.

We represent a used cell with a number from 0 to 3, encoding the direction of
the next used cell (0 for left, 1 for right, 2 for up, and 3 for bottom). We
represent a free cell with its index in the list of free cells, {\em plus 4}.
Hence, a cell state less than 4 represents a used cell, while a cell state
greater than or equal to 4 denotes a free cell. To support ``large'' grids (up
to 65532 cells) we store each state on 16 bits. We store them one after the
other in a buffer, from left to right and from top to bottom. Hence, if the
grid width is $W$ cells, the state of the $(x,y)$ cell is at a offset
$2(x+y.W)$ from the beginning of this buffer.

\section{Implementation}

\rust{
  let mut t = Transpiler5::default();
}

The screen has 30 rows of 100 characters, but each character is $8\times16$
pixels. To get square cells we represent each cell with 2 characters, yielding
30 rows of 50 cells. We reserve the first line to display the current score,
which leaves a play area of 29 rows and 50 columns:

\toy{
const WIDTH: u32 = 50;
const HEIGHT: u32 = 29;
const LEFT: u32 = 0;
const RIGHT: u32 = 1;
const UP: u32 = 2;
const DOWN: u32 = 3;
}%toy

The following struct implements the data structures discussed above. The first
two fields are pointers to the grid of cell states and to the list of free
cells. The {\tt seed} field contains the last $v_t$ value computed with the
Knuth \& Lewis random number generator:

\toy{
struct Game {
  cell_states: &u32,
  free_cells: &u32,
  num_free_cells: u32,
  head_x: u32,
  head_y: u32,
  tail_x: u32,
  tail_y: u32,
  apple_x: u32,
  apple_y: u32,
  direction: u32,
  score: u32,
  seed: u32
}
}%toy

The following functions are used to read and write values in the grid of cell
states and in the list of free cells. They use the offsets explained in the
previous section:

\toy{
fn game_get_cell_state(self: &Game, x: u32, y: u32) -> u32 {
  return load16(self.cell_states + 2 * (x + y * WIDTH));
}
fn game_set_cell_state(self: &Game, x: u32, y: u32, state: u32) {
  store16(self.cell_states + 2 * (x + y * WIDTH), state);
}
fn game_get_free_cell_coords(self: &Game, index: u32, x: &u32, y: &u32) {
  *x = load8(self.free_cells + 2 * index);
  *y = load8(self.free_cells + 2 * index + 1);
}
fn game_set_free_cell_coords(self: &Game, index: u32, x: u32, y: u32) {
  store8(self.free_cells + 2 * index, x);
  store8(self.free_cells + 2 * index + 1, y);
}
}%toy

The next functions use them to change a used cell into a free cell, and vice
versa (with the algorithms and encodings presented in
\cref{subsection:snake-operations,subsection:snake-encodings}):

\toy{
fn game_free_cell(self: &Game, x: u32, y: u32) {
  let free_cell_index = self.num_free_cells;
  game_set_cell_state(self, x, y, free_cell_index + 4);
  game_set_free_cell_coords(self, free_cell_index, x, y);
  self.num_free_cells = free_cell_index + 1;
}

fn game_use_cell(self: &Game, x: u32, y: u32, direction: u32) {
  let free_cell_index = game_get_cell_state(self, x, y) - 4;
  game_set_cell_state(self, x, y, direction);
  let last_free_cell_index = self.num_free_cells - 1;
  let last_x = 0;
  let last_y = 0;
  game_get_free_cell_coords(self, last_free_cell_index, &last_x, &last_y);
  game_set_free_cell_coords(self, free_cell_index, last_x, last_y);
  game_set_cell_state(self, last_x, last_y, free_cell_index + 4);
  self.num_free_cells = last_free_cell_index;
}
}%toy

We can use them to create a new {\tt Game} data structure, with its cell grid
and free cells list buffers, and to initialize all the cells to free cells:

\toy{
fn game_new(heap_p: &&u32, heap_limit: &u32) -> &Game {
  let game = mem_allocate(sizeof(Game), heap_p, heap_limit) as &Game;
  let cell_states = mem_allocate(2 * WIDTH * HEIGHT, heap_p, heap_limit);
  let free_cells = mem_allocate(2 * WIDTH * HEIGHT, heap_p, heap_limit);
  if game == null || cell_states == null || free_cells == null {
    return null;
  }
  game.cell_states = cell_states;
  game.free_cells = free_cells;
  game.num_free_cells = 0;
  let y = 0;
  let x = 0;
  while y < HEIGHT {
    x = 0;
    while x < WIDTH {
      game_free_cell(game, x, y);
      x = x + 1;
    }
    y = y + 1;
  }
  return game;
}
}%toy

The following functions implement the Knuth \& Lewis random number generator,
and use it to place a new apple (they assume there is at least one free cell).
The generated numbers are in $[0, 2^{32}[$, but we need a value in $[0, n[$,
where $n$ is the size of the free cells list ($n \le 50*29 = 1450$). For this
we use the 11 most significant bits\footnote{The least significant bits are
``less random''. For instance, in the 0, 3, 2, 13, 4, 7, 6, 1, 8, 11, 10, 5,
12, 15, 14, 9 sequence shown above, numbers are alternatively even and odd,
\ie, bit 0 is 0, 1, 0, 1, 0, 1, \ldots} of a random number, modulo $n$:

\toy{
fn random(seed: &u32) -> u32 {
  *seed = *seed * 1664525 + 1013904223;
  return *seed;
}
fn modulo(x: u32, m: u32) -> u32 { return x - (x / m) * m; }

fn game_new_apple(self: &Game) {
  let index = modulo(random(&self.seed) >> 21, self.num_free_cells);
  game_get_free_cell_coords(self, index, &self.apple_x, &self.apple_y);
}
}%toy

To draw the apple, and to move the snake, we just need to draw a red cell, a
green cell (for the new head), or a black cell (to erase the old tail). We thus
implement a function to draw a cell in a given color. Since we
reserved the first line for the score, and use 2 characters per cell, the
$(x,y)$ cell corresponds to row $y+1$ and to columns $2x$ and $2x+1$. To fill
these characters with the given color we draw two spaces with this color as
background color (the graphics card has 3 registers to set it, using the same
format as for the foreground color -- see \cref{subsection:text-config} and
\cite{RA8875}).

\toy{
fn gpu_set_background(r: u32, g: u32, b: u32) {
  gpu_set_register(96 /*Background Color 0*/, r);
  gpu_set_register(97 /*Background Color 1*/, g);
  gpu_set_register(98 /*Background Color 2*/, b);
}
fn draw_cell(x: u32, y: u32, r: u32, g: u32, b: u32) {
  gpu_set_background(r, g, b);
  gpu_set_cursor(2 * x, y + 1);
  gpu_draw_char(' ');
  gpu_draw_char(' ');
  gpu_set_background(0, 0, 0);
}
}%toy

The next 3 functions are used to draw the score in the top-left corner:

\toy{
static SCORE = ['S','c','o','r','e',':',' '];
static GAME_OVER = ['G','a','m','e',' ','o','v','e','r','!'];

fn draw_integer(x: u32) {
  let quotient = x / 10;
  if quotient > 0 { draw_integer(quotient); }
  gpu_draw_char(x - 10 * quotient + '0');
}
fn draw_string(src: &u32, length: u32) {
  let i = 0;
  while i < length {
    gpu_draw_char(load8(src + i));
    i = i + 1;
  }
}
fn draw_score(score: u32) {
  gpu_set_cursor(0, 0);
  draw_string(SCORE, 7);
  draw_integer(score);
}
}%toy

The following function places an initial snake in the middle of screen, using 6
cells and moving left, draws it, places and draws the apple, and finally
initializes and draws the score. It does not initialize the {\tt seed} on
purpose, otherwise the apple would always be at the same initial position.
Instead, the seed gets whatever value was here in memory (most likely the last
{\tt seed} value from a previous run of this game).

\toy{
fn game_init(self: &Game) {
  self.head_x = WIDTH / 2;
  self.head_y = HEIGHT / 2;
  self.tail_x = self.head_x + 5;
  self.tail_y = self.head_y;
  self.direction = LEFT;
  self.score = 0;
  let x = self.head_x;
  while x <= self.tail_x {
    game_use_cell(self, x, self.head_y, LEFT);
    draw_cell(x, self.head_y, 0, 7, 0);
    x = x + 1;
  }
  game_new_apple(self);
  draw_cell(self.apple_x, self.apple_y, 7, 0, 0);
  draw_score(0);
}
}%toy

The function below moves the snake in the current direction, and returns {\tt
OK} if this is valid, or {\tt INVALID\_STATE} otherwise. It starts by updating
the head position and detecting collisions with the ``walls'':

\toy{
fn game_update(self: &Game) -> u32 \{
  let old_head_x = self.head_x;
  let old_head_y = self.head_y;
  if self.direction == LEFT {
    if self.head_x == 0 { return INVALID_STATE; }
    self.head_x = self.head_x - 1;
  } else if self.direction == RIGHT {
    if self.head_x == WIDTH - 1 { return INVALID_STATE; }
    self.head_x = self.head_x + 1;
  } else if self.direction == UP {
    if self.head_y == 0 { return INVALID_STATE; }
    self.head_y = self.head_y - 1;
  } else if self.direction == DOWN {
    if self.head_y == HEIGHT - 1 { return INVALID_STATE; }
    self.head_y = self.head_y + 1;
  }
}%toy

It then checks if the head hits the body. If not, it updates the old and new
head cells, and draws the new one:

\toy{
  if game_get_cell_state(self, self.head_x, self.head_y) < 4 {
    return INVALID_STATE;
  }
  game_set_cell_state(self, old_head_x, old_head_y, self.direction);
  game_use_cell(self, self.head_x, self.head_y, 0);
  draw_cell(self.head_x, self.head_y, 0, 7, 0);
}%toy

If the head just moved in the apple cell, it places and draws a new apple (if
there is at least one free cell left), and then updates the score:

\toy{
  if self.head_x == self.apple_x && self.head_y == self.apple_y {
    if self.num_free_cells == 0 { return INVALID_STATE; }
    game_new_apple(self);
    draw_cell(self.apple_x, self.apple_y, 7, 0, 0);
    self.score = self.score + 1;
    draw_score(self.score);
    return OK;
  }
}%toy

Finally, if the apple was not eaten, this function ends by freeing the current
tail cell, erasing it on screen, and updating the tail position to the next
used cell (using the direction stored in the old tail cell):

\toy{
  let tail_direction = game_get_cell_state(self, self.tail_x, self.tail_y);
  game_free_cell(self, self.tail_x, self.tail_y);
  draw_cell(self.tail_x, self.tail_y, 0, 0, 0);
  if tail_direction == LEFT {
    self.tail_x = self.tail_x - 1;
  } else if tail_direction == RIGHT {
    self.tail_x = self.tail_x + 1;
  } else if tail_direction == UP {
    self.tail_y = self.tail_y - 1;
  } else if tail_direction == DOWN {
    self.tail_y = self.tail_y + 1;
  }
  return OK;
\}
}%toy

We can finally implement the main function of the game. This function starts by
clearing the screen, hiding the cursor (see \cref{subsection:ra8875-config}),
creating and initializing the game data structures, and drawing this initial
state:

\toy{
const ESCAPE_KEY: u32 = 27;
const ARROW_LEFT_KEY: u32 = 235;
const ARROW_RIGHT_KEY: u32 = 244;
const ARROW_UP_KEY: u32 = 245;
const ARROW_DOWN_KEY: u32 = 242;

fn main(args: &u32, args_end: &u32, heap: &u32, heap_limit: &u32) -> u32 \{
  gpu_set_single_buffer();
  gpu_clear_screen();
  gpu_set_register(64 /*Memory Write Control 0*/, 128 /*Hide cursor*/);
  gpu_set_color(0, 7, 0);
  let game = game_new(&heap, heap_limit);
  if game == null { return OUT_OF_MEMORY; }
  game_init(game);
}%toy

It continues by waiting for a short time, updating the current direction if an
arrow key was pressed, updating the game state in memory and on screen, and by
repeating this until the game is over:

\toy{
  let c = 0;
  loop {
    sleep(125);
    read(KEYBOARD, &c, 1);
    if c == ESCAPE_KEY { break; }
    else if c == ARROW_LEFT_KEY { game.direction = LEFT; }
    else if c == ARROW_RIGHT_KEY { game.direction = RIGHT; }
    else if c == ARROW_UP_KEY { game.direction = UP; }
    else if c == ARROW_DOWN_KEY { game.direction = DOWN; }
    if game_update(game) != OK {
      gpu_set_cursor(44, 0);
      gpu_set_color(7, 0, 0);
      draw_string(GAME_OVER, 10);
      read(STANDARD_INPUT, &c, 1);
      break;
    }
  }
  gpu_set_register(64 /*Memory Write Control 0*/, 224 /*Show cursor*/);
  return OK;
\}
}%toy

\section{Compilation and play}

\rust{
  let display = Rc::new(RefCell::new(TextDisplay::default()));
  context.set_display(display.clone());
  context.micro_controller().borrow_mut().reset();
  context.run_until_get_char();

  context.type_ascii("EDIT SRC/SNAKE/SNAKE.TOY\n");
  t.write_toy5("website/sources/snake.txt")?;
  context.enter_text_editor_text(&t.get_toy5());
  context.type_keys(vec!["Escape"]);
  assert_eq!(display.borrow().get_text(), "Save (y/n)?");
  context.type_ascii("Y");

  let mut t = Transpiler5::default();
}

Type ``{\tt edit src/snake/snake.toy}'' and enter the above code. For
reference, we also provide this code in the {\tt snake.txt} file in
\toypcurl{sources.zip}. Then use the text editor again to create the following
build file:

\medskip \noindent {\tt\bfseries src/snake/BUILD}:
\toy{
toyc snake src/base.toy src/memory.toy src/gpu.toy src/snake/snake.toy
}%toy

\rust{
  context.type_ascii("EDIT SRC/SNAKE/");
  context.type_keys(vec!["Shift"]);
  context.type_ascii("BUILD");
  context.type_keys(vec!["~Shift", "Enter"]);
  context.enter_text_editor_text(&t.get_toy5());
  context.type_keys(vec!["Escape"]);
  assert_eq!(context.get_display().borrow().get_text(), "Save (y/n)?");
  context.type_ascii("Y");
}

Finally, type ``{\tt shell src/snake/BUILD}'' to compile the game and ``{\tt
snake}'' to play with it!

\rust{
  context.type_ascii("SHELL SRC/SNAKE/");
  context.type_keys(vec!["Shift"]);
  context.type_ascii("BUILD");
  context.type_keys(vec!["~Shift", "Enter"]);
  assert_eq!(context.get_display().borrow().get_text(),
      ">shell src/snake/BUILD\n>");

  context.write_backup("website/backups", "final.txt")?;
}
