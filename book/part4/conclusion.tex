% This work is licensed under the Creative Commons Attribution NonCommercial
% ShareAlike 4.0 International License. To view a copy of the license, visit
% https://creativecommons.org/licenses/by-nc-sa/4.0/

\chapter*{Conclusion}
\addcontentsline{toc}{chapter}{Conclusion}

An operating system manages the resources and the peripherals of a computer,
such as the memory, the disk, the keyboard, the screen, etc. It also provides a
simple and safe way for processes to use them, via system calls. In this part
we built a very basic monotasking operating system, in several steps. Starting
from a basic input output system and a compiler, in the Flash1 memory bank, we
progressively built and stored in the Flash0 bank an initial, self-hosting
operating system. Then, using this system alone, we improved its kernel and
added some utility programs.

The resulting computer is much easier to use than what it was at the beginning
of this part, but can still be improved in many ways. For instance, if a
process enters in an infinite loop because of a bug, we currently have no way
to stop it other than rebooting the computer. To solve this we could use a
combination of keys, detected in the keyboard interrupt handler, to stop the
current process. Incidentally, we could also improve the keyboard driver to
support combinations of keys, such as Ctrl+C or Ctrl+V. Another possible
improvement is to format the Flash1 memory bank, no longer needed, into free
blocks added to the Flash0 file system (to get more disk space). Alternatively,
we could format the Flash1 bank with a better file system format, for instance
with hierarchical directories. We could then build a new kernel version,
supporting this new file system, and store it in this new disk. By successively
switching between the two memory banks like this, we could then improve our
kernel in arbitrary ways (in the limit of the available memory and disk space).

\subsubsection{Further readings}

Another possible improvement is to switch to a multitasking operating system,
capable of running several processes concurrently. But this is a complex task,
which impacts almost all parts of the system. To know how this can be done, or
to know more about operating systems in general, you can read one of the
following books:
\begin{itemize}
  \item ``Modern Operating Systems'' \cite{ModernOSes}. This book presents the
  fundamental concepts used in operating systems, and discusses several
  strategies to implement each aspect (processes, memory management, file
  systems, peripherals, etc). It starts with processes, discusses all the
  problems related to their concurrent execution, and presents methods to solve
  them.

  \item ``Operating Systems Design and Implementation'' \cite{Minix}. This book
  is from the same author and contains many sections also present
  in~\cite{ModernOSes}. But it has a more practical point of view and presents
  the source code of a real operating system, small enough to be included in
  the book.
\end{itemize}

You can also visit the OSDev website (\url{https://wiki.osdev.org}), which
provides many more resources about operating systems. It contains for instance
a list of operating systems, including educational ones such as MentOS
(\url{https://mentos-team.github.io/}) and xv6
(\url{https://pdos.csail.mit.edu/6.828/2023/xv6.html}), used for teaching in
universities. Finally, you can also read ``Project Oberon: the design of an
operating system and compiler'' \cite{Oberon} (a second edition is available
online), which describes a small self-hosting operating system and its compiler.

