% This work is licensed under the Creative Commons Attribution NonCommercial
% ShareAlike 4.0 International License. To view a copy of the license, visit
% https://creativecommons.org/licenses/by-nc-sa/4.0/

\renewcommand{\rustfile}{chapter6}
\setcounter{rustid}{0}

\rust{
  context.write_backup("website/backups", "memory_protection.txt")?;
}

\chapter{Memory Protection}\label{chapter:memory-protection}

Our operating system is now self-hosting, which means that we can edit and
recompile its entire source code with itself. But the kernel is not finished.
Indeed, a bug in a process can currently crash everything, instead of exiting
this process with an error code, as required in \cref{chapter:processes}. To
meet this requirement the plan was to make sure that a process can only access
its own memory region (see \cref{subsection:process-design}). This chapter
implements this strategy with the help of the microprocessor's Memory
Protection Unit, which is presented first.

\section{Memory Protection Unit}

The Memory Protection Unit (MPU) is a component of the Cortex M3 microprocessor
inside the microcontroller (see \cref{fig:sam3x8e}). As briefly introduced in
\cref{section:cortex-m3}, it can divide the memory into regions, and can
associate different access rights with each region. For instance, one region
can be made inaccessible, another read-only, etc. If an instruction tries to
read memory in an inaccessible region, or to write a value in a read-only one,
a Memory Management exception is triggered. If the handler for this type of
exceptions is not explicitly enabled, the generic Hard Fault handler is called
instead.

\subsection{Regions}

The MPU uses 8 configurable regions, numbered from 0 to 7, plus a
non-configurable ``background region''. Each of the 8 regions has the following
configurable properties:
\begin{itemize}
  \item a {\em base address} and a {\em size}. These properties define the
  region boundaries, namely $[\mathit{base\ address}, \mathit{base\
  address}+\mathit{size}[$. {\em size} must be a power of 2 larger than or
  equal to 32, and {\em base address} must be a multiple of {\em size}.

  \item some {\em attributes}, which include {\em access permissions} such as
  ``no access'', ``read-only'' or ``full access''.

  \item an {\em enabled} bit. A disabled region does not have any effect on any
  memory access.
\end{itemize}

The background region can only be enabled or disabled. If enabled, it allows
full access to all memory addresses (\ie, its base address is 0, its size is
4~GB, and its access permissions are ``full access'').

Finally, the MPU as a whole can be enabled or not. If it is disabled, which is
the default after a reset, all memory accesses are permitted. If it is enabled,
a memory access at an address $a$ is checked as follows:
\begin{itemize}
  \item if $a$ is inside at least one enabled region, access is checked against
  the permissions of the enabled region containing $a$ with the largest number.
  For instance, if $a$ is inside region 1, enabled with ``no access'', and also
  inside region 3, enabled as ``read-only'', then reading the value at $a$ is
  permitted. As another example, if the background region is enabled and if $a$
  is inside region 2, enabled with ``no access'', then any access to $a$ is
  rejected (configurable regions take precedence over the background region).

  \item if $a$ is outside the boundaries of all enabled regions, access is
  rejected.
\end{itemize}

\subsection{Subregions}

Regions whose size is larger than or equal to 256 bytes are divided in 8
subregions of equal size. For instance, a 32~KB region starting at address
512~KB is divided in 8 subregions of 4~KB each: subregion 0 in
$[512~\rm{KB},516~\rm{KB}[$, subregion 1 in $[516~\rm{KB},520~\rm{KB}[$, etc.
These subregions have the same attributes as the enclosing region, but can be
independently enabled or disabled. For the above access checking rules, an
address $a$ is inside an enabled region $R$ if $R$ is enabled, and if $a$ is
inside an enabled subregion of $R$.

\subsection{Privilege level}

The MPU regions are configured with some registers, presented in the next
section, inside the ``System'' memory region (see \cref{fig:boot-memory-map}).
If the MPU is configured to forbid all memory accesses outside the RAM region
of a process, then these registers become inaccessible. In other words, it
becomes impossible to reconfigure the MPU to switch to another process. A
solution to this problem could be to allow only memory accesses to the memory
region of a process {\em or} to the ``System'' region. But then, due to a bug
or an intentional ``attack'', the process could access the MPU registers and
reconfigure them to allow all memory accesses!

To solve these issues the Cortex M3 can run at one of two {\em privilege
levels}: privileged or unprivileged. At the privileged level the MPU registers
are always accessible, even if the MPU would otherwise forbid this. The above
problems can thus be solved if 1) the kernel executes at the privileged level
and 2) processes execute at the unprivileged level and have no way to change
this level. Indeed, in this case:
\begin{itemize}
  \item the MPU can be configured to forbid any memory access outside a
  process's memory region, while this process is running. The process is then
  unable to reconfigure the MPU because it cannot access its registers, and
  because it cannot switch to privileged level.

  \item when the kernel is running it can access the MPU registers thanks to
  its privileges, and can thus reconfigure the MPU to switch to another process.
\end{itemize}

Exception and interrupt handlers always execute at the privileged level, which
ensures 1) above (after its initialization, the kernel always runs inside some
exception or interrupt handler, such as the USART or SVC handler). Outside
exception and interrupt handlers, \ie, in Thread mode, the microprocessor can
run at any privilege level. This is configured with a bit of the CONTROL
register (see \cref{subsection:process-stack-pointer}), which can only be
modified at the privileged level (thus ensuring 2) above).

In fact the CONTROL register, which can be accessed with the \arm{MRS} and
\arm{MSR} instructions (see \cref{subsection:process-stack-pointer}), has the
following binary format:

\rs{CONTROL.bit_pattern(true)}

\noindent where $p$ specifies the privilege level in Thread mode, and $s$
selects the current Stack Pointer. More precisely, the default value $p=0$
configures Thread mode to run at privileged level, and $p=1$ to run at
unprivileged level. $s=1$ selects the Process Stack Pointer, and $s=0$ selects
the Main Stack Pointer (see \cref{subsection:process-stack-pointer}). Setting
$s$ in Handler mode has no effect since this mode always uses the Main Stack
Pointer.

\subsection{MPU registers}

\begin{Table}
  \begin{tabular}{|l|l|l|}\hline
    \makecell{\thead{Name}} & \thead{Type} & \thead{Address} \\ \hline
    \makecell{Control Register} & Read-Write & \rs{hex(MPU_CONTROL.address)} \\
    \makecell{Region Number Register} & Read-Write &
    \rs{hex(MPU_RNR.address)} \\
    \makecell{Region Base Address Register} & Read-Write &
    \rs{hex(MPU_RBAR.address)} \\
    \makecell{Region Attribute and Size Register} & Read-Write &
    \rs{hex(MPU_RASR.address)} \\ \hline
  \end{tabular}
  \caption{The Memory Protection Unit registers used in this
    book.}\label{table:mpu-registers}
\end{Table}

The Memory Protection Unit is configured with the following registers (see
\cref{table:mpu-registers}):
\begin{itemize}
  \item The Control Register enables or disables the MPU and the background
  region. Its binary format is the following (we show only the bits that we
  use):

  \rs{MPU_CONTROL.bit_pattern(true)}

  where $e=1$ enables the MPU, $e=0$ disables it, $b=1$ enables the background
  region at the privileged level, and $b=0$ disables it. {\em The background
  region is always disabled at the unprivileged level}.

  \item The Region Number Register defines a ``current region number'', used by
  the next registers ($\mathit{region}$ must be between 0 and 7 included):

  \rs{MPU_RNR.bit_pattern(true)}

  \item The Region Base Address Register defines the base address of a region:

  \rs{MPU_RBAR.bit_pattern(true)}

  If $v=1$ then writing into this register stores $\mathit{region}$ in the
  Region Number Register, and sets the base address of this region. If $v=0$
  writing into this register sets the base address of the region whose number
  is stored in the Region Number Register. In both cases, the region's base
  address is set to $32 * \mathit{base\ address}$. For instance, setting this
  register to \hexa{B3}, \ie, setting $\mathit{base\ address}$ to 5, $v$ to 1
  and $\mathit{region}$ to 3 sets the base address of region 3 to $32*5=160$.

  \item The Region Attribute and Size Register defines the size, access
  permissions and subregions of the region whose number is stored in the Region
  Number Register. Its binary format is the following (we show only the bits
  that we use):

  \rs{MPU_RASR.bit_pattern(true)}

  \begin{itemize}
    \item $e=1$ enables this region, and $e=0$ disables it.

    \item $\mathit{size}$ sets the region size to $2^{\mathit{size}+1}$
    ($\mathit{size}$ must be larger than or equal to 4).

    \item $\mathit{subregions}$ specifies which subregions are {\em disabled}:
    the $i^{th}$ subregion, counting from 0, is disabled if the $i^{th}$ bit of
    $\mathit{subregions}$ is 1. For instance,
    $\mathit{subregions}=5=\bina{101}$ disables subregions 0 and 2 and enables
    the others.

    \item $\mathit{attributes}$ depends on the region type (Flash, RAM, etc).
    For a RAM region, the recommended value is 6 (see Sections 10.23.5 and
    10.23.9.1 in \cite{SAM3X8E}).

    \item $\mathit{access}$ defines the region's access permissions.
    $\mathit{access}=0$ means ``no access'', $3$ means ``full access'', and $7$
    means ``read-only'' (see Table 10.39 in \cite{SAM3X8E}).
  \end{itemize}
\end{itemize}

Note that, with the above registers, it is possible to set a base address which
is not a multiple of the region size, as required. In this case the MPU uses
this base address, rounded down to a multiple of the region size, to perform
the access checks. For instance, if a region is set to $[96,96+64[$, the MPU
internally uses $[64, 64+64[$ to check memory accesses.

\section{Algorithm}

Thanks to the Memory Protection Unit and the privilege levels we can make sure
that a process can only access its own memory region as follows:
\begin{itemize}
  \item enable the MPU and the background region with the MPU Control Register.
  The background region, only enabled in privileged mode, allows the kernel to
  access any memory address, and in particular its own data structures.

  \item set the privilege level of Thread mode to unprivileged with the CONTROL
  register, so that processes run at this privilege level. This must not be
  done during the kernel initialization, which runs in Thread mode. Otherwise
  the kernel would loose its privileges. In particular, this would disable the
  background region, and the kernel might no longer be able to access its own
  data structures. Instead, this step can be done just before spawning the
  initial process, in the spawn system call handler.

  \item just before switching to a child process, or back to a parent process,
  configure the MPU registers to forbid any memory access outside this
  process's memory region, hereafter noted $[\mathit{begin},\mathit{end}[$.
\end{itemize}

The first two steps are trivial to implement, but the last one is not. Indeed,
we cannot simply configure an MPU region with base address $\mathit{begin}$ and
size $\mathit{end}-\mathit{begin}$. This is because the base address must be a
multiple of the region size, which must itself be a power of 2 (larger than or
equal to 32). For instance, $[\mathit{begin},\mathit{end}[\ = [32,128[$ is not
a valid MPU region because $32$ is not a multiple of $96=128-32$, and because
$96$ is not a power of 2. However, this interval can be {\em covered} with two
MPU regions, namely $[32,64[$ and $[64,128[$. It can also be covered with
subregions 1, 2, and 3 of the MPU region $[0,256[$. This example shows that, in
general, several MPU regions or subregions are needed to cover a given interval.

The problem is now to find a way to compute the base address, size, and enabled
subregions of each MPU region so that they cover a given
$[\mathit{begin},\mathit{end}[$ interval. A prerequisite is that ${begin}$ and
$\mathit{end}$ are multiple of 32, since the smallest possible MPU region or
subregion is 32 bytes. This is why we rounded these values to multiples of 32
in \cref{chapter:processes}. If the MPU had a large enough number of regions we
could then cover the $[\mathit{begin},\mathit{end}[$ interval with many 32
bytes regions. But it has only 8. To solve this problem we need to use regions
with the largest possible sizes.

Consider for instance an interval $I_0=[32,\ldots[$, ignoring the
$\mathit{end}$ part for now. The largest region which can cover its beginning
part is $[0,256[$, with subregion $[0,32[$ disabled. This leaves an interval
$I_1=[256,\ldots[$ to cover. The largest region which can cover its beginning
part is $[0,2048[$, with subregion $[0,256[$ disabled. By repeating this
process we then get the MPU regions $[0,16~\rm{KB}[$ and $[0,128~\rm{KB}[$,
with their subregion 0 disabled. Note that the $4^{th}$ region is larger than
the RAM (96~KB). Hence, with this method, we never need more than 4 regions to
cover the beginning part of a process's memory interval (contained in RAM). The
same method, applied to the end part, shows that the 4 remaining MPU regions
are sufficient to cover the rest.

\subsection{Definition}\label{subsection:mpu-regions-algorithm}

The above ideas can be formalized as follows. To cover the
$[\mathit{begin},\mathit{end}[$ interval, where $\mathit{begin}$ and
$\mathit{end}$ are multiples of $2^{\mathit{level}-3}$ (with $\mathit{level}
\ge 8$ and $\mathit{begin} < \mathit{end}$):
\begin{itemize}
  \item configure a region of size $s=2^{\mathit{level}}$, with base address
  $b_0$ equal to $\mathit{begin}$ rounded down to a multiple of $s$, and with
  subregions covering $[\mathit{begin}, \mathit{gap\_begin}[$, where
  $\mathit{gap\_begin} = \min(b_0 + s, \mathit{end}) > \mathit{begin}$.

  \item configure a region of size $s$, with base address $b_1$ equal to
  $\mathit{end}$ rounded up to a multiple of $s$, minus $s$, and with
  subregions covering $[\mathit{gap\_end}, \mathit{end}[$, where
  $\mathit{gap\_end} = \max(\mathit{begin}, \mathit{b_1}) < \mathit{end}$.

  \item if $\mathit{gap\_begin} < \mathit{gap\_end}$, repeat the above steps
  with $\mathit{begin}$, $\mathit{end}$, and $\mathit{level}$ replaced with
  $\mathit{gap\_begin}$, $\mathit{gap\_end}$, and $\mathit{level}+3$,
  respectively.
\end{itemize}

The first two steps configure regions which satisfy the MPU constraints since,
by construction, $s$ is a power of 2, and $b_0$ and $b_1$ are multiples of $s$.
Moreover, by construction too, $\mathit{gap\_begin}$ and $\mathit{gap\_end}$
are inside their respective region, and are multiples of
$2^{\mathit{level}-3}=s/8$ (since $\mathit{begin}$, $\mathit{end}$, $b_0$,
$b_1$, and $s$ are). This ensures that $[\mathit{begin}, \mathit{gap\_begin}[$
and $[\mathit{gap\_end}, \mathit{end}[$ can be covered with subregions.

Finally, if $\mathit{gap\_begin}=\mathit{end}$ then $\mathit{gap\_begin} >
\mathit{gap\_end}$ since $\mathit{end} > \mathit{gap\_end}$. Hence, if
$\mathit{gap\_begin}$ is less than $\mathit{gap\_end}$ it is necessarily equal
to $b_0+s$, and is thus a multiple of $s$. A similar argument shows that, in
this case, $\mathit{gap\_end}$ is necessarily a multiple of $s$ too. The third
step above thus satisfies the required hypotheses to repeat the process with
$\mathit{level}+3$.

For an interval inside the 96~KB of RAM, and starting with $\mathit{level}=8$,
this process can be repeated at most 4 times. Indeed, a fifth repetition would
imply a non-empty $[\mathit{gap\_begin}, \mathit{gap\_end}[$ region whose size
is a multiple of $s=2^{17}=128$~KB. And this is not possible since this is
larger than the RAM.

\subsection{Example}

To illustrate this, consider the case
$[\mathit{begin},\mathit{end}[\ = [672,4448]$ (see \cref{fig:mpu-regions}).

At the first iteration, with $\mathit{level}=8$, the first step gives $s=256$,
$b_0 = \lfloor 672/256 \rfloor * 256 = 512$, and $\mathit{gap\_begin} = b_0 + s
= 768$. We thus configure a region with base address 512, size 256, and with
its last 3 subregions enabled so that they cover $[\mathit{begin},
\mathit{gap\_begin}[\ = [672,768[$. The second step gives $b_1 = \lfloor
(4448+255)/256 \rfloor * 256 - 256 = \lfloor (4448-1)/256 \rfloor * 256 = 4352$
and $\mathit{gap\_end} = b_1 = 4352$. We thus configure a second region with
base address 4352, size 256, and with its first 3 subregions enabled so that
they cover $[\mathit{gap\_end}, \mathit{end}[\ = [4352,4448[$.

At the second iteration, with $\mathit{begin}=768$, $\mathit{end}=4352$, and
$\mathit{level}=11$, the first step gives $s=2048$, $b_0 = \lfloor 768/2048
\rfloor * 2048 = 0$, and $\mathit{gap\_begin} = b_0 + s = 2048$. We thus
configure a third region with base address 0, size 2048, and with its last 5
subregions enabled so that they cover $[\mathit{begin}, \mathit{gap\_begin}[ =
[768,2048[$. The second step gives $b_1 = \lfloor (4352-1)/2048 \rfloor * 2048
= 4096$ and $\mathit{gap\_end}=b_1=4096$. We thus configure a fourth region
with base address 4096, size 2048, and with its first subregion enabled so that
it covers $[\mathit{gap\_end}, \mathit{end}[\ = [4096,4352[$.

At the third and last iteration, with $\mathit{begin}=2048$,
$\mathit{end}=4096$, and $\mathit{level}=14$, the first step gives $s=16384$,
$b_0 = \lfloor 2048/16384 \rfloor * 16384 = 0$, and $\mathit{gap\_begin} =
\min(b_0 + s, \mathit{end}) = 4096$. We thus configure a fifth region with base
address 0, size 16384, and with its subregion 1 enabled so that it covers
$[\mathit{begin}, \mathit{gap\_begin}[\ = [2048,4096[$. The second step gives
exactly the same region and subregion.

\begin{Figure}
  \input{figures/chapter6/mpu-regions.tex}

  \caption{The MPU regions used to forbid any memory access outside the
  $[672,4448[$ memory interval (blue). Enabled regions are shown with their
  enabled subregions in green, and their disabled subregions in
  gray.}\label{fig:mpu-regions}
\end{Figure}

\section{Implementation}

\rust{
  let mut t = Transpiler5::new_str(&context.get_file_content("toys.toy"));
  t.add_unchanged("fn os_init(code: &u32, heap: &u32, stack: &u32);",
      "fn get_process_stack_pointer(");
}

The above algorithm uses the $\min$ and $\max$ mathematical functions. The
former is already implemented in our kernel, but the latter is not. We also
need a function to set the CONTROL register with a Move to Special Register
from Register (MSR) instruction (see \cref{subsection:process-stack-pointer}).
We add them just after the {\tt min} function:

\toy{
fn max(x: u32, y: u32) -> u32 {
  if x > y { return x; } else { return y; }
}

fn set_control_register(value: u32) [
  /*MSR_CONTROL_R0*/ 2283074432;
  /*MOV_PC_LR*/ 18167;
]
}%toy

\rust{
  t.add_unchanged("fn get_process_stack_pointer(",
      "fn os_set_current_process(");
}

The MPU regions must be configured when the current process is changed to a new
one (a child or a parent). This is done in {\tt os\_set\_current\_process}. We
thus implement the above algorithm just before this function.

The following function configures the MPU region whose number is $\it{id}$. It
enables it, sets its size to $s=2^{\mathit{level}}$, and sets its base address
and subregions in order to cover the $[\mathit{begin},\mathit{end}[$ interval.
It assumes that $\mathit{begin}$ and $\mathit{end}$ are multiple of $s/8$, that
$\mathit{begin} > \mathit{end}$, and that $\mathit{end}-\mathit{begin} \le s$.
It sets the base address to $\mathit{begin}$ rounded down to a multiple of $s$,
\ie, to $\lfloor \mathit{begin}/s \rfloor * s = (\mathit{begin} \gg
\mathit{level}) \ll \mathit{level}$. It then sets the attributes, size and
disabled sub regions as follows:
\begin{itemize}
  \item $\mathit{access}$ is set to 3 (``full access''), $\mathit{attributes}$
  to 6 (``RAM''), and $e$ to 1 (``enabled''), yielding
  \rs{hex_dec(0x03060001u32)},

  \item $\mathit{size}$ is set to $\mathit{level}-1$ (recall that the region
  size is $2^{\mathit{size}+1}$),

  \item the disabled subregions are set to $255 - ((2^e-1) - (2^b-1)) = 255 -
  2^e + 2^b = 255 - (1 \ll e) + (1 \ll b)$, where $[b,e[$ are
  the indices of the enabled subregions. For instance, if $b=2$ and $e=5$, this
  gives $2^e-1=\bina{11111}$, $2^b-1=\bina{11}$, $(2^e-1) - (2^b-1) =
  \bina{11100}$, and $255 - ((2^e-1) - (2^b-1)) = \bina{11111111} -
  \bina{11100} = \bina{11100011}$, which disables the subregions other than 2,
  3, or 4.
\end{itemize}

\toy{
fn mpu_set_region(id: u32, begin: u32, end: u32, level: u32) {
  const MPU_REGION_BASE_ADDRESS_REGISTER: &u32 = 3758157212;
  const MPU_REGION_ATTRIBUTE_AND_SIZE_REGISTER: &u32 = 3758157216;
  const RAM_FULL_ACCESS_ENABLED: u32 = 50724865;
  let base_address = (begin >> level) << level;
  begin = (begin - base_address) >> (level - 3);
  end = (end - base_address) >> (level - 3);
  *MPU_REGION_BASE_ADDRESS_REGISTER = base_address | 16 | id;
  *MPU_REGION_ATTRIBUTE_AND_SIZE_REGISTER = RAM_FULL_ACCESS_ENABLED |
      (255 - (1 << end) + (1 << begin)) << 8 | (level - 1) << 1;
}
}%toy

We use this function in the following one, which performs one iteration of the
algorithm in \cref{subsection:mpu-regions-algorithm}. It assumes that {\tt
*}$\mathit{begin}$ and {\tt *}$\mathit{end}$ are multiple of
$2^{\mathit{level}-3}$, and that {\tt *}$\mathit{begin} >$ {\tt
*}$\mathit{end}$. It configures the MPU regions with number $\mathit{id}$ and
$\mathit{id}+1$ to cover as much as possible of the $[${\tt *}$\mathit{begin},$
{\tt *}$\mathit{end}[$ interval, and updates these variables to the (possibly
empty) uncovered gap.

\toy{
fn mpu_set_end_regions(id: u32, begin: &u32, end: &u32, level: u32) {
  let gap_begin = min(((*begin >> level) + 1) << level, *end);
  let gap_end = max(*begin, ((*end - 1) >> level) << level);
  mpu_set_region(id, *begin, gap_begin, level);
  mpu_set_region(id + 1, gap_end, *end, level);
  *begin = min(gap_begin, gap_end);
  *end = gap_end;
}
}%toy

The next function calls it 4 times, with increasing $\mathit{level}$ values and
distinct region numbers, to cover the whole $[\mathit{begin}, \mathit{end}[$
interval (whose bounds must be multiple of 32 and whose size must be less than
128~KB). Note that, for some of these calls, $\mathit{begin}$ and
$\mathit{end}$ might be equal. The above functions actually support this, even
if this common value is not a multiple of $2^{\mathit{level}-3}$. In such
cases, they give MPU regions whose all subregions are disabled. Thus, by always
calling {\tt mpu\_set\_end\_regions} exactly 4 times, we make sure that
unneeded regions are disabled (not doing so could keep unwanted regions
configured for the previous process).

\toy{
fn mpu_set_regions(begin: u32, end: u32) {
  mpu_set_end_regions(0, &begin, &end, 8);
  mpu_set_end_regions(2, &begin, &end, 11);
  mpu_set_end_regions(4, &begin, &end, 14);
  mpu_set_end_regions(6, &begin, &end, 17);
}
}%toy

We finally call the above function in {\tt os\_set\_current\_process} to forbid
all memory accesses outside the memory region of the new current process:

\toy{
@fn os_set_current_process(kernel: &Kernel, process: &Process) {
@  set_process_stack_pointer(process.saved_context as &u32);
  mpu_set_regions(process.begin as u32, process.end as u32);
@  kernel.current_process = process;
@}
}%toy

\rust{
  t.add_unchanged("const FALSE: u32 = 0;",
      "\tcontext.status_register = 1 << 24;");
}

To make sure that this process cannot change the MPU registers to grant itself
more access rights, we also need to set the privilege level of Thread mode to
unprivileged. For that we set the $p$ bit of the CONTROL register to 1 in {\tt
os\_spawn}, just before spawning the initial process (which has no parent by
definition):

\toy{
@  context.status_register = 1 << 24;
@  if parent != null {
@    parent.saved_context = get_process_stack_pointer() as &Context;
  } else {
    set_control_register(/*unprivileged*/1);
  }
}%toy

\rust{
  t.add_unchanged("\tlet process = kernel.heap as &Process;",
    "\tconst SVC_HANDLER_PRIORITY_REGISTER:");
}

Last but not least, we need to enable the MPU and the background region with
the MPU Control Register. We do this at the end of the {\tt os\_init} function,
but only if the user presses 'y':

\toy{
@  const SVC_HANDLER_PRIORITY_REGISTER: &u32 = 3758157084;
  const MPU_CONTROL_REGISTER: &u32 = 3758157204;
@  *SVC_HANDLER_PRIORITY_REGISTER = 255 << 24;
  if keyboard_wait_char() == 'y' { *MPU_CONTROL_REGISTER = 5; }
@\}
}%toy

Without this precaution, a bug in the above implementation could cause a crash
in the kernel, the shell, the text editor or the compiler. We would then have
no way to fix the error. Instead, with this precaution, and if a problem
occurs, we can reboot with the MPU disabled and fix the issue.

\rust{
  t.add_unchanged("fn system_call(", "EOF");
}

\section{Compilation and tests}

Type ``{\tt edit toys.toy}'' and Enter to edit the current kernel source code,
and update it as described above. For reference, we also provide this code in
the {\tt toys\_v3.txt} file in \toypcurl{sources.zip}. Then save it and compile
it with ``{\tt toyc toys toys.toy}''. Repeat these steps until the compilation
is successful.

\rust{
  t.write_toy5("website/sources/toys_v3.txt")?;

  let display = Rc::new(RefCell::new(TextDisplay::default()));
  context.set_display(display.clone());
  context.micro_controller().borrow_mut().reset();
  context.run_until_get_char();
  assert_eq!(display.borrow().get_text(), ">");

  // Recompile kernel with MPU protection added.
  context.type_ascii("EDIT TOYS.TOY\n");
  let kernel_source = t.get_toy5();
  context.enter_text_editor_text(&kernel_source);
  context.type_keys(vec!["Escape"]);
  assert_eq!(display.borrow().get_text(), "Save (y/n)?");
  context.type_ascii("Y");
  context.type_ascii("TOYC TOYS TOYS.TOY\n");
  assert_eq!(display.borrow().get_text(), ">toyc toys toys.toy\n>");

  let mut t = Transpiler5::new_str(&context.get_file_content("hello.toy"));
  t.add_unchanged("fn main();", "fn main() ");
}

To test these changes we can introduce a voluntary bug in our ``{\tt hello}''
program. Type ``{\tt edit hello.toy}'' to edit it and update its main function
as follows:

\toy{
@fn main() {
@  write(1 /*standard output*/, HELLO, 13);
  const KERNEL_POINTER_REGISTER: &u32 = 1074666140;
  *KERNEL_POINTER_REGISTER = 0;
@  exit(0);
@}
}%toy

Then type ``{\tt toyc hello hello.toy}'' to compile it. This bug sets to 0 the
General Purpose Backup Register containing the address of the {\tt Kernel} data
structure (see its definition in ``{\tt toys.toy}''). It could thus make the
kernel unable to access its own data structures. To confirm this, type ``{\tt
hello}'' to run this program. The shell should become unresponsive, because the
kernel crashed. Indeed, we haven't restarted it yet, and the MPU is thus not
yet enabled.

\rust{
  context.type_ascii("EDIT HELLO.TOY\n");
  context.enter_text_editor_text(&t.get_toy5());
  context.type_keys(vec!["Escape"]);
  assert_eq!(display.borrow().get_text(), "Save (y/n)?");
  context.type_ascii("Y");
  // Step 2: compile it
  context.type_ascii("TOYC HELLO HELLO.TOY\n");
  assert_eq!(display.borrow().get_text(), ">toyc hello hello.toy\n>");
  // Step 3: run it -> crash of kernel

  let result = std::panic::catch_unwind(std::panic::AssertUnwindSafe(|| {
    context.type_ascii("HELLO\n");
  }));
  assert!(result.is_err());
  assert_eq!(
      context.micro_controller().borrow_mut().debug_get32(1074666140), 0);
}

Now reset the Arduino, and type ``{\tt y}'' to enable the MPU. This should
launch the shell, and you should be able to launch the text editor and compile
programs. If not reset the Arduino again and type any key other than ``{\tt
y}'' to launch the kernel without MPU. Then repeat the steps from the beginning
of this section.

\rust{
  // Restart to test new kernel with MPU
  context.micro_controller().borrow_mut().reset();
  context.run_until_get_char();
  context.type_ascii("Y"); // enable MPU
  assert_eq!(display.borrow().get_text(), ">");
}

Finally, type ``{\tt hello}'' to run this program again. This time the MPU
should trigger a Hard Fault when the process attempts to set the {\tt
KERNEL\_POINTER\_REGISTER} to 0. This should cause the kernel to terminate this
process with an {\tt INTERNAL\_ERROR} status. In turn, the shell should print
``{\tt Hello, World!}'' followed by the message ``{\tt hello crashed}''. But
the shell and the kernel should now continue to work.

\rust{
  // Retry re-run, no crash of kernel
  let result = std::panic::catch_unwind(std::panic::AssertUnwindSafe(|| {
    context.type_ascii("HELLO\n");
  }));
  assert!(result.is_err());
  assert_ne!(
      context.micro_controller().borrow_mut().debug_get32(1074666140), 0);

  // Restart (only needed in emulator)
  context.micro_controller().borrow_mut().reset();
  context.run_until_get_char();
  context.type_ascii("Y"); // enable MPU

  let mut t = Transpiler5::default();
}

At this stage we no longer need the possibility to disable the MPU when
launching the kernel. Type ``{\tt edit toys.toy}'' and replace the line

\toy{
  if keyboard_wait_char() == 'y' { *MPU_CONTROL_REGISTER = 5; }
}%toy

\noindent with

\toy{
  *MPU_CONTROL_REGISTER = 5; /*Enable MPU and background region*/
}%toy

\noindent Then recompile the kernel with ``{\tt toyc toys toys.toy}'' and reset
the Arduino. The shell should start without having to press any key.

\rust{
  // Remove temporary security measure.
  context.type_ascii("EDIT TOYS.TOY\n");
  context.enter_text_editor_text(&kernel_source.replace(
    "if keyboard_wait_char() == 'y' { *MPU_CONTROL_REGISTER = 5; }",
    "*MPU_CONTROL_REGISTER = 5; /*Enable MPU and background region*/"));
  context.type_keys(vec!["Escape"]);
  assert_eq!(display.borrow().get_text(), "Save (y/n)?");
  context.type_ascii("Y");
  context.type_ascii("TOYC TOYS TOYS.TOY\n");
  assert_eq!(display.borrow().get_text(), ">toyc toys toys.toy\n>");

  // Check security measure removed.
  context.micro_controller().borrow_mut().reset();
  context.run_until_get_char();
  assert_eq!(display.borrow().get_text(), ">");
}
