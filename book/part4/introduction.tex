% This work is licensed under the Creative Commons Attribution NonCommercial
% ShareAlike 4.0 International License. To view a copy of the license, visit
% https://creativecommons.org/licenses/by-nc-sa/4.0/

\chapter*{Introduction}
\addcontentsline{toc}{chapter}{Introduction}

We now have a toy programming language, hereafter called Toy, which makes it
much easier to program our computer (compared to what it was at the end of
\cref{part:computer}). In particular, we no longer need to manually keep track
of function addresses, instruction offsets, local variable indices, etc.
However, we still need to manually keep track of the flash memory content with
maps such as the one in \cref{fig:final-toyc-memory-map}. This is necessary to
find unused memory regions where new programs can be stored, without overriding
existing programs or data. Likewise, we need to manually keep track of the RAM
content with maps such as the one in \cref{fig:command-editor-memory-map}. This
is necessary to find unused RAM regions where programs can store their data,
without overriding the data of other programs. This manual work is quite
tedious and error prone. The main goal of this part is thus to implement a
program which can do it for us. Such a program is called an {\em operating
system}.

In the previous part we removed the need of function addresses and local
variable indices by using symbolic names for functions and local variables, and
by letting the compiler maintain a map between the two. We can do the same for
programs or other data stored in flash memory. More precisely, we can use {\em
file names} to refer to pieces of data stored in flash memory, hereafter called
{\em files}, and let the operating system maintain a map between file names and
flash memory addresses. This map must obviously be stored in flash memory too,
otherwise it would be lost after a reset. The precise way of storing it, as
well as the files themselves, is called a {\em file system}.

In summary, the main goal of this part is to implement a toy operating system,
based on a file system. Another goal is to eventually get rid of our bytecode
interpreter, since we can now compile programs into native code, which is much
faster than bytecode. This implies rewriting the drivers and programs from the
previous parts, written directly in binary bytecode form, into Toy source code.
We achieve these goals in seven steps, presented in as many chapters:

\begin{itemize}
\item \cref{chapter:file-system} defines of a toy file system and provides
corresponding functions to create, read, write, and delete files. We use
them at the end to initialize a new, empty file system in the Flash0 memory
bank (so far we only used the Flash1 bank -- see
\cref{fig:sam3x8e,fig:boot-memory-map}).

\item \cref{chapter:boot-loader-and-drivers} provides a new implementation, in
Toy, of the clock, graphics card, and keyboard drivers from
\cref{part:computer}. We test them at the end with a small program which simply
displays on the screen each key typed on the keyboard, as in
\cref{chapter:keyboard}. This time, however, we compile it in native code, and
we store it in our file system. We also implement a {\em boot loader} to start
it directly after reset, without going through the memory editor.

\item \cref{chapter:processes} provides functions to start and stop programs
stored in the file system, while keeping track of the RAM used by each running
program, called {\em processes}. Together with the file system and driver
functions, they constitute the first version of a program called the operating
system {\em kernel}. This chapter also provides a way for processes to use
services provided by the kernel.

\item \cref{chapter:streams} extends the kernel with new services providing a
simple, unified, and safe way to use the computer resources. These include the
keyboard, the graphics card, and the files. All these resources are
used via byte sequences called {\em streams}.

\item \cref{chapter:shell} provides a better and easier to use version of the
command editor, called a {\em command-line interpreter}, or {\em shell}. This
program is automatically started by the kernel after a reset. Its role is to
start other programs, with commands typed by the user. This chapter also
provides a new implementation of the text editor from
\cref{chapter:text-editor}, in Toy. Together with the Toy compiler and the
shell, stored in the file system, this gives an autonomous, {\em self-hosted}
operating system. This means that we can edit and recompile its entire source
code with itself, without needing the bytecode interpreter, the memory editor,
or any other program written in the previous parts.

\item \cref{chapter:memory-protection} illustrates this self-hosting property by
using the shell, the text editor, and the compiler, running with the operating
system, to replace its kernel with a new version. This new version uses the
Memory Protection Unit from the microcontroller to protect the kernel and each
program from bugs in other programs.

\item \cref{chapter:utilities} completes our operating system with a few small
utility programs, in particular to list, copy, and delete files. It also
provides a better shell version.
\end{itemize}

Finally, just for fun, we conclude this book with a small game implemented with
our toy computer, in \cref{chapter:snake-game}.
