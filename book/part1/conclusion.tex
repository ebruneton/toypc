% This work is licensed under the Creative Commons Attribution NonCommercial
% ShareAlike 4.0 International License. To view a copy of the license, visit
% https://creativecommons.org/licenses/by-nc-sa/4.0/

\chapter*{Conclusion}
\addcontentsline{toc}{chapter}{Conclusion}

Microprocessors are made of electrically controlled switches, assembled into
logic gates, themselves assembled into arithmetic and logic circuits,
registers, memories, and control circuits. Most of them use the architecture
presented in this part, made of a control unit, a processing unit, input and
output mechanisms, and a memory containing data and instructions. It is called
the von Neumann architecture because it was first published by John von Neumann
in 1945.

The toy microprocessor presented in the previous chapter is extremely limited,
especially because it can only address 32 bytes of memory. To solve this
problem, most modern microprocessors use a 32 or 64-bit architecture. A 32-bit
architecture is based on 32-bit registers, buses, and Arithmetic and Logic
Units. A 32-bit microprocessor generally use 32-bit addresses, and can thus use
up to $2^{32} =$ 4~GB of memory. It can also perform computations on 32-bit
values, called {\em words}\footnote{A word is a 32-bit value in a 32-bit
microprocessor, a 64-bit value in a 64-bit microprocessor, etc.}. However, even
with more memory, microprocessors can only do computations on numbers. But then
how can computers create or edit text, music or videos?

\subsubsection{Data representation}

The answer is that text, sound, images, videos, or in fact any information, can
be represented with numbers. And doing some computations on these numbers can
edit this information.

For instance, one could represent letters from ``a'' to ``z'' with numbers from
$0$ to $26$ (excluded), letters from ``A'' to ``Z'' with numbers from $26$ to
$52$ (excluded), a space with $52$, a dot with $53$, etc. Then, for instance, a
program reading bytes and adding $26$ to those smaller than $26$ can
capitalize text. A program can also edit text based on keys typed by the user,
as shown in \cref{chapter:text-editor}. Or transform text into program
instructions which, as explained in \cref{section:toy-insn-set}, can also be
represented with numbers (see \cref{part:compiler}).

Similarly, one can {\em digitize} sound, which is a pressure wave, by measuring
the pressure many times per second, and by storing each measurement in one byte
(or more). Then, for instance, a program reading pairs of bytes, and computing
their average, can mix two sound samples to produce a new one.

Likewise, one can digitize an image by decomposing it in a grid of {\em pixels}
(for ``picture element''), and by representing the luminosity of each pixel
with one byte (from $0$ for black to $255$ for white). These values can then be
stored in memory one after the other, for example from left to right and from
top to bottom. Then, for instance, a program reading bytes and subtracting them
from $255$ can compute the negative of an image. A color image can be
represented in a similar way, with 3 values per pixel (for the intensity of the
red, green, and blue components). And a video can be represented as a sequence
of images (typically 24 or 25 per second).

\subsubsection{Further readings}

The design of our toy microprocessor, although sufficient to explain the main
ideas, is not fully representative of real ones. For instance, real
microprocessors do not wait for input values. Instead, when a new input value
is available, it interrupts the microprocessor (like a notification or a phone
call interrupts what you are doing). To learn more about the design and
architecture of real microprocessors, you can read the following books:
\begin{itemize}
  \item ``Digital Design and Computer Architecture, RISC-V Edition''
  \cite{DigitalDesignAndComputerArchitecture}. This book presents the CMOS
  technology (the most frequently used to build real microprocessors), gives
  more details about logical operations (negation, conjunctions, disjonctions,
  etc), explains how programs can be used to design complex circuits, and
  presents some advanced microprocessor architectures used to improve
  performance.

  \item ``Computer Architecture'' \cite{ComputerArchitecture}. This book gives
  a historic perspective on computer design, gives more details about number
  systems, data representation, input and output mechanisms, etc. It also
  presents several microprocessor architecture styles.
\end{itemize}
