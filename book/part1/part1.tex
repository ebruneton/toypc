% This work is licensed under the Creative Commons Attribution NonCommercial
% ShareAlike 4.0 International License. To view a copy of the license, visit
% https://creativecommons.org/licenses/by-nc-sa/4.0/

\tikzstyle{cv_font} = [font={\footnotesize\sffamily},color=green0]
\tikzstyle{cv_smallfont} = [font={\tiny\sffamily}]

\tikzset{
  AndGate/.pic={
    \path[draw=black] (-10, -20) -- (-10, 20) -- (0, 20)
    .. controls (12, 20) and (20, 11) .. (20, 0)
    .. controls (20, -11) and (12, -20) .. (0, -20) -- cycle;
  },
  Button/.pic={
    \path[draw=black] (13, 0) -- (30, 0);
    \path[draw=black,fill=gray1] (0, 0) circle[radius=13];
  },
  Clock/.pic={
    \path[draw=black] (-10, -10) rectangle +(20, 20);
    \path[draw=green2,line width=0.3mm] (-6, 0) -- (-6, -5.5) -- (0, -5.5) --
    (0, 5.5) -- (6, 5.5) -- (6, 0);
  },
  Demultiplexer/.pic={
    \path[draw=black] (10, -10) -- (10, 10) -- (-10, 20) -- (-10, -20) -- cycle;
    \node[cv_smallfont,anchor=center] at (-4,-10){1};
    \node[cv_smallfont,anchor=center] at (-4,10){0};
  },
  DflipFlop/.pic={
    \path[draw=black] (-20, -20) rectangle +(40, 40);
    \path[draw=black] (-20, 5) -- (-15, 10) -- (-20, 15);
  },
  DigitalLed/.pic={
    \path[draw=gray1] (-20, 0) -- (-40, 0);
    \path[fill=gray1] (-20, -9) rectangle +(20,18);
    \path[fill=gray1] (0, 0) circle[radius=9];
  },
  DigitalLedOn/.pic={
    \path[draw=gray1] (-20, 0) -- (-40, 0);
    \path[fill=red0] (-20, -9) rectangle +(20,18);
    \path[fill=red0] (0, 0) circle[radius=9];
  },
  Ground/.pic={
    \path[draw=black] (0, -10) -- (0, 0) (-10, 0) -- (10, 0)
    (-6, 5) -- (6, 5) (-3, 10) -- (3, 10);
  },
  Input/.pic={
    \path[draw=black] (-10, -10) rectangle +(20, 20);
  },
  Multiplexer/.pic={
    \path[draw=black] (10, -10) -- (10, 10) -- (-10, 20) -- (-10, -20) -- cycle;
    \node[cv_smallfont,anchor=center] at (-4,-10){0};
    \node[cv_smallfont,anchor=center] at (-4,10){1};
  },
  NandGate/.pic={
    \path[draw=black] (-10, -20) -- (-10, 20) -- (0, 20)
    .. controls (12, 20) and (20, 11) .. (20, 0)
    .. controls (20, -11) and (12, -20) .. (0, -20) -- cycle;
    \path[draw=black] (25, 0) circle[radius=5];
  },
  NorGate/.pic={
    \path[draw=black] (-10, -20) .. controls (0, 0) .. (-10, 20)
    .. controls (0, 20) and (15, 10) .. (20, 0)
    .. controls (15, -10) and (0, -20) .. cycle;
    \path[draw=black] (25, 0) circle[radius=5];
  },
  NotGate/.pic={
    \path[draw=black] (-10, -10) -- (-10, 10) -- (10, 0) -- cycle;
    \path[draw=black] (15, 0) circle[radius=5];
  },
  OrGate/.pic={
    \path[draw=black] (-10, -20) .. controls (0, 0) .. (-10, 20)
    .. controls (0, 20) and (15, 10) .. (20, 0)
    .. controls (15, -10) and (0, -20) .. cycle;
  },
  Output/.pic={
    \path[draw=blue0] (-10, -10) rectangle +(20, 20);
  },
  OutputX/.pic={
    \path[draw=red0] (-10, -10) rectangle +(20, 20);
  },
  Power/.pic={
    \path[draw=black] (0, 10) -- (0, 0) -- (10, 0) -- (0, -10)
    -- (-10, 0) -- (0, 0);
  },
  SRflipFlop/.pic={
    \path[draw=black] (-20, -20) rectangle +(40,40);
  },
  TriState/.pic={
    \path[draw=black] (-10, -15) -- (20, 0) -- (-10, 15) -- cycle;
  },
  XorGate/.pic={
    \path[draw=black] (-10, -20)
    .. controls (0, 0) .. (-10, 20)
    .. controls (0, 20) and (15, 10) .. (20, 0)
    .. controls (15, -10) and (0, -20) .. cycle;
    \path[draw=black] (-20, -20)
    .. controls (-6, -10) and (-6, 10) .. (-20, 20);
  },
}

\newcommand*{\cvmath}[1]{\normalsize $#1$}

% This work is licensed under the Creative Commons Attribution NonCommercial
% ShareAlike 4.0 International License. To view a copy of the license, visit
% https://creativecommons.org/licenses/by-nc-sa/4.0/

\renewcommand{\rustfile}{introduction}
\setcounter{rustid}{0}

\chapter*{Introduction}
\addcontentsline{toc}{chapter}{Introduction}

Our toy computer is now fully assembled and autonomous. However, it is still
very hard to use. Indeed, even if our virtual machine instructions are much
simpler than ARM instructions, they remain difficult to use. The goal of this
part is thus to provide an easier way to program our toy computer.

To illustrate how this can be done, consider the task of writing a program to
compute the factorial of a number $n$, defined by $factorial(n)=1*2*\ldots*n$
if $n > 0$, and 1 otherwise. One way to write such a program is to use the
property that, for $n>0$, $factorial(n)=factorial(n-1) * n$. This leads to the
following bytecode instructions:

\rust{
  let address = 0xC1000;
  let mut b = BytecodeAssembler::new(RegionKind::Default, address);
}

\begin{TwoColumns}
\rs{b.func("factorial", &["n"], "", &[])}\\
\bytecode{
  b.get("n");
  b.cst_0();
  b.ifne("not_zero");
  b.cst_1();
  b.retv();
  b.label("not_zero");
  b.get("n");
  b.cst_1();
  b.sub();
  b.call("factorial");
  b.get("n");
  b.mul();
  b.retv();
}
\end{TwoColumns}

The first part, in the left column, compares $n$, the $0^{th}$ word in the
function's stack frame, with 0. If it is not equal to 0, it jumps to the second
part, in the right column. Otherwise it returns 1. The second part, in the
right column, subtracts 1 from $n$, calls the factorial function with this
argument (we assume here that it is stored at \rs{hex(address)}), and returns
the result multiplied by $n$. In order to write this program we need to type on
the keyboard the following {\em numbers}:

\rs{b.get_bytecode_listing(0..b.get_instruction_count() as usize, false)}

Typing numbers is error prone, and understanding their meaning requires a lot
of effort. It would be much easier if we could type the following {\em text}
instead:

\begin{Code}
fn 1
  get 0 cst_0 ifne 10 cst_1 retv
  get 0 cst_1 sub call 4096 get 0 mul retv
\end{Code}

In fact this text still contains some numbers, such as 0, 10=\hexa{A} and
4096=\hexa{1000}. The last two are quite tedious to compute. For instance, one
must sum the size of all the instructions up to the first \insn{retv}
(included) to get the value 10. Function addresses such as 4096 also require to
compute the bytecode size of each function to keep track of their addresses. To
avoid having to do this, it would be simpler if we could use {\em labels} to
designate instructions, and {\em identifiers} to designate functions:

\begin{Code}
fn \textbf{factorial} 1
  get 0 cst_0 ifne \textbf{not_zero} cst_1 retv
  \textbf{:not_zero} get 0 cst_1 sub call \textbf{factorial} get 0 mul retv
\end{Code}

Similarly, to get rid of the last numbers, it would be useful to be able to
give names to the function parameters. We could then replace 1, the number of
parameters of the factorial function, with a list of parameter names. And we
could replace 0 with $n$:

\begin{Code}
fn factorial\textbf{(n)}
  get \textbf{n} cst_0 ifne not_zero cst_1 retv
  :not_zero get \textbf{n} cst_1 sub call factorial get \textbf{n} mul retv
\end{Code}

This text would already be much easier to type and to understand than the above
numbers. However, it still contains long sequences of bytecode instructions
which are more complex than the mathematical {\em expressions} they compute.
For instance, \verb!get n cst_1 sub call factorial get n mul! computes
$factorial(n-1)*n$. It would be much easier if we could use these expressions
directly (this example also adds some ``punctuation'' signs, namely curly
braces, commas, and semi-colons):

\begin{Code}
fn factorial(n) \textbf{\{}
  \textbf{n, 0} ifne not_zero; \textbf{1} retv;
  :not_zero \textbf{factorial(n - 1) * n} retv;
\textbf{\}}
\end{Code}

It would also be more natural if we could write the last remaining bytecode
instructions in a different order, closer to the order of words in English. And
instead of writing ``if $n$ is not 0, jump to :not\_zero to not return 1'', it
would be simpler to write ``if $n$ is 0, return 1''. We could even get rid of
the label by putting the instructions to execute when $n$ is 0 inside curly
braces:

\begin{Code}
fn factorial(n) \{
  \textbf{if} n \textbf{==} 0 \textbf{\{ return} 1; \textbf{\}}
  \textbf{return} factorial(n - 1) * n;
\}
\end{Code}

Finally, to make it clear that this function returns a value (some do not), and
takes a number as parameter, it would be practical to have some {\em type
declarations}, such as the following (where \verb!u32! means an ``unsigned 32
bit'' value):

\begin{Code}
fn factorial(n: \textbf{u32}) \textbf{-> u32} \{
  if n == 0 \{ return 1; \}
  return factorial(n - 1) * n;
\}
\end{Code}

In fact the goal of this part is to be able to type programs in this form
(inspired from Rust~\cite{RustProgrammingLanguage}), and to {\em automatically}
get the corresponding bytecode instructions, in numerical form (also called
{\em binary} form). For this we write a program, called a {\em compiler}, which
transforms the program text, called the {\em source code}, into {\em compiled
code}, \ie, bytecode instructions in binary form. This compiler is a large
program, and we can't write it in source code because we don't have a compiler
yet! On the other hand, writing it in binary form would be very hard. To solve
this problem we write the compiler in several steps:

\begin{enumerate}
\item Write a small compiler to compile textual bytecode instructions (\eg,
{\tt fn 1 get 0 $\ldots$}). Write this {\em opcodes compiler} in binary form.

\item Write a compiler for programs using textual bytecode instructions with
function names and instruction labels. Write this {\em labels compiler}, also
called an {\em assembler}, with pure textual bytecode instructions. Compile it
with the opcodes compiler.

\item Rewrite the labels compiler with function names and labels, and improve
it so that it can compile programs using expressions such as {\tt factorial(n -
1) * n}. Compile this {\em expressions compiler} with the labels compiler.

\item Rewrite the expressions compiler with expressions, and improve it in
order to support programs using {\em statements} such as {\tt if} and {\tt
return}. Compile this {\em statements compiler} with the expressions compiler.

\item Rewrite the statements compiler with statements, and improve it to accept
programs with type declarations such as {\tt factorial(n: u32) -> u32 \{ ...
\}}. Compile this {\em types compiler} with the statements compiler.
\end{enumerate}

With this method, only the first compiler needs to be written in binary form.
And this compiler is small because its task is quite simple. Moreover, each
step is easier to do than the previous one because it can use simplifying
features introduced in the previous steps. However, in order to implement this
method, we need a way to type text, \ie, a {\em text editor}. Indeed, all we
have for now is the memory editor, and entering text in memory with it would
require typing the ASCII code {\em numbers} corresponding to each character of
the text! Obviously, this would be even worse than entering programs directly
in binary form, since a program in text form is usually longer than in binary
form. Unfortunately, we can't write a text editor program in any textual form,
since we don't have a text editor yet! We thus start by writing a simple text
editor, directly in binary form.

Before all this, however, we also need a way to store programs in source or
binary form in flash memory. Otherwise, we would loose everything we typed if
we ever do a mistake causing a crash. For this we implement one more driver,
called the flash memory driver (we didn't need this in the previous part
because flashing programs was done from the external computer). This driver
also needs to be implemented in binary form. The rest of this part presents the
above steps in detail. It is organized as follows:

\begin{itemize}
\item \cref{chapter:flash-driver} presents the flash memory driver, used to
read and write data in flash memory. We use it at the end to save itself in
flash memory.

\item \cref{chapter:text-editor} explains how our text editor works, and
presents its implementation in binary form.

\item \cref{chapter:opcodes-compiler} explains how the opcodes compiler works,
and presents its implementation in binary form.

\item \cref{chapter:labels-compiler} explains how the labels compiler works, and
presents its implementation in textual form.

\item \cref{chapter:expressions-compiler} explains what expressions are, how
they can be compiled, and presents the ones supported by our toy compiler. It
then gives the corresponding implementation.

\item \cref{chapter:statements-compiler} does the same with statements.

\item \cref{chapter:types-compiler} does the same with type declarations.

\item \cref{chapter:native-compiler} finally provides a new version of our toy
compiler which produces ARM instructions instead of bytecode instructions. We
use it in the next part to eventually get rid of our virtual machine.
\end{itemize}

% This work is licensed under the Creative Commons Attribution NonCommercial
% ShareAlike 4.0 International License. To view a copy of the license, visit
% https://creativecommons.org/licenses/by-nc-sa/4.0/

\chapter{Binary Numbers}\label{chapter:binary-numbers}

As its name implies, a computer performs computations, on numbers. A number is
an abstract concept which can be represented in many different concrete ways.
For example, the number of days in a week can be represented with ``seven'',
``7'', ``VII'', etc. Some representation methods, also called numeral systems,
are more practical than others to perform computations. For instance, doing
additions and multiplications is easier in the arabic numeral system than in
the roman one. In fact they are even easier to do in the so called {\em binary
numeral system}. Computers use it for this reason. In order to understand how
they work it is thus necessary to know first what binary numbers are, and how
to compute with them. This is the goal of this chapter.

\section{Binary numbers}

An arabic number such as 237 represents 2 times 100, plus 3 times 10, plus 7
times 1. In mathematical notation this gives
$$237 = 2*100+3*10+7*1 = 2*10^2+3*10^1+7*10^0$$
where $x^n$ denotes 1 if $n=0$, or $x * x * \ldots\,x$ ($n$ times) otherwise.
In other words, an arabic number is a sequence of digits between 0 and 9, where
the $i^{th}$ digit from the right (counting from 0) represents a quantity of
$10^i$.

A binary number is similar but uses two digits instead of ten, namely 0 and 1,
called {\em bits}. It is thus a sequence of bits, where the $i^{th}$ bit from
the right (counting from 0) represents a quantity of $2^i$. For example
$$101_2 = 1*2^2+0*2^1+1*2^0 = 1*4+0*2+1*1 = 5$$
where the subscript 2 indicates a binary number (to avoid confusions with
arabic numbers; $101_2 = 5 \ne 101$ = ``one hundred one''). Another example is
\begin{align*}
11101101_2 &= 1*2^7+1*2^6+1*2^5+0*2^4+1*2^3+1*2^2+0*2^1+1*2^0\\
  &=1*128+1*64+1*32+0*16+1*8+1*4+0*2+1*1\\
  &=237
\end{align*}

The leftmost bit of a binary number is called its {\em most significant} bit.
Conversely, the rightmost bit is called the {\em least significant}. The
$i^{th}$ bit from the right (counting from 0), is called bit number $i$, or
simply bit $i$.

Some numbers have a very simple binary representation and are frequently used.
For instance, $2^n$ is a one followed by $n$ zeros in binary, like $10^n$ in
arabic notation. Another example is $2^n-1$, which is simply $n$ ones (like
$10^n-1$ is $n$ nines in arabic). \cref{table:powers-of-two} gives some
examples of these numbers.

\begin{Table}
  \begin{tabular}{|r|r|r|r|r|}\hline
    \makecell{\thead[r]{n}} &
      \thead[r]{$2^n$} & \thead[r]{$2^n$} &
      \thead[r]{$2^n-1$} & \thead[r]{$2^n-1$} \\ \hline
    \makecell[r]{$0$} & $1$ & $1_2$ & $0$ & $0$ \\
    \makecell[r]{$1$} & $2$ & $10_2$ & $1$ & $1$ \\
    \makecell[r]{$2$} & $4$ & $100_2$ & $3$ & $11_2$ \\
    \makecell[r]{$3$} & $8$ & $1000_2$ & $7$ & $111_2$ \\
    \makecell[r]{$4$} & $16$ & $10000_2$ & $15$ & $1111_2$ \\
    \makecell[r]{$5$} & $32$ & $100000_2$ & $31$ & $11111_2$ \\
    \makecell[r]{$6$} & $64$ & $1000000_2$ & $63$ & $111111_2$ \\
    \makecell[r]{$7$} & $128$ & $10000000_2$ & $127$ & $1111111_2$ \\
    \makecell[r]{$8$} & $256$ & $100000000_2$ & $255$ & $11111111_2$ \\
    \makecell[r]{$16$} & $65536$ & $10000000000000000_2$
      & $65535$ & $1111111111111111_2$ \\ \hline
  \end{tabular}
  \caption{Some frequently used powers of 2, in arabic and binary
  notation.}\label{table:powers-of-two}
\end{Table}

\section{Arithmetic operations}

\subsection{Addition}\label{subsection:binary-add}

Adding two binary numbers can be done as with arabic numbers. Namely one column
at a time, from right to left. For instance, adding $1101010_2$ and $101110_2$
can be done as follows:

\newlength{\defaulttabcolsep}
\setlength{\defaulttabcolsep}{\tabcolsep}
\newcommand*{\carry}[2]{\color{red0} \overset{#1}{\textcolor{black}{#2}}}
\newcommand*{\borrow}[2]{\color{red0} \underset{#1}{\textcolor{black}{#2}}}

\setlength{\tabcolsep}{1pt}
\begin{center}
\begin{tabular}{rrrrrrrrr}
  & $\carry{1}{\phantom{0}}$ & $\carry{1}{1}$ & $1$ & $\carry{1}{0}$ &
                 $\carry{1}{1}$ & $\carry{1}{0}$ & $1$ & $0$ \\
$+$ &              &     & $1$ & $0$ & $1$ & $1$ & $1$ & $0$ \\
\hline \makecell{~}
             & $1$ & $0$ & $0$ & $1$ & $1$ & $0$ & $0$ & $0$
\end{tabular}
\hspace{16mm}
\begin{tabular}{rrrr}
    & $1$ & $\carry{1}{0}$ & $6$ \\
$+$ &                & $4$ & $6$ \\
\hline \makecell{~}
               & $1$ & $5$ & $2$
\end{tabular}
\end{center}
\setlength{\tabcolsep}{\defaulttabcolsep}

\noindent Starting from the right, we add $0$ and $0$, which gives $0$. We then
add $1$ and $1$, which gives $2 = 10_2$. Since this is more than one bit, we
put the least significant one, here $0$, in the current column, and we {\em
carry} the most significant one, here $1$, in the column on the left (shown in
red). This is similar to the addition of the equivalent arabic numbers, shown
on the right, where $6 + 6$ gives $12$, leading to a carry of $1$.

We continue by adding $0$ and $1$, plus the carry from the previous column,
which gives $2$ again. In the next step we add $1$ and $1$, plus the carry from
the previous column, which gives $3 = 11_2$. We thus put $1$ at the bottom of
this column, and carry $1$ in the next one. And so on for the remaining columns.

Although the overall process is the same for binary and arabic numbers, adding
binary numbers is much easier, as stated above. Indeed, there are only $2*2=4$
possible cases when adding two bits, but $10*10=100$ cases when adding two
decimal digits. These four cases are summarized in \cref{table:addition-table}.

\begin{Table}
  \begin{tabular}{r|r|r|r|} \cline{2-4} result bit:
    & \makecell{\thead[r]{a}} & \thead[r]{b} & \thead[r]{a+b} \\ \cline{2-4}
    & \makecell{0} & 0 & 0 \\
    & \makecell{0} & 1 & 1 \\
    & \makecell{1} & 0 & 1 \\
    & \makecell{1} & 1 & 0 \\ \cline{2-4}
  \end{tabular}
  \hspace{8mm}
  \begin{tabular}{r|r|r|r|} \cline{2-4} carry bit:
    & \makecell{\thead[r]{a}} & \thead[r]{b} & \thead[r]{a+b} \\ \cline{2-4}
    & \makecell{0} & 0 & 0 \\
    & \makecell{0} & 1 & 0 \\
    & \makecell{1} & 0 & 0 \\
    & \makecell{1} & 1 & 1 \\ \cline{2-4}
  \end{tabular}
  \caption{The binary addition tables.}\label{table:addition-table}
\end{Table}

\subsection{Subtraction}

Similarly, subtracting two binary numbers can be done as with arabic numbers.
For instance, subtracting $101110_2$ from $1101010_2$ can be done as follows:

\setlength{\tabcolsep}{1pt}
\begin{center}
\begin{tabular}{rrrrrrrr}
             & $1$ & $1$ & $0$ & $1$ & $0$ & $1$ & $0$ \\
$-$ & $\borrow{1}{\phantom{0}}$ & $\borrow{1}{1}$ &
   $\borrow{1}{0}$ & $\borrow{1}{1}$ & $1$ & $1$ & $0$ \\
\hline \makecell{~}
             & $0$ & $1$ & $1$ & $1$ & $1$ & $0$ & $0$
\end{tabular}
\hspace{16mm}
\begin{tabular}{rrrr}
    &                       $1$ & $0$ & $6$ \\
$-$ & $\borrow{1}{\phantom{0}}$ & $4$ & $6$ \\
\hline \makecell{~}
                          & $0$ & $6$ & $0$
\end{tabular}
\end{center}
\setlength{\tabcolsep}{\defaulttabcolsep}

Starting from the right, we subtract $0$ from $0$, and then $1$ from $1$, which
gives $0$ in both cases. In the next step, since we cannot subtract $1$ from
$0$, we subtract it from $10_2=2$ instead, which gives $1$. We thus put a $1$
in the current column, and a carry of $1$ in the subtrahend on the left column
(shown in red). This is similar to the subtraction of the equivalent arabic
numbers, shown on the right, where $0 - 4$ is replaced with $10 - 4$, yielding
the result $6$ and the carry $1$.

We continue by subtracting $1$, plus the carry from the previous column (\ie, a
total of $2$), from $1$. Since this is not possible we subtract them from
$11_2=3$ instead, which gives the result $1$ and the carry $1$. And so on for
the remaining columns.

As with additions, there are only four possible cases when subtracting two
bits, which is much simpler than the hundred possible cases for decimal
digits. These four cases are summarized in \cref{table:subtraction-table}.

\begin{Table}
  \begin{tabular}{r|r|r|r|} \cline{2-4} result bit:
    & \makecell{\thead[r]{a}} & \thead[r]{b} & \thead[r]{a-b} \\ \cline{2-4}
    & \makecell{0} & 0 & 0 \\
    & \makecell{0} & 1 & 1 \\
    & \makecell{1} & 0 & 1 \\
    & \makecell{1} & 1 & 0 \\ \cline{2-4}
  \end{tabular}
  \hspace{8mm}
  \begin{tabular}{r|r|r|r|} \cline{2-4} carry bit:
    & \makecell{\thead[r]{a}} & \thead[r]{b} & \thead[r]{a-b} \\ \cline{2-4}
    & \makecell{0} & 0 & 0 \\
    & \makecell{0} & 1 & 1 \\
    & \makecell{1} & 0 & 0 \\
    & \makecell{1} & 1 & 0 \\ \cline{2-4}
  \end{tabular}
  \caption{The binary subtraction tables.}\label{table:subtraction-table}
\end{Table}

\subsection{Multiplication}\label{subsection:binary-mult}

Multiplying two binary numbers can also be done as with arabic numbers. Namely
by multiplying the first by each bit / digit of the second. And by adding the
results, each shifted by one bit / digit to the left from the previous one. For
instance, multiplying $1101010_2$ by $101110_2$ can be done as follows:

\setlength{\tabcolsep}{1pt}
\begin{center}
  \begin{tabular}{rrrrrrrrrrrrrr}
                        & & & & & & & $1$ & $1$ & $0$ & $1$ & $0$ & $1$ & $0$ \\
                        & $*$ & & & & & & & $1$ & $0$ & $1$ & $1$ & $1$ & $0$ \\
\cline{2-14} \makecell{~}
                        & & & & & & & $0$ & $0$ & $0$ & $0$ & $0$ & $0$ & $0$ \\
                    & & & & & & $1$ & $1$ & $0$ & $1$ & $0$ & $1$ & $0$ & \\
                & & & & & $1$ & $1$ & $0$ & $1$ & $0$ & $1$ & $0$ & & \\
            & & & & $1$ & $1$ & $0$ & $1$ & $0$ & $1$ & $0$ & & & \\
        & & & $0$ & $0$ & $0$ & $0$ & $0$ & $0$ & $0$ & & & & \\
    & & $1$ & $1$ & $0$ & $1$ & $0$ & $1$ & $0$ & & & & & \\
\cline{2-14} \makecell{~}
& $1$ & $0$ & $0$ & $1$ & $1$ & $0$ & $0$ & $0$ & $0$ & $1$ & $1$ & $0$ & $0$
  \end{tabular}
  \hspace{16mm}
  \begin{tabular}{rrrrr}
 &     & $1$ & $0$ & $6$ \\
 & $*$ &     & $4$ & $6$ \\
\cline{2-5} \makecell{~}
 &     & $6$ & $3$ & $6$ \\
 & $4$ & $2$ & $4$ &     \\
\cline{2-5} \makecell{~}
 & $4$ & $8$ & $7$ & $6$
  \end{tabular}
\end{center}
\setlength{\tabcolsep}{\defaulttabcolsep}

Here again, although the process is the same, multiplying binary numbers is
much easier than arabic numbers. Indeed, multiplying the first number by each
bit of the second boils down to multiplications by $0$ or $1$, which are
trivial. By contrast, multiplying an arabic number by a decimal digit requires
using a multiplication table with 100 possible cases. It also involves carries.

Some multiplications are even easier to do than with the general method
described above. In particular, multiplying $x$ by $2^n$ can be done by simply
shifting $x$ by $n$ bits to the left, \ie, by adding $n$ zeros on the right.
For instance, $1101010_2=106$ multiplied by $2^3=8$ is simply
$1101010\mathbf{000}_2=848$. This is similar to multiplications by $10^n$ in
arabic notation (for example, $46$ times $10^3=1000$ is $46\mathbf{000}$).
Shifting a binary number $x$ by $n$ bits to the left is noted $x \ll n$.

The opposite operation, shifting $x$ by $n$ bits to the right, \ie, dropping
the $n$ least significant bits, is noted $x \gg n$. It corresponds to dividing
$x$ by $2^n$. For instance, shifting $1101010_2=106$ by 3 bits to the right
gives $1101_2=13=\lfloor 106 / 2^3 \rfloor$\footnote{The $\lfloor x \rfloor$
notation designates the integer part of $x$. For instance, $106 / 8 = 13.25$
and $\lfloor 13.25 \rfloor = 13$.}. This is similar to dropping the $n$ least
significant digits of an arabic number, which divide it by $10^n$ (for example,
$4876$ shifted to the right by 2 digits is $48=\lfloor 4876 / 100 \rfloor$).
Dividing arbitrary binary numbers can be done as with arabic numbers, but is
not presented here.

\subsection{Conversions}

Computers do all their computations with binary numbers because, as shown
above, this is much easier to do than with arabic numbers. However, humans
prefer to specify inputs with arabic numbers, and to get results in arabic too.
This requires converting arabic numbers to binary ones, and vice versa.

One method to convert an arabic number to binary is to convert each digit from
left to right, and to multiply the result by $10$ before adding the next digit.
For instance, to convert $46$ to binary, we start by converting $4$, which
gives $100_2$. We multiply this by $10=8+2$, which can be done by shifting
$100_2$ by $3$ bits and by $1$ bit to the left, and by adding the results:
$100\mathbf{000}_2+100\mathbf{0}_2=101000_2$. Finally we convert $6$, which
gives $110_2$ and we add this to the previous result, yielding $101110_2$. This
method is well suited for computers since it only involves computations on
binary numbers (plus a small conversion table for each digit from $0$ to $9$).

Another method consists in dividing the arabic number by 2 repeatedly. The
remainders give the bits of the equivalent binary number, from right to left.
For instance, dividing $46$ by $2$ repeatedly gives $23$ (remainder $0$), $11$
(remainder $1$), $5$ (remainder $1$), $2$ (remainder $1$), $1$ (remainder $0$),
and $0$ (remainder $1$). The corresponding binary number is thus $101110_2$.
Since this method involves divisions on arabic numbers, it is more adapted for
humans than for computers.

Similarly, one method to convert a binary number to arabic is to ``convert''
each bit from left to right, and to multiply the result by $2$ before adding
the next bit. For instance, converting $101110_2$ gives successively $1$,
$1*2+0=2$, $2*2+1=5$, $5*2+1=11$, $11*2+1=23$, and $23*2+0=46$. Since this
method involves multiplications of arabic numbers, it is more adapted for
humans. But it can also be used on computers, if necessary.

Another method to convert a binary number is to divide it by 10 repeatedly. The
remainders, converted to arabic, give the digits of the equivalent arabic
number, from right to left. It is well suited for computers since it only
involves computations on binary numbers (plus a small conversion table for each
binary number from $0$ to $1001_2=9$).

\section{Logical operations}

Binary numbers can also be used to perform {\em logical operations}, unlike
arabic numbers. A logical operation computes whether some {\em proposition} is
true of false, depending on the status of one or more other propositions. A
proposition is a statement which is either true or false.

Consider for example a keyboard. A proposition might be ``the E key is
currently pressed'', ``the left Shift key is currently released'', or ``the e
letter is currently pressed''. They are either true or false, depending on the
current state of the keyboard. These propositions, noted
$\mathrm{KeyPressed}(k)$, $\mathrm{KeyReleased}(k)$, and
$\mathrm{LetterPressed}(l)$, are not completely independent. Some can be
computed from the others. For example, we can compute $\mathrm{KeyReleased}(k)$
as the opposite of $\mathrm{KeyPressed}(k)$. This logical operation is
the {\em negation}, also called {\em not}, and is noted $\neg$:
\begin{equation*}
  \mathrm{KeyReleased}(k) = \neg \mathrm{KeyPressed}(k)
\end{equation*}

We can also compute whether the proposition ``a Shift key is pressed'' is true
from the above propositions. Indeed, this is the case if at least one of the
two Shift keys is pressed. This logical operation is the {\em disjunction},
also called {\em or}, and is noted $\vee$:
\begin{equation*}
  \mathrm{ShiftPressed} = \mathrm{KeyPressed}(\mathrm{LeftShift}) \vee
    \mathrm{KeyPressed}(\mathrm{RightShift})
\end{equation*}

The keyboard is in ``uppercase mode'' if a Shift key is currently pressed, or
if caps are locked, but not both (a Shift key reverses the effect of
CapsLock). This logical operation is the {\em exclusive disjunction},
also called {\em exclusive or}, and is noted $\oplus$:
\begin{equation*}
  \mathrm{UppercaseMode} = \mathrm{ShiftPressed} \oplus \mathrm{CapsLocked}
\end{equation*}

As a last example, we can also compute whether
$\mathrm{LetterPressed}(\mathrm{E})$ is true from
$\mathrm{KeyPressed}(\mathrm{E})$ and $\mathrm{UppercaseMode}$. Indeed, this is
the case if both are true. This logical operation is the {\em conjunction},
also called {\em and}, and is noted $\wedge$:
\begin{align*}
  \mathrm{LetterPressed}(\mathrm{E}) &=
    \mathrm{KeyPressed}(\mathrm{E}) \wedge \mathrm{UppercaseMode} \\
  \mathrm{LetterPressed}(\mathrm{e}) &=
    \mathrm{KeyPressed}(\mathrm{E}) \wedge \neg \mathrm{UppercaseMode}
\end{align*}

The above logical operations do not depend on the meaning of the propositions,
but only on whether they are true or false. And their result is either true or
false. For instance, $\neg$ true is false, $p \wedge q$ is true if and only if
both $p$ and $q$ are true, $p \vee q$ is true if at least one of $p$ and $q$ is
true, etc. By representing true with 1 and false with 0, they can be seen as
operations on individual bits. This gives, for example, $\neg 1 = 0$, $1 \wedge
0 = 0$, $1 \wedge 1 = 1$, etc. By doing this for all possible cases we get the
{\em truth table} of each operation, represented in
\cref{table:logical-tables}. Note that the truth tables of $\oplus$ and
$\wedge$ are identical to those giving the result and carry bit of $a+b$,
respectively (see \cref{table:addition-table}). The result bit of $a-b$ is also
equal to $a \oplus b$, and the carry bit is $b \wedge \neg a$ (see
\cref{table:subtraction-table}). Hence, it suffice to know how to implement
these logical operations with electric circuits, or other technologies, in
order to be able to implement arithmetic circuits.

\begin{Table}
  \begin{tabular}[t]{|c|c|} \hline
    \makecell{\thead[c]{p}} & \thead[c]{$\neg p$} \\ \hline
    \makecell{0} & 1 \\
    \makecell{1} & 0 \\ \hline
  \end{tabular}
  \hspace{8mm}
  \begin{tabular}[t]{|c|c|c|} \hline
    \makecell{\thead[c]{p}} & \thead[c]{q} & \thead[c]{$p \wedge q$} \\ \hline
    \makecell{0} & 0 & 0 \\
    \makecell{0} & 1 & 0 \\
    \makecell{1} & 0 & 0 \\
    \makecell{1} & 1 & 1 \\ \hline
  \end{tabular}
  \hspace{8mm}
  \begin{tabular}[t]{|c|c|c|} \hline
    \makecell{\thead[c]{p}} & \thead[c]{q} & \thead[c]{$p \vee q$} \\ \hline
    \makecell{0} & 0 & 0 \\
    \makecell{0} & 1 & 1 \\
    \makecell{1} & 0 & 1 \\
    \makecell{1} & 1 & 1 \\ \hline
  \end{tabular}
  \hspace{8mm}
  \begin{tabular}[t]{|c|c|c|} \hline
    \makecell{\thead[c]{p}} & \thead[c]{q} & \thead[c]{$p \oplus q$} \\ \hline
    \makecell{0} & 0 & 0 \\
    \makecell{0} & 1 & 1 \\
    \makecell{1} & 0 & 1 \\
    \makecell{1} & 1 & 0 \\ \hline
  \end{tabular}
  \caption{The truth tables of not ($\neg$), and ($\wedge$), or ($\vee$), and
  exclusive or ($\oplus$).}\label{table:logical-tables}
\end{Table}

We can then generalize these logical operations from individual bits to whole
binary numbers. By definition, a {\em bitwise} logical operation on two binary
numbers is done by applying it on each bit separately, column by column. Thus,
for instance:

\setlength{\tabcolsep}{1pt}
\begin{center}
  \begin{tabular}{rrrrr}
             & $1$ & $1$ & $0$ & $0$ \\
    $\wedge$ & $1$ & $0$ & $1$ & $0$ \\
    \hline \makecell{~}
             & $1$ & $0$ & $0$ & $0$
  \end{tabular}
  \hspace{16mm}
  \begin{tabular}{rrrrr}
             & $1$ & $1$ & $0$ & $0$ \\
      $\vee$ & $1$ & $0$ & $1$ & $0$ \\
    \hline \makecell{~}
             & $1$ & $1$ & $1$ & $0$
  \end{tabular}
  \hspace{16mm}
  \begin{tabular}{rrrrr}
             & $1$ & $1$ & $0$ & $0$ \\
    $\oplus$ & $1$ & $0$ & $1$ & $0$ \\
    \hline \makecell{~}
             & $0$ & $1$ & $1$ & $0$
  \end{tabular}
\end{center}
\setlength{\tabcolsep}{\defaulttabcolsep}

This can be used to perform several logical operations in parallel (since there
is no carry each column can be computed independently of the others, possibly
at the same time). For instance, we can represent the current state of a 100
keys keyboard with a 100 bits binary number $S$, using one bit per key. We can
then do the following operations, which are commonly used in many similar
contexts:
\begin{itemize}
  \item to check whether at least one letter key is pressed, we can compute $S
  \wedge L$, where $L$ is the binary number whose $i^{th}$ bit is $1$ if and
  only if the $i^{th}$ key is a letter. If the result is $0$ no letter key is
  pressed, otherwise at least one is pressed.

  \item if a new set of keys if pressed, we can compute the representation of
  the new keyboard state with $S' = S \vee P$, where $P$ represents the newly
  pressed keys. For instance, if the $0^{th}$ and $3^{rd}$ keys are currently
  pressed, and if the user presses the $0^{th}$ and $2^{nd}$ keys\footnote{A
  pressed key can be ``pressed'' again due to autorepeat.}, we get $S=1001_2$,
  $P=101_2$ and $S'=1101_2$. This correctly represents the fact that the
  $0^{th}$, $2^{nd}$, and $3^{rd}$ keys are now pressed.

  \item if a new set of keys if released, we can compute the representation of
  the new keyboard state with $S' = S \wedge \neg R$, where $R$ represents the
  newly released keys. For instance, if the $0^{th}$ and $3^{rd}$ keys are
  currently pressed, and if the user releases the third, we get $S=1001_2$,
  $R=1000_2$ and $S'=1$. This correctly represents the fact that only the
  $0^{th}$ key remains pressed.
\end{itemize}

\section{Hexadecimal numbers}

Binary numbers are very practical to perform computations, but are not very
compact. Arabic numbers are much more compact (a given number has about $3.3$
less digits than bits on average), but converting between binary and arabic is
not so easy. To solve these issues {\em hexadecimal} numbers are commonly used.

Hexadecimal numbers are like arabic numbers, but use 16 digits instead of 10.
They are called hex digits and are noted $0$, $1$, $2$, $3$, $4$, $5$, $6$,
$7$, $8$, $9$, $A$ ($=10$), $B$ ($=11$), $C$ ($=12$), $D$ ($=13$), $E$ ($=14$),
and $F$ ($=15$). An hexadecimal number is thus a sequence of hex digits, where
the $i^{th}$ hex digit from the right (counting from 0) represents a quantity
of $16^i$. For instance
$$ED_{16} = E_{16}*16^1+D_{16}*16^0 = 14*16+13 = 237$$
where the subscript 16 indicates an hexadecimal number (to avoid confusions with
words or arabic numbers; $10_{16} = 16 \ne 10$ = ``ten'').

\begin{Table}
  \begin{tabular}{|r|r|r|r|r|r|r|r|} \hline
    \makecell{\thead[r]{binary}} & \thead[r]{hex} &
    \makecell{\thead[r]{binary}} & \thead[r]{hex} &
    \makecell{\thead[r]{binary}} & \thead[r]{hex} &
    \makecell{\thead[r]{binary}} & \thead[r]{hex} \\ \hline
    \makecell[r]{0000} & 0 & 0100 & 4 & 1000 & 8 & 1100 & C \\
    \makecell[r]{0001} & 1 & 0101 & 5 & 1001 & 9 & 1101 & D \\
    \makecell[r]{0010} & 2 & 0110 & 6 & 1010 & A & 1110 & E \\
    \makecell[r]{0011} & 3 & 0111 & 7 & 1011 & B & 1111 & F \\ \hline
  \end{tabular}
  \caption{Conversion between binary and
  hexadecimal.}\label{table:hex-binary-conversion}
\end{Table}

Each hex digit can be represented with up to 4 bits, and each group of 4 bits
can be represented with an hex digit, as shown in
\cref{table:hex-binary-conversion}. It is thus very easy to convert a binary
number to hexadecimal: simply convert each group of 4 bits independently, with
\cref{table:hex-binary-conversion}. For instance, to convert $11101101_2$, we
convert $1110_2$ ($E_{16}$), $1101_2$ ($D_{16}$), and concatenate the results,
yielding $ED_{16}$. Conversely, to convert $ED_{16}$ to binary we simply
concatenate the conversions of $E_{16}$ ($1110_2$) and $D_{16}$ ($1101_2$),
yielding $11101101_2$.

Hexadecimal numbers are thus compact (a given number has about 4 times less hex
digits and bits) and easy to convert to and from binary, which solves the above
issues. On the other hand, doing arithmetic computations with them is harder
than with arabic numbers (this involves tables with $16*16=256$ entries). But
this is not necessary since we can convert them to binary, do computations in
binary, and convert the result back to hexadecimal.


% This work is licensed under the Creative Commons Attribution NonCommercial
% ShareAlike 4.0 International License. To view a copy of the license, visit
% https://creativecommons.org/licenses/by-nc-sa/4.0/

\renewcommand{\rustfile}{chapter2}
\setcounter{rustid}{0}

\chapter[First Steps with the Cortex M3]{First Steps with the\\Cortex
M3}\label{chapter:microprocessor}

This chapter gives a short overview of the Arduino Due microprocessor, the ARM
Cortex M3. This knowledge is necessary to go beyond our first steps with the
Arduino, in the previous chapter. We use it at the end to write our first
program, to blink a LED.

\section{Overview of the Cortex M3}\label{section:cortex-m3}

The Cortex M3 microprocessor has a core instruction processing part and a few
other internal components (see \cref{fig:sam3x8e}). The instruction processing
part loads instructions from memory, decodes them and executes them, in an
endless loop. These instructions form the Cortex M3 {\em machine language}.
They are encoded in 16 or 32 bits, and must always start at even addresses.
They can be divided in 3 main categories, namely {\em data processing}, {\em
load and store}, and {\em conditional and jump} instructions:
\begin{itemize}
  \item The data processing instructions perform arithmetic and logic
  operations (addition, multiplication, bitwise and, etc). They can only
  operate on values stored in registers (or encoded in the instruction itself),
  and can only store the result in registers. The Cortex M3 has 16 32-bit
  registers available for this purpose, named R0 to R15. An example data
  processing instruction is ``add the values in R1 and R2, and store the result
  in R5''.

  \item The load and store instructions load values from memory into registers,
  and store values from registers in memory. An example load instruction is
  ``load the value from memory at the address in R3, offset by 10, and put the
  result in R0''. If R3 contains $60$, this instruction loads the value at
  address $70$ and store it in R0.

  \item The conditional and jump instructions can modify the normal flow of
  execution. Instructions are normally executed sequentially, in increasing
  address order. In other words, in the normal case, after the instruction $i$
  at address $a$ is executed, the instruction at address $a+2$ (or $a+4$ if $i$
  is a 32-bit instruction) is executed. This can be changed with conditional
  instructions or jump instructions. As their name suggest, conditional
  instructions are skipped if some condition is met. And jump instructions
  cause the execution flow to {\em jump}, \ie, to continue at an arbitrary
  address.
\end{itemize}

The other components of the Cortex M3 include a timer, an interrupt controller,
and a memory protection unit (they are presented in more details later in this
book):
\begin{itemize}
  \item The timer component, called SysTick, can be used to measure time. It
  decrements a 24 bit counter by 1 at each clock cycle, and restarts it to a
  configurable value when it reaches 0.

  \item The interrupt controller, called the Nested Vector Interrupt Controller
  (NVIC), provides another way to modify the normal execution flow, other than
  the conditional and jump instructions. It handles errors, also called {\em
  exceptions}, such as trying to read the memory at a reserved address. It also
  handles external events, also called {\em interrupts}, such as the reception
  of some data on an input pin. When an error or event occurs, the NVIC causes
  the execution flow to jump to a predefined {\em handler} address, associated
  with each error or event source.

  \item The memory protection unit can be used to control memory accesses. It
  can divide the memory into regions, and can associate different access rights
  with each region. For instance, one region can be made inaccessible, another
  read-only, etc. This can be used to protect programs from each other, \ie, to
  avoid a crash in a program to crash another one, or to prevent a program from
  reading sensitive data (\eg, passwords) in another program's memory (see
  \cref{chapter:memory-protection}).
\end{itemize}

\section{Registers}\label{section:registers}

As said above, the Cortex M3 instructions can use 16 registers named R0 to R15.
The first 13 registers can be used for any purpose, but the last 3 have
specific usages. They are called the Program Counter, Link Register and Stack
Pointer, and their main goal is to make it easier to split programs into
smaller building blocks called {\em subroutines}. This section presents these
registers and how subroutines work.

\subsection{Program Counter}\label{subsection:program-counter}

The R15 register is also called the {\em Program Counter} (PC). This register
contains the address of the currently executing instruction. More precisely,
just before executing an instruction at address $a$, it contains the value $a$.
However, {\em during the execution of an instruction at address $a$, the PC
contains the value $a+4$}.

Writing into the PC is also possible, and causes a jump. Since instructions
must start at even addresses, the least significant bit of the jump address is
always ignored. For instance, attempting to write 17 into the PC actually
writes 16, and jumps to address 16. In fact, due to historical
reasons\footnote{Some ARM processors support two instruction sets. They use
this bit to specify the instruction set to use after the jump. The Cortex M3
supports only one instruction set but still uses this mechanism.}, some
instructions {\em require} writing $a+1$ into the PC in order to jump to the
instruction at address $a$. $a+1$ is called the instruction's {\em interworking
address}.

\subsection{Link Register}

\begin{Figure}
  \input{figures/chapter2/branch-with-link.tex}

  \caption{The Link Register. Branch with Link (\arm{BL}) instructions set the
  LR to the interworking address of the next instruction (here \arm{MUL}) when
  jumping to another instruction (here \arm{SUB}). Copying the LR into the PC
  ``resumes'' execution (here at \arm{MUL}).}\label{fig:branch-with-link}
\end{Figure}

The R14 register is also called the {\em Link Register} (LR). Some jump
instructions, called {\em branch with link}, set this register to the {\em
interworking} address of the next instruction in sequential order. For
instance, a 32-bit branch with link instruction at address $a$, jumping to
address $b$, sets the LR to $a+5$. This allows the code starting at $b$, when
its task is done, to ``resume'' the execution of the initial code sequence,
\ie, to jump to the instruction at $a+4$. For this, it just needs to write the
value stored in the LR into the PC (see \cref{fig:branch-with-link}).

\subsection{Stack Pointer}\label{subsection:stack-pointer}

\begin{Figure}
  \input{figures/chapter2/descending-stack.tex}

  \caption{The Stack Pointer. When the SP contains the value $a$, pushing a
    32-bit value, here \hexa{31415926}, stores it at address $a-4$ and updates
    the SP to $a-4$. Popping gets the value stored at the address given by the
    SP, and adds 4 to the SP.}\label{fig:descending-stack}
\end{Figure}

The R13 register is also called the {\em Stack Pointer} (SP). A {\em stack} is
similar to a pile of plates: one can only add, or {\em push}, a value (a plate)
on top of the stack (the pile), not inside it. Similarly, one can only remove,
or {\em pop}, a value from the top of the stack. A {\em pointer} is a register
or memory location which contains the address of some value, called the pointer
target. The Stack Pointer contains the address of the top stack value. It is
used by two instructions called \arm{PUSH} and \arm{POP}, which can push the
values in some registers into the stack, and pop values from the stack into
registers. This stack is called {\em descending}\footnote{Note that it may {\em
look} ascending, if addresses are representing in ascending order from top to
bottom.} because pushing a new value {\em decreases} the Stack Pointer's value.
For instance, if the SP contains the value $a$, pushing $x$ means storing $x$
in memory at address $a-4$, and updating the SP value to $a-4$ (see
\cref{fig:descending-stack}).

Note: from now on we use register names to designate either the register
itself, or the value it contains, depending on the context. For instance, in
``add R1 and R2 and store the result in R3'', R1 and R2 designate the values
stored in these registers, while R3 designates the register itself.

\subsection{Subroutines}\label{subsection:subroutines}

\begin{Figure}
  \input{figures/chapter2/subroutine-calls.tex}

  \caption{Nested subroutine calls. Before doing a branch with link to $c$ (3),
  subroutine B pushes the LR (\ie, the interworking return address $a+5$ in A)
  onto the stack (2).}\label{fig:subroutine-calls}
\end{Figure}

The main goal of the above registers is to organize programs into smaller
subprograms of increasingly higher abstraction levels, in a similar way as
digital circuits, or even living organisms, are organized (made of atoms,
grouped in molecules, grouped in cells, grouped in organs, grouped into
organisms). For instance, a program to draw figures could have a subprogram to
draw text, which could itself use a ``sub'' subprogram to draw characters. The
advantage of this method is that each level can be understood and
programmed without knowing how the lower levels work internally.

At the machine code level considered here, these subprograms are called {\em
subroutines}. A subroutine is a list of instructions, some of which can {\em
call} other subroutines. After it has been executed, a subroutine {\em returns}
to the subroutine which called it. A call from an instruction at address $a$ in
subroutine A (\eg, ``draw figure''), to subroutine B (\eg, ``draw text''), is a
jump to B's first instruction, preceded by instructions allowing B to return in
subroutine A. This includes storing the (interworking) {\em return address}
$a+5$ in the Link Register, as described above. However, this is generally not
sufficient. Consider the case where subroutine B needs to call a third
subroutine C (\eg, ``draw character''), from address $b+4$ (see
\cref{fig:subroutine-calls}). It cannot simply store its own (interworking)
return address $b+9$ in the Link Register, otherwise it would loose the value
$a+5$ already stored there, and would no longer be able to return to A! To
solve this, B needs to save the current LR value first. The solution is to push
it on the stack. This is generally done at the very beginning of each
subroutine (see \cref{fig:subroutine-calls}). Returning to the {\em caller},
\ie, to A, can then be done by popping this value from the stack, into the PC.

\section{Instruction set}\label{section:instruction-set}

The Cortex M3 machine language has about 100 instructions. However, some
instructions have several variants and several encodings, on 16 or 32 bits. For
instance, the \arm{ADD} instruction has 4 variants and 14 encodings. In total,
there are more than 350 different encodings. Here we present only the main
instructions and encodings used in this book (about 40 -- a few more are
presented later), and we give only a short overview for each. All the details
about all the instructions can be found in \cite{ARMv7}.

\subsection{Data processing instructions}\label{subsection:data-insns}

\begin{Paragraph}[]
\rs{ADD_RDN_IMM8.definition()}\\
\rs{ADD_RD_RN_RM.definition()}\\
\rs{ADD_SP_SP_IMM7.definition()}\\
\rs{ADD_RD_SP_IMM8.definition()}
\end{Paragraph}

\rust{
  let add = &ADD_RDN_IMM8;
  let mov = &MOV_RD_RM;
  let z = 3;
  let c = 45;
  let encoding = add.encode(&[z, c]);
  let operation = &add.concrete_semantics(encoding, false);
}

The \arm{ADD} instruction adds two registers, or a register and a constant,
depending on the variant used, and stores the result in a register. For
instance, the first variant, \rs{ADD_RDN_IMM8.semantics()}, adds the constant
$\rs{add.fields[1].name}$ to register $\mathrm{R}\rs{add.fields[0].name}$ and
stores the result in $\mathrm{R}z$. It is encoded on 16 bits. The most
significant ones are fixed to \bina{00110}. The next 3 bits define which
register is used, from R0 to R7. The last 8 bits define the constant to add to
this register. For instance, by replacing $\rs{add.fields[0].name}$ with
\rs{bin_dec(z)} and $\rs{add.fields[1].name}$ with \rs{bin_dec(c)} we get the
instruction \rs{operation}. Its encoding is obtained by replacing
$\rs{add.fields[0].name}$ and $\rs{add.fields[1].name}$ with their values in
the encoding schema, yielding \rs{bin_hex16(encoding)}.

\begin{Paragraph}
\rs{SUB_RDN_IMM8.definition()}\\
\rs{SUB_RD_RN_RM.definition()}\\
\rs{SUB_SP_SP_IMM7.definition()}\\
\rs{MUL_RDM_RN.definition()}
\end{Paragraph}

The \arm{SUB} and \arm{MUL} instructions perform subtractions and
multiplications. Note that additions, subtractions, and multiplications are
done modulo $2^{32}$, and can thus overflow (see
\cref{subsection:int-overflow}).

\begin{Paragraph}
\rs{UDIV_RD_RN_RM.definition()}
\end{Paragraph}

The \arm{UDIV} instruction, encoded on 32-bit, performs integer divisions, such
as $\lfloor 14 / 4 \rfloor = 3$. $\mathrm{R}x$, $\mathrm{R}y$, and
$\mathrm{R}z$ must not be the SP or the PC.

\begin{Paragraph}
\rs{AND_RDN_RM.definition()}\\
\rs{ORR_RDN_RM.definition()}
\end{Paragraph}

The \arm{AND} instruction performs bitwise AND operations, such as \bina{1010}
$\wedge$ \bina{1100} = \bina{1000}. Similarly, the \arm{ORR} instruction
performs bitwise OR operations, such as \bina{1010} $\vee$ \bina{1100} =
\bina{1110}.

\begin{Paragraph}
\rs{LSL_RD_RM_IMM5.definition()}\\
\rs{LSL_RDN_RM.definition()}
\end{Paragraph}

The Logical Shift Left (\arm{LSL}) instruction shifts the 32 bits of a register
to the left by a certain amount. For instance, shifting the value
\bina{1011101} to the left by 3 inserts 3 zeros on the right (and drops the 3
most significant bits), yielding \bina{1011101{\em 000}}. Shifting to the left
by $n$ bits is equivalent to multiplying by $2^n$ (modulo $2^{32}$).

\begin{Paragraph}
\rs{LSR_RD_RM_IMM5.definition()}\\
\rs{LSR_RDN_RM.definition()}
\end{Paragraph}

The Logical Shift Right (\arm{LSR}) instruction shifts the 32 bits of a
register to the right by a certain amount. For instance, shifting the value
\bina{1011101} to the right by 3 drops the 3 least significant bits, yielding
\bina{1011}. Shifting to the right by $n$ bits is equivalent to dividing by
$2^n$.

\begin{Paragraph}
\rs{MOV_RD_IMM8.definition()}\\
\rs{MOV_RD_RM.definition()}
\end{Paragraph}

The Move (\arm{MOV}) instructions ``move'' (or more precisely copy) a value
from a register or a constant into a register. The
$\rs{mov.fields[0].name}{:}\rs{mov.fields[1].name}$ notation above denotes a
concatenation of binary numbers. For instance, replacing
$\rs{mov.fields[0].name}$ with \bina{1} and $\rs{mov.fields[1].name}$ with
\bina{011} gives \bina{1011}, and thus
$\mathrm{R}\rs{mov.fields[0].name}{:}\rs{mov.fields[1].name}=\mathrm{R}11$. The
\rs{mov.operation()} instruction can thus access the 16 registers R0 to R15.

\begin{Paragraph}
\rs{MOVW_RD_IMM16.definition()}
\end{Paragraph}

The Move Wide (\arm{MOVW}) instruction copies a 16-bit constant into a
register, which must not be the SP or the PC. It can copy values up to 65535,
whereas the \arm{MOV} instruction is restricted to values up to 255.

\begin{Paragraph}
\rs{MOVT_RD_IMM16.definition()}
\end{Paragraph}

The Move Top (\arm{MOVT}) instruction copies a 16-bit constant into the 16 most
significant bits of a register (which must not be the SP or the PC), leaving
the others unchanged. Together with \arm{MOVW}, it can be used to load a 32-bit
constant in a register. For this, first load the least significant bits with
\arm{MOVW} (which erases the most significant bits), then load the most
significant bits with \arm{MOVT}.

\begin{Paragraph}
\rs{CMP_RN_IMM8.definition()}\\
\rs{CMP_RN_RM.definition()}
\end{Paragraph}

The Compare (\arm{CMP}) instructions compare two registers, or a register and a
constant. They store the comparison result, \eg, whether $\mathrm{R}x$ is less
than, equal to, or greater than $\mathrm{R}y$ in a special register, different
from the R0-R15 ones. This special register, similar to the Carry register in
\cref{subsection:alu-and-ram-example}, is used by conditional
instructions\footnote{Other instructions can store results in this special
register. For simplicity, we don't use this feature.} (see
\cref{subsection:conditional-insns}).

\begin{Paragraph}
\rs{ADR_RD_MINUS_IMM12.definition()}
\end{Paragraph}

The Address To Register (\arm{ADR}) instruction subtracts a constant from the
PC and stores the result in a register (which must not be the SP or the PC).
More precisely, it subtracts a constant from the largest multiple of 4 which is
less than or equal to the PC, noted $\lfloor\mathrm{PC}\rfloor_4$. For
instance, if the PC value is $102=4*25+2$, then
$\lfloor\mathrm{PC}\rfloor_4=100$. If the PC value is $100$, then
$\lfloor\mathrm{PC}\rfloor_4=100$. Do not forget also that when an instruction
executes, the PC value is the address of this instruction plus 4. Hence, an
\arm{ADR} instruction at address $a$ subtracts a constant from $\lfloor a+4
\rfloor_4=\lfloor a\rfloor_4+4$.

Incidentally, this raises the question of how 16 and 32-bit instructions are
stored in memory. We saw in \cref{section:boot-assistant-first-steps} that
32-bit values are stored in little endian order. In fact everything is stored
in this order, including instructions. When a 16-bit instruction of the form

\vspace{1pt}
\noindent\hfill\rs{GENERIC_16BIT.bit_pattern()}
\vspace{3pt}

\noindent is stored at address $a$, then $byte_0$ is stored at address $a$, and
$byte_1$ at address $a+1$. Similarly, when a 32-bit instruction of the form

\vspace{1pt}
\noindent\hfill\rs{GENERIC_32BIT.bit_pattern()}
\vspace{3pt}

\noindent is stored at address $a$, then $byte_0$ is stored at address $a$,
$byte_1$ at address $a+1$, $byte_2$ at address $a+2$, and $byte_3$ at address
$a+3$\footnote{These bytes are displayed in a different order in \cite{ARMv7},
see Figure A3-5, pA3-68.}.

\subsection{Load and store instructions}\label{subsection:load-store-insns}

\begin{Paragraph}[]
\rs{LDR_RT_RN_IMM5.definition()}\\
\rs{LDR_RT_SP_IMM8.definition()}\\
\rs{LDR_RT_PC_IMM8.definition()}\\
\rs{LDRH_RT_RN_IMM5.definition()}\\
\rs{LDRB_RT_RN_IMM5.definition()}
\end{Paragraph}

The Load Register (\arm{LDR}*) instructions load a value from memory and store
it in a register. There are 3 variants, \arm{LDR}, \arm{LDRH}(alf), and
\arm{LDRB}(yte), to load 32-bit, 16-bit, and 8-bit values from memory. The
address from which the value must be loaded can be in one of the R0-R7
registers, in the SP, or in the PC (in this case the memory address actually
used is $\lfloor \mathrm{PC} \rfloor_4 = \lfloor a+4 \rfloor_4$, where $a$ is
the instruction's address). In all cases, a constant offset $c$ can be added to
this address.

\begin{Paragraph}
\rs{STR_RT_RN_IMM5.definition()}\\
\rs{STR_RT_SP_IMM8.definition()}\\
\rs{STRH_RT_RN_IMM5.definition()}\\
\rs{STRB_RT_RN_IMM5.definition()}
\end{Paragraph}

The Store Register (\arm{STR}*) instructions read a value in a register and
store it in memory. There are 3 variants, \arm{STR}, \arm{STRH}(alf), and
\arm{STRB}(yte), to store the 32-bits of the register's value, only its 16
least significant bits, or only its 8 least significant bits. The address at
which these bits must be stored can be in one of the R0-R7 registers, or in the
SP. In all cases, a constant offset $c$ can be added to this address.

\begin{Paragraph}
\rs{PUSH.definition()}
\end{Paragraph}

The \arm{PUSH} instruction pushes one or more of the R0-R7 registers, and
optionally the LR, onto the stack. A register R$n$ is pushed if and only if bit
$n$ of the $\rs{PUSH.fields[0].name}$ field is 1. For instance, if
$\rs{PUSH.fields[0].name}$=\bina{11010} then registers R1, R3 and R4 are
pushed. The LR is pushed if the $\rs{PUSH.fields[1].name}$ field is 1. These
registers are pushed in {\em decreasing index order} (\ie, first the LR if it
is selected, then R7 if selected, and so on down to R0). The selected register
with the smallest index thus ends up on top of the stack (see
\cref{fig:push-pop-insns}).

\begin{Paragraph}
\rs{POP.definition()}
\end{Paragraph}

The \arm{POP} instruction pops some values from the stack, and stores them in
one or more of the R0-R7 registers, and optionally in the PC. A popped value is
stored in register R$n$ if and only if bit $n$ of the $\rs{POP.fields[0].name}$
field is 1. A popped value is stored in the PC if the $\rs{POP.fields[1].name}$
field is 1. In this case, it must be an interworking address. The number of
values popped from the stack is equal to the number of bits set to 1 in the
previous fields. The popped values are stored in registers in {\em increasing
index order} (\ie, first in R0 if it is selected, then in R1 if selected, and
so on up to the PC -- see \cref{fig:push-pop-insns}).

\begin{Figure}
  \input{figures/chapter2/push-pop-insns.tex}

  \caption{The \arm{PUSH} and \arm{POP} instructions. Registers are pushed in
  decreasing index order (here LR, then R3, then R2). They are popped in
  increasing index order (here R0, then R1, then PC -- set to 18 = 19 - 1
  because of interworking addresses).}\label{fig:push-pop-insns}
\end{Figure}

\subsection{Jump instructions}\label{subsection:jump-insns}

\begin{Paragraph}[]
\rs{B_IMM11.definition()}
\end{Paragraph}

The Branch (\arm{B}) instruction jumps to an address which is obtained by
adding a constant $2 * c$ to the PC. More precisely:
\begin{itemize}
  \item if $2 * c < 2^{12-1}$, \ie, if $2 * c < 2048$, $2 * c$ is added to the
  PC,

  \item otherwise, $2 * c - 2^{12} = 2 * c - 4096$, which is negative, is added
  to the PC.
\end{itemize}

\begin{Paragraph}
\rs{BX_RM.definition()}
\end{Paragraph}

The Branch and Exchange (\arm{BX}) instruction jumps to an {\em interworking}
target address stored in a register. It thus jumps to the instruction at
address $t$ if the register contains an odd value $t+1$.

\begin{Paragraph}
\rs{BLX_RM.definition()}
\end{Paragraph}

The Branch with Link and Exchange (\arm{BLX}) instruction does a branch {\em
with link} operation. It does the same jump as the \arm{BX} instruction, but it
also sets the LR to the {\em interworking address} of the instruction just
after itself. This is done so that the LR can be directly copied into the PC to
return from the called subroutine, without having to think about
interworking addresses. $\mathrm{R}x$ must not be the PC.

\begin{Paragraph}
\rs{BL_IMM22.definition()}
\end{Paragraph}

The Branch with Link (\arm{BL}) instruction\footnote{For simplicity, we use a
restricted version of the \arm{BL} instruction, with 2 bits fixed to 1. The
unrestricted instruction can support $\pm$16~MB jumps.} jumps to an address
which is obtained by adding a constant $2 * c$ to the PC (where $c=c_1{:}c_0$).
More precisely:
\begin{itemize}
  \item if $2 * c < 2^{23-1}$, \ie, if $2 * c < 4\ \mathrm{MB}$, $2 * c$ is
  added to the PC,

  \item otherwise, $2 * c - 2^{23} = 2 * c - 8\ \mathrm{MB}$, which is
  negative, is added to the PC.
\end{itemize}
In addition, the \arm{BL} instruction sets the LR to the {\em interworking
address} of the instruction just after itself. This is done so that the LR can
be directly copied into the PC to return from the called subroutine, without
having to think about interworking addresses.

\begin{Paragraph}
\rs{TBB_RN_RM.definition()}
\end{Paragraph}

The Table Branch Byte (\arm{TBB}) instruction jumps to an address which is
obtained by adding to the PC (the double of) a byte offset, read in a table.
For instance, suppose there is a list of 5 bytes, [$12, 34, 56, 78, 90$],
starting at address $a$ in memory. If
$\mathrm{R}\rs{TBB_RN_RM.fields[0].name}=a$ and
$\mathrm{R}\rs{TBB_RN_RM.fields[1].name}=3$, then this instruction adds to the
PC the double of the $3^{rd}$ value in this table (counting from 0), \ie,
$2*78$. Note that $\mathrm{R}\rs{TBB_RN_RM.fields[0].name}$ can be the PC (but
not $\mathrm{R}\rs{TBB_RN_RM.fields[1].name}$ -- none of them can be the SP).
In this case the table must start at the PC, \ie, just after the \arm{TBB}
instruction itself (because the PC is the instruction's address plus 4).

\subsection{Conditional instructions}\label{subsection:conditional-insns}

\begin{Paragraph}[]
\rs{IT.definition()}
\end{Paragraph}

The If Then (\arm{IT}) instruction makes the following one to four instructions
conditional. This means that these instructions, noted I1, I2, I3, and I4, can
either be executed normally, or skipped. For instance, I1 and I3 could be
executed, while I2 and I4 are skipped (\ie, after I1, execution would continue
with I3, ignoring I2). Whether I$n$ is executed or skipped depends on $c_0$,
$c_1$, etc, and on the result of the last \arm{CMP} instruction\footnote{For
simplicity, we use \arm{IT} instructions only immediately after a \arm{CMP}
instruction.}, as explained below.

First, I1 is always made conditional by the \arm{IT} instruction, but I2, I3,
and I4 can be made conditional or not. This depends on $c_2{:}c_3{:}c_4{:}c_5$,
which must not be 0:
\begin{itemize}
  \item if $c_2{:}c_3{:}c_4{:}c_5=1000_2$, then only I1 is made conditional,

  \item if $c_2{:}c_3{:}c_4{:}c_5=c_2100_2$, then only I1 and I2 are made
  conditional,

  \item if $c_2{:}c_3{:}c_4{:}c_5=c_2c_310_2$, then only I1, I2 and I3 are made
  conditional,

  \item if $c_2{:}c_3{:}c_4{:}c_5=c_2c_3c_41_2$, then I1, I2, I3 and I4 are
  made conditional.
\end{itemize}

Second, if an instruction I$n$ is made conditional, it is executed if and only
if the last comparison result is the one corresponding to $c_0{:}c_n$ (noted
$cond_n$), as defined in \cref{table:cond-values}. As an example, if the
\arm{IT} instruction follows a \arm{CMP} R1 R2 instruction, then to execute I1
if R1$\ <\ $R2, and I2 and I3 otherwise (with I4 unconditional), we must use
$c_0=001_2$, $c_1=1_2$ and $c_2{:}c_3{:}c_4{:}c_5=0010_2$.

Finally, it should be noted that I1, I2, I3, I4 cannot be arbitrary
instructions. For instance, they cannot be \arm{IT} instructions themselves.
Jump instructions are also forbidden, except for the last conditional
instruction.

\begin{Table}
  \renewcommand\cellgape{\Gape[2pt][0pt]}
  \begin{tabular}{|c|c|c|c|c|c|c|}
    \hline
    \makecell{\thead{Value}} &
    \rs{bin4(Condition::EQ as u32)} &
    \rs{bin4(Condition::NE as u32)} &
    \rs{bin4(Condition::GE as u32)} &
    \rs{bin4(Condition::LT as u32)} &
    \rs{bin4(Condition::GT as u32)} &
    \rs{bin4(Condition::LE as u32)}\\ \hline
    \makecell{\thead{Meaning}} &
    $a = b$ &
    $a \ne b$ &
    $a \ge b$ &
    $a < b$ &
    $a > b$ &
    $a \le b$\\ \hline
  \end{tabular}
  \caption{Meaning of the $cond_n=c_0{:}c_n$ values used in this book, for an
  \arm{IT} instruction following a \arm{CMP} $compare(a,b)$
  instruction.}\label{table:cond-values}
\end{Table}

\section{Vector Table}\label{section:vector-table}

When it {\em boots}, \ie, when it is powered on or after a reset, the Cortex M3
starts by reading the {\em Vector Table}, a list of 32 bit words beginning at
address \hexa{0}. More precisely, the microprocessor initializes the Link
Register to \hexa{FFFFFFFF}, the Stack Pointer to the value at offset \hexa{0}
in this table, and the Program Counter to the interworking address at offset
\hexa{4}, also called the {\em Reset handler}. It then starts executing
instructions from there.

When the boot selection bit (\cf \cref{subsection:boot-mode}) is 0, addresses
\hexa{0} and \hexa{4} are mapped to the ROM, and the Vector Table is thus read
from ROM. The Cortex M3 then starts executing the boot assistant program stored
in ROM. When the boot selection bit is 1, these addresses are mapped to the
flash memory, and the Vector Table is thus read from flash. In this case, the
flash memory must contain valid initial values for the Stack Pointer and the
Program Counter, at addresses \hexa{80000} and \hexa{80004}, respectively.

The other values in the Vector Table are either reserved for future use, or
contain interworking addresses used to handle errors and external events. There
is one value for each type of error (also called exceptions) and external
events (also called interrupts). For instance, the value at offset \hexa{14} is
used to handle ``bus faults'', such as trying to access a reserved memory
address. When such an error occurs, execution jumps to the interworking address
stored at this offset (this process is described in more details in
\cref{section:nvic}). Another value, at offset \hexa{84}, is used to handle
events occuring in the Universal Synchronous Asynchronous Receiver Transmitter
(USART) -- such as the reception of new data. When such an interrupt occurs,
execution jumps to the interworking address stored at offset \hexa{84}. By
default, however, these specific exception and interrupt handlers are not
enabled (they must be enabled explicitly). Instead, all exceptions and
interrupts are handled by a generic ``hard fault'' handler, at offset \hexa{C}.
In other words, by default, if any exception or interrupt occurs (other than a
reset), execution jumps to the interworking address stored at offset \hexa{C}.

The Vector Table can be moved to another location in memory, for instance in
RAM. For this a new table must first be written somewhere in memory, starting
at an address which is multiple of 256. This address must then be written in
the Vector Table Offset Register, at address \hexa{E000ED08} in the ``System''
memory region (see \cref{fig:boot-memory-map}).

\section{First program}\label{section:blink-led}

We now know enough about the Arduino to be able to write our first program. We
have already seen how to turn a LED on and off ``manually'', with a wire or by
writing values in memory with the boot assistant. Lets now do the same with a
program. More precisely, lets write a program to blink the ``L'' LED on the
Arduino. At a high level, this program should execute the following steps:
\begin{enumerate}
  \item initialize the PIO controller to control the LED,
  \item turn the LED on with the PIO controller,
  \item wait some time,
  \item turn the LED off with the PIO controller,
  \item wait some time,
  \item go back to step 2.
\end{enumerate}

We see that steps 3 and 5 are the same. To avoid duplicating the instructions
doing this, we can put them in a subroutine, called twice from the main program.

\subsection{Subroutine}

Lets start by writing the subroutine. The most basic method to wait for some
time is to count to some value, as in a hide and seek game. To do this we can
use two registers, say R0 and R1, with R0 containing the current count, and R1
the value at which the count must be stopped. At a high level, we can thus
use the following algorithm for our subroutine:
\begin{enumerate}
  \item set R0 to 0,
  \item set R1 to the maximum counter value,
  \item add 1 to R0,
  \item if R0 $\ne$ R1 go back to step 3,
  \item return to the caller.
\end{enumerate}

Each step can be implemented with Cortex M3 instructions, as follows:
\begin{itemize}
  \item Step 1 can be done with a \arm{MOV} R0 $\leftarrow$ 0 instruction.

  \item Step 2 needs to store a large value in R1, lets say 1 million (the
  Cortex M3 counts fast). This can't be done with a \arm{MOV} instruction, nor
  with a \arm{MOVW} instruction. We could use \arm{MOVW} and a \arm{MOVT}, but
  using a \arm{LDR} instruction with the PC as a base address is simpler and
  shorter. The maximum counter value can then be put at the end of the
  subroutine, after the instruction for step 5.

  \item Step 3 is a simple \arm{ADD} R0 $\leftarrow$ R0 + 1 instruction.

  \item Step 4 must be done with a \arm{CMP} R0 R1 instruction, followed by an
  \arm{IT} instruction to make the ``go back to step 3'' part conditional. This
  optional jump is a simple \arm{B} instruction.

  \item Step 5 must move the LR into the PC. This could be done with a
  \arm{MOV} instruction, but lets use a \arm{POP} instruction instead, for
  illustrative purpose. For this the LR must be pushed on the stack first. We
  can add a step 0 for this. This step can also push R0 and R1 at the same
  time. These registers can then be restored at the end, so that calling the
  subroutine has no ``side effect''.
\end{itemize}

\rust{
  let mut a = Assembler::default();
  const R0: u32 = 0;
  const R1: u32 = 1;
  const WAIT_CYCLES: u32 = 1000000;
  const START_ADDRESS: u32 = 0x20071000;

  // subroutine:
  a.label("wait");
  a.push_list(&[R0, R1], true);
  a.mov_rd_imm8(R0, 0);
  a.ldr_rt_pc_imm8(R1, "wait_cycles");
  a.label("loop");
  a.add_rdn_imm8(R0, 1);
  a.cmp_rn_rm(R0, R1);
  a.if_then(Condition::NE, &[]);
  a.b_imm11("loop");
  a.pop_list(&[R0, R1], true);
  a.label("wait_cycles");
  a.u32_data(WAIT_CYCLES, "maximum counter value");

  // main:
  a.label("start");
  a.ldr_rt_pc_imm8(R0, "controller");
  a.ldr_rt_pc_imm8(R1, "pb27");
  a.str_rt_rn_imm5(R1, R0, 0); // PIOB_PER = PIO_PB27;
  a.str_rt_rn_imm5(R1, R0, 0x10); // PIOB_OER = PIO_PB27;
  a.str_rt_rn_imm5(R1, R0, 0x60); // PIOB_PUDR = PIO_PB27;
  a.label("main_loop");
  a.str_rt_rn_imm5(R1, R0, 0x30); // PIOB_SODR = PIO_PB27;
  a.bl_imm22("wait");
  a.str_rt_rn_imm5(R1, R0, 0x34); // PIOB_CODR = PIO_PB27;
  a.bl_imm22("wait");
  a.b_imm11("main_loop");
  a.label("controller");
  a.u32_data(PIOB_PER, "Start address of PIO B registers");
  a.label("pb27");
  a.u32_data(1 << 27, "Pin 27");
}

Going down to the machine code level, steps 0 and 1 give

\rs{a.get_listing(0..2)}

\noindent where the first column is the instruction's name and high level
description, the second one is its binary encoding (obtained from the patterns
in \cref{section:instruction-set}), the third is the corresponding hexadecimal
value, and the fourth is the instruction's offset from the beginning of the
program, in hexadecimal too. For step 2 we need to know where the maximum
counter value will be stored. Since we don't know this yet, lets skip this
instruction for now. We know it will use 2 bytes, so the instruction for step
3 starts at offset $6$:

\rs{a.get_listing(3..4)}

For step 4, the optional jump must go back to step 3, \ie, at offset $6$. This
can be done with a \arm{B} instruction after the CMP and IT instructions, at
offset $12$. When this instruction executes, the PC contains $12+4=16$.
Hence, this instruction must subtract $16-6=10$ from the PC to jump back to
step 3. Due to the encoding format of the \arm{B} instruction, we must then use
$c=2043$ (because $2*2043-4096=-10$):

\rs{a.get_listing(4..7)}

We finish the subroutine with a POP instruction for step 5, as discussed above,
followed by the maximum counter value (\hexa{F4240}=1000000):

\rs{a.get_listing(7..9)}

\noindent We now know that this maximum value is at offset $16$, and we want to
load it with a \arm{LDR} instruction at offset $4$. When this instruction
executes, the PC contains $4+4=8$. {\em Assuming that the subroutine starts at
an address which is a multiple of 4}, then $\lfloor PC \rfloor_4$ is also equal
to $8$. This instruction must thus add $16-8=8=4*2$ to the PC to load the above
value. This gives our missing \arm{LDR} instruction:

\rs{a.get_listing(2..3)}

Putting all this together, we get the following machine code for the subroutine:

\rs{a.get_machine_code_listing(0..9)}

\noindent where bytes are shown, in increasing address order, {\em from right
to left}. Indeed, the left to right order would show all the bytes in the
reverse order, which would make it hard to recognize the above instruction
encodings. It would also make it harder to store this program in memory with
the boot assistant word by word. Indeed, we would have to reverse each group of
4 bytes to get a value which could be entered in the boot assistant. With the
right to left order, we get these values directly.

\subsection{Main program}\label{subsection:blink-led-main}

We can now implement the main program. Its first step consists in initializing
the PIO controller so that the pin to which the LED is connected, PB27, is
configured as an output pin, controlled by the microprocessor. In such cases we
don't need the pull-up resistor, so we want to disable it as well. Going back
to \cref{section:pio}, this requires to write the value $2^{27}$ in the PIO
Enable Register, Output Enable Register, and Pull-up Disable Register of the
PIO B controller (\ie, at addresses \rs{hex(PIOB_PER)}, \rs{hex(PIOB_OER)}, and
\rs{hex(PIOB_PUDR)}). For this we must load this value and the 3 addresses in
registers first\footnote{In the following, for brevity, we often use ``we''
instead of ``the program'' or ``some instructions''. For instance, here, this
sentence means ``For this {\em the program} must load $\ldots$''.}. We can then
use 3 \arm{STR} instructions to store the value at the 3 addresses. In fact,
since \arm{STR} instructions can store a value at an address {\em plus an
offset}, we just need to load one address in a register, namely
\rs{hex(PIOB_PER)}. We can then get the other 2 addresses with the offsets
\hexa{10} and \hexa{60}. As for the subroutine, lets use 2 \arm{LDR}
instructions to load the \rs{hex(PIOB_PER)} and $2^{27}$ values, stored at the
end of the program, in registers R0 and R1 respectively. Since we don't know
the program size yet, lets skip these two \arm{LDR} instructions for now. We
thus start with the 3 \arm{STR} instructions instead, at offset 24=\hexa{18}
(\ie, 4 bytes after the end of the subroutine, to leave space for the 2
\arm{LDR} instructions):

\rs{a.get_listing(11..14)}

The second step of the main program is to turn the LED on. This requires
writing the value $2^{27}$ in the Set Output Data Register, \ie, at address
\rs{hex(PIOB_SODR)}. This can be done with another \arm{STR} instruction:

\rs{a.get_listing(14..15)}

We now want to wait some time, by calling the subroutine. For this we need a
Branch with Link instruction. This instruction is at offset $32$, and the
subroutine starts at offset $0$, so we need to subtract $32+4=36$ from the PC
to jump there. Due to the encoding format of the \arm{BL} instruction, this
means we need to use $c=4194286$ (because $2*4194286-8*1024*1024=-36$):

\rs{a.get_listing(15..16)}

After that we want to turn the LED off and wait some time again. This requires
writing the value $2^{27}$ in the Clear Output Data Register, \ie, at address
\rs{hex(PIOB_CODR)}, and calling the subroutine again. Following the same
method as
above, we get:

\rs{a.get_listing(16..18)}

The last step of the main program is go back to step 2, with a \arm{B}
instruction. Following the same reasoning as we did for the subroutine, we find
that we need to use $c=2040$, to jump from offset 42 to offset 30. With the two
values \rs{hex(PIOB_PER)} and $2^{27}$ just after this last instruction we
obtain:

\rs{a.get_listing(18..21)}

Knowing the location of these two values, \ie, offsets 44 and 48, we can now
implement the two \arm{LDR} instructions we skipped at the beginning (for the
two \arm{LDR} instructions, $\lfloor PC \rfloor_4$ contains $24=14_{16}+4$):

\rs{a.get_listing(9..11)}

Putting everything together, we obtain the complete machine code for the main
program and its subroutine, with the main program starting at offset \hexa{14}:

\rs{a.get_machine_code_listing(0..21)}

\subsection{Run from RAM}\label{subsection:blink-led-boot-assistant}

As mentioned in \cref{section:boot-assistant}, the boot assistant has a
\code{G}$address$\code{\#} command to run a program. We show here how to use it
to run our ``blink LED'' program.

The \code{G}$address$\code{\#} command seems to take as argument the start
address of a ``mini Vector Table'' (this is not documented in \cite{SAM3X8E}).
Indeed, this command does not jump to $address$, but to the {\em interworking
address} stored at $address+4$ (just as the Cortex M3 starts by jumping to the
interworking address stored at address $4$). Presumably, the value at $address$
is meant to contain an initial Stack Pointer value, as in the Vector Table.
However, experiments show that this is not the case, \ie, programs run with
this command execute on the stack used by the boot assistant itself. They can
return to the boot assistant by moving the LR to the PC, but the SP must be the
same on return as what it was on entry.

To run our program with the boot assistant, the easiest way is to store it in
RAM. Lets put it after the first 4~KB, used by the boot assistant, at address
\hexa{20071000} (a multiple of 4, as we assumed when we wrote it). To run it
with the \code{G} command we must have the interworking address of its first
instruction (\hexa{20071000} + \hexa{14} + 1) somewhere in memory too. We can
put it just after our program, at address \hexa{20071034}. We should then be
able to run the program with the \code{G20071030\#} command. To verify this,
connect the Arduino to your computer and open a terminal as you did in
\cref{section:boot-assistant-first-steps}. Then write the program into RAM with
the following commands:

\rust{
  let micro_controller = RefCell::new(MicroController::default());
  let blink_counter = Rc::new(RefCell::new(BlinkLedCounter::default()));
  micro_controller.borrow_mut().set_max_boot_program_go_cycles(1000);
  micro_controller.borrow_mut().set_pio_device(blink_counter.clone());
  let mut boot_helper = BootHelper::new(&micro_controller);
  let words = a.machine_code();
  for line in boot_assistant_commands(&words, START_ADDRESS) {
    boot_helper.write(line.as_str());
  }
  boot_helper.write(format!("W{:08X},{:08X}#",
    START_ADDRESS + 4 * words.len() as u32,
    START_ADDRESS + a.label_offset("start") + 1).as_str());
}
\rs{host_log_multicols(&boot_helper.read(), 2)}

\noindent And launch it with

\rust{
  boot_helper.write(&format!("W{:X},0000000A#",
      START_ADDRESS + a.label_offset("wait_cycles")));
  boot_helper.read();
  boot_helper.write("G20071030#");
  assert!(blink_counter.borrow().blink_count >= 8);
}
\rs{host_log(&boot_helper.read())}

\indent If you didn't make any typo in the above commands, you should see the
LED blinking! After 5 seconds the \code{boot\_helper.py} program exits because
it hasn't received any response from the boot assistant, but our program
continues to run on the Arduino. You can then reset the Arduino to go back to
the boot assistant (at this point our program is lost).

\subsection{Run from flash}\label{subsection:blink-led-flash}

To avoid losing our program after it is run, we can store it in flash memory.
We can even run it directly when the Arduino boots, without going through the
boot assistant. This requires two things: storing a proper Vector Table in
flash memory (\cf \cref{section:vector-table}), and setting the boot mode
selection bit to boot from flash (\cf \cref{section:flash-controller}).

By default, only the first, second and fourth words of the Vector Table are
used (for the initial SP, initial PC, and the “hard fault” handler -- the third
one is “reserved”), since the exception and interrupt handlers are disabled by
default. We can thus store our program after these 4 words, \ie, starting at
offset \hexa{10} (a multiple of 4, as needed). The initial PC must then be set
to \hexa{80000} (the beginning of the flash memory region), plus \hexa{10},
plus \hexa{14} (the offset of the main program after the subroutine), plus 1
(an interworking address is required here). The initial SP can be set to almost
any address in RAM (our program pushes at most 3 words on the stack, so any
value larger than 12 bytes after the beginning of the RAM region is fine). Lets
use the end of the contiguous RAM region, \hexa{20088000} (see
\cref{fig:boot-memory-map}). Finally, the ``hard fault'' handler can be set to
the same value as the initial PC, \hexa{80025}, so that, in case of errors, our
program restarts from the beginning.

The above Vector Table and our program use 17 words in total, much less than
the 64 words of a flash memory page. However, as we have seen in
\cref{section:flash-controller}, writing a page in flash memory requires
writing 64 words in all cases. To avoid this extra work, we provide a program
called \code{flash\_helper.py}. This program extends \code{boot\_helper.py}
with an additional command named \code{flash\#}. When it runs on the host
computer, this program does the following:
\begin{itemize}
  \item When it receives a \code{W}{\em address},{\em value}\code{\#} command
  with an address in flash memory, {\em instead of sending it to the Arduino},
  \code{flash\_helper.py} sends 64 \code{w} commands to read the corresponding
  page. It stores the result in memory (on the host computer), and writes {\em
  value} in this copy of the page.

  \item When it receives the \code{flash\#} command, \code{flash\_helper.py}
  writes the (modified) page copies in the Arduino's flash memory. For this,
  for each page, it sends 64 \code{W} commands, followed by a \code{W} command
  in the Flash Controller Command Register to write the page, followed by
  \code{w} commands to read the Status Register until the write is done (\cf
  \cref{section:flash-controller}).

  \item When it receives any other boot assistant command,
  \code{flash\_helper.py} sends it directly to the Arduino.
\end{itemize}

Note that with this program, we can modify a single word in flash memory,
without modifying the 63 other words of the page, with only two commands
(namely a \code{W} and a \code{flash\#} -- with \code{boot\_helper.py} more
than 128 commands would be needed). Lets use it to write our program:

\rust{
  const FLASH_ADDRESS: u32 = 0x80010;
  let micro_controller = RefCell::new(MicroController::default());
  let blink_counter = Rc::new(RefCell::new(BlinkLedCounter::default()));
  micro_controller.borrow_mut().set_pio_device(blink_counter.clone());
  let mut flash_helper = FlashHelper::new(&micro_controller);
  let init_pc = FLASH_ADDRESS + a.label_offset("start") + 1;
  flash_helper.write("W80000,20088000#");
  flash_helper.write(format!("W80004,{init_pc:08X}#").as_str());
  flash_helper.write(format!("W8000C,{init_pc:08X}#").as_str());
  let words = a.machine_code();
  for line in boot_assistant_commands(&words, FLASH_ADDRESS) {
    flash_helper.write(line.as_str());
  }
  let mut log = flash_helper.read();
  flash_helper.write(&format!("W{:X},0000000A#",
      FLASH_ADDRESS + a.label_offset("wait_cycles")));
  flash_helper.read();
  flash_helper.write("flash#");
  log.push('\n');
  log.push_str(flash_helper.read().as_str());
  context.add_program("machine_code_blink", a);
}
\rs{host_log_multicols(&log, 2)}

At this point we could run our program with a \code{G80000\#} command, and we
would no longer loose it after a reset. Instead, lets set the boot mode
selection bit to boot from flash (\cf \cref{section:flash-controller}):

\rust{
  flash_helper.write(&format!("W{:X},5A00010B#", EEFC0_FCR.address));
  micro_controller.borrow_mut().reset();
  micro_controller.borrow_mut().run(1000);
  assert!(blink_counter.borrow().blink_count >= 8);
}
\rs{host_log(&flash_helper.read())}

\noindent Then press the RESET button on the Arduino: if you didn't make any
typo in the above commands, you should see the LED blinking. However, two
things can be noticed. First, the LED blinks slower than before (about 1 blink
every 2 seconds). This is because, by default, the Arduino's clock runs at
4MHz. But when the boot assistant starts, it sets the clock to a larger
frequency. Second, if you watch the LED at least 30 seconds, you can see that
some blinks are shorter than others. This is because our program is reset by
the Watchdog Timer (\cf \cref{section:overview-sam3x8e}). The boot assistant
disables it when it starts, which is why we didn't have this issue before. We
explain in \cref{chapter:clock} how to configure the clock and how to disable
the Watchdog Timer.

Our program is now persistent and runs autonomously, without needing the boot
assistant. But what if we want to modify it, for instance to make it blink
faster? For this we need to go back to the boot assistant. The only way to do
this at this point is to do a full ERASE, in order to reset the boot mode
selection bit to boot from ROM. Unfortunately, this also erases the flash
memory, and thus our program too. We would then need to flash the whole program
again, even if we just need to change one word (the one containing the maximum
counter value). To avoid this issue, the solution is to add a few instructions
at the beginning of our program, in order to set the boot mode selection to boot
from ROM as soon as our program starts. In this way, when we reset it, we will
automatically run the boot assistant again, without needing to do an ERASE.

% This work is licensed under the Creative Commons Attribution NonCommercial
% ShareAlike 4.0 International License. To view a copy of the license, visit
% https://creativecommons.org/licenses/by-nc-sa/4.0/

\renewcommand{\rustfile}{chapter3}
\setcounter{rustid}{0}

\rust{
  context.write_backup("website/backups", "opcodes_compiler.txt")?;
}

\chapter{Opcodes Compiler}\label{chapter:opcodes-compiler}

We now have everything we need to implement our compiler. We start in this
chapter with a very simple version, whose main role is to convert opcode names
into their numerical value. Indeed, this initial compiler must be written in
binary form, and should thus be as small as possible, in order to simplify our
task. Hopefully, this is the last program we need to write in such form
(besides a few small functions to launch programs with the memory editor). We
use it at the end to write a command editor, namely a small program to make it
easier to run other programs.

\section{Requirements}\label{section:toyc0-requirements}

The goal of our initial compiler is to convert opcode instructions from textual
to binary form. For instance, given the text ``\insn{fn} \insn{1} \insn{get}
\insn{0} \insn{cst\_0} \insn{ifne} \insn{10} $\ldots$'', it should produce, in
increasing address order, \hexa{19} \hexa{01} \hexa{16} \hexa{00} \hexa{00}
\hexa{10} \hexa{000A} $\ldots$ The programs that it should accept as input can
be described as ``zero or more instructions, one after the other'', where each
instruction is one of ``\insn{cst\_0}'', ``\insn{cst\_1}'', ``\insn{cst8}''
followed by an 8-bit value, ``\insn{cst}'' followed by a 32-bit value, and so
on for the remaining opcodes. These rules define the {\em grammar} of a {\em
programming language}, that valid programs must follow. They can be summarized
with:

\begin{Paragraph}
program: instruction*\\
instruction: ``\insn{cst\_0}'' | ``\insn{cst\_1}'' | ``\insn{cst8}'' INTEGER |
``\insn{cst32}'' INTEGER | $\ldots$
\end{Paragraph}

\noindent where ``*'' means ``zero or more times'' and ``|'' means ``or''. Text
between quotes, as well as names in capital letters, refer to individual
``words'' or ``punctuation signs'' of the program, called {\em tokens}. As in
English, tokens are generally separated by spaces. Here INTEGER designates an
integer value, \ie, a token made of one or more decimal digit characters. In
this context, we define the precise requirements of our initial compiler as
follows:
\begin{itemize}
  \item The compiler should take as input a source code address, noted
  $\it{src\_buffer}$, and a destination address where to store the compiled
  code, noted $\it{dst\_buffer}$.

  \item The source code should be in a data buffer (see
  \cref{subsection:data-buffer}), and should follow the above grammar.

  \item The compiled code must be produced in a data buffer. It must be the
  binary form of the bytecode instructions provided as input.

  \item The compiler should return 0 if the compilation was successful, and a
  non-zero value otherwise. In the latter case, the location of the error in
  the source code should be stored at the $\it{dst\_buffer}$ address.
\end{itemize}

Many errors could occur in the source code, such as an undefined opcode name
(``cst\_3''), an opcode without argument followed by integer value (``cst\_0
10''), an opcode with argument not followed by an integer value (``cst8
cst8''), an opcode with an 8-bit argument followed by an integer greater than
255, a jump instruction opcode followed by an invalid instruction offset, etc.
To simplify our task, in this chapter, we only require the detection of (most
of the) undefined opcode names.

\section{Algorithms}

A compiler can generally be divided in at least 3 parts: a {\em scanner}, a {\em
parser}, and a {\em backend} (see \cref{fig:compiler-parts}). The scanner reads
the source code and extracts its individual tokens. The parser calls the scanner
to read tokens, and checks that they follow the programming language's grammar.
The backend provides functions to generate the compiled code. In very simple
compilers such as ours, the parser uses the backend to directly produce the
compile code, while analyzing the source code.

Our compiler uses 5 main variables, shown in \cref{fig:compiler-parts}. Besides
$\it{src\_buffer}$ and $\it{dst\_buffer}$, already defined, the most
important ones are $src$ and $dst$. $src$ points to the next character to read.
$dst$ points to the next byte where compiled code must be written. Finally,
$src\_end$ points to the next byte after the end of the source code. When $src$
reaches $src\_end$ the whole program has been read and the compiler returns.

\begin{Figure}
  \input{figures/chapter3/compiler-parts.tex}

  \caption{The 3 parts of our compiler (top) and its 2 data structures
  (bottom), here with 3 tokens of a 15 bytes program already read (left) and
  compiled (right).}\label{fig:compiler-parts}
\end{Figure}

\bigskip \paragraph*{Scanner} The scanner splits the source code in tokens,
detects invalid tokens, and returns some data about each token. For instance, it
should detect that ``\insn{cst\_3}'' is invalid, and it could return 42 for the
token ``\insn{42}'' (\hexa{34}\hexa{32} in ASCII). To simplify, in this chapter,
we move the error detection in the backend. A token is then any sequence of
characters which does not contain a space, a tabulation, or a ``new line''. To
compute the numerical value $v$ of an integer token ``$c_{n-1}\ldots c_1c_0$'',
we can initialize $v$ to 0 and update $v$ to $10v+(c_i-\hexa{30})$, for each
character $c_i$ from left to right. In fact, to simplify the initial compiler,
the scanner returns such a value for {\em all} tokens. For instance, for the
``\insn{fn}'' token (\hexa{66}\hexa{6E} in ASCII), it returns
$10(\hexa{66}-\hexa{30})+(\hexa{6E}-\hexa{30})=602$. In summary, a token is read
as described in \cref{alg:scanner0}, which also corresponds to the finite state
machine in \cref{fig:toyc0-automaton}.

\begin{Algorithm}
\caption{Reading a token and returning its value $v$.}\label{alg:scanner0}
\begin{algorithmic}[1]
\Begin while $src<src\_end$ and the character at $src$ is a space, tab or ``new
line''
  \State increment $src$ by 1 to skip this character
\End
\State if $src=src\_end$ return nothing
\State initialize $v$ to 0
\Begin while $src<src\_end$ and the character $c$ at $src$ is not a space, tab
or ``new line''
  \State update $v$ to $10v+(c-\hexa{30})$
  \State increment $src$ by 1
\End
\State return $v$
\end{algorithmic}
\end{Algorithm}

\begin{Figure}
  \newcommand\vsp[1][.75em]{%
    \makebox[#1]{%
      \kern.07em
      \vrule height.3ex
      \hrulefill
      \vrule height.3ex
      \kern.07em
    }%
  }
  \input{figures/chapter3/automaton.tex}

  \caption{The scanner and parser can be modeled with Finite State Machines (see
  \cref{subsection:keyboard-driver-design}) reading characters $c$ (left) and
  token values $v$ (right), respectively. \vsp~represents a space, tab or ``new
  line''. Many parser transitions are not shown.}\label{fig:toyc0-automaton}
\end{Figure}

\bigskip \paragraph*{Backend} The backend provides functions to write opcodes
and their arguments in the output buffer. Here we mostly need functions to
write 8-bit and 16-bit values in memory, plus some code to detect invalid
opcodes. Valid opcodes are between 0 and 31 included, but we also add here a
pseudo opcode \insn{d} (for ``data'', with value 32), with an 8-bit argument
$x$. Once compiled, a \insn{d} $x$ instruction simply produces the byte $x$. It
can be used to mix code and data (such as the transition table of our keyboard
driver). In summary, a function to write an $opcode$ (without its argument)
should return an error if $opcode>32$, do nothing if $opcode=32$, or write the
$opcode$ byte otherwise.

\bigskip \paragraph*{Parser} The parser calls the scanner to read the source
code one token at a time, and generates the corresponding compiled code with the
backend. For our very simple initial compiler, the parser can be modeled with a
Finite State Machine, represented in \cref{fig:toyc0-automaton}. There are 4
states, corresponding to the expected ``type'' of the next token returned by the
scanner. State 0 corresponds to opcode tokens, such as \insn{add}. States 1, 2,
and 4 correspond to 1, 2, and 4-byte opcode arguments, respectively (such as the
argument of \insn{cst8}, \insn{iflt}, and \insn{cst}, respectively). In these 3
states, any token value $v$ should simply be written at $dst$ in 1, 2, or 4
bytes, and the next state is state 0. In state 0, the opcode corresponding to
the token value $v$, noted $opcode(v)$, should be written at $dst$. And the next
state, noted $S(v)$, should be either 0, 1, 2, or 4, depending on this opcode.
By listing the opcode names and computing their token values $v$ with
\cref{alg:scanner0}, we get $opcode(v)$ and $S(v)$, shown in
\cref{table:parser0}.

In order to implement this Finite State Machine we need functions to compute
$opcode(v)$ and $S(v)$. For this, the easiest is to store \cref{table:parser0}
in memory. $opcode(v)$ can then be computed by finding the row corresponding to
$v$, and then returning the value in its $opcode$ column -- and similarly for
$S(v)$. In fact, since the $opcode$ of the $i^{th}$ row is $i$, we don't need
to store this column. Notice also that the least significant byte $lsb(v)$ of
the token values $v$ are all unique. We can thus store only one byte per value
in this column. In summary, $opcode(v)$ and $S(v)$ can be computed as described
in \cref{alg:parser0-getopcode}, where {\tt LSB} and {\tt S} are the $lsb(v)$
and $S(v)$ value lists.

\rust{
  let opcode_names = vec!["cst_0", "cst_1", "cst8", "cst", "add", "sub", "mul",
  "div", "and", "or", "lsl", "lsr", "iflt", "ifeq", "ifgt", "ifle", "ifne",
  "ifge", "goto", "load", "store", "ptr", "get", "set", "pop", "fn", "call",
  "callr", "calld", "ret", "retv", "blx", "d"];
  let opcode_args = [0, 0, 1, 4, 0, 0, 0, 0, 0, 0, 0, 0, 2, 2, 2, 2, 2, 2,
  2, 0, 0, 1, 1, 1, 0, 1, 2, 2, 0, 0, 0, 0, 1];
  assert_eq!(opcode_args.len(), opcode_names.len());
  let token_value = |opcode:&&str| {
    let mut v = 0;
    for c in opcode.chars() {
      v = 10 * v + (c as u32 - '0' as u32);
    }
    v
  };
  let token_values : Vec<u32> =
      opcode_names.iter().map(token_value).collect();
  let table_row = |i: usize| {
    format!("{{\\makecell\\tt {}}} & {:X} & {:02X} & {}",
        opcode_names[i].replace('_', "\\_"), token_values[i], i, opcode_args[i])
  };
}

\begin{Table}
\begin{tabular}[t]{|l|r|r|c|} \hline
  \makecell{\thead{token}} & $v$ & $opcode$ & $S$ \\ \hline
  \rs{table_row(0)} \\
  \rs{table_row(1)} \\
  \rs{table_row(2)} \\
  \rs{table_row(3)} \\
  \rs{table_row(4)} \\
  \rs{table_row(5)} \\
  \rs{table_row(6)} \\
  \rs{table_row(7)} \\
  \rs{table_row(8)} \\
  \rs{table_row(9)} \\
  \rs{table_row(10)} \\
  \rs{table_row(11)} \\
  \rs{table_row(12)} \\
  \rs{table_row(13)} \\
  \rs{table_row(14)} \\
  \rs{table_row(15)} \\
  \rs{table_row(16)} \\ \hline
\end{tabular}
\hspace{1cm}
\begin{tabular}[t]{|l|r|r|c|} \hline
  \makecell{\thead{token}} & $v$ & $opcode$ & $S$ \\ \hline
  \rs{table_row(17)} \\
  \rs{table_row(18)} \\
  \rs{table_row(19)} \\
  \rs{table_row(20)} \\
  \rs{table_row(21)} \\
  \rs{table_row(22)} \\
  \rs{table_row(23)} \\
  \rs{table_row(24)} \\
  \rs{table_row(25)} \\
  \rs{table_row(26)} \\
  \rs{table_row(27)} \\
  \rs{table_row(28)} \\
  \rs{table_row(29)} \\
  \rs{table_row(30)} \\
  \rs{table_row(31)} \\
  \rs{table_row(32)} \\ \hline
\end{tabular}
  \caption{The token value $v$ and the corresponding compiled $opcode$ and next
  state $S$ for each valid opcode token.}\label{table:parser0}
\end{Table}

\rust{
  let lsb = |v:u32| { format!("{:02X}", v & 0xFF) };
  let lsb_set : HashSet<u32> = token_values.iter().map(|v| v & 0xFF).collect();

  assert_eq!(lsb_set.len(), token_values.len());
  assert!(!lsb_set.contains(&(token_value(&"cst_3") & 0xFF)));
  assert!(lsb_set.contains(&(token_value(&"cst_2") & 0xFF)));
}

\begin{Algorithm}
\caption{Computing \{$opcode(v)$, $S(v)$\} for a token value
$v$.}\label{alg:parser0-getopcode}
\begin{algorithmic}[1]
\Statex {\tt LSB} = [\rs{lsb(token_values[0])}, \rs{lsb(token_values[1])},
\rs{lsb(token_values[2])}, \rs{lsb(token_values[3])}, $\ldots$], {\tt S} = [0,
0, 1, 4, $\ldots$]
\State initialize $i$ to 0
\Begin while $i \le 32$ and the $i^{th}$ value in {\tt LSB} is not equal to $(v
\wedge 255)$
  \State increment $i$ by 1
\End
\State return \{$i$, $i^{th}$ value in {\tt S}\}
\end{algorithmic}
\end{Algorithm}

Note that for the invalid token \insn{cst\_3}, $lsb(v)$ is equal to
\rs{lsb(token_value(&"cst_3"))}, which is not in {\tt LSB}. In such cases,
\cref{alg:parser0-getopcode} returns the invalid opcode 33. Hence, most invalid
tokens can be detected by checking for invalid opcodes. However, some invalid
tokens, such as \insn{cst\_2}, cannot be detected like this because the least
significant byte of their token value {\em is} in {\tt LSB}. We fix this in
\cref{chapter:labels-compiler}, at the price of a greater complexity.

\section{Implementation}\label{section:toyc0-implementation}

\rust{
  let compiler_address = next_page_address(
      context.memory_region("text_editor").end());
  let mut b = BytecodeAssembler::new(RegionKind::DataBuffer, compiler_address);
  b.import_labels(context.memory_region("graphics_card_driver"));
  b.import_labels(context.memory_region("flash_driver"));
}

We can now implement this initial compiler. We do this in a new data buffer, in
the next flash memory page after the text editor (\ie, at address
\rs{hex(compiler_address)}). We start with the scanner, with a function
returning 1 if a given character $c$ is a space, a tabulation, or a ``new line''
(\hexa{20}, \hexa{09}, and \hexa{0A} in ASCII, respectively), and 0 otherwise:

\begin{TwoColumns}
\rs{b.func("tc_is_space", &["c"], "bool", &[])}\\
\bytecode{
  b.get("c");
  b.cst8(32);
  b.ifeq("is_space_true");
  b.get("c");
  b.cst8(9);
  b.ifeq("is_space_true");
  b.get("c");
  b.cst8(10);
  b.ifeq("is_space_true");
  b.cst_0();
  b.retv();
  b.label("is_space_true");
  b.cst_1();
  b.retv();
}
\end{TwoColumns}

We then implement \cref{alg:scanner0} in two parts. Steps 1 and 2 are
implemented in the following function, which returns the new $src$ value:

\begin{Paragraph}
\begin{paracol}{2}
\rs{b.func("tc_skip_spaces", &["src", "src_end"], "src'", &[])}

Initialize $src'$ to $src$.

\bytecode[switchcolumn]{
  b.get("src");
  b.def("src'");
}

Step 1. If $src'$ (in the $6^{th}$ stack frame slot) is greater than or equal to
$src\_end$, go to the last instruction.

\bytecode[switchcolumn]{
  b.label("tc_skip_spaces_loop");
  // while src' < end && tc_is_space(load_byte(src')) == 1 :
  b.get("src'");
  b.get("src_end");
  b.ifge("tc_skip_spaces_end");
}

If the character at $src'$ is not a spacing character, go to the last
instruction.

\bytecode[switchcolumn]{
  b.get("src'");
  b.call("load_byte");
  b.call("tc_is_space");
  b.cst_1();
  b.ifne("tc_skip_spaces_end");
}

Step 2. Increment $src'$ (the top stack value) by 1 and go back to step 1.

\bytecode[switchcolumn]{
  // src' = src' + 1;
  b.cst_1();
  b.add();
  b.goto("tc_skip_spaces_loop");
}

Return the top stack value $src'$.

\bytecode[switchcolumn]{
  b.label("tc_skip_spaces_end");
  // return src';
  b.retv();
}
\end{paracol}
\end{Paragraph}

Steps 5 to 7 are implemented in the next function, which also returns the new
$src$ value (we assume that steps 3 and 4 are done by the caller). Since a
function can't return several values, it can't return $v$ as described in
\cref{alg:scanner0}. Instead, it takes as parameter a {\em pointer} $v^p$ to a
memory word where $v$ can be read and modified:

\begin{Paragraph}
\begin{paracol}{2}
\rs{b.func("tc_read_token", &["src", "src_end", "v^p"], "src'", &[])}

Initialize $src'$ to $src$.

\bytecode[switchcolumn]{
  b.get("src");
  b.def("src'");
}

Step 5. If $src'$ (in the $7^{th}$ stack frame slot) is greater than or equal to
$src\_end$, go to the last instruction.

\bytecode[switchcolumn]{
  b.label("tc_read_token_loop");
  // while src' < src_end && tc_is_space(load_byte(src')) == 0 :
  b.get("src'");
  b.get("src_end");
  b.ifge("tc_read_token_end");
}

If the character at $src'$ is a spacing character, go to the last instruction.

\bytecode[switchcolumn]{
  b.get("src'");
  b.call("load_byte");
  b.call("tc_is_space");
  b.cst_1();
  b.ifeq("tc_read_token_end");
}

Step 6. Update $v$, at address $v^p$, to $10v+(c-\hexa{30})$, where $c$ is the
character at $src'$. To simplify, we do not check if this new value actually
fits in a word.

\bytecode[switchcolumn]{
  // *v = *v * 10 + (load8(src') - '0');
  b.get("v^p");
  b.get("v^p");
  b.load();
  b.cst8(10);
  b.mul();
  b.get("src'");
  b.call("load_byte");
  b.cst8(48);
  b.sub();
  b.add();
  b.store();
}

Step 7. Increment $src'$ (the top stack value) by 1 and go back to step 5.

\bytecode[switchcolumn]{
  // src' = src' + 1;
  b.cst_1();
  b.add();
  b.goto("tc_read_token_loop");
}

Return the top stack value $src'$.

\bytecode[switchcolumn]{
  b.label("tc_read_token_end");
  // return src';
  b.retv();
}
\end{paracol}
\end{Paragraph}

This concludes the scanner part. We continue with the backend part. As said
above, we mostly need here functions to store 8-bit and 16-bit values in memory.
We already have a \verb!store_byte! function (see
\cref{table:flash_driver_functions}), hence we only need a new \verb!store_half!
function (very similar to \verb!store_byte!, already explained):

\begin{TwoColumns}
\rs{b.func("store_half", &["ptr", "value"], "", &[])}\\
\bytecode{
  b.get("ptr");
  b.get("ptr");
  b.load();
  b.cst(0xFFFF0000);
  b.and();
  b.get("value");
  b.or();
  b.store();
  b.ret();
}
\end{TwoColumns}

We finish the backend part with a function to write an $opcode$ byte at $dst$,
which returns the new $dst$ value, $dst'$. As described above, this function
returns an error (represented with $dst=0$) if $opcode>32$, and does nothing if
$opcode=32$:

\bigskip \rs{b.func("tc_write_opcode", &["dst", "opcode"], "dst'", &[])}
\vspace{-0.9\baselineskip}
\begin{TwoColumns}
\bytecode{
  // if opcode == 33 { return 0; }
  b.get("opcode");
  b.cst8(33);
  b.ifne("tc_write_opcode_valid");
  b.cst_0();
  b.retv();
  b.label("tc_write_opcode_valid");
  // if opcode == 32 { return dst; }
  b.get("opcode");
  b.cst8(32);
  b.ifne("tc_write_opcode_end");
  // return dst;
  b.get("dst");
  b.retv();
  b.label("tc_write_opcode_end");
  // store_byte(dst, opcode);
  b.get("dst");
  b.get("opcode");
  b.call("store_byte");
  // return dst + 1;
  b.get("dst");
  b.cst_1();
  b.add();
  b.retv();
}
\end{TwoColumns}

\rust{
  b.label("LSB");
  for x in token_values {
    b.u8_data(x as u8);
  }
  b.u8_data(0);
  b.label("ARGUMENT");
  for x in opcode_args {
    b.u8_data(x);
  }
  b.u8_data(0);
}

We continue the implementation with the parser part, starting with
\cref{alg:parser0-getopcode}. We first store the {\tt LSB} and {\tt S} tables,
at addresses \rs{hex(b.label_address("LSB"))} and
\rs{hex(b.label_address("ARGUMENT"))}, respectively (note that we end each
table with a $33^{rd}$ 0 value, since $i$ can be equal to 33 at step 4 of
\cref{alg:parser0-getopcode}):

\bytecode[binary]{
  b.label("print-the-above-data");
}

We then implement \cref{alg:parser0-getopcode} in the following function. Since
a function can't return several values, it returns the $opcode$ only, and
stores $S$ at an address $S^p$ passed as parameter. Note also that this
function takes the least significant byte $lsb$ of $v$ as parameter (instead of
$v$ in \cref{alg:parser0-getopcode}):

\begin{Paragraph}
\begin{paracol}{2}
\rs{b.func("tc_get_opcode", &["lsb", "S^p"], "opcode", &[])}

Step 1. Initialize $i$ to 0.

\bytecode[switchcolumn]{
  b.cst_0();
  b.def("i");
}

Step 2. If $i$ (in the $6^{th}$ stack frame slot) is greater than 32, go to
step 4.

\bytecode[switchcolumn]{
  // while i <= 32 && get8(LSB + i) != lsb
  b.label("tc_get_opcode_loop");
  b.get("i");
  b.cst8(32);
  b.ifgt("tc_get_opcode_loop_end");
}

If the $i^{th}$ value in {\tt LSB} is equal to $lsb$, go to step 4.

\bytecode[switchcolumn]{
  b.cst(b.label_address("LSB"));
  b.get("i");
  b.add();
  b.call("load_byte");
  b.get("lsb");
  b.ifeq("tc_get_opcode_loop_end");
}

Step 3. Increment the top stack value $i$ by 1 and go back to step 2.

\bytecode[switchcolumn]{
  b.cst_1();
  b.add();
  b.goto("tc_get_opcode_loop");
}

Step 4. Store the $i^{th}$ value of {\tt S} at $S^p$ and return the top stack
value $i$.

\bytecode[switchcolumn]{
  b.label("tc_get_opcode_loop_end");
  // *S^p = get8(ARGUMENT + i);
  b.get("S^p");
  b.cst(b.label_address("ARGUMENT"));
  b.get("i");
  b.add();
  b.call("load_byte");
  b.store();
  // return opcode;
  b.retv();
}
\end{paracol}
\end{Paragraph}

With this we can now implement a function to perform a transition of the
parser's Finite State Machine. The following function takes a pointer $S^p$ to
the current state $S$ as parameter, as well as a token value $v$ and the
current value of $dst$. It performs the corresponding action, updates the value
at $S^p$ to the next state, and returns the new $dst$ value. It has 4 main
parts, corresponding to the 4 possible values of the current state, plus a
shared $5^{th}$ part:

\begin{Paragraph}
\begin{paracol}{2}
\rs{b.func("tc_parse_token", &["dst", "v", "S^p"], "dst'", &[])}

Get the value $S$ at address $S^p$.

\bytecode[switchcolumn]{
  // let S = *S^p;
  b.get("S^p");
  b.load();
  b.def("S");
}

Part 1. If $S$ (in the $7^{th}$ stack frame slot) is not 0, go to part 2.

\bytecode[switchcolumn]{
  // if S == OPCODE:
  b.get("S");
  b.cst_0();
  b.ifne("tc_parse_token_not_opcode");
}

Otherwise, compute $opcode(v)$ and store $S(v)$ at $S^p$ by calling
\verb!tc_get_opcode! on the least significant byte of $v$, $v \wedge 255$.
Write this opcode at $dst$ by calling \verb!tc_write_opcode!, and return the
result.

\bytecode[switchcolumn]{
  // return tc_write_opcode(dst, tc_get_opcode(v & 255, S^p));
  b.get("dst");
  b.get("v");
  b.cst8(255);
  b.and();
  b.get("S^p");
  b.call("tc_get_opcode");
  b.call("tc_write_opcode");
  b.retv();
}

Part 2. If $S$ is not 1, go to part 3.

\bytecode[switchcolumn]{
  b.label("tc_parse_token_not_opcode");
  // if S == DATA8:
  b.get("S");
  b.cst_1();
  b.ifne("tc_parse_token_not_data8");
}

Otherwise, store the byte $v$ at $dst$ and go to part 5 (to simplify we do
not check if $v$ actually fits in a byte).

\bytecode[switchcolumn]{
  // store_byte(dst, v);
  b.get("dst");
  b.get("v");
  b.call("store_byte");
  b.goto("tc_parse_token_end");
}

Part 3. If $S$ is not 2, go to part 4.

\bytecode[switchcolumn]{
  b.label("tc_parse_token_not_data8");
  // else if S == DATA16:
  b.get("S");
  b.cst8(2);
  b.ifne("tc_parse_token_not_data16");
}

Otherwise, store the half word $v$ at $dst$ and go to part 5 (to simplify we
do not check if $v$ actually fits in a half word).

\bytecode[switchcolumn]{
  // store_half(dst, v);
  b.get("dst");
  b.get("v");
  b.call("store_half");
  b.goto("tc_parse_token_end");
}

Part 4. $S$ is necessarily equal to 4. Store the word $v$ at $dst$ and continue
to part 5.

\bytecode[switchcolumn]{
  b.label("tc_parse_token_not_data16");
  // *dst = v;
  b.get("dst");
  b.get("v");
  b.store();
}

Part 5. Update the value at $S^p$ to 0, the next state after a transition from
state 1, 2, or 4.

\bytecode[switchcolumn]{
  b.label("tc_parse_token_end");
  // *S^p = OPCODE;
  b.get("S^p");
  b.cst_0();
  b.store();
}

Return the new $dst$ value, $dst+S$ (since $S$ is the number of bytes just
written).

\bytecode[switchcolumn]{
  // return dst + S;
  b.get("dst");
  b.get("S");
  b.add();
  b.retv();
}
\end{paracol}
\end{Paragraph}

We can finally implement the compiler's main function. It starts by
initializing $src$ to $\it{src\_buffer}+4$, $src\_end$ to
$src+\mathrm{mem32}[\it{src\_buffer}]$, $dst$ to $\it{dst\_buffer}+4$,
the token value $v$ to 0, and the Finite State Machine state $S$ to 0, in stack
frame slots 6, 7, 8, 9, and 10, respectively:

\bigskip \rs{b.func("tc_main", &["src_buffer", "dst_buffer"], "error", &[])}
\vspace{-0.9\baselineskip}
\begin{TwoColumns}
\bytecode{
  // let src = src_buffer + 4;
  b.get("src_buffer");
  b.cst8(4);
  b.add();
  b.def("src");
  // let end = src + *src_buffer;
  b.get("src");
  b.get("src_buffer");
  b.load();
  b.add();
  b.def("src_end");
  // let dst = dst_buffer + 4;
  b.get("dst_buffer");
  b.cst8(4);
  b.add();
  b.def("dst");
  // let v = 0;
  b.cst_0();
  b.def("v");
  // let S = OPCODE;
  b.cst_0();
  b.def("S");
}
\end{TwoColumns}

It continues with a loop which 1) skips spaces and returns 0 if $src\_end$ is
reached, 2) reads a token and performs the corresponding Finite State Machine
transition, 3) returns 1 if an invalid token was found:

\begin{Paragraph}
\begin{paracol}{2}
Step 1. Update $src$ to the result of \verb!tc_skip_spaces!($src$, $src\_end$).

\bytecode[switchcolumn]{
  b.label("tc_main_loop");
  // loop :
  // src = tc_skip_spaces(src, src_end);
  b.get("src");
  b.get("src_end");
  b.call("tc_skip_spaces");
  b.set("src");
}

Step 2. If $src<src\_end$, go to step 3.

\bytecode[switchcolumn]{
  // if src >= src_end :
  b.get("src");
  b.get("src_end");
  b.iflt("tc_main_ok");
}

Otherwise, \ie, if the end of the program is reached, set the value at
$\it{dst\_buffer}$ to the number of bytes written,
$dst-\it{dst\_buffer}-4$, and return 0 (meaning ``no error''). To simplify,
we do not check if $S$ is 0 (if not the program ends in the middle of an
instruction, which is an error).

\bytecode[switchcolumn]{
  // *dst_buffer = dst - dst_buffer - 4;
  b.get("dst_buffer");
  b.get("dst");
  b.get("dst_buffer");
  b.sub();
  b.cst8(4);
  b.sub();
  b.store();
  // return 0;
  b.cst_0();
  b.retv();
}

Step 3. Call the scanner to read a token and store its value in $v$. Update
$src$ to the result of \verb!tc_read_token!.

\bytecode[switchcolumn]{
  b.label("tc_main_ok");
  // src = tc_read_token(src, src_end, &v);
  b.get("src");
  b.get("src_end");
  b.ptr("v");
  b.call("tc_read_token");
  b.set("src");
}

Step 4. Perform the Finite State Machine transition corresponding to $v$.
Update $dst$ to the result of \verb!tc_parse_token!.

\bytecode[switchcolumn]{
  // dst = tc_parse_token(dst, v, &S);
  b.get("dst");
  b.get("v");
  b.ptr("S");
  b.call("tc_parse_token");
  b.set("dst");
}

Step 5. If $dst \ne 0$, go to step 6.

\bytecode[switchcolumn]{
  // if dst == 0 :
  b.get("dst");
  b.cst_0();
  b.ifne("tc_main_end_loop");
}

Otherwise, \ie, if an invalid token has been read, set the value at
$\it{dst\_buffer}$ to the location of the error,
$src-\it{src\_buffer}-4$, and return 1 (meaning ``error'').

\bytecode[switchcolumn]{
  // *dst_buffer = src - src_buffer - 4;
  b.get("dst_buffer");
  b.get("src");
  b.get("src_buffer");
  b.sub();
  b.cst8(4);
  b.sub();
  b.store();
  // return 1;
  b.cst_1();
  b.retv();
}

Step 6. Reinitialize $v$ to 0 for the next loop iteration, and go back to step
1.

\bytecode[switchcolumn]{
  b.label("tc_main_end_loop");
  // v = 0;
  b.cst_0();
  b.set("v");
  b.goto("tc_main_loop");
}
\end{paracol}
\end{Paragraph}

In summary the full code of our initial compiler is the following:

\rs{b.get_bytecode_listing(0..b.get_instruction_count() as usize, false)}

\rust{
  // generate command file to flash toyc0 with SAMBA
  let mut commands = Vec::new();
  commands.extend(b.boot_assistant_commands());
  commands.push(String::from("flash#"));
  commands.push(String::from("reset#"));
  write_lines("website/part3", "opcodes_compiler.txt", &commands)?;
}

\rust{
  // enter toyc0 code in RAM with memory editor
  let display = Rc::new(RefCell::new(TextDisplay::default()));
  context.set_display(display.clone());

  context.add_memory_region("toyc0", b.memory_region());
  context.micro_controller().borrow_mut().reset();
  context.run_until_get_char();
  let mut context1 = context.clone();

  const COMMAND_ADDRESS: u32 = 0x20080000;
  const BUFFER_RAM_ADDRESS: u32 = 0x20070000;
  context.type_ascii(&b.memory_editor_commands(BUFFER_RAM_ADDRESS));
}

To store it in flash memory we must enter it in RAM first, lets say at address
\rs{hex(BUFFER_RAM_ADDRESS)}, and then save it in flash. In the memory editor,
type ``w\rs{hex_word_low(BUFFER_RAM_ADDRESS)}''+Enter, and then store the
compiler size in bytes at this address by typing
``w\rs{hex_word_low(b.bytecode_size())}''+Enter. Continue by entering each word
of the compiler code, listed above, by typing its value followed by Enter.
Finally, save this code in flash memory (starting at
$page=\rs{dec(page_number(compiler_address))}$) by running the following
function:

\rust{
  // small function to store toyc0 in flash at 'compiler_address'
  let mut c = BytecodeAssembler::default();
  c.import_labels(context.memory_region("flash_driver"));
}
\begin{TwoColumns}
\rs{c.func("save", &[], "", &["nolink"])}\\
\bytecode{
  c.cst(BUFFER_RAM_ADDRESS);
  c.cst8(page_number(compiler_address).try_into().unwrap());
  c.call("buffer_flash");
  c.ret();
}
\end{TwoColumns}

For this enter the full code of the above function in an unused RAM region, for
instance starting at address \rs{hex(COMMAND_ADDRESS)}:

\rs{c.get_bytecode_listing(0..c.get_instruction_count() as usize, false)}

\rust{
  // enter above function in RAM and execute it.
  context.type_ascii(&c.memory_editor_commands(COMMAND_ADDRESS));
  context.type_ascii(&format!("W{:08X}\n", COMMAND_ADDRESS));
  context.type_ascii("R");

  let boot_mode_address = context.memory_region("foundations")
      .label_address("boot_mode_select_rom");
}

Then type ``w\rs{hex_word_low(COMMAND_ADDRESS)}'' followed by ``r'' to run it.
Alternatively, if you don't want to enter the full compiler code manually with
the memory editor, which is a bit tedious, you can ``cheat'' by saving it via
an external computer, as follows. First run the \verb!boot_mode_select_rom!
function by typing ``w\rs{hex_word_low(boot_mode_address)}''+Enter, followed by
``r''. Then reset the Arduino and, on the host computer, run the following
command to flash the compiler code and reset the Arduino again:

\rust{
  context1.type_ascii(&format!("W{:08X}\n", boot_mode_address));
  context1.type_ascii("R");
  context1.micro_controller().borrow_mut().reset();
  let mut flash_helper1 = FlashHelper::from_file(
      context1.micro_controller(), "website/", "part3/opcodes_compiler.txt")?;
}
\rs{host_log(&flash_helper1.read())}

\section{Command editor}

We can now write and compile our very first program in textual form. For this
we first need to enter it in memory with the text editor. This requires calling
the text editor, and then the compiler, with specific arguments. In turn, this
currently requires typing a few bytecode instructions {\em in binary form} with
the memory editor, as we did above to call {\tt buffer\_flash}. To avoid having
to do this in the next chapters, our first program is a {\em command editor}.
Its goal is to edit, compile and run small functions, called {\em commands},
such as the {\tt save} function above.

\subsection{User interface}

A task such as writing and compiling a program requires less than a dozen
distinct commands to edit the program, save it, compile it, save the compiled
code, etc. However, each command must usually be run several times (if the
compiler returns an error, the program must be edited, saved, and compiled
again). In order to avoid having to repeatedly type the same commands, the
command editor should be able to save up to 12 distinct commands in flash
memory. We number them from 1 to 12. It should then be able to load an existing
command, and to edit it if necessary. Finally, it should be able to compile and
run a command. To fulfill these requirements we define the command editor user
interface as follows:
\begin{itemize}
  \item typing a ``F$i$'' key between ``F1'' and ``F12'' included should load
  command number $i$ and display it. This command becomes the {\em current
  command}.

  \item typing ``e'' should run the text editor to edit the current command.
  Each command must be a function without argument, returning an integer value.

  \item typing ``s'' should save the current command in flash memory.

  \item typing ``r'' should compile the current command, run it, display its
  result, and wait until Enter is pressed (and not until any key press
  because releasing ``r'' can appear as a key press for commands using the
  flash memory driver -- see \cref{section:flash-memory-driver-impl}). If the
  compilation fails, the compiler result should be displayed instead.

  \item typing Escape should exit the command editor.
\end{itemize}

Finally, when launched, the command editor should load and display command
number 1. All commands are initially empty in flash memory.

\subsection{Implementation}\label{subsection:command-editor-implementation}

\rust{
  const PAGE_SIZE: u32 = 256;

  // Address of command editor code and source code in flash memory
  // (sizes include buffer header).
  const MAX_COMMAND_EDITOR_CODE_SIZE: u32 = 256;
  const MAX_COMMAND_EDITOR_SOURCE_SIZE: u32 = 1024;
  const COMMAND_SOURCE: u32 = 0xD0000;
  const NUM_COMMAND_EDITOR_COMMANDS: u32 = 12;
  const MAX_COMMAND_CODE_SIZE: u32 = PAGE_SIZE;
  const MAX_COMMAND_SOURCE_SIZE: u32 = PAGE_SIZE;
  let command_editor_code =
      next_page_address(context.memory_region("toyc0").end());
  let command_editor_source =
      COMMAND_SOURCE + NUM_COMMAND_EDITOR_COMMANDS * MAX_COMMAND_SOURCE_SIZE;

  // Address of current command source code and compiled code in RAM.
  const RAM_START: u32 = 0x20070000;
  const RAM_COMMAND_SOURCE: u32 = RAM_START;
  const RAM_COMMAND_CODE: u32 = RAM_COMMAND_SOURCE + MAX_COMMAND_SOURCE_SIZE;

  // Address of compiler code, backup code, and source code in flash
  // memory (sizes include buffer header).
  const MAX_COMPILER_CODE_KB: u32 = 12;
  const MAX_COMPILER_CODE_SIZE: u32 = MAX_COMPILER_CODE_KB * 1024;
  const MAX_COMPILER_SOURCE_KB: u32 = 48;
  const MAX_COMPILER_SOURCE_SIZE: u32 = MAX_COMPILER_SOURCE_KB * 1024;
  let compiler_code = command_editor_code + MAX_COMMAND_EDITOR_CODE_SIZE;
  // bytecode call instructions only support of 64 KB range after 0xC0000.
  assert!(compiler_code + MAX_COMPILER_CODE_SIZE < 0xC0000 + 65536);
  let compiler_code_page = page_number(compiler_code);
  let compiler_code_backup = 0xE0000;
  let compiler_code_backup_page = page_number(compiler_code_backup);
  let compiler_source = command_editor_source + MAX_COMMAND_EDITOR_SOURCE_SIZE;
  let compiler_source_page = page_number(compiler_source);

  // Address of compiler code and source code in RAM.
  const RAM_COMPILER_SOURCE: u32 = RAM_COMMAND_CODE + MAX_COMMAND_CODE_SIZE;
  const RAM_COMPILER_CODE: u32 = RAM_COMPILER_SOURCE + MAX_COMPILER_SOURCE_SIZE;

  let mut compiler_labels = HashMap::<String, Label>::new();
  compiler_labels.insert(
      String::from("main"),
      Label {
          offset: 0,
          description: String::default(),
      },
  );
  context.add_memory_region(
      "compiler_code",
      MemoryRegion::new(
          RegionKind::DataBuffer,
          compiler_code,
          MAX_COMPILER_CODE_SIZE,
          &compiler_labels,
          0,
          0,
          0,
          Vec::default(),
      ),
  );
  context.add_memory_region(
      "compiler_code_backup",
      MemoryRegion::new(
          RegionKind::DataBuffer,
          compiler_code_backup,
          MAX_COMPILER_CODE_SIZE,
          &HashMap::default(),
          0,
          0,
          0,
          Vec::default(),
      ),
  );
  context.add_memory_region(
      "compiler_source",
      MemoryRegion::new(
          RegionKind::DataBuffer,
          compiler_source,
          MAX_COMPILER_SOURCE_SIZE,
          &HashMap::default(),
          0,
          0,
          0,
          Vec::default(),
      ),
  );

  // generate source code of command editor program
  let mut b =
      BytecodeAssembler::create(RegionKind::DataBuffer, command_editor_code,
      true);
  b.import_labels(context.memory_region("graphics_card_driver"));
  b.import_labels(context.memory_region("keyboard_driver"));
  b.import_labels(context.memory_region("memory_editor"));
  b.import_labels(context.memory_region("flash_driver"));
  b.import_labels(context.memory_region("text_editor"));
  b.import_labels(context.memory_region("toyc0"));
}

We can now write the command editor source code. For this we assume that its
compiled code will eventually be stored in the next page after the opcodes
compiler, \ie, at address \rs{hex(command_editor_code)}%
=\hexa{C0000}+\rs{dec(command_editor_code-0xC0000)}.

To implement the above requirements we reserve 12 pages of flash memory, one
for each command, starting at address \rs{hex(COMMAND_SOURCE)}. This gives
$256-4=252$ bytes for the source code of each command, stored as a data buffer
(see \cref{subsection:data-buffer}). We can then write a function to load
command number $\it{command}$ (here numbered from 0 to 11) at address $\it{dst}$
(the right column shows source code; in particular, all numbers are in decimal
form):

\begin{Paragraph}
\begin{paracol}{2}
\rs{b.func("ced_load", &["command", "dst"], "", &[])}

Initialize $\it{dst}$ to an empty buffer.

\bytecode[switchcolumn]{
  // *dst = 0;
  b.new_line();
  b.get("dst");
  b.cst_0();
  b.store();
}

Compute the $\it{src}$ address of $\it{command}$. This is
$\rs{hex(COMMAND_SOURCE)}+256*\it{command}$.

\bytecode[switchcolumn]{
  // let src = (0xD0000 as *u32) + (command << 8);
  b.new_line();
  b.cst(COMMAND_SOURCE);
  b.get("command");
  b.cst8(8);
  b.lsl();
  b.add();
  b.def("src");
}

If the $\it{src}$ buffer size is greater than 252 this means that no command
has ever been stored here (each flash memory bit is initialized to 1). Then
return directly.

\bytecode[switchcolumn]{
  // if *src <= 252 { buffer_copy(src, dst); }
  b.new_line();
  b.get("src");
  b.load();
  b.cst8(252);
  b.ifgt("ced_load_end");
}

Otherwise copy the $\it{src}$ buffer to $\it{dst}$ and return.

\bytecode[switchcolumn]{
  b.get("src");
  b.get("dst");
  b.call("buffer_copy");
  b.new_line();
  b.label("ced_load_end");
  b.ret();
}
\end{paracol}
\end{Paragraph}

We continue with a function to display the command at $\it{src}$. For this we
simply reuse the {\tt ted\_draw} function of the text editor:

\begin{Paragraph}
\begin{paracol}{2}
\rs{b.func("ced_draw", &["src"], "", &[])}

Set the color to yellow, to make it easier to distinguish the command editor
and the text editor (which draws text in green).

\bytecode[switchcolumn]{
  // gpu_set_color(7, 7, 0);
  b.new_line();
  b.cst8(7);
  b.cst8(7);
  b.cst_0();
  b.call("gpu_set_color");
}

Compute the $\it{begin}$ address of the text, which is 4 bytes after $\it{src}$.

\bytecode[switchcolumn]{
  // let begin = src + 4;
  b.new_line();
  b.get("src");
  b.cst8(4);
  b.add();
  b.def("begin");
}

Compute the $\it{end}$ address of the text, which is $n$ bytes after
$\it{begin}$ (where $n$, the $\it{src}$ buffer size, is the value at address
$\it{src}$).

\bytecode[switchcolumn]{
  // let end = begin + *src;
  b.new_line();
  b.get("begin");
  b.get("src");
  b.load();
  b.add();
  b.def("end");
}

Draw the text with a zero $\it{gap}$ and a $\it{cursor}$ at the end (see
\cref{chapter:text-editor}).

\bytecode[switchcolumn]{
  // ted_draw(begin, end, 0, end);
  b.new_line();
  b.get("begin");
  b.get("end");
  b.cst_0();
  b.get("end");
  b.call("ted_draw");
  b.new_line();
  b.ret();
}
\end{paracol}
\end{Paragraph}

The next function compiles the source code at $\it{src}$, writes the compiled
code at $\it{dst}$, and runs it. It then displays the result and waits until
Enter is pressed.

\begin{Paragraph}
\begin{paracol}{2}
\rs{b.func("ced_run", &["src", "dst"], "", &[])}

Compile the code. The result, noted $\it{error}$, is pushed in the $6^{th}$
stack frame slot.

\bytecode[switchcolumn]{
  // let error = tc_main(src, dst);
  b.new_line();
  b.get("src");
  b.get("dst");
  b.call("tc_main");
  b.def("error");
}

If the compilation is successful (\ie, if $error=0$), run the compiled code
(which starts after the 4 bytes $\it{dst}$ header) and store its result in
$\it{error}$. Otherwise skip this step.

\bytecode[switchcolumn]{
  // if error == 0 { error = "calld" (dst + 4); }
  b.new_line();
  b.get("error");
  b.cst_0();
  b.ifne("ced_run_end");
  b.get("dst");
  b.cst8(4);
  b.add();
  b.calld();
  b.set("error");
}

Clear the screen, set the cursor to the top-left corner, draw $\it{error}$ in
hexadecimal, and wait until Enter is pressed.

\bytecode[switchcolumn]{
  // gpu_clear_screen();
  b.new_line();
  b.label("ced_run_end");
  b.call("gpu_clear_screen");
  // gpu_set_cursor(0, 0);
  b.new_line();
  b.cst_0();
  b.cst_0();
  b.call("gpu_set_cursor");
  // gpu_draw_hex_word(error);
  b.new_line();
  b.get("error");
  b.call("gpu_draw_hex_word");
  // while keyboard_get_char() != 10 {}
  b.new_line();
  b.label("ced_run_wait_enter");
  b.call("keyboard_get_char");
  b.cst8(10);
  b.ifne("ced_run_wait_enter");
  b.new_line();
  b.ret();
}
\end{paracol}
\end{Paragraph}

We can finally write the main command editor function. This function loops
until Escape is pressed, and performs the appropriate action for any other
typed key. It loads the current command in the 256 bytes region starting at
address \rs{hex(RAM_COMMAND_SOURCE)}, and compiles and runs it the next 256
bytes.

\begin{Paragraph}
\begin{paracol}{2}
\rs{b.func("command_editor", &[], "", &[])}

Initialize $\it{src}$ to \rs{hex(RAM_COMMAND_SOURCE)}.

\bytecode[switchcolumn]{
  // let src = 0x20070000 as *u32;
  b.new_line();
  b.cst(RAM_COMMAND_SOURCE);
  b.def("src");
}

Initialize $\it{command}$ to 0.

\bytecode[switchcolumn]{
  // let command = 0;
  b.new_line();
  b.cst_0();
  b.def("command");
}

Initialize $c$ to ``F1'' (see \cref{table:code_tables_choices}).

\bytecode[switchcolumn]{
  // let c = 0x80;
  b.new_line();
  b.cst8(0x80);
  b.def("c");
}

Step 1. If $c$ is not the Escape key go to step 2. Otherwise return.

\bytecode[switchcolumn]{
  // loop
  b.new_line();
  b.label("ced_loop");
  //  if c == 0x1B { return; }
  b.get("c");
  b.cst8(0x1B);
  b.ifne("ced_not_escape");
  b.ret();
}

Step 2. If $c$ is not between ``F1'' and ``F12'' included go to step 3.

\bytecode[switchcolumn]{
  //  if c >= 0x80 && c <= 0x8B
  b.new_line();
  b.label("ced_not_escape");
  b.get("c");
  b.cst8(0x80);
  b.iflt("ced_not_load");
  b.get("c");
  b.cst8(0x8B);
  b.ifgt("ced_not_load");
}

Otherwise set $\it{command}$ to $c-$``F1''.

\bytecode[switchcolumn]{
  // command = c - 0x80;
  b.new_line();
  b.get("c");
  b.cst8(128);
  b.sub();
  b.set("command");
}

Then load this new command and go to step 6 to display it.

\bytecode[switchcolumn]{
  // ced_load(command, src);
  b.new_line();
  b.get("command");
  b.get("src");
  b.call("ced_load");
  b.goto("ced_redraw");
}

Step 3. If $c$ is not equal to ``e'' go to step 4.

\bytecode[switchcolumn]{
  //  else if c == 'e' { text_editor(src, 0, 252); }
  b.new_line();
  b.label("ced_not_load");
  b.get("c");
  b.cst8(b'e');
  b.ifne("ced_not_edit");
}

Otherwise call the text editor to edit the current command (with a maximum text
length of 252 bytes). Then go to step 6 to display it.

\bytecode[switchcolumn]{
  b.get("src");
  b.cst_0();
  b.cst8(252);
  b.call("text_editor");
  b.goto("ced_redraw");
}

Step 4. If $c$ is not equal to ``e'' go to step 5.

\bytecode[switchcolumn]{
  //  else if c == 's' { buffer_flash(src, 256 + command); }
  b.new_line();
  b.label("ced_not_edit");
  b.get("c");
  b.cst8(b's');
  b.ifne("ced_not_save");
}

Otherwise save the current command at address
$\rs{hex(COMMAND_SOURCE)}+256*\it{command}$, which corresponds to page
$\rs{dec(page_number(COMMAND_SOURCE))}+\it{command}$.

\bytecode[switchcolumn]{
  b.get("src");
  b.cst(page_number(COMMAND_SOURCE));
  b.get("command");
  b.add();
  b.call("buffer_flash");
  b.goto("ced_redraw");
}

Step 5. If $c$ is not equal to ``r'' go to step 6.

\bytecode[switchcolumn]{
  //  else if c == 'r' { ced_run(src, src + 256); }
  b.new_line();
  b.label("ced_not_save");
  b.get("c");
  b.cst8(b'r');
  b.ifne("ced_redraw");
}

Otherwise compile and run the current command. The compiled code is written at
$\it{dst}=\it{src}+256$. Then continue to step 6.

\bytecode[switchcolumn]{
  b.get("src");
  b.get("src");
  b.cst(RAM_COMMAND_CODE - RAM_COMMAND_SOURCE);
  b.add();
  b.call("ced_run");
}

Step 6. Draw the current command, wait for a key to be pressed, store it in
$c$, and go back to step 1 to handle it.

\bytecode[switchcolumn]{
  b.label("ced_redraw");
  //  ced_draw(src);
  b.get("src");
  b.call("ced_draw");
  //  c = keyboard_wait_char();
  b.new_line();
  b.call("keyboard_wait_char");
  b.set("c");
  b.goto("ced_loop");
}
\end{paracol}
\end{Paragraph}

The command editor implementation is now complete, and is summarized below:

\rs{code(&b.get_toy0_source_code())}

\rust{
  let command_editor_source_code = b.get_toy0_source_code();
  assert!(command_editor_source_code.len() + 4 <
      MAX_COMMAND_EDITOR_SOURCE_SIZE as usize);

  let mut command_editor_labels = HashMap::new();
  command_editor_labels.insert(
      String::from("ram_command_source"),
      Label {
          offset: RAM_COMMAND_SOURCE - command_editor_source - 4,
          description: String::from("command source code in RAM"),
      },
  );
  command_editor_labels.insert(
      String::from("ram_compiler_source"),
      Label {
          offset: RAM_COMPILER_SOURCE - command_editor_source - 4,
          description: String::from("compiler source code in RAM"),
      },
  );
  command_editor_labels.insert(
      String::from("ram_compiler_code"),
      Label {
          offset: RAM_COMPILER_CODE - command_editor_source - 4,
          description: String::from("compiler code in RAM"),
      },
  );
  context.add_memory_region(
      "command_editor_commands",
      MemoryRegion::new(
          RegionKind::DataBuffer,
          COMMAND_SOURCE,
          NUM_COMMAND_EDITOR_COMMANDS * MAX_COMMAND_SOURCE_SIZE,
          &HashMap::new(),
          0,
          0,
          0,
          Vec::default(),
      ),
  );
  context.add_memory_region(
      "command_editor_source",
      MemoryRegion::new(
          RegionKind::DataBuffer,
          command_editor_source,
          command_editor_source_code.len() as u32 + 4,
          &command_editor_labels,
          0,
          0,
          0,
          Vec::default(),
      ),
  );

  context.add_memory_region("command_editor", b.memory_region());
}

\subsection{Compilation}

We now need to type this source code with the text editor, save it, compile it,
and store the compiled code. These 4 steps are explained below.

\subsubsection{Edit}

\rust{
  const MAX_SOURCE_LENGTH: u32 = 0x1000;
  const SOURCE_ADDRESS: u32 = 0x20070000;
  const CODE_ADDRESS: u32 = SOURCE_ADDRESS + MAX_SOURCE_LENGTH;

  const MAX_COMMAND_SIZE: u32 = 0x20;
  const EDIT_COMMAND_ADDRESS: u32 = 0x20080000;

  // Temporary program to run the text editor on SOURCE_ADDRESS.
  let mut c = BytecodeAssembler::default();
  c.import_labels(context.memory_region("text_editor"));
  c.func("temp", &[], "", &[]);
  c.cst(SOURCE_ADDRESS);
  c.cst_0();
  c.cst(MAX_SOURCE_LENGTH);
  c.call("text_editor");
  c.ret();
  assert!(c.bytecode_size() < MAX_COMMAND_SIZE);
}

Typing the source code requires launching the text editor first. For this, in
the memory editor, type ``w\rs{hex_word_low(EDIT_COMMAND_ADDRESS)}''+Enter,
followed by the code below (see \cref{section:text-editor-experiments}):

\rs{c.get_bytecode_listing(0..c.get_instruction_count() as usize, false)}

Then initialize an empty text buffer by typing
``w\rs{hex_word_low(SOURCE_ADDRESS)}''+Enter, followed by ``00000000''+Enter.
Run the text editor on this empty buffer by typing
``\rs{hex_word_low(EDIT_COMMAND_ADDRESS)}''+Enter, followed by ``r''. Finally,
type the command editor source code listed above, followed by Escape to return
in the memory editor.

Alternatively, if you don't want to type this source code, you can ``cheat'' by
saving it via an external computer, as follows. First run the
\verb!boot_mode_select_rom! function by typing
``w\rs{hex_word_low(boot_mode_address)}''+Enter, followed by ``r''. Then reset
the Arduino and, on the host computer, run the following command to flash the
source code and reset the Arduino again (you can then skip the ``Save'' step
below):

\rust{
  // Test the above instructions with a short text.
  context.type_ascii(&c.memory_editor_commands(EDIT_COMMAND_ADDRESS));
  // initialize empty buffer
  context.type_ascii(&format!("W{:08X}\n", SOURCE_ADDRESS));
  context.type_ascii("00000000\n");
  // run text editor
  context.type_ascii(&format!("W{:08X}\n", EDIT_COMMAND_ADDRESS));
  context.type_ascii("R");
  context.type_ascii("HELLO");
  assert_eq!(display.borrow().get_text(), "hello");
  context.type_keys(vec!["Escape"]);
  context.type_ascii(&format!("W{:08X}\n", SOURCE_ADDRESS));
  assert!(display.borrow().get_text().lines().next().unwrap()
      .ends_with("6F 6C6C6568 00000005 20070000"));

  // Then store the source code in RAM as if edited with the text editor.
  context.store_text(SOURCE_ADDRESS, command_editor_source_code.as_str());

  write_lines("website/part3", "command_editor.txt",
      &flash_helper_commands(command_editor_source_code.as_str(),
      command_editor_source))?;

  context1.run_until_get_char();
  context1.type_ascii(&format!("W{:08X}\n", boot_mode_address));
  context1.type_ascii("R");
  context1.micro_controller().borrow_mut().reset();
  flash_helper1 = FlashHelper::from_file(
      context1.micro_controller(), "website/", "part3/command_editor.txt")?;
}
\rs{host_log(flash_helper1.read().lines().next().unwrap())}

\subsubsection{Save}

\rust{
  const SAVE_SOURCE_COMMAND_ADDRESS: u32 =
      EDIT_COMMAND_ADDRESS + MAX_COMMAND_SIZE;

  // Temporary program to flash source code of command editor.
  let mut c = BytecodeAssembler::default();
  c.import_labels(context.memory_region("flash_driver"));
}

Before compiling this code we want to save it, in case something goes wrong. We
can save it after the 12 pages reserved for the commands, at address
\rs{hex(command_editor_source)}, which corresponds to page
\rs{dec_hex(page_number(command_editor_source))}. This can be done with the
following function:

\begin{TwoColumns}
\rs{c.func("save_source", &[], "", &["nolink"])}\\
\bytecode{
  c.cst(SOURCE_ADDRESS);
  c.cst(page_number(command_editor_source));
  c.call("buffer_flash");
  c.ret();
}
\end{TwoColumns}

Enter it in RAM after the ``edit'' function, at address
\rs{hex(SAVE_SOURCE_COMMAND_ADDRESS)}, by typing
``w\rs{hex_word_low(SAVE_SOURCE_COMMAND_ADDRESS)}''+Enter, followed by the full
code of this function, listed below. Then run it by typing
``w\rs{hex_word_low(SAVE_SOURCE_COMMAND_ADDRESS)}''+Enter, followed by ``r''.

\rs{c.get_bytecode_listing(0..c.get_instruction_count() as usize, false)}

\rust{
  assert!(c.bytecode_size() < MAX_COMMAND_SIZE);
  context.type_ascii(&c.memory_editor_commands(SAVE_SOURCE_COMMAND_ADDRESS));
  context.type_ascii(&format!("W{:08X}\n", SAVE_SOURCE_COMMAND_ADDRESS));
  context.type_ascii("R");

  // Check that the "cheat" above stores the correct values in flash.
  context.check_equal_buffer(&mut context1, command_editor_source);
}

\subsubsection{Compile}

\rust{
  const COMPILE_COMMAND_ADDRESS: u32 =
      SAVE_SOURCE_COMMAND_ADDRESS + MAX_COMMAND_SIZE;
  const COMPILE_RESULT_ADDRESS: u32 =
      COMPILE_COMMAND_ADDRESS + MAX_COMMAND_SIZE;

  // Temporary program to compile the source code.
  let mut c = BytecodeAssembler::default();
  c.import_labels(context.memory_region("toyc0"));
}

Compiling the code can be done with the following function, which writes
the compiled code at address \rs{hex(CODE_ADDRESS)} and the compiler's result
value at address \rs{hex(COMPILE_RESULT_ADDRESS)}:

\begin{TwoColumns}
\rs{c.func("compile_source", &[], "", &["nolink"])}\\
\bytecode{
  c.cst(COMPILE_RESULT_ADDRESS);
  c.cst(command_editor_source);
  c.cst(CODE_ADDRESS);
  c.call("tc_main");
  c.store();
  c.ret();
}
\end{TwoColumns}

Enter it in RAM after the ``save'' function, at address
\rs{hex(COMPILE_COMMAND_ADDRESS)}, by typing
``w\rs{hex_word_low(COMPILE_COMMAND_ADDRESS)}''+Enter, followed by the full
code of this function, listed below. Then run it by typing
``w\rs{hex_word_low(COMPILE_COMMAND_ADDRESS)}''+Enter, followed by ``r''.

\rs{c.get_bytecode_listing(0..c.get_instruction_count() as usize, false)}

\rust{
  assert!(c.bytecode_size() < MAX_COMMAND_SIZE);
  context.type_ascii(&c.memory_editor_commands(COMPILE_COMMAND_ADDRESS));
  context.type_ascii(&format!("W{:08X}\n", COMPILE_COMMAND_ADDRESS));
  context.type_ascii("R");
  assert!(display.borrow().get_text().lines().nth(6).unwrap()
      .ends_with("00000000 20080060"));

  // Check that the compiled code is equal to b's code.
  let words = b.bytecode_words();
  for i in 0..b.bytecode_size() / 4 {
    let word = context.micro_controller().borrow_mut().debug_get32(
        CODE_ADDRESS + 4 * (i + 1));
    assert_eq!(word, words[i as usize]);
  }
}

If all goes well the value at address \rs{hex(COMPILE_RESULT_ADDRESS)} should
be 0, because the compiler returns 0 if and only if the compilation is
successful. If this is not the case, run the ``edit'' function again, double
check the source code and fix any error found (you can also get the location of
the error at address \rs{hex(CODE_ADDRESS)}). Then save and compile the code
again. And repeat this until success.

\subsubsection{Store}

\rust{
  const SAVE_CODE_COMMAND_ADDRESS: u32 =
      COMPILE_RESULT_ADDRESS + MAX_COMMAND_SIZE;

  // - command to flash compiled command editor code
  let mut c = BytecodeAssembler::default();
  c.import_labels(context.memory_region("flash_driver"));
}

Once the compilation is successful, the compiled code can be stored in flash
memory. The following function stores it in the next page after the compiler
itself, \ie, at address \rs{hex(command_editor_code)}, which corresponds to
page \rs{dec_hex(page_number(command_editor_code))}:

\begin{TwoColumns}
\rs{c.func("save_code", &[], "", &["nolink"])}\\
\bytecode{
  c.cst(CODE_ADDRESS);
  c.cst8(page_number(command_editor_code).try_into().unwrap());
  c.call("buffer_flash");
  c.ret();
}
\end{TwoColumns}

Enter it in RAM after the ``compile'' function, at address
\rs{hex(SAVE_CODE_COMMAND_ADDRESS)}, by typing
``w\rs{hex_word_low(SAVE_CODE_COMMAND_ADDRESS)}''+Enter, followed by the full
code of this function, listed below. Then run it by typing
``w\rs{hex_word_low(SAVE_CODE_COMMAND_ADDRESS)}''+Enter, followed by ``r''.

\rs{c.get_bytecode_listing(0..c.get_instruction_count() as usize, false)}

\rust{
  context.type_ascii(&c.memory_editor_commands(SAVE_CODE_COMMAND_ADDRESS));
  context.type_ascii(&format!("W{:08X}\n", SAVE_CODE_COMMAND_ADDRESS));
  context.type_ascii("R");
}

\subsection{First commands}\label{subsection:first-commands}

\rust{
  let command_editor_main = context.memory_region("command_editor")
      .label_address("command_editor");
}

We can now try our command editor. Start it by typing
``w\rs{hex_word_low(command_editor_main)}''+Enter, followed by ``r'' (its main
function is at address \hexa{C0000}+%
\rs{dec(command_editor_main - 0xC0000)}=\rs{hex(command_editor_main)} -- see
\cref{subsection:command-editor-implementation}). The screen should now be
empty, because it displays command number 1, initially empty. Lets use this
command to show a welcome message when the command editor starts. Type ``e'' to
edit it, then type ``Welcome to the command editor." followed by Escape. At
this point the message you typed should be displayed in yellow. For now it is
only in RAM. Type ``s'' to save it in flash memory. We now want to define some
commands to create, load, edit, save, and compile a program.

\rust{
  context.type_ascii(&format!("W{:08X}\n", command_editor_main));
  context.type_ascii("R");
  assert!(display.borrow().get_text().is_empty());

  context.type_ascii("E");
  context.type_keys(vec!["Shift", "W", "~Shift"]);
  context.type_ascii("ELCOME TO THE COMMAND EDITOR.");
  context.type_keys(vec!["Escape"]);
  context.type_keys(vec!["S"]);
}
\bigskip

\paragraph*{New (F2)} initializes an empty text buffer at address
\rs{dec_hex(RAM_COMPILER_SOURCE)}, just after the memory region used by the
command editor (see \cref{fig:command-editor-memory-map}). Type ``F2'' followed
by ``e'' to edit it, then type its source code followed by Escape and ``s''
(the dummy data at the end describes the command):

\rust{
  context.type_keys(vec!["F2"]);
  let mut c = BytecodeAssembler::default();
  c.func("new_source_code", &[], "", &[]);
  c.new_line();
  c.cst(RAM_COMPILER_SOURCE);
  c.cst_0();
  c.store();
  c.new_line();
  c.cst_0();
  c.retv();
  let c_source = format!("{}\nd NEW_SOURCE_CODE", c.get_toy0_source_code());
  context.store_text(RAM_COMMAND_SOURCE, &c_source);
  context.type_keys(vec!["S"]);
}
\rs{code(&c_source)}

\paragraph*{Load (F3)} calls \hyperlink{buffer-copy}{buffer\_copy} to load a
program stored in flash memory at address \rs{dec_hex(compiler_source)}, just
after the command editor source code (see
\cref{fig:command-editor-memory-map}). Store its source code in command number
3:

\rust{
  context.type_keys(vec!["F3"]);
  let mut c = BytecodeAssembler::default();
  c.import_labels(context.memory_region("flash_driver"));
  c.func("load_source_code", &[], "", &[]);
  c.new_line();
  c.cst(compiler_source);
  c.cst(RAM_COMPILER_SOURCE);
  c.call("buffer_copy");
  c.new_line();
  c.cst_0();
  c.retv();
  let c_source = format!("{}\nd LOAD_SOURCE_CODE", c.get_toy0_source_code());
  context.store_text(RAM_COMMAND_SOURCE, &c_source);
  context.type_keys(vec!["S"]);
}
\rs{code(&c_source)}

\paragraph*{Edit (F4)} calls \hyperlink{text-editor}{text\_editor} to edit the
text buffer at address \rs{hex(RAM_COMPILER_SOURCE)}, with the word at address
\rs{dec_hex(RAM_COMPILER_CODE)} as initial offset, and a maximum length of
\rs{dec(MAX_COMPILER_SOURCE_KB)}~KB (including the 4 bytes header). The initial
offset corresponds to the header of a compiled code buffer (see
\cref{fig:command-editor-memory-map}) which, in case of a compilation error,
contains the error location. Hence, editing the source code after a compilation
error opens the text editor at the location of this error. Store the following
source code in command number 4:

\rust{
  context.type_keys(vec!["F4"]);
  let mut c = BytecodeAssembler::default();
  c.import_labels(context.memory_region("text_editor"));
  c.func("edit_source_code", &[], "", &[]);
  c.new_line();
  c.cst(RAM_COMPILER_SOURCE);
  c.cst(RAM_COMPILER_CODE); // 1st word contains offset of error if applicable
  c.load();
  c.cst(MAX_COMPILER_SOURCE_SIZE - 4);
  c.call("text_editor");
  c.new_line();
  c.cst_0();
  c.retv();
  let c_source = format!("{}\nd EDIT_SOURCE_CODE", c.get_toy0_source_code());
  context.store_text(RAM_COMMAND_SOURCE, &c_source);
  context.type_keys(vec!["S"]);
}
\rs{code(&c_source)}

\paragraph*{Save (F5)} calls \hyperlink{buffer-flash}{buffer\_flash} to save
the edited program in flash memory at address \rs{hex(compiler_source)}, which
corresponds to page \rs{dec(compiler_source_page)}. Store it in command number
5:

\rust{
  context.type_keys(vec!["F5"]);
  let mut c = BytecodeAssembler::default();
  c.import_labels(context.memory_region("flash_driver"));
  c.func("save_source_code", &[], "", &[]);
  c.new_line();
  c.cst(RAM_COMPILER_SOURCE);
  c.cst(compiler_source_page);
  c.call("buffer_flash");
  c.new_line();
  c.cst_0();
  c.retv();
  let c_source = format!("{}\nd SAVE_SOURCE_CODE", c.get_toy0_source_code());
  context.store_text(RAM_COMMAND_SOURCE, &c_source);
  context.type_keys(vec!["S"]);
}
\rs{code(&c_source)}

\paragraph*{Compile (F6)} calls \hyperlink{tc-main}{tc\_main} to compile the
source code at address \rs{hex(compiler_source)}, and to write the compiled
code at address \rs{dec_hex(RAM_COMPILER_CODE)} (just after the source code in
RAM, see \cref{fig:command-editor-memory-map}). It returns the compiler's
result, which is non-zero if a compilation error occurs. Store it in command
number 6:

\rust{
  context.type_keys(vec!["F6"]);
  let mut c = BytecodeAssembler::default();
  c.import_labels(context.memory_region("toyc0"));
  c.func("compile_source_code", &[], "", &[]);
  c.new_line();
  c.cst(compiler_source); // src_buffer
  c.cst(RAM_COMPILER_CODE); // dst_buffer
  c.call("tc_main");
  c.retv();
  let c_source = format!("{}\nd COMPILE_SOURCE_CODE",
  c.get_toy0_source_code());
  context.store_text(RAM_COMMAND_SOURCE, &c_source);
  context.type_keys(vec!["S"]);
}
\rs{code(&c_source)}

\paragraph*{Store (F7)} calls \hyperlink{buffer-flash}{buffer\_flash} to store
the compiled code in flash memory at address \rs{dec_hex(compiler_code)}, which
corresponds to page \rs{dec(compiler_code_page)} (after the command editor --
see \cref{fig:command-editor-memory-map}). Before that, this command backs up
the current compiled code by saving a copy of it at address
\rs{dec_hex(compiler_code_backup)}, which corresponds to page
\rs{dec(compiler_code_backup_page)}. Store it in command number 7:

\rust{
  context.type_keys(vec!["F7"]);
  let mut c = BytecodeAssembler::default();
  c.import_labels(context.memory_region("flash_driver"));
  c.func("save_compiled_code", &[], "", &[]);
  c.new_line();
  c.cst(compiler_code);
  c.cst(compiler_code_backup_page);
  c.call("buffer_flash"); // backup current compiled code
  c.new_line();
  c.cst(RAM_COMPILER_CODE);
  c.cst(compiler_code_page);
  c.call("buffer_flash"); // saved compiled code in RAM into flash
  c.new_line();
  c.cst_0();
  c.retv();
  let c_source = format!("{}\nd STORE_COMPILED_CODE", c.get_toy0_source_code());
  context.store_text(RAM_COMMAND_SOURCE, &c_source);
  context.type_keys(vec!["S"]);
}
\rs{code(&c_source)}

\paragraph*{Restore (F8)} calls \hyperlink{buffer-flash}{buffer\_flash} to
restore the backup created by the previous command, in case something goes
wrong. Store it in command number 8:

\rust{
  context.type_keys(vec!["F8"]);
  let mut c = BytecodeAssembler::default();
  c.import_labels(context.memory_region("flash_driver"));
  c.func("restore_backup_code", &[], "", &[]);
  c.new_line();
  c.cst(compiler_code_backup);
  c.cst(compiler_code_page);
  c.call("buffer_flash"); // backup current compiled code
  c.new_line();
  c.cst_0();
  c.retv();
  let c_source = format!("{}\nd RESTORE_BACKUP_CODE",
  c.get_toy0_source_code());
  context.store_text(RAM_COMMAND_SOURCE, &c_source);
  context.type_keys(vec!["S"]);
}
\rs{code(&c_source)}

\begin{Figure}
  \rs{define("mmapa", &hex(context.memory_region("toyc0").start))}
  \rs{define("mmapb", &hex(command_editor_code))}
  \rs{define("mmapc", &hex(compiler_code))}
  \rs{define("mmapd", &hex(COMMAND_SOURCE))}
  \rs{define("mmape", &hex(command_editor_source))}
  \rs{define("mmapf", &hex(compiler_source))}
  \rs{define("mmapg", &hex(compiler_code_backup))}
  \rs{define("mmaph", &hex(RAM_COMMAND_SOURCE))}
  \rs{define("mmapi", &hex(RAM_COMMAND_CODE))}
  \rs{define("mmapj", &hex(RAM_COMPILER_SOURCE))}
  \rs{define("mmapk", &hex(RAM_COMPILER_CODE))}
  \input{figures/chapter3/command-editor-memory-map.tex}

  \caption{The flash memory and RAM regions used by the command editor, and by
  the commands defined in \cref{subsection:first-commands}. White, blue and
  gray areas represent source code, bytecode and unused memory, respectively
  (not to scale).}\label{fig:command-editor-memory-map}
\end{Figure}

\subsection{Tests}

In order to test the above commands, type ``F2''+``r'' to create a new program,
and press Enter to return in the command editor. Then type ``F4''+``r'' to
edit this program, and type the following code, which contains an error on
purpose:

\rust{
  context.type_keys(vec!["F2"]);
  context.type_ascii("R");
  assert_eq!(display.borrow().get_text(), "00000000");
  context.type_ascii("\n");

  let c_source = "fn 0 cst_3 retv";

  context.type_keys(vec!["F4"]);
  context.type_ascii("R");
  context.type_ascii("FN 0 CST");
  context.type_keys(vec!["Shift", "-", "~Shift"]);
  context.type_ascii("3 RETV");
  assert_eq!(display.borrow().get_text(), c_source);
  context.type_keys(vec!["Escape"]);
  assert_eq!(display.borrow().get_text(), "00000000");
  context.type_ascii("\n");
}
\rs{code(c_source)}

\noindent Then type Escape to exit the text editor. The command's result, 0,
should be displayed. Press Enter to return in the command editor's main loop
(in the following we omit these ``press Enter'' instructions, for brevity).

Type ``F5''+``r'' to save this program, ``F2''+``r'' to create a new one, and
``F4''+``r'' to edit it. The screen should be empty. Type Escape to return
in the memory editor, then type ``F3''+``r'' to load the previously saved
program. Type ``F4''+``r'' to check that it is now loaded in RAM, and
Escape to return in the command editor.

\rust{
  context.type_keys(vec!["F5"]);
  context.type_ascii("R\n");
  context.type_keys(vec!["F2"]);
  context.type_ascii("R\n");
  context.type_keys(vec!["F4"]);
  context.type_ascii("R");
  assert!(display.borrow().get_text().is_empty());
  context.type_keys(vec!["Escape"]);
  context.type_ascii("\n");
  context.type_keys(vec!["F3"]);
  context.type_ascii("R\n");
  context.type_keys(vec!["F4"]);
  context.type_ascii("R");
  assert_eq!(display.borrow().get_text(), c_source);
  context.type_keys(vec!["Escape"]);
  context.type_ascii("\n");
}

Type ``F6''+``r'' to compile this program. The result should be 1, because {\tt
cst\_3} is an invalid opcode. Then type ``F4''+``r'' to edit the program. The
cursor should be just after this invalid opcode. Enter the correct code below:

\rust{
  context.type_keys(vec!["F6"]);
  context.type_ascii("R");
  assert_eq!(display.borrow().get_text(), "00000001");
  context.type_ascii("\n");

  context.type_keys(vec!["F4"]);
  context.type_ascii("R");
  let c_source = "fn 0 cst8 3 retv";
  context.type_keys(vec!["Backspace", "Backspace"]);
  context.type_ascii("8 3");
  assert_eq!(display.borrow().get_text(), c_source);
}
\rs{code(c_source)}

Finally, type Escape to exit the text editor, ``F5''+``r'' to save the
corrected code, and ``F6''+``r'' to compile it. The result should be 0 this
time.

\rust{
  context.type_keys(vec!["Escape"]);
  context.type_ascii("\n");
  context.type_keys(vec!["F5"]);
  context.type_ascii("R\n");
  context.type_keys(vec!["F6"]);
  context.type_ascii("R");
  assert_eq!(display.borrow().get_text(), "00000000");
  context.type_ascii("\n");
}

% This work is licensed under the Creative Commons Attribution NonCommercial
% ShareAlike 4.0 International License. To view a copy of the license, visit
% https://creativecommons.org/licenses/by-nc-sa/4.0/

\renewcommand{\rustfile}{chapter4}
\setcounter{rustid}{0}

\chapter[Control Circuits]{Control Circuits}\label{chapter:control-circuits}

Thanks to registers and memory circuits we can use an Arithmetic and Logic Unit
to perform computations without having to mentally memorize intermediate
results. Instead, as shown in the previous chapter, we can simply send a series
of pulse signals on the correct inputs, and in the correct order. But this
requires to memorize this procedure. And executing it manually is very slow and
error-prone, even if the circuit does each operation very quickly. To solve the
first issue a solution is to store some description of the desired procedure in
Random Access Memory. To solve the second one we can use new circuits to
execute this procedure for us, by sending the appropriate pulses. This chapter
explains how this can be done.

\section{Instructions}

A procedure such as the one presented in \cref{subsection:alu-and-ram-example}
could be described in an abstract way as ``read 3 numbers $a$, $b$, and $c$ in
input, compute $a+b-c$, and write the result in RAM at address $x$''. However,
representing such descriptions with one or more numbers which can be stored in
RAM is not easy. And figuring out which pulses to send to execute them would
also be quite complicated.

This procedure can also be described as a sequence of elementary actions:
``wait an input value'', ``send a pulse on selectInput'', ''send a pulse on
writeR0'', ``wait an input value'', ''send a pulse on writeR1'', etc. Each
action can easily be represented with a small number (\eg, 0 for ``wait an
input value'', 1 for ``send a pulse on selectInput'', etc). And each action is
easy to execute. However, such a description is hard to design and to
understand for humans (because its high level meaning is lost in the details).

A trade-off is to describe this procedure with more abstract actions, but not
too abstract either, called {\em instructions}. For instance, an instruction
could be ``wait an input value and store it in R0'', ``add the values in R0 and
R1 and store the result in R0'', or ``copy the value in R0 in RAM, at address
3''. As shown below, such instructions are not too complex to represent with a
number, called their {\em encoding} (to store them in memory). And they are
still quite simple to execute (each instruction only requires sending 2 or 3
pulses at most). Finally, a sequence of instructions is less hard to design and
to understand than the corresponding sequence of pulses (but still quite hard;
we address this problem in \cref{part:compiler}).

Simple procedures, also called {\em programs}, can be described with a sequence
of instructions, to be executed one after the other. For this we can store
their encoding one after the other in memory, \ie, at consecutive addresses.
Then, after the instruction at address $a$ is executed, the one at address
$a+1$ should execute\footnote{Assuming that each encoded instruction can fit in
the $n$ bits between two consecutive addresses.}.

\subsection{Jump instructions}

Some programs need to repeat the same sequence of instructions two or more
times. For instance, a ``calculator'' program needs to repeat forever the same
sequence (read two numbers in input, compute and output their sum, repeat). In
other words, after the last instruction of the sequence is executed, the
instruction at the next address should {\em not} be executed. Instead,
execution should restart at the first instruction of the sequence. This can be
described with a so called {\em jump instruction}. A ``jump to $a$''
instruction specifies that the next instruction to execute is the one at
address $a$.

In many cases a sequence of instructions must be repeated a precise number of
times. For instance, to compute $a*b$ with the circuit of
\cref{fig:alu-and-ram}, we can repeat $b$ times a sequence adding $a$ to R0
(initially set to 0). Then, after $a$ has been added to R0, there are two
cases: either we need to repeat the sequence again, or we need to continue with
the rest of the program (\eg, output the result $a*b$). This can be
described with a {\em conditional jump} instruction. Such an instruction either
jumps to a given address, or continues to the instruction at the next address,
depending on some condition (for instance, whether R0 is equal to 0 or not).

\section{A toy instruction set}\label{section:toy-insn-set}

To illustrate the above discussion we define in this section a concrete set of
instructions for a circuit such as the one in \cref{fig:alu-and-ram} (\ie, with
a RAM and two registers R0 and R1 as input of a very basic Arithmetic Unit).
These instructions are the following:
\begin{itemize}
  \item Memory:

  \begin{itemize}
    \item the Load instruction copies the value at a given address $a$ into the
    R0 register.

    \item the Store instruction copies the value in the R0 register at a
    given address $a$.
  \end{itemize}

  \item Arithmetic:

  \begin{itemize}
    \item the Add instruction adds the value at address $a$ to the value in the
    R0 register, and stores the result in R0.

    \item the Subtract instruction subtracts the value at address $a$ from the
    value in the R0 register, and stores the result in R0.
  \end{itemize}

  \item Jumps:

  \begin{itemize}
    \item the Jump instruction specifies that the next instruction to execute
    is the one at address $a$.

    \item the Jump If Zero instruction specifies that the next instruction to
    execute is the one at address $a$ if the value in R0 is equal to $0$.
    Otherwise execution continues with the instruction at the next address.

    \item the Jump If Carry instruction specifies that the next instruction to
    execute is the one at address $a$ if the last Add or Subtract instruction
    produced a non-zero carry bit. Otherwise execution continues with the
    instruction at the next address.
  \end{itemize}

  \item Input and output:

  \begin{itemize}
    \item the Input instruction waits for the user to press a button, and then
    copies the value on the input wires into the R0 register.

    \item the Output instruction displays the value in the R0 register, and
    then waits until the user presses a button.
  \end{itemize}
\end{itemize}

\subsection{Encoding}

The above {\em instruction set} contains 9 instructions. We can thus give them
numbers from 0 to 8, called {\em operation codes}, or {\em opcodes}. This
requires at least 4 bits to encode each instruction. But all instructions
except the last two have an associated address $a$, called an {\em operand}.
This operand must also be encoded as part of the instruction, which requires
more bits.

In the following we assume that the memory contains $2^5=32$ bytes, each with
their own address, and that R0, R1, and the Arithmetic Unit work on 8-bit
values. We then use 5 bits per address, and we encode each instruction in one
byte, as follows:

\bigskip \noindent
\rs{T8Instruction::Ldr(0).definition()}\\
\rs{T8Instruction::Str(0).definition()}\\
\rs{T8Instruction::Add(0).definition()}\\
\rs{T8Instruction::Sub(0).definition()}\\
\rs{T8Instruction::Jump(0).definition()}\\
\rs{T8Instruction::IfZ(0).definition()}\\
\rs{T8Instruction::IfC(0).definition()}\\
\rs{T8Instruction::In.definition()}\\
\rs{T8Instruction::Out.definition()}\\

The left column is the {\em instruction mnemonic}, an abbreviation of the
instruction name. The middle column is a symbolic description of the effect
each instruction. Here $\mathit{dst} \leftarrow \mathit{src}$ or $\mathit{src}
\rightarrow \mathit{dst}$ means a copy of the value in $\mathit{src}$ into
$\mathit{dst}$, and $\mathrm{mem8}[a]$ means the 8-bit value at address $a$.
Finally, the right column is the binary number corresponding to this
instruction, \ie, its encoding. For instance, the encoding of the LDR 7
instruction, which copies the byte at address $7=111_2$ into R0, is
$001_2$ followed by $7$ encoded in $5$ bits, $00111_2$, which gives
$00100111_2=39$.

\subsection{Example program}\label{subsection:adder-program}

With the above instruction set a ``calculator'' program adding numbers in an
endless loop can be implemented as follows:

\rust{
  let mut a = T8Program::default();
  a.input();
  a.str(6);
  a.input();
  a.add(6);
  a.output();
  a.jump(0);
  a.data(0, "the $a$ number");
}
\rs{a.get_listing()}

\rust{
  let outputs = T8Emulator::new().emulate(&a.get_machine_code(),
      &[7, 13, 17, 19], 2);
  assert_eq!(outputs, &[7 + 13, 17 + 19]);
}

\noindent where the left part gives the symbolic description of each
instruction, and the right part their encoding and their address (in gray).

The first two instructions read a number $a$ as input and store it at address
$6$. The next two instructions read a second number $b$, and add $a$ to it. The
last two instructions output the value in R0, which at this stage contains
$a+b$, and jump back to the first instruction to add two new numbers. The next
byte after these five instructions is the one used to store $a$.

\subsection{Notes}\label{subsection:int-overflow}

Adding two 8-bit numbers can give a 9-bit number. For instance,
$11111111_2=255$ plus $1$ gives the 9-bit number $100000000_2=256$. However,
the registers and the memory can only store 8-bit numbers, and the Arithmetic
Unit can only use 8-bit numbers as input. Hence, in practice, and unless a
program does something special with the carry bit (with the IFC instruction),
all additions are {\em modulo} $2^8=256$. This means that adding $a$ and $b$
does not give $a+b$ but the remainder of the division of $a+b$ by 256. It is
noted $(a+b) \mod 256$, where $x \mod m$ is defined as $x - \lfloor x / m
\rfloor * m$. For instance, adding $255$ and $1$ gives $0$\footnote{This {\em
modular arithmetic} is used in everyday life with hours. For example, 10 a.m
plus 5 hours is 3 p.m because $(10+5) \mod 12 = 3$.}.

Similarly, subtracting two numbers can give a negative result, but the
registers and the memory can only store nonnegative numbers. Hence, in
practice, and unless a program does something special with the carry bit, all
subtractions are modulo $256$ too. For instance, subtracting $1$ from $0$ gives
$255$ because $-1 \mod 256 = -1 - \lfloor -1 / 256 \rfloor * 256 = -1 -
(-1)*256 = 255$ (recall that $\lfloor y \rfloor$ means the integer part of $y$).

When $a+b$ differs from $(a+b) \mod 2^n$ we say that there is an (integer) {\em
overflow} (where $n$ is the Arithmetic Unit's ``bit width''). We say the same
when $a-b \ne (a-b) \mod 2^n$, $a*b \ne (a*b) \mod 2^n$, etc. With an
Arithmetic Unit such as the one in \cref{fig:alu}, there is an overflow if and
only if the carry output is $1$.

\section{Control circuits}

We now have a way to describe a sequence of instructions with some numbers
stored in memory. The next step, as described in the introduction of this
chapter, is to build a circuit to automatically {\em execute} these
instructions. Which means sending a corresponding sequence of pulse signals on
the registers, memory, and bus circuits. For instance, to execute an IN
instruction with the circuit in
\cref{fig:alu-and-ram}, one needs to:
\begin{itemize}
  \item connect the input wires to the bus by sending a pulse on
  ``selectInput''.

  \item wait for the user to press a button.

  \item store the input value in R0 by sending a pulse on ``writeR0''.
\end{itemize}

More generally, all instructions can be executed by sending appropriate pulses
1) on the correct wires, 2) in the correct order, and 3) at appropriate times
(signals must have time to propagate throughout the circuit between two
pulses). The first two items can be ensured with circuits of the following
form:

\begin{center}
  \input{figures/chapter4/sequencer1.tex}
\end{center}

When this circuit is powered on ``wire1'' and ``wire2'' change from 0 to 1 due
to the NOT gate. Changing $c$ from $0$ to $1$ (and back to $0$) makes the first
and second D flip-flops memorize 1. This resets ``wire1'' and ``wire2'' to 0,
and sets ``wire3'' to 1:

\begin{center}
  \input{figures/chapter4/sequencer2.tex}
\end{center}

Changing $c$ from $0$ to $1$ (and back to $0$) again makes the second and third
flip-flops memorize 0 and 1, respectively. This resets ``wire3'' to 0, and sets
``wire4'' to 1:

\begin{center}
  \input{figures/chapter4/sequencer3.tex}
\end{center}

Finally, changing $c$ from $0$ to $1$ (and back to $0$) one more time resets
``wire4'' to 0:

\begin{center}
  \input{figures/chapter4/sequencer4.tex}
\end{center}

In other words, with a series of pulses on the $c$ input, one gets two
simultaneous pulses on ``wire1'' and ``wire2'', followed by a pulse on
``wire3'' and then on ``wire4'':

\begin{center}
  \input{figures/chapter4/sequencer-signals.tex}
\end{center}

Each wire pulse starts and ends at the precise moment when $c$ switches from
$0$ to $1$. This shows that, with circuits like the one above, it is possible
to send pulses on specific wires, in a specific order. The only requirement is
the ability to send a series of pulses on a shared input $c$, which can be done
with a {\em clock}.

\subsection{Clock}

A clock is a circuit which generates a signal switching between $0$ and $1$ at
a constant frequency. A clock can be implemented in many ways. For instance,
one could use a pendulum acting on a switch. But this would not be very
practical, and can not produce high frequencies. Instead, a frequently used
method is to use the oscillations of a crystal. Crystals can oscillate one
million times per second or more (\ie, at $1$~MHz or more). In the following we
represent a clock with the symbol on the left:

\begin{center}
  \input{figures/chapter4/clock-symbol.tex}
  \input{figures/chapter4/clock-signal.tex}
\end{center}

A clock generates the signal shown above (right). A {\em period}, also called a
{\em clock cycle}, is the time between successive pulses. The clock frequency
is the number of pulses per second, \ie, the inverse of its period. Increasing
the frequency increases the number of instructions which are executed per
second. However, the frequency cannot be increased without limit. Indeed, there
must be enough time between two pulses for signals to propagate throughout the
circuit. For instance, computing an addition in an Arithmetic Unit takes some
time, because the input values have to propagate through all its logic gates,
up to the carry output. If a pulse is sent to write the sum in a register
before this delay, a wrong result will be stored.

\subsection{Control loop}

A circuit like the one above can generate a sequence of pulses to execute one
instruction. But each type of instruction needs a different sequence of pulses
to be executed. The solution is to use several circuits like this, one per type
of instruction. And to connect them to a binary decoder, so that the correct
subcircuit is used depending on the instruction opcode. For instance, if there
are only 4 different opcodes, we can use a circuit similar to the following:

\begin{center}
  \input{figures/chapter4/decode-and-execute.tex}
\end{center}

Depending on the two bits of the instruction opcode, $op_1op_0$, the above
circuit sends pulses on ``wire0-1'' to ``wire0-3'', or on ``wire1-1'' to
``wire1-3'', etc. Before this the instruction must be read in memory, so that
$op_1op_0$ contain the correct values. This can be done, as shown later, with a
so called FETCH circuit sending an appropriate sequence of pulses. Finally,
after the instruction has been executed, the next one must be fetched, decoded,
and executed. For this it suffice to connect the outputs of the EXECUTE
subcircuit back to the input of the FETCH circuit:

\begin{center}
  \input{figures/chapter4/fetch-decode-execute.tex}
\end{center}

In this way we get a pulse which loops forever in the FETCH, DECODE and EXECUTE
circuits, each time going through a specific EXECUTE subcircuit.

\section{A toy control unit}\label{section:toy-control-unit}

To illustrate the above discussions we design in this section a very basic {\em
control unit} for the circuit of \cref{fig:alu-and-ram-schema} (with an 8-bit
{\em architecture}, \ie, an 8-bit Arithmetic Unit, 8-bit registers, etc). As
its name implies, a control unit controls the rest of the circuit, called the
{\em processing unit} (\ie, the Arithmetic and Logic Unit, the registers, the
bus, etc). It does so by executing instructions stored in memory. We assume
here that these instructions are those defined in \cref{section:toy-insn-set}.

\begin{Figure}
  \input{figures/chapter4/T8-schema.tex}

  \caption{A basic control unit (yellow background) for the circuit in
  \cref{fig:alu-and-ram-schema} (white background), with the instruction set of
  \cref{section:toy-insn-set}.}\label{fig:T8-schema}
\end{Figure}

The core part of our example control unit is a control loop circuit with FETCH,
DECODE, and EXECUTE subcircuits, as presented above. To implement it we need
two new registers, in addition to R0 and R1 (see \cref{fig:T8-schema}):
\begin{itemize}
  \item the {\em Program Counter} (PC) register stores the address of the
  instruction being currently executed or, once it has been executed, the
  address of the next instruction to execute. Since addresses use only 5 bits,
  this register is a 5-bit register.

  \item the {\em Instruction Register} (IR) stores the encoding of the
  instruction being currently executed. This 8-bit register stores a copy of
  the original instruction in memory. This is necessary to have access to its
  value during its execution, which might require reading or writing values at
  other addresses in memory.
\end{itemize}

Once an instruction has been executed, the Program Counter value must be
incremented by one to execute the next instruction. Unless the last instruction
was a jump. In this case the Program Counter value must be replaced with the
operand of this jump instruction. To do this we include two more circuits in
our control unit (see \cref{fig:T8-schema}):
\begin{itemize}
  \item a 5-bit incrementer, which computes ``PC+1'', \ie, the value in the
  Program Counter register plus $1$. This is an adder circuit similar to the one
  in \cref{subsection:adder-circuit}, simplified for the case where one
  input is always $1$.

  \item a 5-bit {\em address bus}, to which we connect the Program Counter, the
  output of the above incrementer, and the 5 least significant bits of the
  Instruction Register (\ie, the address operand). This bus is also connected
  to the address decoder of the RAM and thus selects which address to read or
  write to.
\end{itemize}

Thanks to these components, we can increment the Program Counter by connecting
the output of the incrementer to the address bus, and by sending a pulse on
``writePC'' to store this value (see \cref{fig:T8-schema}). Likewise, we can
replace the Program Counter with the operand of a jump instruction by
connecting the Instruction Register to the address bus, and by sending a pulse
on ``writePC''.

\subsection{FETCH circuit}

With the above architecture, fetching an instruction can be done as follows:
\begin{itemize}
  \item send simultaneous pulses on ``selectPC'' and ``selectRAM'' to read the
  value in memory at the address stored in the Program Counter, and to get it
  on the data bus.

  \item send a pulse on ``writeIR'' to write this value in the Instruction
  Register.

  \item send a pulse on ``selectIR'' to prepare reading or writing a value at
  the address operand of the new instruction. Technically this step is part of
  the instruction's execution, but we include it in the FETCH circuit to
  avoid duplications (it is common to all instructions except IN and OUT).
\end{itemize}

\subsection{DECODE circuit}

Decoding an instruction can be done with a binary decoder with 3 inputs, namely
the 3 most significant bits of the Instruction Register. Plus a single
demultiplexer, controlled by the $4^{th}$ most significant bit, in order to
distinguish the IN and OUT instructions (the 3 most significant bits are
$111_2$ for both instructions).

\subsection{EXECUTE circuit}

The EXECUTE circuit has 9 subcircuits, one per type of instruction:

\medskip \paragraph{LDR} This subcircuit sends a pulse on ``writeR0'' to store
the value read from memory at the instruction's address operand (selected by
the last step of the FETCH circuit). It then increments the PC value with a
pulse on ``selectPC+1'', followed by one on ``writePC''.

\medskip \paragraph{STR} This subcircuit sends a pulse on ``selectR0'',
followed by a pulse on $w$ to store R0's value in memory, at the instruction's
address operand (selected by the last step of the FETCH circuit). It then
increments the PC value, as above.

\medskip \paragraph{ADD} This subcircuit sends a pulse on ``writeR1'' to store
the value at the instruction's address operand in R1. It then sends a pulse on
``selectALU'' to get the sum of the values in R0 and R1 on the bus, followed by
simultaneous pulses on ``writeR0'' and ``writeCarry'' to write it in R0 and
Carry. It then increments the PC value as above.

\medskip \paragraph{SUB} This subcircuit is almost the same as the ADD
subcircuit. It just sends an additional pulse on ``subtract'', at the same time
as the ``selectALU'' pulse. These pulses last until the one on ``writeR0''
starts. This ensures that the correct result, the difference of R0 and R1
values, is written in R0.

\medskip \paragraph{JMP} This subcircuit just sends a pulse on ``writePC'' to
replace the Program Counter with the instruction's address operand (selected by
the last step of the FETCH circuit).

\medskip \paragraph{IFZ} This subcircuit has two branches. The first, executed
if the value in R0 is $0$, is the same as the JMP subcircuit. The second,
executed if R0's value is not $0$, increments the PC value as for non-jump
instructions. The two branches are connected to a demultiplexer controlled by
the ``$\ne 0$'' signal (see \cref{fig:T8-schema}):

\begin{center}
  \input{figures/chapter4/jump-insn.tex}
\end{center}

\paragraph{IFC} This subcircuit is almost the same as the IFZ one,
except that its demultiplexer is controlled by the value of the Carry register.

\medskip \paragraph{IN} This subcircuit sends a pulse on ``selectInput'', and
then waits until a button is pressed. This can be done with a loop similar to
the control loop, with a demultiplexer to either wait, or to continue with the
next instruction:

\begin{center}
  \input{figures/chapter4/in-insn.tex}
\end{center}

\noindent In the latter case, this subcircuit sends a pulse on ``writeR0'', and
then increments the Program Counter as above.

\medskip \paragraph{OUT} This subcircuit sends a pulse on ``selectR0'' and then
waits until a button is pressed, with the same method as above. It then
increments the Program Counter's value.

% This work is licensed under the Creative Commons Attribution NonCommercial
% ShareAlike 4.0 International License. To view a copy of the license, visit
% https://creativecommons.org/licenses/by-nc-sa/4.0/

\renewcommand{\rustfile}{chapter5}
\setcounter{rustid}{0}

\rust{
  context.write_backup("website/backups", "expressions_compiler.txt")?;
}

\chapter{Expressions Compiler}\label{chapter:expressions-compiler}

The labels compiler written in the previous chapter removes the need to
manually compute function addresses and instruction offsets, which is a huge
improvement. However, it still requires us to use numbers to refer to function
arguments or values on the stack. This is easier to do than to use function
addresses and instruction offsets, but it would be better if we could avoid
this. Another issue is that computing simple expressions such as $2+3*4$
requires a ``lot'' of code, in an unnatural order (\insn{cst8} {\tt 2}
\insn{cst8} {\tt 3} \insn{cst8} {\tt 4} \insn{mul} \insn{add}). It would be
better if we could just type {\tt 2+3*4} instead. This chapter extends our toy
programming language and its compiler in order to solve these issues.

\section{Requirements}

\subsection{Function parameters}\label{subsubsection:toyc5-fn-params}

So far we used comments to give a symbolic name to each function parameter. We
also used these symolic names in comments next to each \insn{get}, \insn{set}
or \insn{ptr} instruction, in order to make them easier to understand. To avoid
using numbers in these instructions, the same solution as in the previous
chapter can be used: we can use symbolic names directly in the source code,
instead of in comments. The compiler can then keep track of the value of these
symbols (\ie, their index in the stack frame), like it does for function names
and labels. To this end, we now require function parameters to be declared
after the function name, between parentheses and separated by commas, as in the
following example:

\insn{fn} {\tt gpu\_set\_color}{\tt (}{\tt red,} {\tt green,} {\tt blue}{\tt )}
$\ldots$

\noindent We could then use these names in instructions such as \insn{get} {\tt
red}, \insn{set} {\tt green} or \insn{ptr} {\tt blue} (instead of writing
\insn{get} {\tt 0}, \insn{set} {\tt 1} or \insn{ptr} {\tt 2}). In fact, to get
shorter programs we simply use ``{\tt red}'' instead of \insn{get} {\tt red}
and ``{\tt \&blue}'' instead of \insn{ptr} {\tt blue} (see below).

\subsection{Expressions}\label{subsubsection:toyc5-exprs}

A well-formed series of arithmetic and logic instructions (see
\cref{subsection:alu-insns}) computes a single value on the stack which can be
written in a shorter mathematical form. For instance, as noted above,
\insn{cst8} {\tt 2} \insn{cst8} {\tt 3} \insn{cst8} {\tt 4} \insn{mul}
\insn{add} computes {\tt 2+3*4}, which is much shorter to write and is called
an {\em expression}. For this reason, we now require our programming language
to support expressions. This means that it should be possible to write {\tt
2+3*4} in a program, for instance, and that the compiler should automatically
compile this into \insn{cst8} {\tt 2} \insn{cst8} {\tt 3} \insn{cst8} {\tt 4}
\insn{mul} \insn{add} (in binary form).

More precisely, our programming language should support the following
expressions (where $e_i$ is an expression and {\tt code[$e_i$]} the
corresponding compiled code, $x$ is a symbolic name corresponding to the
$i^{th}$ stack frame slot, and $f$ is the name of a function whose address
is $a+\hexa{C0000}$):
\begin{itemize}
  \item integer constants: compiled to \insn{cst\_0}, \insn{cst\_1},
    \insn{cst8}, or \insn{cst}, depending on the value.
  \item {\tt $e_1$ + $e_2$}: compiled to {\tt code[$e_1$]} {\tt code[$e_2$]}
    \insn{add}.
  \item {\tt $e_1$ - $e_2$}: compiled to {\tt code[$e_1$]} {\tt code[$e_2$]}
    \insn{sub}.
  \item {\tt $e_1$ * $e_2$}: compiled to {\tt code[$e_1$]} {\tt code[$e_2$]}
    \insn{mul}.
  \item {\tt $e_1$ / $e_2$}: compiled to {\tt code[$e_1$]} {\tt code[$e_2$]}
    \insn{div}.
  \item {\tt $e_1$ \& $e_2$}: compiled to {\tt code[$e_1$]} {\tt code[$e_2$]}
    \insn{and}.
  \item {\tt $e_1$ | $e_2$}: compiled to {\tt code[$e_1$]} {\tt code[$e_2$]}
    \insn{or}.
  \item {\tt *$e_1$}: compiled to {\tt code[$e_1$]} \insn{load}.
  \item {\tt \&$x$}: compiled to \insn{ptr} $i$.
  \item {\tt $x$}: compiled to \insn{get} $i$.
  \item $f${\tt($e_1$, $e_2$, $\ldots$)}: compiled to {\tt code[$e_1$]}
    {\tt code[$e_2$]} $\ldots$ \insn{call} $a$.
\end{itemize}

These expressions correspond to all the bytecode instructions which produce a
value on the stack, except the ones we don't need for now (namely \insn{lsl},
\insn{lsr}, \insn{callr}, and \insn{calld}). The \insn{call} instruction is a
special case: the callee might not return a value. However, functions used in
subexpressions {\em must} return a value.

\subsection{Local variables}\label{subsubsection:toyc5-local-vars}

The result of most expressions is immediately consumed in other expressions or
instructions. Some results, however, are left on the stack and used later on
with \insn{get}, \insn{set} or \insn{ptr} instructions. This currently requires
keeping track of which value is stored in which stack frame slot. To avoid
this, our programming language should provide a way to give a symbolic name to
an expression whose value is stored on the stack. In this chapter we use the
``{\tt let} $x$ $e${\tt ;}'' syntax, where $x$ is an identifier, and $e$ an
expression. $x$ can then be used in other expressions, such as `{\tt `$x$+1}''
or ``{\tt \&$x$}'' (unlike labels, $x$ must be {\em declared} with {\tt let}
before it can be used). It is called a {\em local variable} because it can only
be used in the current function (since \insn{get}, \insn{set} and \insn{ptr}
can only refer to slots in the top stack frame). This does not prevent another
function to declare a variable with the same name, but it is then independent
(\ie, refers to a different slot).

\subsection{Grammar}

We can now extend the grammar of our programming language in order to support
the above requirements. Lets look at expressions first. The above definitions
might suggest a grammar rule of the following form:

\begin{Paragraph}
expr: expr (``{\tt +}'' | ``{\tt -}'' | ``{\tt *}'' | ``{\tt /}'' | ...) expr |
``{\tt *}'' expr | ``{\tt \&}'' IDENTIFIER | ...
\end{Paragraph}

\noindent meaning that an expression is either a {\em binary} expression made
of two subexpressions with an operator in between, or an {\em unary} expression
with an operator followed by an expression, etc. However, this rule has several
issues:
\begin{itemize}
  \item It is ambiguous. Consider for instance the $2+3*4$ expression. It can
  be seen as a binary $+$ expression with subexpressions $2$ and $3*4$. But it
  can also be seen as a binary $*$ expression with subexpressions $2+3$ and
  $4$. Both interpretations are valid for the above grammar, but they don't
  give the same value! In practice anyone would give the value $14$ because we
  use implicit {\em operator precedence} rules. One such rule is that
  multiplications have a higher precedence than additions, meaning that they
  must be performed first. When we want to use the other interpretation we use
  parentheses, which have the highest precedence: $(2+3)*4$.

  Another ambiguity of the above rule is that $2-3-4$ can be seen as ``$2-3$''
  minus $4$, or as $2$ minus ``$3-4$''. Here again, both interpretations are
  valid for the grammar (and for the precedence rules), but they don't give the
  same value. In practice one always uses the first interpretation, because we
  use another implicit rule saying that operations are done from left to right.

  \item It can not be implemented with a recursive descent parser. This is
  because the ``expr'' rule is used on the leftmost position in the definition
  of one of its alternatives (such grammars are called {\em left recursive}).
  This would give a {\tt parse\_expr} function which would call itself
  recursively without reading any token in between, \ie, indefinitely (if the
  stack was unbounded).
\end{itemize}

To solve these issues, a solution is to use several rules, one per precedence
level. For instance, considering only the four basic operations for now, we can
use

\begin{Paragraph}
expr: term ((``{\tt +}'' | ``{\tt -}'') term)*\\
term: factor ((``{\tt *}''| ``{\tt /}'') factor)*\\
factor: INTEGER | ``{\tt (}'' expr ``{\tt )}''
\end{Paragraph}

\noindent where precedence increases from top to bottom. Indeed, with these
rules, $2+3*4$ can be interpreted in only one way, the one we are used to
(because $2+3$ is not a factor). Similarly, $2-3-4$ can only be seen as an
expression with an unambiguously ordered list of 3 terms. This does not tell in
itself whether these terms must be evaluated from left to right or right to
left, but this choice can be enforced in the compiler implementation. Finally,
these rules can be implemented with a recursive descent parser (``expr''
indirectly uses itself recursively, but only after the ``{\tt (}'' token; hence
there is no left recursion).

The following grammar applies this idea to all the expressions in
\cref{subsubsection:toyc5-exprs}, and takes into account the requirements in
\cref{subsubsection:toyc5-fn-params,subsubsection:toyc5-local-vars}. It does
this by extending the previous grammar as follows (unchanged parts are in gray):

\begin{Paragraph}
\unchanged{program: (fn | static} | const\unchanged{)* END}\\
\unchanged{fn: ``{\tt fn}'' fn\_name} fn\_parameters \unchanged{fn\_body}\\
\unchanged{fn\_name: IDENTIFIER}\\
fn\_parameters: ``{\tt (}'' (IDENTIFIER (``{\tt ,}'' IDENTIFIER)*)?
  ``{\tt )}''\\
\unchanged{fn\_body:} ``{\tt \{}'' statement* ``{\tt \}}''
  \unchanged{| ``{\tt ;}''}\\
statement: label | let\_stmt | (expr (``{\tt ,}'' expr)*)? instruction?
  ``{\tt ;}''\\
let\_stmt: ``{\tt let}'' IDENTIFIER expr ``{\tt ;}''\\
instruction: ``{\tt iflt}'' argument | ``{\tt ifeq}'' argument | ... |
  ``{\tt store}'' | ``{\tt pop}'' | ...\\
expr: bit\_and\_expr (``{\tt |}'' bit\_and\_expr)*\\
bit\_and\_expr: add\_expr (``{\tt \&}'' add\_expr)*\\
add\_expr: mult\_expr ((``{\tt +}'' | ``{\tt -}'') mult\_expr)*\\
mult\_expr: pointer\_expr ((``{\tt *}'' | ``{\tt /}'') pointer\_expr)*\\
pointer\_expr: ``{\tt *}'' pointer\_expr | ``{\tt \&}'' IDENTIFIER |
primitive\_expr\\
primitive\_expr:  INTEGER | IDENTIFIER fn\_arguments? |
  ``{\tt (}'' expr ``{\tt )}''\\
fn\_arguments: ``{\tt (}'' (expr (``{\tt ,}'' expr)*)? ``{\tt )}''\\
\unchanged{label: ``{\tt :}'' IDENTIFIER}\\
argument: IDENTIFIER\\
\unchanged{static: ``{\tt static}'' IDENTIFIER INTEGER*}\\
const: ``{\tt const}'' IDENTIFIER INTEGER
\end{Paragraph}

\noindent where ``?'' denotes an optional element. Thus, for instance, the
``fn\_parameters'' rule means ``a left parenthesis, optionally followed by an a
non-empty list of parameters, followed by a right parenthesis'' (the non-empty
list of parameters being defined as an identifier, followed by any number of
``comma identifier'' groups).

The body of a function is now defined as a list of {\em statements} between
curly braces (added to more clearly separate functions from each other, but
also to simplify the parser). Each statement is either a label, a local
variable declaration, or a comma separated list of expressions\footnote{It is
possible to use more strict rules to enforce a precise number of expressions
before each instruction (\eg, 0 before \insn{ret}, 1 before \insn{set}, or 2
before \insn{ifeq}). We use this less strict rule to simplify the grammar, and
thus the implementation.} followed by an optional instruction and ending with a
semi-colon. Examples of the latter case include ``\insn{ret}{\tt ;}'', ``{\tt 0}
\insn{set} {\tt x}{\tt ;}'', ``{\tt x}, {\tt y} \insn{ifeq} {\tt ok}{\tt ;}''
(an instruction preceded by 0, 1 or 2 expressions, respectively), or ``{\tt
panic(1){\tt ;}''} (an expression not followed by any instruction).

Instructions are defined as in the previous chapter, except that all the
instructions listed in \cref{subsubsection:toyc5-exprs} (\insn{add},
\insn{sub}, etc) are now removed (expressions must be instead). Similarly,
instruction arguments can no longer be integers: label, function parameter or
local variable names must be used instead.

Expressions are defined as explained above, with 6 levels of precedence.
Constants, identifiers, function calls, parentheses, and the {\em address-of}
operator ``{\tt \&}'' have the same highest precedence\footnote{We put the
address-of operator in the pointer\_expr rule for convenience, but it could be
moved in the primitive\_expr rule instead, thus showing that it has the same
precedence as the others.}. They are followed by the {\em dereference operator}
``{\tt *}'' which loads the value at some address. Then comes multiplicative
expressions, additive expressions, bitwise and expressions, and finally bitwise
or expressions. Thus, for instance, ``{\tt a + *b * c \& d}'' is equivalent to
``{\tt (a + ((*b) * c)) \& d}'' and not to ``{\tt a + ((*b) * (c \& d))}'' or
``{\tt a + (*(b * c) \& d)}'', for instance. However, when in doubt, it is
preferable to use explicit parentheses.

Finally, the new ``const'' rule adds a syntax to give a symbolic name to a
constant value. For instance, ``{\tt const} {\tt RESOLVED} {\tt 0}'' makes it
possible to use ``{\tt RESOLVED}'' in an expression, which is more meaningful
than using ``{\tt 0}''. To simplify the implementation, a constant must be
defined before it is used.

\subsection{Scanner}

As a last new requirement, our programming language should support INTEGER
tokens of the form ``{\tt '$c$'}'', where $c$ is a printable character (ASCII
code between 32 and 127, excluded), and whose numeric value is the ASCII code
of $c$. For instance, it should be possible to write {\tt 'a'} instead of a's
ASCII code {\tt 97}, or {\tt '\hspace{0pt}'\hspace{0pt}'} instead of the
quote's ASCII code {\tt 39}.

\section{Algorithms}\label{subsection:toyc2-algorithms}

Compiling function parameters, local variable names and {\tt const}
declarations can be done with the same algorithms used for function names and
labels. Namely those to add a symbol in the list of symbols, and to use this
list to find the value of a symbol. Similarly, compiling expressions can be
done with the same recursive descent method already used, namely with one
function per grammar rule. Hence we do not really need any new algorithm in
this chapter. Instead, we list here the main implementation differences
compared with the previous compiler version.

The scanner needs an updated {\tt CHAR\_TYPES} table to support the new single
character tokens (parentheses, curly braces, +, -, *, /, \&, | and ,). It also
needs an updated {\tt KEYWORDS} table ({\tt let} and {\tt const} are added,
\insn{add}, \insn{sub}, etc are removed). Finally, a new function is needed to
read the new quoted character tokens defined above.

The backend should provide new functions to write the opcode instructions
needed to compile expressions (\insn{cst\_0}, $\ldots$ \insn{add}, $\ldots$).
The goal is to simplify the parser by {\em encapsulating} the low level
instruction encoding details in simple to use functions.

The compiler needs to keep track of the stack frame slot index corresponding to
each function parameter and local variable. The former is easy: the $i^{th}$
parameter is in the $i^{th}$ slot. For the latter we assume that {\tt let}
statements are the only ones which leave a value on the stack (this forbids,
for instance, statements such as ``1;''). We also assume, without verification,
that {\tt let} statements are executed in the same order as in the source code,
and exactly once (unless the function returns before). Then the $i^{th}$ local
variable is in the $(i+4)^{th}$ slot (recall that 4 saved register values are
pushed after the function arguments).

Compiling a {\tt const} $c$ $v$ declaration does not need to produce any code.
Instead, we can simply add a corresponding symbol in the symbols list. Then,
each time $c$ is used, a \insn{cst}* instruction to push $v$ on the stack can
be produced. Note however that, in order to do this, the parser must know that
$c$ refers to a constant. Indeed, if $c$ is referring to a local variable or
function argument, a \insn{get} instruction must be produced instead. To this
end, we introduce a new symbol kind, {\tt VARIABLE} (2), in addition to the
{\tt RESOLVED} and {\tt UNRESOLVED} kinds. And we use {\tt RESOLVED} for {\tt
const} symbols and {\tt VARIABLE} for local variables and function arguments.

\section{Implementation}

\rust{
  let mut t = Transpiler2::new();
}

We can now extend the labels compiler in order to support expressions. We first
need to write it without using expressions, so that it can be compiled with the
labels compiler. We then compile this source code, which gives us the
expressions compiler bytecode. Finally, we rewrite this source code with
expressions, and we compile it with the expressions compiler bytecode. As
before, to save space, we give the two compiler versions at the same time
(without expressions in red, with in green).

The start of the compiler does not change in the $1^{st}$ version (changes are
indicated with a vertical bar in the margin), but can be rewritten in a clearer
way in the $2^{nd}$:

\toy{
@fn tc_main(src_buffer, dst_buffer, flash_buffer);
@fn main(src_buffer, dst_buffer, flash_buffer) {
@  tc_main(src_buffer, dst_buffer, flash_buffer) retv;
@}

@fn load8(ptr) { (*ptr) & 255 retv; }
@fn load16(ptr) { (*ptr) & 65535 retv; }
@fn store8(ptr, value) { ptr, (*ptr) & 4294967040 | value store; ret; }
@fn store16(ptr, value) { ptr, (*ptr) & 4294901760 | value store; ret; }

@const PANIC_BUFFER 1074666152
@fn panic_copy(src, dst) {
@  dst, *src store;
@  (dst + 4), *(src + 4) store;
@  (dst + 8), *(src + 8) store;
@  (dst + 12), *(src + 12) store;
@  ret;
@}
@fn panic_result(ptr) {
@  panic_copy(&ptr - 16, ptr);
@  PANIC_BUFFER, ptr store;
@  0 retv;
@}
@fn panic(error) {
@  panic_copy(*PANIC_BUFFER, &error - 16);
@  error retv;
@}
}%toy

\subsection{Shared constants}

We then declare, in the $2^{nd}$ version, a set of constants for $\it{token}$
values, symbol's $\it{kind}$ values, for the offset of each symbol's variable
(and their total size), and for the offset of the compiler variables from
$\it{self}$. For convenience, we set the token values of +, -, *, /, \&, and |
to their corresponding opcode (4, 5, 6, 7, 8, and 9, respectively).

\toy{
const TC_INTEGER 2
const TC_IDENTIFIER 3
const TC_ADD 4
const TC_SUB 5
const TC_MUL 6
const TC_DIV 7
const TC_BIT_AND 8
const TC_BIT_OR 9
const TC_FN 'f'
const TC_LET 'l'
const TC_CONST 'c'
const TC_STATIC 's'

const SYM_RESOLVED 0
const SYM_UNRESOLVED 1
const SYM_VARIABLE 2

const sym_name 0
const sym_length 4
const sym_kind 8
const sym_value 12
const sym_next 16
const sizeof_symbol 20

const tc_src 0
const tc_src_end 4
const tc_next_char 8
const tc_next_char_type 12
const tc_next_token 16
const tc_next_token_data 20
const tc_next_token_length 24
const tc_dst 28
const tc_heap 32
const tc_symbols 36
const tc_flash_offset 40
const tc_fn_dst 44
}%toy

\subsection{Scanner}

The {\tt CHAR\_TYPES} table must be updated to support the new single character
tokens (+, -, *, /, \&, |, comma, parentheses and curly braces). For
convenience, we set the character types of +, -, *, /, \&, and | to the
corresponding token values. And we set the type of the others, including the
quote, to their ASCII code. The {\tt KEYWORDS} table must also be updated to
remove instructions now handled with expressions (\insn{add}, \insn{sub}, etc)
and to add the new {\tt let} and {\tt const} keywords.

\toy{
@static TC_CHAR_TYPES
@  1 1 1 1 1 1 1 1 1 32 32 1 1 1 1 1 1 1 1 1 1 1 1 1 1 1 1 1 1 1 1 1
  32 1 1 1 1 1 8 39 40 41 6 4 44 5 1 7 2 2 2 2 2 2 2 2 2 2 58 59 1 1 1 1
@  1 3 3 3 3 3 3 3 3 3 3 3 3 3 3 3 3 3 3 3 3 3 3 3 3 3 3 1 1 1 1 3
  1 3 3 3 3 3 3 3 3 3 3 3 3 3 3 3 3 3 3 3 3 3 3 3 3 3 3 123 9 125 1 1
@  1 1 1 1 1 1 1 1 1 1 1 1 1 1 1 1 1 1 1 1 1 1 1 1 1 1 1 1 1 1 1 1
@  1 1 1 1 1 1 1 1 1 1 1 1 1 1 1 1 1 1 1 1 1 1 1 1 1 1 1 1 1 1 1 1
@  1 1 1 1 1 1 1 1 1 1 1 1 1 1 1 1 1 1 1 1 1 1 1 1 1 1 1 1 1 1 1 1
@  1 1 1 1 1 1 1 1 1 1 1 1 1 1 1 1 1 1 1 1 1 1 1 1 1 1 1 1 1 1 1 1
static TC_KEYWORDS
@  4 'i' 'f' 'l' 't' 140
@  4 'i' 'f' 'e' 'q' 141
@  4 'i' 'f' 'g' 't' 142
@  4 'i' 'f' 'l' 'e' 143
@  4 'i' 'f' 'n' 'e' 144
@  4 'i' 'f' 'g' 'e' 145
@  4 'g' 'o' 't' 'o' 146
  5 's' 't' 'o' 'r' 'e' 148
  3 's' 'e' 't' 151
  3 'p' 'o' 'p' 152
  3 'r' 'e' 't' 157
  4 'r' 'e' 't' 'v' 158
  2 'f' 'n' 102
  3 'l' 'e' 't' 108
  5 'c' 'o' 'n' 's' 't' 99
@  6 's' 't' 'a' 't' 'i' 'c' 115
@  0
}%toy

The first scanner functions are unchanged compared with the labels compiler:

\toy{
@fn mem_compare(ptr1, ptr2, size) {
@  let i 0;
@:step2
@  i, size ifge step4;
@  load8(ptr1 + i), load8(ptr2 + i) ifne step4;
@  i + 1 set i; goto step2;
@:step4
@  size - i retv;
@}

@fn tc_get_keyword(start, length) {
@  let len 0;
@  let ptr TC_KEYWORDS;
@:step2
@  load8(ptr) set len;
@  len, 0 ifne step4;
@  TC_IDENTIFIER retv;
@:step4
@  length, len ifne step7;
@  mem_compare(start, ptr + 1, length), 0 ifne step7;
@  load8(ptr + len + 1) retv;
@:step7
@  ptr + len + 2 set ptr; goto step2;
@}

@fn tc_read_char(self) {
@  let src *(self+tc_src);
@  let src_end *(self+tc_src_end);
@  src, src_end iflt step2;
@  panic(10);
@:step2
@  src + 1 set src;
@  let c 0;
@  let type 0;
@  src, src_end ifge end;
@  load8(src) set c;
@  load8(TC_CHAR_TYPES + c) set type;
@:end
@  (self+tc_src), src store;
@  (self+tc_next_char), c store;
@  (self+tc_next_char_type), type store;
@  type retv;
@}

@fn tc_read_integer(self) {
@  let type *(self+tc_next_char_type);
@  let v 0;
@:step2
@  type, TC_INTEGER ifne step5;
@  v * 10 + (*(self+tc_next_char) - '0') set v;
@  tc_read_char(self) set type;
@  goto step2;
@:step5
@  (self+tc_next_token_data), v store;
@  TC_INTEGER retv;
@}
}%toy

To support the new quoted characters tokens such as ``{\tt 'a'}'' we add the
following function. It starts by reading the first quote (the caller should
check that the next character is a quote). It then checks if the second
character, in $\it{value}$, is printable, and panics otherwise. Finally, it
checks that the third character is a quote, sets the $\it{next\_token\_data}$ to
$value$ and return the {\tt INTEGER} token type:

\toy{
fn tc_read_quoted_char(self) {
  tc_read_char(self) pop;
  let value *(self+tc_next_char);
  value, 32 iflt not_printable;
  value, 127 iflt printable;
:not_printable
  panic(11);
:printable
  tc_read_char(self), ''' ifeq ok;
  panic(12);
:ok
  tc_read_char(self) pop;
  (self+tc_next_token_data), value store;
  TC_INTEGER retv;
}
}%toy
\rust{
  context.add_error_code(11, "Printable character expected");
  context.add_error_code(12, "Quote character expected");
}

The remaining scanner functions are essentially unchanged compared with the
labels compiler. We just add a new case in {\tt tc\_read\_token}, which calls
the above function if the next character is a quote:

\toy{
@fn tc_read_identifier(self) {
@  let start *(self+tc_src);
@  let type *(self+tc_next_char_type);
@:step2
@  type, TC_IDENTIFIER ifeq step3;
@  type, TC_INTEGER ifne step4;
@:step3
@  tc_read_char(self) set type; goto step2;
@:step4
@  let length *(self+tc_src) - start;
@  (self+tc_next_token_data), start store;
@  (self+tc_next_token_length), length store;
@  tc_get_keyword(start, length) retv;
@}

@fn tc_read_token(self) {
@  let type *(self+tc_next_char_type);
@:step1
@  type, ' ' ifne step3;
@  tc_read_char(self) set type; goto step1;
@:step3
@  let token type;
  type, TC_INTEGER ifne step4;
@  tc_read_integer(self) set token; goto end;
:step4
  type, ''' ifne step5;
  tc_read_quoted_char(self) set token; goto end;
@:step5
@  type, TC_IDENTIFIER ifne step6;
@  tc_read_identifier(self) set token; goto end;
@:step6
@  type, 0 ifeq end;
@  tc_read_char(self) pop;
@:end
@  (self+tc_next_token), token store;
@  ret;
@}
}%toy

\subsection{Backend}

The backend is extended with new functions to write the opcode instructions
needed by the parser. Its first functions are unchanged compared with the
labels compiler:

\toy{
@fn mem_allocate(size, ptr_p) {
@  let ptr *ptr_p;
@  ptr_p, ptr + size store;
@  ptr retv;
@}
@fn tc_write8(self, value) {
@  store8(mem_allocate(1, self+tc_dst), value);
@  ret;
@}
@fn tc_write16(self, value) {
@  store16(mem_allocate(2, self+tc_dst), value);
@  ret;
@}
@fn tc_write32(self, value) {
@  mem_allocate(4, self+tc_dst), value store;
@  ret;
@}
}%toy

We then add a generic utility function to write an instruction with a single
byte argument, used later on:

\toy{
fn tc_write_insn(self, opcode, argument) {
  tc_write8(self, opcode);
  tc_write8(self, argument);
  ret;
}
}%toy

The next two functions are unchanged ({\tt last\_placeholder} and {\tt
new\_placeholder} refer to $x$ and $y$ in \cref{alg:toyc1-add-ref},
respectively):

\toy{
@fn tc_add_placeholder(self, placeholder_p) {
@  let new_placeholder *(self+tc_dst);
@  let last_placeholder *placeholder_p;
@  placeholder_p, new_placeholder store;
@  last_placeholder, 0 ifne step4; new_placeholder set last_placeholder;
@:step4
@  new_placeholder - last_placeholder retv;
@}

@fn tc_fill_placeholders(placeholder, value) {
@  let offset 0;
@:step2
@  placeholder, 0 ifeq end;
@:step3
@  load16(placeholder) set offset;
@  store16(placeholder, value);
@  offset, 0 ifne step6; ret;
@:step6
@  placeholder - offset set placeholder;
@  goto step2;
@:end
@  ret;
@}
}%toy

The following new function writes an instruction to push a given value on the
stack. It encapsulates the details related to the \insn{cst\_0}, \insn{cst\_1},
\insn{cst8} and \insn{cst} instructions by writing the appropriate instruction
depending on $\it{value}$:

\toy{
fn tc_write_cst_insn(self, value) {
  value, 1 ifgt not0_or_1;
  tc_write8(self, value); ret;
:not0_or_1
  value, 256 ifge not_byte;
  tc_write_insn(self, 2, value); ret;
:not_byte
  tc_write8(self, 3);
  tc_write32(self, value); ret;
}
}%toy

The next function writes the instruction to perform the arithmetic operation
specified by $\it{token}$, which must be one of {\tt TC\_ADD}, {\tt TC\_SUB},
{\tt TC\_MUL}, {\tt TC\_DIV}, {\tt TC\_BIT\_AND}, or {\tt TC\_BIT\_OR}. It is
trivial since these values are equal to the corresponding opcodes:

\toy{
fn tc_write_binary_insn(self, token) {
  tc_write8(self, token);
  ret;
}
}%toy

The following 3 functions write the instruction corresponding to their name.
They encapsulate the details related to their encoding.

\toy{
fn tc_write_load_insn(self) {
  tc_write8(self, 19);
  ret;
}
fn tc_write_ptr_insn(self, variable) {
  tc_write_insn(self, 21, variable);
  ret;
}
fn tc_write_get_insn(self, variable) {
  tc_write_insn(self, 22, variable);
  ret;
}
}%toy

The next function is simplified by using the new {\tt tc\_write\_insn} function:

\toy{
fn tc_write_fn_insn(self, arity) {
  tc_write_insn(self, 25, arity);
  ret;
}
}%toy

The last backend function writes the instruction to call a given
$\it{function}$, specified with a symbol. It writes the \insn{call} opcode,
followed either by a new placeholder if the symbol is unresolved, or by the
symbol's value. By hypothesis, this value is the \insn{call} instruction
argument which must be used to call $\it{function}$. The following function
encapsulates the details of its computation.

\toy{
fn tc_get_fn_value(self, fn_dst) {
  fn_dst - *(self+tc_flash_offset) - 786432 retv;
}
fn tc_write_call_insn(self, function) {
  tc_write8(self, 26);
  *(function+sym_kind), SYM_UNRESOLVED ifne resolved;
  tc_write16(self, tc_add_placeholder(self, function+sym_value));
  ret;
:resolved
  tc_write16(self, *(function+sym_value));
  ret;
}
}%toy

\subsection{Parser}

The start of the parser is the same as in the labels compiler:

\toy{
@fn sym_lookup(symbol, name, length) {
@:step2
@  symbol, 0 ifeq step7;
@  *(symbol+sym_length), length ifne step6;
@  mem_compare(*(symbol+sym_name), name, length), 0 ifne step6;
@  symbol retv;
@:step6
@  *(symbol+sym_next) set symbol; goto step2;
@:step7
@  0 retv;
@}

@fn tc_add_symbol(self, name, length, kind, value) {
@  let symbol mem_allocate(sizeof_symbol, self+tc_heap);
@  sym_lookup(*(self+tc_symbols), name, length), 0 ifeq ok;
@  panic(30);
@:ok
@  (symbol+sym_name), name store;
@  (symbol+sym_length), length store;
@  (symbol+sym_kind), kind store;
@  (symbol+sym_value), value store;
@  (symbol+sym_next), *(self+tc_symbols) store;
@  (self+tc_symbols), symbol store;
@  symbol retv;
@}

@fn tc_add_or_resolve_symbol(self, name, length, value) {
@  let symbol sym_lookup(*(self+tc_symbols), name, length);
@  symbol, 0 ifne found;
@  tc_add_symbol(self, name, length, SYM_RESOLVED, value) retv;
@:found
@  *(symbol+sym_kind), SYM_UNRESOLVED ifeq ok;
@  panic(31);
@:ok
@  tc_fill_placeholders(*(symbol+sym_value), value);
@  (symbol+sym_kind), SYM_RESOLVED store;
@  (symbol+sym_value), value store;
@  symbol retv;
@}

@fn tc_parse_token(self, token) {
@  *(self+tc_next_token), token ifeq ok;
@  panic(20);
@:ok
@  tc_read_token(self);
@  ret;
@}
@fn tc_parse_integer(self) {
@  *(self+tc_next_token), TC_INTEGER ifeq ok;
@  panic(21);
@:ok
@  let value *(self+tc_next_token_data);
@  tc_read_token(self);
@  value retv;
@}
@fn tc_parse_identifier(self, length_p) {
@  *(self+tc_next_token), TC_IDENTIFIER ifeq ok;
@  panic(22);
@:ok
@  let name *(self+tc_next_token_data);
@  length_p, *(self+tc_next_token_length) store;
@  tc_read_token(self);
@  name retv;
@}
}%toy

Here we add a new utility function to parse an identifier which must correspond
to an existing symbol. This function parses an identifier and returns its
corresponding symbol in the list passed as argument in $\it{symbol}$ (or panics
if no symbol is found):

\toy{
fn tc_parse_symbol(self, symbol) {
  let length 0;
  let name tc_parse_identifier(self, &length);
  sym_lookup(symbol, name, length) set symbol;
  symbol, 0 ifne ok;
  panic(33);
:ok
  symbol retv;
}
}%toy
\rust{
  context.add_error_code(33, "Undefined symbol");
}

The {\tt tc\_parse\_static} function is unchanged, but a new trivial {\tt
  tc\_parse\_const} function is added for the new ``{\tt const} $x$ $v$''
syntax. This function simply adds a new symbol for $x$, with value $v$.

\toy{
fn tc_parse_const(self) {
  tc_parse_token(self, TC_CONST);
  let length 0;
  let name tc_parse_identifier(self, &length);
  tc_add_symbol(self, name, length, SYM_RESOLVED, tc_parse_integer(self)) pop;
  ret;
}
@fn tc_parse_static(self) {
@  tc_parse_token(self, TC_STATIC);
@  let length 0;
@  let name tc_parse_identifier(self, &length);
@  let value *(self+tc_dst) - *(self+tc_flash_offset);
@  tc_add_symbol(self, name, length, SYM_RESOLVED, value) pop;
@:loop
@  *(self+tc_next_token), TC_INTEGER ifne end;
@  tc_write8(self, tc_parse_integer(self)); goto loop;
@:end
@  ret;
@}
}%toy

The {\tt parse\_argument} function is updated to match the new ``argument''
grammar rule, which no longer allows INTEGER arguments. As a consequence, the
start of this function, which was calling {\tt tc\_parse\_integer}, is removed.
On the other hand, the {\tt tc\_parse\_label} function is unchanged (the
``label'' rule has not changed):

\toy{
fn tc_parse_argument(self) {
@  let length 0;
@  let name tc_parse_identifier(self, &length);
@  let symbol sym_lookup(*(self+tc_symbols), name, length);
@  symbol, 0 ifne found;
@  tc_add_symbol(self, name, length, SYM_UNRESOLVED, 0) set symbol;
@:found
@  *(symbol+sym_kind), SYM_UNRESOLVED ifne resolved;
@  tc_add_placeholder(self, symbol+sym_value) retv;
@:resolved
@  *(symbol+sym_value) retv;
@}

@fn tc_parse_label(self) {
@  tc_parse_token(self, ':');
@  let length 0;
@  let name tc_parse_identifier(self, &length);
@  let value *(self+tc_dst) - *(self+tc_fn_dst);
@  tc_add_or_resolve_symbol(self, name, length, value);
@  ret;
@}
}%toy

The following functions are the main new part of the compiler. They implement
the expressions rules in reverse order, starting with ``fn\_arguments''. This
rule uses ``expr'' but since {\tt tc\_parse\_expr} is implemented last, we need
to declare it first.

\toy{
fn tc_parse_expr(self);
}%toy

The {\tt tc\_parse\_fn\_arguments} function parses the arguments of a function
call $f(e_0,e_1,\ldots)$, and takes as argument the symbol corresponding to
$f$. It first checks that this symbol is not a local variable, and panics
otherwise. It then parses the arguments with the recursive descent method:
after parsing the opening parenthesis, it parses a first expression unless the
next token is a closing parenthesis. Then, while the next token is a comma, it
reads it and parses another expression. This generates the compiled code for
the arguments, after which we just need to write a \insn{call} instruction.

\toy{
fn tc_parse_fn_arguments(self, function) {
  *(function+sym_kind), SYM_VARIABLE ifne ok;
  panic(34);
:ok
  tc_parse_token(self, '(');
  *(self+tc_next_token), ')' ifeq end;
  tc_parse_expr(self);
:loop
  *(self+tc_next_token), ',' ifne end;
  tc_read_token(self);
  tc_parse_expr(self);
  goto loop;
:end
  tc_parse_token(self, ')');
  tc_write_call_insn(self, function);
  ret;
}
}%toy
\rust{
  context.add_error_code(34, "Function name expected");
}

A primitive expression can either start with an integer, an identifier, or an
opening parenthesis. In the first case we generate the code to push this
integer on the stack:

\toy{
fn tc_parse_primitive_expr(self) \{
  let symbol 0;
  *(self+tc_next_token), TC_INTEGER ifne not_integer;
  tc_write_cst_insn(self, tc_parse_integer(self));
  ret;
}%toy

In the second case, there must be a symbol for this identifier. If it is
followed by an opening parenthesis this is a function call, and we parse it
with the above function.

\toy{
:not_integer
  *(self+tc_next_token), TC_IDENTIFIER ifne parentheses;
  tc_parse_symbol(self, *(self+tc_symbols)) set symbol;
  *(self+tc_next_token), '(' ifne identifier;
  tc_parse_fn_arguments(self, symbol);
  ret;
}%toy

\noindent Otherwise we write the appropriate instruction depending on the kind
of symbol referred to by the identifier (or panic if the symbol is unresolved).

\toy{
:identifier
  *(symbol+sym_kind), SYM_VARIABLE ifne not_variable;
  tc_write_get_insn(self, *(symbol+sym_value));
  ret;
:not_variable
  *(symbol+sym_kind), SYM_RESOLVED ifne error;
  tc_write_cst_insn(self, *(symbol+sym_value));
  ret;
:error
  panic(35);
}%toy
\rust{
  context.add_error_code(35, "Illegal identifier expression");
}

In the last case, we simply need to parse an expression between parentheses:

\toy{
:parentheses
  tc_parse_token(self, '(');
  tc_parse_expr(self);
  tc_parse_token(self, ')');
  ret;
\}
}%toy

A pointer expression either starts with ``{\tt *}'' or ``{\tt \&}'', or is a
primitive expression. Following the recursive descent method, the first case is
trivial: we just need to read the ``{\tt *}'' token, parse a pointer expression
recursively, and finally generate a \insn{load} instruction (recall that
``*$e$'' means ``the value at address $e$''). The second case is also simple:
the ``{\tt \&}'' must be followed by an identifier which must correspond to a
local variable or function parameter $x$. If it is we must generate a
\insn{ptr} $x$ instruction (recall that ``{\tt \&$x$}'' means ``the address of
$x$'s stack frame slot''), otherwise this is an error:

\toy{
fn tc_parse_pointer_expr(self) {
  let symbol 0;
  *(self+tc_next_token), TC_MUL ifne not_mul;
  tc_read_token(self);
  tc_parse_pointer_expr(self);
  tc_write_load_insn(self);
  ret;
:not_mul
  *(self+tc_next_token), TC_BIT_AND ifne not_bit_and;
  tc_read_token(self);
  tc_parse_symbol(self, *(self+tc_symbols)) set symbol;
  *(symbol+sym_kind), SYM_VARIABLE ifne error;
  tc_write_ptr_insn(self, *(symbol+sym_value));
  ret;
:error
  panic(36);
:not_bit_and
  tc_parse_primitive_expr(self);
  ret;
}
}%toy
\rust{
  context.add_error_code(36, "Illegal address-of operator argument");
}

The remaining expression parsing functions are straightforward, again by
following the recursive descent method. After parsing a first subexpression, a
loop is used, while the next token is a permitted operator (\eg, ``{\tt +}'' or
``{\tt -}'' for the ``add\_expr'' rule), to read the operator, parse a
subexpression, and write the operator's instruction:

\toy{
fn tc_parse_mult_expr(self) {
  tc_parse_pointer_expr(self);
  let next_token *(self+tc_next_token);
:loop
  next_token, TC_MUL ifeq mul_or_div;
  next_token, TC_DIV ifne end;
:mul_or_div
  tc_read_token(self);
  tc_parse_pointer_expr(self);
  tc_write_binary_insn(self, next_token);
  *(self+tc_next_token) set next_token;
  goto loop;
:end
  ret;
}
fn tc_parse_add_expr(self) {
  tc_parse_mult_expr(self);
  let next_token *(self+tc_next_token);
:loop
  next_token, TC_ADD ifeq add_or_sub;
  next_token, TC_SUB ifne end;
:add_or_sub
  tc_read_token(self);
  tc_parse_mult_expr(self);
  tc_write_binary_insn(self, next_token);
  *(self+tc_next_token) set next_token;
  goto loop;
:end
  ret;
}
fn tc_parse_bit_and_expr(self) {
  tc_parse_add_expr(self);
:loop
  *(self+tc_next_token), TC_BIT_AND ifne end;
  tc_read_token(self);
  tc_parse_add_expr(self);
  tc_write_binary_insn(self, TC_BIT_AND);
  goto loop;
:end
  ret;
}
fn tc_parse_expr(self) {
  tc_parse_bit_and_expr(self);
:loop
  *(self+tc_next_token), TC_BIT_OR ifne end;
  tc_read_token(self);
  tc_parse_bit_and_expr(self);
  tc_write_binary_insn(self, TC_BIT_OR);
  goto loop;
:end
  ret;
}
}%toy

The {\tt tc\_parse\_instruction} function is similar to its previous version in
the labels compiler and is even simpler, since an instruction can no longer be
a label (labels have been moved to the ``statement'' rule):

\toy{
@static ARG_SIZES
@  0 0 1 4 0 0 0 0 0 0 0 0 2 2 2 2 2 2 2 0 0 1 1 1 0 0 2 0 0 0 0

fn tc_parse_instruction(self) {
  let opcode *(self+tc_next_token) - 128;
  tc_write8(self, opcode);
  tc_read_token(self);
  let arg_size load8(ARG_SIZES + opcode);
  arg_size, 0 ifne not0;
  ret;
:not0
  let arg tc_parse_argument(self);
  arg_size, 1 ifne not1;
  tc_write8(self, arg); ret;
:not1
  arg_size, 2 ifne not2;
  tc_write16(self, arg); ret;
:not2
  tc_write32(self, arg); ret;
}
}%toy

The following function parses the new ``{\tt let $x$ $e$;}'' syntax. It takes
as parameter the stack frame slot index of $x$ and returns the index to use for
the next {\tt let}. The parsing itself is trivial. No code needs to be
generated besides the one generated while parsing $e$. We just need to add $x$
in the list of symbols, with value $\it{variable}$.

\toy{
fn tc_parse_let_stmt(self, variable) {
  tc_parse_token(self, TC_LET);
  let length 0;
  let name tc_parse_identifier(self, &length);
  tc_parse_expr(self);
  tc_parse_token(self, ';');
  tc_add_symbol(self, name, length, SYM_VARIABLE, variable) pop;
  variable + 1 retv;
}
}%toy

The next function implements the ``statement'' rule. It takes as parameter the
stack frame slot to use if the statement is a {\tt let} construct, and returns
the slot to use for the next statement. If the next token is a colon or the
{\tt let} keyword, we just need to call the function to parse a label or a let
statement. Otherwise, if the next token is not an opcode keyword ($\it{token}
\ge 128$) or a semicolon, we need to parse an expression. Then, while the next
token is a comma, we should read it and parse another expression. Finally, we
should parse an instruction if the next token is not a semicolon.

\toy{
fn tc_parse_statement(self, next_variable) {
	*(self+tc_next_token), ':' ifne not_label;
	tc_parse_label(self);
	goto end;
:not_label
	*(self+tc_next_token), TC_LET ifne expr_or_insn;
	tc_parse_let_stmt(self, next_variable) retv;
:expr_or_insn
  *(self+tc_next_token), ';' ifeq insn;
  *(self+tc_next_token), 128 ifge insn;
	tc_parse_expr(self);
:loop
	*(self+tc_next_token), ',' ifne insn;
	tc_read_token(self);
	tc_parse_expr(self);
	goto loop;
:insn
	*(self+tc_next_token), ';' ifeq insn_end;
	*(self+tc_next_token), 128 ifge ok;
	panic(24);
:ok
	tc_parse_instruction(self);
:insn_end
	tc_parse_token(self, ';');
:end
	next_variable retv;
}
}%toy
\rust{
	context.add_error_code(24, "Instruction opcode expected");
}

The {\tt tc\_parse\_fn\_name} function is the same as in the labels compiler,
except that it now computes the symbol's value with the new {\tt
tc\_get\_fn\_value} function:

\toy{
@fn tc_parse_fn_name(self) {
@  let length 0;
@  let name tc_parse_identifier(self, &length);
@  let fn_dst *(self+tc_dst);
@  (self+tc_fn_dst), fn_dst store;
  let value tc_get_fn_value(self, fn_dst);
@  tc_add_or_resolve_symbol(self, name, length, value) retv;
@}
}%toy

The overall algorithm of the {\tt tc\_parse\_fn\_parameters} function is the
same as the one parsing function arguments, and derives from the recursive
descent method. This function parses an opening parenthesis and then, while the
next token is not a closing parenthesis, parses a comma (except at the first
iteration) and an identifier. Each identifier is added to the list of symbols
with its index $i$ as value. $i$ is initialized to 0, incremented after each
identifier, and finally returned to the caller.

\toy{
fn tc_parse_fn_parameters(self) {
  let i 0;
  let name 0;
  let length 0;
  tc_parse_token(self, '(');
:loop
  *(self+tc_next_token), ')' ifeq end;
  i, 0 ifle identifier;
  tc_parse_token(self, ',');
:identifier
  tc_parse_identifier(self, &length) set name;
  tc_add_symbol(self, name, length, SYM_VARIABLE, i) pop;
  i + 1 set i;
  goto loop;
:end
  tc_read_token(self);
  i retv;
}
}%toy

The {\tt parse\_fn\_body} function is updated to parse the new curly braces,
and no longer parses the function's arity, now passed as argument. It also
parses statements instead of instructions, and keeps track of the stack frame
slot to use for {\tt let} statements (initialized to $\it{arity}+4$ -- \cf
\cref{subsection:toyc2-algorithms}).

\toy{
fn tc_parse_fn_body(self, function, arity) {
@  *(self+tc_next_token), ';' ifne body;
@  tc_read_token(self);
@  (function+sym_kind), SYM_UNRESOLVED store;
@  (function+sym_value), 0 store;
@  ret;
@:body
  tc_parse_token(self, '\{');
  tc_write_fn_insn(self, arity);
  let next_variable arity + 4;
:loop
  *(self+tc_next_token), '\}' ifeq end;
  tc_parse_statement(self, next_variable) set next_variable;
  goto loop;
:end
  tc_read_token(self);
  ret;
}
}%toy

The {\tt tc\_check\_symbols} function is unchanged, while {\tt tc\_parse\_fn}
is updated to parse the function parameters before parsing its body. Note that
restoring the $\it{heap}$ and $\it{symbols}$ variables now also deletes the
symbols added for the function parameters and for the local variables (in
addition to the label symbols). This is what we want, so that parameters and
local variables defined in one function cannot be used in another.

\toy{
@fn tc_check_symbols(symbol, end_symbol) {
@:loop
@  symbol, end_symbol ifeq end;
@  *(symbol+sym_kind), SYM_UNRESOLVED ifne next;
@  panic(32);
@:next
@  *(symbol+sym_next) set symbol; goto loop;
@:end
@  ret;
@}
@fn tc_parse_fn(self) {
@  tc_parse_token(self, TC_FN);
@  let function tc_parse_fn_name(self);
@  let heap *(self+tc_heap);
@  let symbols *(self+tc_symbols);
  let arity tc_parse_fn_parameters(self);
  tc_parse_fn_body(self, function, arity);
@  tc_check_symbols(*(self+tc_symbols), symbols);
@  (self+tc_symbols), symbols store;
@  (self+tc_heap), heap store;
@  ret;
@}
}%toy

Finally, {\tt tc\_parse\_program} is updated to handle the new ``{\tt const}''
case, while the {\tt tc\_main} function is unchanged:

\toy{
@fn tc_parse_program(self) {
@:loop
@  *(self+tc_next_token), TC_FN ifne not_fn;
@  tc_parse_fn(self); goto loop;
@:not_fn
  *(self+tc_next_token), TC_STATIC ifne not_static;
@  tc_parse_static(self); goto loop;
:not_static
  *(self+tc_next_token), TC_CONST ifne end;
  tc_parse_const(self); goto loop;
@:end
@  *(self+tc_next_token), 0 ifeq ok;
@  panic(23);
@:ok
@  tc_check_symbols(*(self+tc_symbols), 0); ret;
@}
@fn tc_main(src_buffer, dst_buffer, flash_buffer) {
@  let fn_dst 0;
@  let flash_offset dst_buffer - flash_buffer;
@  let symbols 0;
@  let heap dst_buffer + 12288;
@  let dst dst_buffer + 4;
@  let next_token_length 0;
@  let next_token_data 0;
@  let next_token 0;
@  let next_char_type 0;
@  let next_char 0;
@  let src_end src_buffer + 4 + *src_buffer;
@  let src src_buffer + 3;
@  let panic3 0;
@  let panic2 0;
@  let panic1 0;
@  let panic0 0;
@  let error 0;
@  panic_result(&panic0) set error;
@  error, 0 ifeq ok;
@  dst_buffer, src - src_buffer - 4 store;
@  error retv;
@:ok
@  tc_read_char(&src) pop;
@  tc_read_token(&src);
@  tc_parse_program(&src);
@  dst_buffer, dst - dst_buffer - 4 store;
@  0 retv;
@}
}%toy

\rust{
  t.write_toy1("website/sources/expressions_compiler_v1.txt")?;
  t.write_toy2("website/sources/expressions_compiler_v2.txt")?;
  t.check_changes("website/sources/labels_compiler_v2.txt")?;
}

\section{Compilation and tests}

\rust{
  let boot_mode_address = context.memory_region("foundations")
      .label_address("boot_mode_select_rom");

  let display = Rc::new(RefCell::new(TextDisplay::default()));
  context.set_display(display.clone());
  context.micro_controller().borrow_mut().reset();
  context.run_until_get_char();

  let mut context1 = context.clone();

  // Launch the command editor
  let command_editor_main =
      context.memory_region("command_editor").label_address("command_editor");
  context.type_ascii(&format!("W{:08X}\n", command_editor_main));
  context.type_ascii("R");
}

To compile the above source code proceed as follows (see also
\cref{fig:compilation-and-test}). First launch the command editor by typing
``w\rs{hex_word_low(command_editor_main)}''+Enter in the memory editor,
followed by ``r''.

\medskip \paragraph*{Edit v1} Type ``F3''+``r'' and ``F4''+``r'' to load and
edit the current compiler version. Then update it to the $1^{st}$ version of
the expressions compiler (that is, the above code with the parts highlighted in
red). For convenience, we also provide this code in the {\tt
expressions\_compiler\_v1.txt} file in \toypcurl{sources.zip}. When you are
done, exit the text editor and type ``F5''+``r'' to save your work.
Alternatively, you can ``cheat'' by running the following command on an
external computer (see \cref{section:toyc1-compilation} for more details):

\rust{
  // Enter source code in RAM, as if edited with F4 command.
  let ram_compiler_source = context.memory_region("command_editor_source")
      .label_address("ram_compiler_source");
  let toyc2_dot_toy1 = std::fs::read_to_string(
      "website/sources/expressions_compiler_v1.txt")?;
  context.store_text(ram_compiler_source, toyc2_dot_toy1.as_str());

  // Save it in flash.
  context.type_keys(vec!["F5"]);
  context.type_ascii("R");
  assert_eq!(display.borrow().get_text(), "00000000");
  context.type_ascii("\n");

  // Alternative method.
  context1.micro_controller().borrow_mut().reset();
  context1.run_until_get_char();
  context1.type_ascii(&format!("W{:08X}\n", boot_mode_address));
  context1.type_ascii("R");
  context1.micro_controller().borrow_mut().reset();

  let compiler_source = context.memory_region("compiler_source").start;
  write_lines("website/part3", "expressions_compiler_v1.txt",
      &flash_helper_commands(&toyc2_dot_toy1, compiler_source))?;
  let mut flash_helper1 = FlashHelper::from_file(context1.micro_controller(),
      "website/", "part3/expressions_compiler_v1.txt")?;
  let log = flash_helper1.read();

  // Check that both methods give the same result.
  context.check_equal_buffer(&mut context1, compiler_source);
}
\rs{host_log(log.lines().next().unwrap())}

\medskip \paragraph{Compile v1} In the command editor, type ``F6''+``r'' to
compile the code you typed. If all goes well, after about 2 seconds, you should
get a result equal to 0 (meaning that no error was found). If this is not the
case use \cref{appendix:compilercodes} to get the error code meaning, fix this
error, save the program and compile it again. Repeat this process until the
compilation is successful. Then type ``F7''+``r'' to save the result.

\rust{
  // Compile it with toyc1.
  context.type_keys(vec!["F6"]);
  context.type_ascii("R");
  assert_eq!(display.borrow().get_text(), "00000000");
  context.type_ascii("\n");

  // Save it in flash memory.
  context.type_keys(vec!["F7"]);
  context.type_ascii("R");
  assert_eq!(display.borrow().get_text(), "00000000");
  context.type_ascii("\n");
}

\medskip \paragraph{Test v1} Type ``F2''+``r'' to create a new program,
``F4''+``r'' to edit it, and type the following small test program, which
computes the factorial of 6:

\rust{
  let test_program = r"fn factorial(n);
fn test() { factorial(6) retv; }
fn factorial(n) {
  n, 0 ifne not_zero; 1 retv;
  :not_zero factorial(n - 1) * n retv;
}";
  // Enter source code in RAM, as if edited with F4 command.
  context.store_text(ram_compiler_source, test_program);
}
\rs{code(test_program)}

\noindent Then type ``F9''+``r'' to compile and run it. If the result is not
\rs{dec_hex(720u32)} this means that the compiler is wrong. In this case, type
``F8''+``r'' to restore the labels compiler. Then repeat the previous steps and
double check everything until this test passes.

\rust{
  context.type_keys(vec!["F9"]);
  context.type_ascii("R");
  assert_eq!(display.borrow().get_text(), "000002D0");
  context.type_ascii("\n");
}

\medskip \paragraph*{Edit v2} Type ``F3''+``r'' to load the $1^{st}$ version of
the expression compiler and ``F4''+``r'' to edit it. Then update it to the
$2^{nd}$ version (that is, the code in the previous section, with the parts
highlighted in green). For convenience, we also provide this code in the {\tt
expressions\_compiler\_v2.txt} file. Then save this new version with the F5
command. Alternatively, run the following command on an external computer:

\rust{
  // Enter source code in RAM, as if edited with F4 command.
  let toyc2_dot_toy2 = std::fs::read_to_string(
      "website/sources/expressions_compiler_v2.txt")?;
  context.store_text(ram_compiler_source, toyc2_dot_toy2.as_str());

  // Save it in flash.
  context.type_keys(vec!["F5"]);
  context.type_ascii("R");
  assert_eq!(display.borrow().get_text(), "00000000");
  context.type_ascii("\n");

  // Alternative method.
  context1.run_until_get_char();
  context1.type_ascii(&format!("W{:08X}\n", boot_mode_address));
  context1.type_ascii("R");
  context1.micro_controller().borrow_mut().reset();

  write_lines("website/part3", "expressions_compiler_v2.txt",
      &flash_helper_commands(&toyc2_dot_toy2, compiler_source))?;
  let mut flash_helper1 = FlashHelper::from_file(context1.micro_controller(),
      "website/", "part3/expressions_compiler_v2.txt")?;
  let log = flash_helper1.read();

  // Check that both methods give the same result.
  context.check_equal_buffer(&mut context1, compiler_source);
}
\rs{host_log(log.lines().next().unwrap())}

\medskip \paragraph*{Compile v2} Type``F6''+``r'' to compile this new code. The
result should be 0, meaning ``no error''. If this is not the case, repeat the
``Edit v2'' and ``Compile v2'' steps until all errors are fixed.

\rust{
  // Compile v2.
  context.type_keys(vec!["F6"]);
  context.type_ascii("R");
  assert_eq!(display.borrow().get_text(), "00000000");
  context.type_ascii("\n");
}

\medskip \paragraph*{Test v2} The compilation of the $2^{nd}$ version of the
expressions compiler should give the same code for each expression as the
manually written code in the $1^{st}$ version. Consequently, the compiled code
of the $2^{nd}$ version should be identical to that of the $1^{st}$ version. To
check this type ``F10''+``r''. The result should be 0. If this is not the case,
repeat the steps from ``Edit v2'' until this test passes.

\rust{
  // Test same code as v1.
  context.type_keys(vec!["F10"]);
  context.type_ascii("R");
  assert_eq!(display.borrow().get_text(), "00000000");
  context.type_ascii("\n");

  // return in memory editor
  context.type_ascii("\n");
}

% This work is licensed under the Creative Commons Attribution NonCommercial
% ShareAlike 4.0 International License. To view a copy of the license, visit
% https://creativecommons.org/licenses/by-nc-sa/4.0/

\chapter*{Conclusion}
\addcontentsline{toc}{chapter}{Conclusion}

An operating system manages the resources and the peripherals of a computer,
such as the memory, the disk, the keyboard, the screen, etc. It also provides a
simple and safe way for processes to use them, via system calls. In this part
we built a very basic monotasking operating system, in several steps. Starting
from a basic input output system and a compiler, in the Flash1 memory bank, we
progressively built and stored in the Flash0 bank an initial, self-hosting
operating system. Then, using this system alone, we improved its kernel and
added some utility programs.

The resulting computer is much easier to use than what it was at the beginning
of this part, but can still be improved in many ways. For instance, if a
process enters in an infinite loop because of a bug, we currently have no way
to stop it other than rebooting the computer. To solve this we could use a
combination of keys, detected in the keyboard interrupt handler, to stop the
current process. Incidentally, we could also improve the keyboard driver to
support combinations of keys, such as Ctrl+C or Ctrl+V. Another possible
improvement is to format the Flash1 memory bank, no longer needed, into free
blocks added to the Flash0 file system (to get more disk space). Alternatively,
we could format the Flash1 bank with a better file system format, for instance
with hierarchical directories. We could then build a new kernel version,
supporting this new file system, and store it in this new disk. By successively
switching between the two memory banks like this, we could then improve our
kernel in arbitrary ways (in the limit of the available memory and disk space).

\subsubsection{Further readings}

Another possible improvement is to switch to a multitasking operating system,
capable of running several processes concurrently. But this is a complex task,
which impacts almost all parts of the system. To know how this can be done, or
to know more about operating systems in general, you can read one of the
following books:
\begin{itemize}
  \item ``Modern Operating Systems'' \cite{ModernOSes}. This book presents the
  fundamental concepts used in operating systems, and discusses several
  strategies to implement each aspect (processes, memory management, file
  systems, peripherals, etc). It starts with processes, discusses all the
  problems related to their concurrent execution, and presents methods to solve
  them.

  \item ``Operating Systems Design and Implementation'' \cite{Minix}. This book
  is from the same author and contains many sections also present
  in~\cite{ModernOSes}. But it has a more practical point of view and presents
  the source code of a real operating system, small enough to be included in
  the book.
\end{itemize}

You can also visit the OSDev website (\url{https://wiki.osdev.org}), which
provides many more resources about operating systems. It contains for instance
a list of operating systems, including educational ones such as MentOS
(\url{https://mentos-team.github.io/}) and xv6
(\url{https://pdos.csail.mit.edu/6.828/2023/xv6.html}), used for teaching in
universities. Finally, you can also read ``Project Oberon: the design of an
operating system and compiler'' \cite{Oberon} (a second edition is available
online), which describes a small self-hosting operating system and its compiler.


