% This work is licensed under the Creative Commons Attribution NonCommercial
% ShareAlike 4.0 International License. To view a copy of the license, visit
% https://creativecommons.org/licenses/by-nc-sa/4.0/

\tikzstyle{cv_font} = [font={\footnotesize\sffamily},color=green0]
\tikzstyle{cv_smallfont} = [font={\tiny\sffamily}]

\tikzset{
  AndGate/.pic={
    \path[draw=black] (-10, -20) -- (-10, 20) -- (0, 20)
    .. controls (12, 20) and (20, 11) .. (20, 0)
    .. controls (20, -11) and (12, -20) .. (0, -20) -- cycle;
  },
  Button/.pic={
    \path[draw=black] (13, 0) -- (30, 0);
    \path[draw=black,fill=gray1] (0, 0) circle[radius=13];
  },
  Clock/.pic={
    \path[draw=black] (-10, -10) rectangle +(20, 20);
    \path[draw=green2,line width=0.3mm] (-6, 0) -- (-6, -5.5) -- (0, -5.5) --
    (0, 5.5) -- (6, 5.5) -- (6, 0);
  },
  Demultiplexer/.pic={
    \path[draw=black] (10, -10) -- (10, 10) -- (-10, 20) -- (-10, -20) -- cycle;
    \node[cv_smallfont,anchor=center] at (-4,-10){1};
    \node[cv_smallfont,anchor=center] at (-4,10){0};
  },
  DflipFlop/.pic={
    \path[draw=black] (-20, -20) rectangle +(40, 40);
    \path[draw=black] (-20, 5) -- (-15, 10) -- (-20, 15);
  },
  DigitalLed/.pic={
    \path[draw=gray1] (-20, 0) -- (-40, 0);
    \path[fill=gray1] (-20, -9) rectangle +(20,18);
    \path[fill=gray1] (0, 0) circle[radius=9];
  },
  DigitalLedOn/.pic={
    \path[draw=gray1] (-20, 0) -- (-40, 0);
    \path[fill=red0] (-20, -9) rectangle +(20,18);
    \path[fill=red0] (0, 0) circle[radius=9];
  },
  Ground/.pic={
    \path[draw=black] (0, -10) -- (0, 0) (-10, 0) -- (10, 0)
    (-6, 5) -- (6, 5) (-3, 10) -- (3, 10);
  },
  Input/.pic={
    \path[draw=black] (-10, -10) rectangle +(20, 20);
  },
  Multiplexer/.pic={
    \path[draw=black] (10, -10) -- (10, 10) -- (-10, 20) -- (-10, -20) -- cycle;
    \node[cv_smallfont,anchor=center] at (-4,-10){0};
    \node[cv_smallfont,anchor=center] at (-4,10){1};
  },
  NandGate/.pic={
    \path[draw=black] (-10, -20) -- (-10, 20) -- (0, 20)
    .. controls (12, 20) and (20, 11) .. (20, 0)
    .. controls (20, -11) and (12, -20) .. (0, -20) -- cycle;
    \path[draw=black] (25, 0) circle[radius=5];
  },
  NorGate/.pic={
    \path[draw=black] (-10, -20) .. controls (0, 0) .. (-10, 20)
    .. controls (0, 20) and (15, 10) .. (20, 0)
    .. controls (15, -10) and (0, -20) .. cycle;
    \path[draw=black] (25, 0) circle[radius=5];
  },
  NotGate/.pic={
    \path[draw=black] (-10, -10) -- (-10, 10) -- (10, 0) -- cycle;
    \path[draw=black] (15, 0) circle[radius=5];
  },
  OrGate/.pic={
    \path[draw=black] (-10, -20) .. controls (0, 0) .. (-10, 20)
    .. controls (0, 20) and (15, 10) .. (20, 0)
    .. controls (15, -10) and (0, -20) .. cycle;
  },
  Output/.pic={
    \path[draw=blue0] (-10, -10) rectangle +(20, 20);
  },
  OutputX/.pic={
    \path[draw=red0] (-10, -10) rectangle +(20, 20);
  },
  Power/.pic={
    \path[draw=black] (0, 10) -- (0, 0) -- (10, 0) -- (0, -10)
    -- (-10, 0) -- (0, 0);
  },
  SRflipFlop/.pic={
    \path[draw=black] (-20, -20) rectangle +(40,40);
  },
  TriState/.pic={
    \path[draw=black] (-10, -15) -- (20, 0) -- (-10, 15) -- cycle;
  },
  XorGate/.pic={
    \path[draw=black] (-10, -20)
    .. controls (0, 0) .. (-10, 20)
    .. controls (0, 20) and (15, 10) .. (20, 0)
    .. controls (15, -10) and (0, -20) .. cycle;
    \path[draw=black] (-20, -20)
    .. controls (-6, -10) and (-6, 10) .. (-20, 20);
  },
}

\newcommand*{\cvmath}[1]{\normalsize $#1$}

% This work is licensed under the Creative Commons Attribution NonCommercial
% ShareAlike 4.0 International License. To view a copy of the license, visit
% https://creativecommons.org/licenses/by-nc-sa/4.0/

\chapter*{Introduction}
\addcontentsline{toc}{chapter}{Introduction}

We now have a toy programming language, hereafter called Toy, which makes it
much easier to program our computer (compared to what it was at the end of
\cref{part:computer}). In particular, we no longer need to manually keep track
of function addresses, instruction offsets, local variable indices, etc.
However, we still need to manually keep track of the flash memory content with
maps such as the one in \cref{fig:final-toyc-memory-map}. This is necessary to
find unused memory regions where new programs can be stored, without overriding
existing programs or data. Likewise, we need to manually keep track of the RAM
content with maps such as the one in \cref{fig:command-editor-memory-map}. This
is necessary to find unused RAM regions where programs can store their data,
without overriding the data of other programs. This manual work is quite
tedious and error prone. The main goal of this part is thus to implement a
program which can do it for us. Such a program is called an {\em operating
system}.

In the previous part we removed the need of function addresses and local
variable indices by using symbolic names for functions and local variables, and
by letting the compiler maintain a map between the two. We can do the same for
programs or other data stored in flash memory. More precisely, we can use {\em
file names} to refer to pieces of data stored in flash memory, hereafter called
{\em files}, and let the operating system maintain a map between file names and
flash memory addresses. This map must obviously be stored in flash memory too,
otherwise it would be lost after a reset. The precise way of storing it, as
well as the files themselves, is called a {\em file system}.

In summary, the main goal of this part is to implement a toy operating system,
based on a file system. Another goal is to eventually get rid of our bytecode
interpreter, since we can now compile programs into native code, which is much
faster than bytecode. This implies rewriting the drivers and programs from the
previous parts, written directly in binary bytecode form, into Toy source code.
We achieve these goals in seven steps, presented in as many chapters:

\begin{itemize}
\item \cref{chapter:file-system} defines of a toy file system and provides
corresponding functions to create, read, write, and delete files. We use
them at the end to initialize a new, empty file system in the Flash0 memory
bank (so far we only used the Flash1 bank -- see
\cref{fig:sam3x8e,fig:boot-memory-map}).

\item \cref{chapter:boot-loader-and-drivers} provides a new implementation, in
Toy, of the clock, graphics card, and keyboard drivers from
\cref{part:computer}. We test them at the end with a small program which simply
displays on the screen each key typed on the keyboard, as in
\cref{chapter:keyboard}. This time, however, we compile it in native code, and
we store it in our file system. We also implement a {\em boot loader} to start
it directly after reset, without going through the memory editor.

\item \cref{chapter:processes} provides functions to start and stop programs
stored in the file system, while keeping track of the RAM used by each running
program, called {\em processes}. Together with the file system and driver
functions, they constitute the first version of a program called the operating
system {\em kernel}. This chapter also provides a way for processes to use
services provided by the kernel.

\item \cref{chapter:streams} extends the kernel with new services providing a
simple, unified, and safe way to use the computer resources. These include the
keyboard, the graphics card, and the files. All these resources are
used via byte sequences called {\em streams}.

\item \cref{chapter:shell} provides a better and easier to use version of the
command editor, called a {\em command-line interpreter}, or {\em shell}. This
program is automatically started by the kernel after a reset. Its role is to
start other programs, with commands typed by the user. This chapter also
provides a new implementation of the text editor from
\cref{chapter:text-editor}, in Toy. Together with the Toy compiler and the
shell, stored in the file system, this gives an autonomous, {\em self-hosted}
operating system. This means that we can edit and recompile its entire source
code with itself, without needing the bytecode interpreter, the memory editor,
or any other program written in the previous parts.

\item \cref{chapter:memory-protection} illustrates this self-hosting property by
using the shell, the text editor, and the compiler, running with the operating
system, to replace its kernel with a new version. This new version uses the
Memory Protection Unit from the microcontroller to protect the kernel and each
program from bugs in other programs.

\item \cref{chapter:utilities} completes our operating system with a few small
utility programs, in particular to list, copy, and delete files. It also
provides a better shell version.
\end{itemize}

Finally, just for fun, we conclude this book with a small game implemented with
our toy computer, in \cref{chapter:snake-game}.

% This work is licensed under the Creative Commons Attribution NonCommercial
% ShareAlike 4.0 International License. To view a copy of the license, visit
% https://creativecommons.org/licenses/by-nc-sa/4.0/

\renewcommand{\rustfile}{chapter1}
\setcounter{rustid}{0}

\rust{
  context.write_backup("website/backups", "flash_memory_driver.txt")?;
}

\chapter{Flash Memory Driver}\label{chapter:flash-driver}

Before writing our toy compiler, as described in introduction, we need a way to
save it in flash memory. In theory, our memory editor provides everything we
need to do this. Indeed, as we have seen in the previous part, we can save a 64
words page in flash memory by writing these words at their final address, and
then by writing an appropriate value in the Enhanced Embedded Flash Controller
(EEFC) Command Register (which has a well defined address in memory). All these
steps can be done manually with the memory editor, but doing so would not be
very practical. To make this is easier we provide in this chapter a few helper
functions, called the flash memory driver. We use them at the end to store
themselves in flash memory.

\section{Overview}

Due to the flash memory usage constraints, it is not practical to directly edit
data here. Instead, what we can do is edit data in RAM, and then save it in
flash memory. The latter step can be done automatically (see
\cref{fig:flash-driver-overview}):
\begin{itemize}
\item copy the first 64 words of data from RAM to flash memory (we assume in
this chapter that data is always saved at the beginning of a flash memory page),
\item save them by writing the appropriate command in the EEFC Command Register,
\item ...
\item copy the last $n$ words of data from RAM to flash memory, and the $64-n$
remaining words from flash memory to itself (recall that we cannot save a page
without writing all its 64 words first),
\item save them by writing the appropriate command in the EEFC Command Register.
\end{itemize}

Conversely, to edit data which is already in flash memory, we can copy it in
RAM, edit it here, and then save it back. Note that these algorithms use two
kinds of steps: copying memory from one address to another, and saving a page
of flash memory. The rest of this section presents them in more detail.

\begin{Figure}
  \input{figures/chapter1/overview.tex}

  \caption{Saving data (in dark blue) from RAM into flash memory must be done
  page by page. Each 64 words page must be copied (dashed arrows) and then
  saved. The unused words of the last page (light blue) must be copied in place
  so that all the page words are written before it is saved (as required by the
  EEFC component).}\label{fig:flash-driver-overview}
\end{Figure}

\subsection{Memory copy}

As explained above, our flash memory driver needs a function to copy some data
from one address to another. Our text editor, presented in the next chapter,
also needs such a function. The former only needs to copy words, between two
distinct memory regions. But the latter needs to copy bytes, between two
regions which can overlap. To support both use cases, we present here a general
memory copy algorithm.

The basic algorithm to copy $n$ bytes starting at address $src$ to address
$dst$ is very simple. We just need to store the byte loaded from address $src +
i$ to address $dst + i$, for all $i \in [0,n[$. However, the order in which
these operations are done is important if the source and destination regions
overlap. Consider for instance the task of copying $n=10$ bytes from $src=4$ to
$dst=7$ (see \cref{fig:mem-copy}). Starting by copying the byte at $src+0$ to
$dst+0$ would override the byte at $src+3$, leading to an incorrect result. The
solution is to copy the bytes in decreasing order, from $i=n-1$ to $i=0$.
Conversely, copying $n=10$ bytes from $src=7$ to $dst=4$ must be done in
increasing order (starting by copying the byte at $src+9$ to $dst+9$ would
override the byte $src+6$). In summary, bytes must be copied in decreasing
order if $dst \ge src$, and in increasing order otherwise. Note also that we
can start by copying words, and use byte copies only for the last 1 to 3
remaining bytes, leading to \cref{alg:mem-copy}.

\begin{Algorithm}
\caption{Copying $n$ bytes from $src$ to $dst$.}\label{alg:mem-copy}
\begin{algorithmic}[1]
\Begin if $dst < src$
  \State initialize $i$ to 0
  \State while $i+4 \le n$, copy the word at $src+i$ to $dst+i$ and then
  increment $i$ by 4
  \State while $i < n$, copy the byte at $src+i$ to $dst+i$ and then increment
  $i$ by 1
\Continue otherwise
  \State initialize $i$ to n
  \State while $i \ge 4$, decrement $i$ by 4 and then copy the word at $src+i$
  to $dst+i$
  \State while $i > 0$, decrement $i$ by 1 and then copy the byte at $src+i$ to
  $dst+i$
\End
\end{algorithmic}
\end{Algorithm}

\begin{Figure}
  \input{figures/chapter1/mem-copy.tex}

  \caption{Copying 10 bytes (dark blue) from address 4 to address 7 in
  increasing order (\ie, from byte 4 to byte 13) leads to incorrect results (in
  red). Copying them in decreasing order, from byte 13 to byte 4, solves the
  problem.}\label{fig:mem-copy}
\end{Figure}

\subsection{Page flash}\label{subsection:flash-subroutine}

Once the 64 words of a page have been copied from RAM to flash memory, they can
be saved by writing the appropriate value in the EEFC Command Register. One
must then wait until the EEFC Status Register value is 1, indicating that the
operation is done. During this time, the flash memory bank must not be used
(see \cref{section:flash-controller}). Unfortunately, our virtual machine {\em
is} in flash memory. Therefore, we can't use a bytecode function to read the
EEFC Status Register (running it would require the microprocessor to read the
ARM instructions of the virtual machine, \ie, would use the flash memory). A
solution is to use a small subroutine made of ARM instructions, stored in RAM,
to save a page in flash memory without using it. This subroutine can be
implemented as follows:

\rust{
  const R0 : u32 = 0;
  const R1 : u32 = 1;
  let mut a = Assembler::new(0);
  a.push_list(&[R0, R1], true);
  a.ldr_rt_pc_imm8(R0, "command_register");
  a.ldr_rt_pc_imm8(R1, "command_value");
  a.str_rt_rn_imm5(R1, R0, 0); // stores R1 in mem[R0 + 0]
  a.label("wait_ready");
  a.ldr_rt_rn_imm5(R1, R0, 4); // loads mem[R0 + 4] into R1
  a.cmp_rn_imm8(R1, 1);
  a.if_then(Condition::NE, &[]);
  a.b_imm11("wait_ready");
  a.pop_list(&[R0, R1], true);
  a.u16_data(0, "padding, unused");
  a.label("command_register");
  a.u32_data(EEFC1_FCR.address, "EEFC1 Command Register");
  a.label("command_value");
  a.u32_data(0x5A000003, "EEFC1 Command"); // Command value (must add page << 8)
}
\rs{a.get_listing(0..a.get_instruction_count() as usize)}

It starts with a \arm{PUSH} to save the $\mathrm{R0}$ and $\mathrm{R1}$
registers, as well as the Link Register (LR). It then loads the address of the
EEFC Command Register in $\mathrm{R0}$, and the value to write into it in
$\mathrm{R1}$, with two \arm{LDR} instructions (this data is stored just after
the function itself). The next instruction actually stores this value in the
Command Register, thereby starting the flashing process. The next \arm{LDR}
instruction loads the value of the EEFC Status Register (whose address is 4
bytes after the Command Register address, \ie, $\mathrm{R0}+4$). The following
\arm{CMP} instruction compares this value with 1. If it is not equal to 1, the
\arm{B} instruction jumps back to the \arm{LDR} instruction to read the Status
Register again. Otherwise this instruction is skipped and the final \arm{POP}
restores the $\mathrm{R0}$ and $\mathrm{R1}$ registers, and returns to the
caller by moving the saved LR into the Program Counter (PC). The first data
word after that contains the address of the EEFC1 Command Register (we want to
store our compiler in the same flash memory bank as our basic input output
system). The final word is the command to write at this address in order to
flash the 0$^{th}$ page. To flash the $p^{th}$ page instead, $p$ should be put
in bytes 1 and 2 of this word (\eg, to flash the $4^{th}$ page, use
\hexa{5A000403} -- see \cref{section:flash-controller}). In summary, the
complete code of this subroutine is:

\rs{a.get_machine_code_listing(0..a.get_instruction_count() as usize)}

\subsection{Data buffers}\label{subsection:data-buffer}

In order to copy or save data we must know their address, but we must also know
their size in bytes. To avoid having to manually keep track of this size, we
can store it in memory too. One way to do this is to store it in a word located
just before the data themselves (see \cref{fig:data-buffer}). This word is
called a {\em header} or a {\em metadata} (because it is some data about other
data). In the following we call this header and its associated data a {\em data
buffer}. And we provide functions to copy and save data buffers.

\begin{Figure}
  \input{figures/chapter1/data-buffer.tex}

  \caption{A data buffer containing 6 bytes of data (light blue), starting at
  address 10, begins with a 4 bytes header (dark blue) containing the size of
  the following data.}\label{fig:data-buffer}
\end{Figure}

\section{Implementation}\label{section:flash-memory-driver-impl}

\rust{
  let driver_address = next_page_address(
      context.memory_region("memory_editor").end());
}

We can now implement the above algorithms. We do this in a data buffer, so that
our flash memory driver can save itself. This buffer must be saved at the start
of a page, as we assumed at the beginning. Lets use the next page after our
memory editor, page \rs{dec(page_number(driver_address))}, starting at
\rs{hex(driver_address)}. The code will thus start at
\rs{hex(driver_address + 4)}, after the header.

For \cref{alg:mem-copy} we need functions to load and store a single byte. We
already have a \verb!load_byte! function (see \cref{table:bios_functions}), but
we don't have a \verb!store_byte! one. We thus provide one, as follows:

\rust{
  let mut b = BytecodeAssembler::new(RegionKind::DataBuffer, driver_address);
  b.import_labels(context.memory_region("graphics_card_driver"));
}

\begin{TwoColumns}
\rs{b.func("store_byte", &["ptr", "value"], "", &[])}\\
\bytecode{
  // *ptr = (*ptr) & 0xFFFFFF00 | value
  b.get("ptr");
  b.get("ptr");
  b.load();
  b.cst(0xFFFFFF00);
  b.and();
  b.get("value");
  b.or();
  b.store();
  b.ret();
}
\end{TwoColumns}

This function loads the word at $ptr$, discards its 8 least significant bits
(while keeping the others unchanged) by computing the bitwise AND of this word
with \hexa{FFFFFF00}, replaces them with $value$ (supposed to be strictly less
than 256) with a bitwise OR, and finally stores the result back at $ptr$. We
then implement \cref{alg:mem-copy} in the following \verb!mem_copy!
function:

\begin{Paragraph}
\begin{paracol}{2}
\rs{b.func("mem_copy", &["src", "dst", "n"], "dst+n", &[])}

Step 1. If $dst \ge src$, go the second half of this function (see below).

\bytecode[switchcolumn]{
  // if dst < src
  b.get("dst");
  b.get("src");
  b.ifge("mem_copy_forward");
}

Step 2. Initialize $i$ to 0.

\bytecode[switchcolumn]{
  b.cst_0();
  b.def("i");
}

Step 3. If $i+4>n$, go to step 4 ($i$ is stored in the $7^{th}$ stack frame
slot).

\bytecode[switchcolumn]{
  b.label("mem_copy_while1");
  // while i + 4 <= n
  b.get("i");
  b.cst8(4);
  b.add();
  b.get("n");
  b.ifgt("mem_copy_while2");
}

Otherwise, load the word at $src+i$ and store it at $dst+i$, $\ldots$

\bytecode[switchcolumn]{
  // *(dst + i) = *(src + i);
  b.get("dst");
  b.get("i");
  b.add();
  b.get("src");
  b.get("i");
  b.add();
  b.load();
  b.store();
}

$\ldots$ increment $i$ by 4, and go back above to check again if $i+4<n$.

\bytecode[switchcolumn]{
  // i = i + 4;
  b.cst8(4);
  b.add();
  // end while
  b.goto("mem_copy_while1");
}

Step 4. If $i \ge n$, go the end of the function (see below).

\bytecode[switchcolumn]{
  b.label("mem_copy_while2");
  // while i < n
  b.get("i");
  b.get("n");
  b.ifge("mem_copy_end");
}

Otherwise, load the byte at $src+i$ and store it at $dst+i$, $\ldots$

\bytecode[switchcolumn]{
  // store_byte(dst + i, load_byte(src + i));
  b.get("dst");
  b.get("i");
  b.add();
  b.get("src");
  b.get("i");
  b.add();
  b.call("load_byte");
  b.call("store_byte");
}

$\ldots$ increment $i$ by 1, and go back above to check again if $i<n$.

\bytecode[switchcolumn]{
  // i = i + 1;
  b.cst_1();
  b.add();
  b.goto("mem_copy_while2");
}
\end{paracol}
\end{Paragraph}

The second half of the function is similar, and handles the case $dst \ge src$
by copying data in decreasing order, as described in \cref{alg:mem-copy}:

\begin{TwoColumns}
\bytecode{
  b.label("mem_copy_forward");
  // let i = n;
  b.get("n");
  b.label("mem_copy_while3");
  // while i >= 4
  b.get("i");
  b.cst8(4);
  b.iflt("mem_copy_while4");
  // i = i - 4;
  b.cst8(4);
  b.sub();
  // *(dst + i) = *(src + i);
  b.get("dst");
  b.get("i");
  b.add();
  b.get("src");
  b.get("i");
  b.add();
  b.load();
  b.store();
  // end while
  b.goto("mem_copy_while3");
  b.label("mem_copy_while4");
  // while i > 0
  b.get("i");
  b.cst_0();
  b.ifle("mem_copy_end");
  // i = i - 1;
  b.cst_1();
  b.sub();
  // store_byte(dst + i, load_byte(src + i));
  b.get("dst");
  b.get("i");
  b.add();
  b.get("src");
  b.get("i");
  b.add();
  b.call("load_byte");
  b.call("store_byte");
  // end while
  b.goto("mem_copy_while4");
  b.label("mem_copy_end");
}
\end{TwoColumns}

Both parts jump to the following final instructions when the copy is done.
These instructions simply return $dst+n$:

\begin{TwoColumns}
\bytecode{
  // return dst + n;
  b.get("dst");
  b.get("n");
  b.add();
  b.retv();
}
\end{TwoColumns}

Using this memory copy function, it is easy to write a function to copy a data
buffer from $src$ to $dst$. Indeed, we simply need to call \verb!mem_copy! with
$src$, $dst$, and $n=\mathrm{mem32}[src]+4$, the total size of the data buffer
(recall that $\mathrm{mem32}[x]$ means ``the 32 bit value at address $x$''):

\begin{TwoColumns}
\rs{b.func("buffer_copy", &["src", "dst"], "", &[])}\\
\bytecode{
  // mem_copy(src, dst, *src + 4);
  b.get("src");
  b.get("dst");
  b.get("src");
  b.load();
  b.cst8(4);
  b.add();
  b.call("mem_copy");
  b.ret();
}
\end{TwoColumns}

We can now implement a function to copy and save a single page in flash memory.
As described above, to save $n \le 256$ bytes, we must first copy them, then
copy the remaining $256-n$ bytes of the page in place, and finally save the
page by calling the subroutine defined in \cref{subsection:flash-subroutine}.
For this, the subroutine must be stored somewhere in RAM first. The easiest
solution is to store it on the stack. The following function uses this method
to save $n$ bytes starting from $src$ in a page of the Flash1 memory bank
specified by its $page$ index:

\begin{Paragraph}
\begin{paracol}{2}
\rs{b.func("page_flash", &["src", "page", "n"], "", &[])}

If $n=0$ there is nothing to do, return right away. Otherwise execute the
following instructions.

\bytecode[switchcolumn]{
  // if n == 0  return;
  b.get("n");
  b.cst_0();
  b.ifne("page_flash_non_empty");
  b.ret();
  b.label("page_flash_non_empty");
}

Copy $n$ bytes from $src$ to $\hexa{C0000}+256.page$, the address of the
$page^{th}$ page of the Flash1 memory bank. The \verb!mem_copy! call returns
$dst=\hexa{C0000}+256.page+n$, in the $7^{th}$ stack frame slot.

\bytecode[switchcolumn]{
  // let dst = mem_copy(src, (0xC0000 as *u32) + (page << 8), n);
  b.get("src");
  b.cst(0xC0000);
  b.get("page");
  b.cst8(8);
  b.lsl();
  b.add();
  b.get("n");
  b.callr("mem_copy");
  b.def("dst");
}

Copy the remaining $256-n$ bytes of the page in place, from $dst$ to $dst$, and
discard the result returned by \verb!mem_copy!.

\bytecode[switchcolumn]{
  // mem_copy(dst, dst, 256 - n);
  b.get("dst");
  b.get("dst");
  b.cst(0x100);
  b.get("n");
  b.sub();
  b.callr("mem_copy");
  b.pop();
}

Disable the USART interrupts with the Nested Vector Interrupt Controller (see
\cref{section:nvic} and below).

\bytecode[switchcolumn]{
  // *(NVIC_ICER0 as *u32) = 0x20000;
  b.cst(NVIC_ICER0);
  b.cst(0x20000);
  b.store();
}

Push the value to store in the EEFC1 Command Register in order to save the
$page^{th}$ page: $\hexa{5A000003}\ |\ (page \ll 8)$.

\bytecode[switchcolumn]{
  // let command = 0x5A000003 | page << 8;
  let asm_words = a.machine_code();
  b.cst(asm_words[asm_words.len() - 1]);
  b.get("page");
  b.cst8(8);
  b.lsl();
  b.or();
  b.def("command");
  b.comment("");
}

Push the remaining words of the subroutine to save this page. These words must
be pushed in reverse order, because each word is pushed 4 bytes {\em before}
the previous one.

\bytecode[switchcolumn]{
  for i in (0..asm_words.len() - 1).rev() {
    b.cst(asm_words[i]);
    b.def(&if i == 0 { String::new() } else { format!("word{i}") });
    b.comment("");
  }
}

Call the subroutine, which starts in the
\rs{dec(7 + asm_words.len() as u32)}$^{th}$ stack frame slot. Its {\em
interworking address} is the address of this slot (given by the \insn{ptr}
instruction), plus 1.

\bytecode[switchcolumn]{
  b.ptr("");
  b.cst_1();
  b.add(); // add one for interworking address
  b.blx();
}

Re-enable the USART interrupts and return.

\bytecode[switchcolumn]{
  // *(NVIC_ISER0 as *u32) = 0x20000;
  b.cst(NVIC_ISER0);
  b.cst(0x20000);
  b.store();
  b.ret();
}
\end{paracol}
\end{Paragraph}

\noindent A few things should be noted:
\begin{itemize}
\item USART interrupts are temporarily disabled while the page is being saved.
Without this, a key press or release during this time would run the
\verb!keyboard_handler!, which would make use of the flash memory. In turn,
this would cause a Hard Fault because flash memory must not be used while a
page is being saved. Unfortunately, flashing a page takes a few milliseconds,
during which several interrupts could occur. In this case they are lost, except
the last one, which can confuse the keyboard driver. For instance, releasing
the ``r'' key causes two interrupts, for the \hexa{F0} and \hexa{2D} scancodes.
If the first is lost, this appears as a key press (see
\cref{appendix:scancodes}). This problem disappears in the next part.

\item we use \insn{callr} instead of \insn{call} instructions to call
\verb!mem_copy!. The next section explains why (these instructions use an
offset from their own address, instead of an offset from \hexa{C0000} -- see
\cref{subsection:fn-instructions}).

\item $n$ must be a multiple of 4, so that \verb!mem_copy! does not call
\verb!store_byte!. Indeed, \verb!store_byte! does not work in flash memory,
because loads do not ``see'' the effect of stores until the page is saved (see
\cref{subsection:page-write}). If it was called several times to store the
bytes of a word, only the last call would have any effect.
\end{itemize}

We can finally implement the last function of our flash memory driver, which
copies a data buffer starting at $src$ and saves it in the Flash1 memory bank,
starting at the $page^{th}$ page. This function simply calls \verb!page_flash!
for each page.

\begin{Paragraph}
\begin{paracol}{2}
\rs{b.func("buffer_flash", &["src", "page"], "", &[])}

Compute the number of bytes $n$ to copy. This is $\mathrm{mem32}[src]+4$,
rounded upwards to a multiple of 4 (as required by \verb!page_flash!), \ie,
$(\mathrm{mem32}[src]+7)\ \wedge$ \hexa{FFFFFFFC}.

\bytecode[switchcolumn]{
  // let n = (*src + 7) & 0xFFFFFFFC;
  b.get("src");
  b.load();
  b.cst8(7);
  b.add();
  b.cst(0xFFFFFFFC);
  b.and();
  b.def("n");
}

If $n$, in the $6^{th}$ stack frame slot, is greater than 255, jump to the next
instructions. Otherwise, call \verb!page_flash! to copy and save $n$ bytes from
$src$ into $page$, and return.

\bytecode[switchcolumn]{
  b.label("buffer_flash_loop");
  // if n <= 255
  b.get("n");
  b.cst8(255);
  b.ifgt("buffer_flash_full_page");
  // page_flash(src, page, n); return;
  b.get("src");
  b.get("page");
  b.get("n");
  b.callr("page_flash");
  b.ret();
  b.label("buffer_flash_full_page");
}

Call \verb!page_flash! to copy and save $256$ bytes from $src$ into $page$.

\bytecode[switchcolumn]{
  // page_flash(src, page, 256);
  b.get("src");
  b.get("page");
  b.cst(0x100);
  b.callr("page_flash");
}

Increment $src$ by 256.

\bytecode[switchcolumn]{
  // src = src + 256;
  b.get("src");
  b.cst(0x100);
  b.add();
  b.set("src");
}

Increment $page$ by 1.

\bytecode[switchcolumn]{
  // page = page + 1;
  b.get("page");
  b.cst_1();
  b.add();
  b.set("page");
}

Decrement $n$ by 256 and go back above to copy the rest of the data buffer.

\bytecode[switchcolumn,bigskip]{
  // n = n - 256;
  b.cst(0x100); // 55
  b.sub();
  // end loop
  b.goto("buffer_flash_loop");
}
\end{paracol}
\end{Paragraph}

In summary, the main functions of our flash memory driver are those listed in
\cref{table:flash_driver_functions}, and its full code is the following:

\rs{b.get_bytecode_listing(0..b.get_instruction_count() as usize, false)}

\rust{
  let mut commands = Vec::new();
  commands.extend(b.boot_assistant_commands());
  commands.push(String::from("flash#"));
  commands.push(String::from("reset#"));
  write_lines("website/part3", "flash_memory_driver.txt", &commands)?;
}

\begin{Table}
  \begin{tabular}{|l|l|} \hline
    \makecell{\thead{Function}} & \thead{Address} \\ \hline
    \rs{MemoryRegion::labels_table_rows(vec![&b.memory_region()])} \\ \hline
  \end{tabular}

  \caption{The most important functions of the flash memory
    driver.}\label{table:flash_driver_functions}
\end{Table}

\section{Storage}\label{section:flash-driver-storage}

\rust{
  const RAM_ADDRESS: u32 = 0x20070000;
  let mut context1 = context.clone();
}

Lets store our driver in flash memory. For this we must first enter it in RAM,
say at address \rs{hex(RAM_ADDRESS)}, and then save it by calling the
\verb!buffer_flash! function. In the memory editor, type
``w\rs{hex_word_low(RAM_ADDRESS)}''+Enter, and then store the size of our
driver at this address by typing
``\rs{hex_word_low(b.bytecode_size())}''+Enter. Continue by entering each word
of the driver code, listed above, by typing its value followed by Enter.

\rust{
  let display = Rc::new(RefCell::new(TextDisplay::default()));
  context.set_display(display.clone());

  context.add_memory_region("flash_driver", b.memory_region());
  context.micro_controller().borrow_mut().reset();
  context.run_until_get_char();

  context.type_ascii(&b.memory_editor_commands(RAM_ADDRESS));
}

Our driver is now in RAM. To save it in flash memory we must
call \verb!buffer_flash!, at address
$\rs{hex(b.label_address("buffer_flash"))} - \rs{hex(driver_address)} +
\rs{hex(RAM_ADDRESS)}$, with $src=\rs{hex(RAM_ADDRESS)}$ and
$page=\rs{dec(page_number(driver_address))}$. This can be done with the
following function:

\rust{
  let mut c = BytecodeAssembler::default();
}
\begin{TwoColumns}
\rs{c.func("save_driver", &[], "", &["nolink"])}\\
\bytecode{
  c.cst(RAM_ADDRESS);
  c.cst8(page_number(driver_address).try_into().unwrap());
  c.cst(b.label_address("buffer_flash") - driver_address + RAM_ADDRESS);
  c.calld();
  c.ret();
}
\end{TwoColumns}

Note that the call to \verb!buffer_flash! causes indirect calls to
\verb!page_flash! and \verb!mem_copy! which, for now, are in RAM. Hence, the
instructions calling these functions cannot use their final address in flash
memory, since they are not stored there yet! This is why we used \insn{callr}
instructions instead of \insn{call} instructions in the above code. Indeed, by
specifying the callee with an offset from the caller, the code works wherever
it is stored, in RAM or in flash memory. Such code is called {\em position
independent code}. The full code of the above function is the following:

\rs{c.get_bytecode_listing(0..c.get_instruction_count() as usize, false)}

\rust{
  const COMMAND_ADDRESS: u32 = 0x20080000;
}

\noindent With the memory editor, enter these values in an unused RAM region,
for instance starting at address \rs{hex(COMMAND_ADDRESS)}. Then type
``w\rs{hex_word_low(COMMAND_ADDRESS)}''+Enter, followed by ``r'', to run this
function. The driver should now be saved in flash memory. To check this, type
``w\rs{hex_word_low(driver_address)}''+Enter. You should see the following
screen, displaying the same words as those listed above, after the data buffer
header:

\rust{
  context.type_ascii(&c.memory_editor_commands(COMMAND_ADDRESS));
  context.type_ascii(&format!("W{:08X}\n", COMMAND_ADDRESS));
  context.type_ascii("R");
  context.type_ascii(&format!("W{:08X}\n", driver_address));
}

\rust{
  let driver_display = display.borrow().get_text();
  let first_line = driver_display.lines().next().unwrap();
  let second_line = driver_display.lines().nth(1).unwrap();
  assert!(first_line.ends_with(&format!("{:08X} {:08X} {:08X}",
      context.memory_region("flash_driver").words[0],
      context.memory_region("flash_driver").len - 4,
      driver_address)));
}

\begin{Paragraph}
\rs{med_page_row(first_line)}\\
\rs{med_page_row(second_line)}\\
{\tt ...}
\end{Paragraph}

\rust{
  let display1 = Rc::new(RefCell::new(TextDisplay::new()));
  context1.set_display(display1.clone());
  context1.micro_controller().borrow_mut().reset();
  context1.run_until_get_char();

  let boot_mode_address = context.memory_region("foundations")
      .label_address("boot_mode_select_rom");
}

Alternatively, if something went wrong or if you don't want to enter all the
driver code with the keyboard, you can ``cheat'' by saving it via an external
computer, as follows. First run the \verb!boot_mode_select_rom! function by
typing ``w\rs{hex_word_low(boot_mode_address)}''+Enter, followed by ``r''. Then
reset the Arduino and, on the host computer, run the following commands to
flash the driver code and reset the Arduino again:

\rust{
  context1.type_ascii(&format!("W{:08X}\n", boot_mode_address));
  context1.type_ascii("R");
  context1.micro_controller().borrow_mut().reset();
  let mut flash_helper = FlashHelper::from_file(
    context1.micro_controller(), "website/", "part3/flash_memory_driver.txt")?;
}
\rs{host_log(&flash_helper.read())}

Finally, on the Arduino, type ``w\rs{hex_word_low(driver_address)}''+Enter to
check that the driver is indeed in flash memory: you should see the same screen
as above.

\rust{
  context1.run_until_get_char();
  context1.type_ascii(&format!("W{:08X}\n", driver_address));
  let driver_display1 = display1.borrow().get_text();
  let first_line1 = driver_display1.lines().next().unwrap();
  assert_eq!(first_line1, first_line);
}

% This work is licensed under the Creative Commons Attribution NonCommercial
% ShareAlike 4.0 International License. To view a copy of the license, visit
% https://creativecommons.org/licenses/by-nc-sa/4.0/

\chapter[Logic Gates and Arithmetic Circuits]
  {Logic Gates and\\ Arithmetic Circuits}\label{chapter:logic-gates}

As explained in the previous chapter, computing arithmetic operations on binary
numbers boils down to the computation of simple logical operations such as
conjunctions and exclusive disjunctions. This chapter explains how these
operations can be implemented with electric circuits, and then how these
circuits can be combined to perform arithmetic operations.

\section{Transistors}

In order to implement a logical operation with an electric circuit we first
need a way to represent 0 and 1 with some electric states. One possibility is
to view a wire connected to the ground as 0, and a wire connected to the power
source (\eg, +5V) as 1. To implement a circuit for $\neg p$, for instance,
we can use an input wire for $p$, and an output wire for the result $\neg p$.
The circuit in the middle must then connect the output wire to the ground
(resp. power source) if the input wire is connected to the power source (resp.
ground). A simple switch can do this, provided it is controlled by the input
wire, instead of manually. In fact, as illustrated in the next sections, such
switches are sufficient to implement any logical operation.

An electric switch itself controlled by electricity connects or disconnects two
terminals, hereafter noted $a$ and $b$, depending on the voltage or current in
a third one, noted $c$. One method to do this is to use a {\em transistor}.
Another method is to use a {\em relay}. Transistors are much more efficient
than relays, and are used virtually everywhere. But relays are simpler to
understand and, for this reason, we use them in this chapter to explain how
logic gates work.

A relay can be built with an electromagnet controlling a metallic connector.
There are two types of relays connecting or disconnecting two terminals (see
\cref{fig:relays}):
\begin{itemize}
  \item in a {\em normally open} relay, the $a$ and $b$ terminals are
  disconnected when no current is flowing through the electromagnet. They are
  connected when the relay is {\em active}, \ie, when there is a current in the
  electromagnet.

  \item in a {\em normally closed} relay, the $a$ and $b$ terminals are
  connected when the relay is {\em inactive}, \ie, when there is no current in
  the electromagnet. They are disconnected when it is active.
\end{itemize}

In the following we represent relays with the symbols illustrated in
\cref{fig:relay-symbols}. We draw input terminals connected to the ground
(resp. power source) with a 0 (resp. 1) inside a black square. Similarly, we
use a 0 (resp. 1) inside a blue square for output terminals connected to the
ground (resp. power source). We represent those which are not connected to
anything with an X inside a red square. Finally, we draw wires connected to the
ground, to the power source, or to nothing in blue, green, and red,
respectively (see \cref{fig:relay-symbols}).

\begin{Figure}
  \input{figures/chapter2/relays.tex}

  \caption{The two types of relays used in this chapter. The electromagnet,
  when active (right), attracts a metallic piece. This connects the $a$ and $b$
  terminals of a normally open relay (top), and disconnects those of a normally
  closed one (bottom). When the electromagnet inactive, a spring moves the
  metallic piece away from it.}\label{fig:relays}
\end{Figure}

\begin{Figure}
  \input{figures/chapter2/relay-symbols.tex}

  \caption{The symbols and colors used for normally closed (left) and normally
  open (right) relays, as well as for wires and input (black) and output (blue)
  terminals connected to the ground, to the power source (up triangle), or to
  nothing (in red).}\label{fig:relay-symbols}
\end{Figure}

\section{Logic gates}

\subsection{NOT}

A {\em NOT gate} is a circuit implementing the logical not operation. This gate
can be built with two relays controlled by the same input\footnote{In practice,
with electromagnet relays, 0 can be represented with a terminal connected to
the ground or to nothing. Then a single normally closed relay is sufficient to
build a NOT gate \cite{MerciaRelay}. In this chapter we do as if it was not the
case. This leads to circuits which are much closer to those built with the most
common technology, namely Complementary Metal Oxide Semi-conductors (CMOS).}.
The first, normally closed, connects the output to the power source by default.
The second, normally open, connects the output to the ground when active (see
\cref{fig:not-gate}). Hence, when the input is 0 the first relay connects the
output to the power source, \ie, sets it to 1 (while the second does nothing).
Conversely, when the input is 1, the first relay has no effect but the second
connects the output to the ground, \ie, sets it to 0 (see \cref{fig:not-gate}).

\begin{Figure}
  \input{figures/chapter2/not-gate.tex}

  \caption{The two possible states of the NOT gate.}\label{fig:not-gate}
\end{Figure}

\subsection{AND and NAND}

\begin{Figure}
  \input{figures/chapter2/and-gate.tex} \\
  \medskip
  \input{figures/chapter2/nand-gate.tex}

  \caption{The four possible states of the AND (top) and NAND (bottom)
    gates.}\label{fig:and-gate}
\end{Figure}

The {\em AND gate} is a circuit implementing the logical and operation. This
circuit must connect the output to the power source when both inputs are 1.
This can be done with two normally open relays connected in series. Conversely,
this gate must connect the output to the ground when at least one input is 0.
This can be done with two normally closed relays connected in parallel (see
\cref{fig:and-gate}).

The {\em NAND gate} implements the negation of the logical and, \ie, it
computes $\neg (p \wedge q)$. It can be obtained by connecting a NOT gate to
the output of an AND gate. But a simpler method is to switch the power source
and the ground of the AND gate or, equivalently, the upper and lower halves of
this circuit\footnote{With the CMOS technology ``normally closed'' (resp.
``open'') transistors are only used in the upper (resp. lower) half of a gate.
Hence a CMOS AND gate is built with a NAND gate followed by a NOT.} (see
\cref{fig:and-gate}).

\subsection{OR and NOR}

The {\em OR gate} is a circuit implementing the logical or operation. This
gate must connect the output to the power source when at least one input is
1. This can be done with two normally open relays connected in parallel.
Conversely, this gate must connect the output to the ground when both inputs
are 0. This can be done with two normally closed relays connected in series
(see \cref{fig:or-gate}).

The {\em NOR gate} implements the negation of the logical or, \ie, it computes
$\neg (p \vee q)$. As the NAND gate, it can be obtained by switching the upper
and lower halves of the OR gate circuit (see \cref{fig:or-gate}).

\begin{Figure}
  \input{figures/chapter2/or-gate.tex} \\
  \medskip
  \input{figures/chapter2/nor-gate.tex}

  \caption{The four possible states of the OR (top) and NOR (bottom)
  gates.}\label{fig:or-gate}
\end{Figure}

\subsection{XOR}

The {\em XOR gate} implements the exclusive or operation. The result of $p
\oplus q$ is 1 when $p$ is 1 and $q$ is 0, or when $p$ is 0 and $q$ is 1. This
gate must thus connect the output to the power source if at least one these two
cases happens. This can be done with two sub circuits, one for each case,
connected in parallel. Each sub circuit must connect its output to the power
source when both inputs have a specific value. This can be done with two relays
connected in series: a normally open for $p$ or $q$, and a normally closed for
$\neg p$ or $\neg q$.

Conversely, the result of $p \oplus q$ is 0 when both ``$p$ is 0 or $q$ is 1''
and ``$p$ is 1 or $q$ is 0'' are true. The same reasoning as above leads to two
sub circuits connected in series, where each sub circuit uses two relays
connected in parallel. This lead to the final circuit shown in
\cref{fig:xor-gate}.

In the following, to simplify figures and to make it easier to distinguish each
logic gate, we represent them with their American National Standards Institute
(ANSI) symbols, shown in \cref{fig:gate-symbols}.

\begin{Figure}
  \input{figures/chapter2/xor-gate.tex}

  \caption{The four possible states of the XOR gate.}\label{fig:xor-gate}
\end{Figure}

\begin{Figure}
  \input{figures/chapter2/gate-symbols.tex}

  \caption{The ANSI symbols of the NOT, AND, NAND, OR, NOR and XOR logic
  gates.}\label{fig:gate-symbols}
\end{Figure}

\section{Multiplexers and demultiplexers}

Logic gates can be assembled to create more and more complex circuits. A simple
example is the {\em demultiplexer}, shown below, and represented with the
symbol on the right:

\begin{center}
  \input{figures/chapter2/demux.tex}
\end{center}

This circuit copies its input $i$ to the $o_s$ output, \ie, to $o_0$ if $s=0$
or to $o_1$ if $s=1$. It sets the other to 0. It can be viewed as a ``railroad
switch'' for signals. The {\em multiplexer}, shown below and represented with
the symbol on the right, does the opposite:

\begin{center}
  \input{figures/chapter2/mux.tex}
\end{center}

This circuit sets its output $o$ to the $i_s$ input, \ie, to $i_0$ if $s=0$, or
to $i_1$ if $s=1$.

\section{Arithmetic circuits}

\subsection{Adder}\label{subsection:adder-circuit}

\begin{Figure}
  \input{figures/chapter2/adder.tex}

  \caption{A circuit to add two 4-bit numbers (bottom) can be built with 4
    full-adder circuits (top right), each made of two half-adders (top left) and
    an OR gate. Here this circuit computes $0111_2+0011_2=1010_2$
    ($7+3=10$).}\label{fig:adder}
\end{Figure}

As shown in the previous chapter, the addition of two bits is simply their
exclusive disjunction, with a carry equal to their conjunction. In other words,
we can add two bits with an XOR gate, plus an AND gate for the carry. The
resulting circuit, called a {\em half-adder}, is illustrated in
\cref{fig:adder}.

As explained in \cref{subsection:binary-add}, adding two binary numbers $a$ and
$b$ must be done step by step, from right to left. At each step, one bit $a_i$
from $a$ must be added to one bit $b_i$ from $b$, and to the carry $c_{i-1}$
from the previous step. In other words, three bits must be added at each step,
but the above circuit can only add two. The solution is to connect two copies
of it: a first copy adds $a_i$ and $b_i$, and a second adds $c_{i-1}$ to the
result of the first. Each copy produces a new carry, but at most one of these
can be 1. Indeed, if $a_i+b_i$ gives a carry then the second stage necessarily
adds $c_{i-1}$ to $0$, which cannot give a carry. Hence the new carry $c_i$
resulting from $a_i+b_i+c_{i-1}$ can be computed with a disjunction of the
carries from the two half-adders. This leads to the {\em full adder} circuit
shown in \cref{fig:adder}.

Finally, to add two binary numbers with $n$ bits each, we simply need to
connect $n$ full-adder circuits, with the output carry $c_i$ of step $i$
connected to the input carry $c_{i-1}$ of step $i+1$ (see \cref{fig:adder}).

\subsection{Subtractor}

\begin{Figure}
  \input{figures/chapter2/subtractor.tex}

  \caption{A circuit to subtract two 4-bit numbers (bottom) can be built with 4
    full-subtractor circuits (top right), each made of two half-subtractors (top
    left) and an OR gate. Here this circuit computes $1010_2-0111_2=0011_2$
    ($10-7=3$).}\label{fig:subtractor}
\end{Figure}

Subtracting two binary numbers can be done with a very similar circuit. As
shown in the previous chapter, subtracting a bit $b$ from $a$ gives their
exclusive disjunction (as their addition), plus a carry equal to the
conjunction of $\neg a$ and $b$ (versus of $a$ and $b$ for an addition). In
other words, a circuit to subtract $b$ from $a$ can be obtained by adding a NOT
gate in a half-adder circuit. The result, called a {\em half-subtractor}, is
illustrated in \cref{fig:subtractor}.

Subtracting two binary numbers $a$ and $b$ must be done step by step, from
right to left. At each step, one bit $b_i$ from $b$, and the carry $c_{i-1}$
from the previous step, must be subtracted from a bit $a_i$ from $a$. In other
words, three bits must be subtracted at each step, but the above circuit can
only subtract two. The solution is to connect two copies of it: a first copy
subtracts $b_i$ from $a_i$, and a second subtracts $c_{i-1}$ from the result of
the first. Each copy produces a new carry, but at most one of these can be 1.
Indeed, if $a_i-b_i$ gives a carry then the second stage necessarily subtracts
$c_{i-1}$ from $1$, which cannot give a carry. Hence the new carry $c_i$
resulting from $a_i-b_i-c_{i-1}$ can be computed with a disjunction of the
carries from the two half-subtractors. This leads to the {\em full subtractor}
circuit shown in \cref{fig:subtractor}.

Finally, to subtract two binary numbers with $n$ bits each, we simply need to
connect $n$ full subtractor circuits, with the output carry $c_i$ of step $i$
connected to the input carry $c_{i-1}$ of step $i+1$ (see
\cref{fig:subtractor}).

\subsection{Arithmetic and Logic Unit}

\begin{Figure}
  \input{figures/chapter2/alu.tex}

  \caption{A simple Arithmetic Unit which can perform additions, subtractions,
    and comparisons of $4$-bit numbers.}\label{fig:alu}
\end{Figure}

As shown in \cref{subsection:binary-add}, multiplying two binary numbers $a$
and $b$ boils down to additions of left shifted copies of $a$, each multiplied
by a bit of $b$. Furthermore, $a_i * b_j$ gives the same result as $a_i \wedge
b_j$. Hence a circuit to multiply two $n$-bit binary numbers (yielding $2n$
bits) can be obtained with $n$ copies of an $n$-bit adder, plus $n^2$ AND gates
to compute the $a_i \wedge b_j$ terms.

Comparing two $n$-bit binary numbers is also easy to do. Indeed:
\begin{itemize}
  \item $a=b$ if and only if the $n$ least significant bits of $a-b$ are equal
  to 0.

  \item $a>b$ if and only if at least one of the $n$ least significant bits of
  $a-b$ is different from 0, and if there is no carry in the $n^{th}$ column
  (counting from 0).

  \item $a<b$ if and only if at least one of the $n$ least significant bits of
  $a-b$ is different from 0, and if there {\em is} a carry in the $n^{th}$
  column (counting from 0).
\end{itemize}
Hence a subtractor circuit, plus a another computing whether its output
(excluding the carry) is different from 0, is sufficient to compare two numbers.

Finally, circuits computing bitwise logical operations on $n$-bit numbers are
trivial to implement. Indeed, we just need $n$ copies of the corresponding
logic gate, each computing one bit of the result, independently of the others
(\ie, in parallel).

All these circuits can be put together into a larger circuit called an {\em
Arithmetic and Logic Unit}. Such a circuit accepts two binary numbers as input,
plus a third one specifying an operation to perform on them. It produces as
output the result of this operation, on the given numbers.

For instance, a very simple Arithmetic ``and Logic'' Unit which can only
perform additions, subtractions, and comparisons is shown in \cref{fig:alu}.
If its {\tt subtract} input is 1 it subtracts its two 4-bit inputs. Otherwise
it adds them. For this it uses a subtractor circuit where the NOT gates are
replaced with XOR gates, connected to the {\tt subtract} input. When this input
is 0, the XOR gates behave as a simple wire ($p \oplus 0 = p$), which gives an
adder circuit. When {\tt subtract} is 1 these gates behave as NOT gates ($p
\oplus 1 = \neg p$), yielding a subtractor. Finally, three OR gates compute
whether at least one bit of the output is 1. Together with the carry bit, this
can be used to compare the inputs, as explained above.

To conclude this chapter, it should be noted that a relay takes some time to
switch between its active and inactive states (because its moving metallic
piece cannot move instantly). This is the case for transistors too.
Consequently, the output of a logic gate does not change instantly when its
inputs change. And this is the same for all circuits. The more logic gates
there is between an input and an output, the longer it takes for an input
change to {\em propagate} to the output. These propagation delays must be taken
into account in some circuits, including some presented in the next chapters.

% This work is licensed under the Creative Commons Attribution NonCommercial
% ShareAlike 4.0 International License. To view a copy of the license, visit
% https://creativecommons.org/licenses/by-nc-sa/4.0/

\renewcommand{\rustfile}{chapter3}
\setcounter{rustid}{0}

\rust{
  context.write_backup("website/backups", "opcodes_compiler.txt")?;
}

\chapter{Opcodes Compiler}\label{chapter:opcodes-compiler}

We now have everything we need to implement our compiler. We start in this
chapter with a very simple version, whose main role is to convert opcode names
into their numerical value. Indeed, this initial compiler must be written in
binary form, and should thus be as small as possible, in order to simplify our
task. Hopefully, this is the last program we need to write in such form
(besides a few small functions to launch programs with the memory editor). We
use it at the end to write a command editor, namely a small program to make it
easier to run other programs.

\section{Requirements}\label{section:toyc0-requirements}

The goal of our initial compiler is to convert opcode instructions from textual
to binary form. For instance, given the text ``\insn{fn} \insn{1} \insn{get}
\insn{0} \insn{cst\_0} \insn{ifne} \insn{10} $\ldots$'', it should produce, in
increasing address order, \hexa{19} \hexa{01} \hexa{16} \hexa{00} \hexa{00}
\hexa{10} \hexa{000A} $\ldots$ The programs that it should accept as input can
be described as ``zero or more instructions, one after the other'', where each
instruction is one of ``\insn{cst\_0}'', ``\insn{cst\_1}'', ``\insn{cst8}''
followed by an 8-bit value, ``\insn{cst}'' followed by a 32-bit value, and so
on for the remaining opcodes. These rules define the {\em grammar} of a {\em
programming language}, that valid programs must follow. They can be summarized
with:

\begin{Paragraph}
program: instruction*\\
instruction: ``\insn{cst\_0}'' | ``\insn{cst\_1}'' | ``\insn{cst8}'' INTEGER |
``\insn{cst32}'' INTEGER | $\ldots$
\end{Paragraph}

\noindent where ``*'' means ``zero or more times'' and ``|'' means ``or''. Text
between quotes, as well as names in capital letters, refer to individual
``words'' or ``punctuation signs'' of the program, called {\em tokens}. As in
English, tokens are generally separated by spaces. Here INTEGER designates an
integer value, \ie, a token made of one or more decimal digit characters. In
this context, we define the precise requirements of our initial compiler as
follows:
\begin{itemize}
  \item The compiler should take as input a source code address, noted
  $\it{src\_buffer}$, and a destination address where to store the compiled
  code, noted $\it{dst\_buffer}$.

  \item The source code should be in a data buffer (see
  \cref{subsection:data-buffer}), and should follow the above grammar.

  \item The compiled code must be produced in a data buffer. It must be the
  binary form of the bytecode instructions provided as input.

  \item The compiler should return 0 if the compilation was successful, and a
  non-zero value otherwise. In the latter case, the location of the error in
  the source code should be stored at the $\it{dst\_buffer}$ address.
\end{itemize}

Many errors could occur in the source code, such as an undefined opcode name
(``cst\_3''), an opcode without argument followed by integer value (``cst\_0
10''), an opcode with argument not followed by an integer value (``cst8
cst8''), an opcode with an 8-bit argument followed by an integer greater than
255, a jump instruction opcode followed by an invalid instruction offset, etc.
To simplify our task, in this chapter, we only require the detection of (most
of the) undefined opcode names.

\section{Algorithms}

A compiler can generally be divided in at least 3 parts: a {\em scanner}, a {\em
parser}, and a {\em backend} (see \cref{fig:compiler-parts}). The scanner reads
the source code and extracts its individual tokens. The parser calls the scanner
to read tokens, and checks that they follow the programming language's grammar.
The backend provides functions to generate the compiled code. In very simple
compilers such as ours, the parser uses the backend to directly produce the
compile code, while analyzing the source code.

Our compiler uses 5 main variables, shown in \cref{fig:compiler-parts}. Besides
$\it{src\_buffer}$ and $\it{dst\_buffer}$, already defined, the most
important ones are $src$ and $dst$. $src$ points to the next character to read.
$dst$ points to the next byte where compiled code must be written. Finally,
$src\_end$ points to the next byte after the end of the source code. When $src$
reaches $src\_end$ the whole program has been read and the compiler returns.

\begin{Figure}
  \input{figures/chapter3/compiler-parts.tex}

  \caption{The 3 parts of our compiler (top) and its 2 data structures
  (bottom), here with 3 tokens of a 15 bytes program already read (left) and
  compiled (right).}\label{fig:compiler-parts}
\end{Figure}

\bigskip \paragraph*{Scanner} The scanner splits the source code in tokens,
detects invalid tokens, and returns some data about each token. For instance, it
should detect that ``\insn{cst\_3}'' is invalid, and it could return 42 for the
token ``\insn{42}'' (\hexa{34}\hexa{32} in ASCII). To simplify, in this chapter,
we move the error detection in the backend. A token is then any sequence of
characters which does not contain a space, a tabulation, or a ``new line''. To
compute the numerical value $v$ of an integer token ``$c_{n-1}\ldots c_1c_0$'',
we can initialize $v$ to 0 and update $v$ to $10v+(c_i-\hexa{30})$, for each
character $c_i$ from left to right. In fact, to simplify the initial compiler,
the scanner returns such a value for {\em all} tokens. For instance, for the
``\insn{fn}'' token (\hexa{66}\hexa{6E} in ASCII), it returns
$10(\hexa{66}-\hexa{30})+(\hexa{6E}-\hexa{30})=602$. In summary, a token is read
as described in \cref{alg:scanner0}, which also corresponds to the finite state
machine in \cref{fig:toyc0-automaton}.

\begin{Algorithm}
\caption{Reading a token and returning its value $v$.}\label{alg:scanner0}
\begin{algorithmic}[1]
\Begin while $src<src\_end$ and the character at $src$ is a space, tab or ``new
line''
  \State increment $src$ by 1 to skip this character
\End
\State if $src=src\_end$ return nothing
\State initialize $v$ to 0
\Begin while $src<src\_end$ and the character $c$ at $src$ is not a space, tab
or ``new line''
  \State update $v$ to $10v+(c-\hexa{30})$
  \State increment $src$ by 1
\End
\State return $v$
\end{algorithmic}
\end{Algorithm}

\begin{Figure}
  \newcommand\vsp[1][.75em]{%
    \makebox[#1]{%
      \kern.07em
      \vrule height.3ex
      \hrulefill
      \vrule height.3ex
      \kern.07em
    }%
  }
  \input{figures/chapter3/automaton.tex}

  \caption{The scanner and parser can be modeled with Finite State Machines (see
  \cref{subsection:keyboard-driver-design}) reading characters $c$ (left) and
  token values $v$ (right), respectively. \vsp~represents a space, tab or ``new
  line''. Many parser transitions are not shown.}\label{fig:toyc0-automaton}
\end{Figure}

\bigskip \paragraph*{Backend} The backend provides functions to write opcodes
and their arguments in the output buffer. Here we mostly need functions to
write 8-bit and 16-bit values in memory, plus some code to detect invalid
opcodes. Valid opcodes are between 0 and 31 included, but we also add here a
pseudo opcode \insn{d} (for ``data'', with value 32), with an 8-bit argument
$x$. Once compiled, a \insn{d} $x$ instruction simply produces the byte $x$. It
can be used to mix code and data (such as the transition table of our keyboard
driver). In summary, a function to write an $opcode$ (without its argument)
should return an error if $opcode>32$, do nothing if $opcode=32$, or write the
$opcode$ byte otherwise.

\bigskip \paragraph*{Parser} The parser calls the scanner to read the source
code one token at a time, and generates the corresponding compiled code with the
backend. For our very simple initial compiler, the parser can be modeled with a
Finite State Machine, represented in \cref{fig:toyc0-automaton}. There are 4
states, corresponding to the expected ``type'' of the next token returned by the
scanner. State 0 corresponds to opcode tokens, such as \insn{add}. States 1, 2,
and 4 correspond to 1, 2, and 4-byte opcode arguments, respectively (such as the
argument of \insn{cst8}, \insn{iflt}, and \insn{cst}, respectively). In these 3
states, any token value $v$ should simply be written at $dst$ in 1, 2, or 4
bytes, and the next state is state 0. In state 0, the opcode corresponding to
the token value $v$, noted $opcode(v)$, should be written at $dst$. And the next
state, noted $S(v)$, should be either 0, 1, 2, or 4, depending on this opcode.
By listing the opcode names and computing their token values $v$ with
\cref{alg:scanner0}, we get $opcode(v)$ and $S(v)$, shown in
\cref{table:parser0}.

In order to implement this Finite State Machine we need functions to compute
$opcode(v)$ and $S(v)$. For this, the easiest is to store \cref{table:parser0}
in memory. $opcode(v)$ can then be computed by finding the row corresponding to
$v$, and then returning the value in its $opcode$ column -- and similarly for
$S(v)$. In fact, since the $opcode$ of the $i^{th}$ row is $i$, we don't need
to store this column. Notice also that the least significant byte $lsb(v)$ of
the token values $v$ are all unique. We can thus store only one byte per value
in this column. In summary, $opcode(v)$ and $S(v)$ can be computed as described
in \cref{alg:parser0-getopcode}, where {\tt LSB} and {\tt S} are the $lsb(v)$
and $S(v)$ value lists.

\rust{
  let opcode_names = vec!["cst_0", "cst_1", "cst8", "cst", "add", "sub", "mul",
  "div", "and", "or", "lsl", "lsr", "iflt", "ifeq", "ifgt", "ifle", "ifne",
  "ifge", "goto", "load", "store", "ptr", "get", "set", "pop", "fn", "call",
  "callr", "calld", "ret", "retv", "blx", "d"];
  let opcode_args = [0, 0, 1, 4, 0, 0, 0, 0, 0, 0, 0, 0, 2, 2, 2, 2, 2, 2,
  2, 0, 0, 1, 1, 1, 0, 1, 2, 2, 0, 0, 0, 0, 1];
  assert_eq!(opcode_args.len(), opcode_names.len());
  let token_value = |opcode:&&str| {
    let mut v = 0;
    for c in opcode.chars() {
      v = 10 * v + (c as u32 - '0' as u32);
    }
    v
  };
  let token_values : Vec<u32> =
      opcode_names.iter().map(token_value).collect();
  let table_row = |i: usize| {
    format!("{{\\makecell\\tt {}}} & {:X} & {:02X} & {}",
        opcode_names[i].replace('_', "\\_"), token_values[i], i, opcode_args[i])
  };
}

\begin{Table}
\begin{tabular}[t]{|l|r|r|c|} \hline
  \makecell{\thead{token}} & $v$ & $opcode$ & $S$ \\ \hline
  \rs{table_row(0)} \\
  \rs{table_row(1)} \\
  \rs{table_row(2)} \\
  \rs{table_row(3)} \\
  \rs{table_row(4)} \\
  \rs{table_row(5)} \\
  \rs{table_row(6)} \\
  \rs{table_row(7)} \\
  \rs{table_row(8)} \\
  \rs{table_row(9)} \\
  \rs{table_row(10)} \\
  \rs{table_row(11)} \\
  \rs{table_row(12)} \\
  \rs{table_row(13)} \\
  \rs{table_row(14)} \\
  \rs{table_row(15)} \\
  \rs{table_row(16)} \\ \hline
\end{tabular}
\hspace{1cm}
\begin{tabular}[t]{|l|r|r|c|} \hline
  \makecell{\thead{token}} & $v$ & $opcode$ & $S$ \\ \hline
  \rs{table_row(17)} \\
  \rs{table_row(18)} \\
  \rs{table_row(19)} \\
  \rs{table_row(20)} \\
  \rs{table_row(21)} \\
  \rs{table_row(22)} \\
  \rs{table_row(23)} \\
  \rs{table_row(24)} \\
  \rs{table_row(25)} \\
  \rs{table_row(26)} \\
  \rs{table_row(27)} \\
  \rs{table_row(28)} \\
  \rs{table_row(29)} \\
  \rs{table_row(30)} \\
  \rs{table_row(31)} \\
  \rs{table_row(32)} \\ \hline
\end{tabular}
  \caption{The token value $v$ and the corresponding compiled $opcode$ and next
  state $S$ for each valid opcode token.}\label{table:parser0}
\end{Table}

\rust{
  let lsb = |v:u32| { format!("{:02X}", v & 0xFF) };
  let lsb_set : HashSet<u32> = token_values.iter().map(|v| v & 0xFF).collect();

  assert_eq!(lsb_set.len(), token_values.len());
  assert!(!lsb_set.contains(&(token_value(&"cst_3") & 0xFF)));
  assert!(lsb_set.contains(&(token_value(&"cst_2") & 0xFF)));
}

\begin{Algorithm}
\caption{Computing \{$opcode(v)$, $S(v)$\} for a token value
$v$.}\label{alg:parser0-getopcode}
\begin{algorithmic}[1]
\Statex {\tt LSB} = [\rs{lsb(token_values[0])}, \rs{lsb(token_values[1])},
\rs{lsb(token_values[2])}, \rs{lsb(token_values[3])}, $\ldots$], {\tt S} = [0,
0, 1, 4, $\ldots$]
\State initialize $i$ to 0
\Begin while $i \le 32$ and the $i^{th}$ value in {\tt LSB} is not equal to $(v
\wedge 255)$
  \State increment $i$ by 1
\End
\State return \{$i$, $i^{th}$ value in {\tt S}\}
\end{algorithmic}
\end{Algorithm}

Note that for the invalid token \insn{cst\_3}, $lsb(v)$ is equal to
\rs{lsb(token_value(&"cst_3"))}, which is not in {\tt LSB}. In such cases,
\cref{alg:parser0-getopcode} returns the invalid opcode 33. Hence, most invalid
tokens can be detected by checking for invalid opcodes. However, some invalid
tokens, such as \insn{cst\_2}, cannot be detected like this because the least
significant byte of their token value {\em is} in {\tt LSB}. We fix this in
\cref{chapter:labels-compiler}, at the price of a greater complexity.

\section{Implementation}\label{section:toyc0-implementation}

\rust{
  let compiler_address = next_page_address(
      context.memory_region("text_editor").end());
  let mut b = BytecodeAssembler::new(RegionKind::DataBuffer, compiler_address);
  b.import_labels(context.memory_region("graphics_card_driver"));
  b.import_labels(context.memory_region("flash_driver"));
}

We can now implement this initial compiler. We do this in a new data buffer, in
the next flash memory page after the text editor (\ie, at address
\rs{hex(compiler_address)}). We start with the scanner, with a function
returning 1 if a given character $c$ is a space, a tabulation, or a ``new line''
(\hexa{20}, \hexa{09}, and \hexa{0A} in ASCII, respectively), and 0 otherwise:

\begin{TwoColumns}
\rs{b.func("tc_is_space", &["c"], "bool", &[])}\\
\bytecode{
  b.get("c");
  b.cst8(32);
  b.ifeq("is_space_true");
  b.get("c");
  b.cst8(9);
  b.ifeq("is_space_true");
  b.get("c");
  b.cst8(10);
  b.ifeq("is_space_true");
  b.cst_0();
  b.retv();
  b.label("is_space_true");
  b.cst_1();
  b.retv();
}
\end{TwoColumns}

We then implement \cref{alg:scanner0} in two parts. Steps 1 and 2 are
implemented in the following function, which returns the new $src$ value:

\begin{Paragraph}
\begin{paracol}{2}
\rs{b.func("tc_skip_spaces", &["src", "src_end"], "src'", &[])}

Initialize $src'$ to $src$.

\bytecode[switchcolumn]{
  b.get("src");
  b.def("src'");
}

Step 1. If $src'$ (in the $6^{th}$ stack frame slot) is greater than or equal to
$src\_end$, go to the last instruction.

\bytecode[switchcolumn]{
  b.label("tc_skip_spaces_loop");
  // while src' < end && tc_is_space(load_byte(src')) == 1 :
  b.get("src'");
  b.get("src_end");
  b.ifge("tc_skip_spaces_end");
}

If the character at $src'$ is not a spacing character, go to the last
instruction.

\bytecode[switchcolumn]{
  b.get("src'");
  b.call("load_byte");
  b.call("tc_is_space");
  b.cst_1();
  b.ifne("tc_skip_spaces_end");
}

Step 2. Increment $src'$ (the top stack value) by 1 and go back to step 1.

\bytecode[switchcolumn]{
  // src' = src' + 1;
  b.cst_1();
  b.add();
  b.goto("tc_skip_spaces_loop");
}

Return the top stack value $src'$.

\bytecode[switchcolumn]{
  b.label("tc_skip_spaces_end");
  // return src';
  b.retv();
}
\end{paracol}
\end{Paragraph}

Steps 5 to 7 are implemented in the next function, which also returns the new
$src$ value (we assume that steps 3 and 4 are done by the caller). Since a
function can't return several values, it can't return $v$ as described in
\cref{alg:scanner0}. Instead, it takes as parameter a {\em pointer} $v^p$ to a
memory word where $v$ can be read and modified:

\begin{Paragraph}
\begin{paracol}{2}
\rs{b.func("tc_read_token", &["src", "src_end", "v^p"], "src'", &[])}

Initialize $src'$ to $src$.

\bytecode[switchcolumn]{
  b.get("src");
  b.def("src'");
}

Step 5. If $src'$ (in the $7^{th}$ stack frame slot) is greater than or equal to
$src\_end$, go to the last instruction.

\bytecode[switchcolumn]{
  b.label("tc_read_token_loop");
  // while src' < src_end && tc_is_space(load_byte(src')) == 0 :
  b.get("src'");
  b.get("src_end");
  b.ifge("tc_read_token_end");
}

If the character at $src'$ is a spacing character, go to the last instruction.

\bytecode[switchcolumn]{
  b.get("src'");
  b.call("load_byte");
  b.call("tc_is_space");
  b.cst_1();
  b.ifeq("tc_read_token_end");
}

Step 6. Update $v$, at address $v^p$, to $10v+(c-\hexa{30})$, where $c$ is the
character at $src'$. To simplify, we do not check if this new value actually
fits in a word.

\bytecode[switchcolumn]{
  // *v = *v * 10 + (load8(src') - '0');
  b.get("v^p");
  b.get("v^p");
  b.load();
  b.cst8(10);
  b.mul();
  b.get("src'");
  b.call("load_byte");
  b.cst8(48);
  b.sub();
  b.add();
  b.store();
}

Step 7. Increment $src'$ (the top stack value) by 1 and go back to step 5.

\bytecode[switchcolumn]{
  // src' = src' + 1;
  b.cst_1();
  b.add();
  b.goto("tc_read_token_loop");
}

Return the top stack value $src'$.

\bytecode[switchcolumn]{
  b.label("tc_read_token_end");
  // return src';
  b.retv();
}
\end{paracol}
\end{Paragraph}

This concludes the scanner part. We continue with the backend part. As said
above, we mostly need here functions to store 8-bit and 16-bit values in memory.
We already have a \verb!store_byte! function (see
\cref{table:flash_driver_functions}), hence we only need a new \verb!store_half!
function (very similar to \verb!store_byte!, already explained):

\begin{TwoColumns}
\rs{b.func("store_half", &["ptr", "value"], "", &[])}\\
\bytecode{
  b.get("ptr");
  b.get("ptr");
  b.load();
  b.cst(0xFFFF0000);
  b.and();
  b.get("value");
  b.or();
  b.store();
  b.ret();
}
\end{TwoColumns}

We finish the backend part with a function to write an $opcode$ byte at $dst$,
which returns the new $dst$ value, $dst'$. As described above, this function
returns an error (represented with $dst=0$) if $opcode>32$, and does nothing if
$opcode=32$:

\bigskip \rs{b.func("tc_write_opcode", &["dst", "opcode"], "dst'", &[])}
\vspace{-0.9\baselineskip}
\begin{TwoColumns}
\bytecode{
  // if opcode == 33 { return 0; }
  b.get("opcode");
  b.cst8(33);
  b.ifne("tc_write_opcode_valid");
  b.cst_0();
  b.retv();
  b.label("tc_write_opcode_valid");
  // if opcode == 32 { return dst; }
  b.get("opcode");
  b.cst8(32);
  b.ifne("tc_write_opcode_end");
  // return dst;
  b.get("dst");
  b.retv();
  b.label("tc_write_opcode_end");
  // store_byte(dst, opcode);
  b.get("dst");
  b.get("opcode");
  b.call("store_byte");
  // return dst + 1;
  b.get("dst");
  b.cst_1();
  b.add();
  b.retv();
}
\end{TwoColumns}

\rust{
  b.label("LSB");
  for x in token_values {
    b.u8_data(x as u8);
  }
  b.u8_data(0);
  b.label("ARGUMENT");
  for x in opcode_args {
    b.u8_data(x);
  }
  b.u8_data(0);
}

We continue the implementation with the parser part, starting with
\cref{alg:parser0-getopcode}. We first store the {\tt LSB} and {\tt S} tables,
at addresses \rs{hex(b.label_address("LSB"))} and
\rs{hex(b.label_address("ARGUMENT"))}, respectively (note that we end each
table with a $33^{rd}$ 0 value, since $i$ can be equal to 33 at step 4 of
\cref{alg:parser0-getopcode}):

\bytecode[binary]{
  b.label("print-the-above-data");
}

We then implement \cref{alg:parser0-getopcode} in the following function. Since
a function can't return several values, it returns the $opcode$ only, and
stores $S$ at an address $S^p$ passed as parameter. Note also that this
function takes the least significant byte $lsb$ of $v$ as parameter (instead of
$v$ in \cref{alg:parser0-getopcode}):

\begin{Paragraph}
\begin{paracol}{2}
\rs{b.func("tc_get_opcode", &["lsb", "S^p"], "opcode", &[])}

Step 1. Initialize $i$ to 0.

\bytecode[switchcolumn]{
  b.cst_0();
  b.def("i");
}

Step 2. If $i$ (in the $6^{th}$ stack frame slot) is greater than 32, go to
step 4.

\bytecode[switchcolumn]{
  // while i <= 32 && get8(LSB + i) != lsb
  b.label("tc_get_opcode_loop");
  b.get("i");
  b.cst8(32);
  b.ifgt("tc_get_opcode_loop_end");
}

If the $i^{th}$ value in {\tt LSB} is equal to $lsb$, go to step 4.

\bytecode[switchcolumn]{
  b.cst(b.label_address("LSB"));
  b.get("i");
  b.add();
  b.call("load_byte");
  b.get("lsb");
  b.ifeq("tc_get_opcode_loop_end");
}

Step 3. Increment the top stack value $i$ by 1 and go back to step 2.

\bytecode[switchcolumn]{
  b.cst_1();
  b.add();
  b.goto("tc_get_opcode_loop");
}

Step 4. Store the $i^{th}$ value of {\tt S} at $S^p$ and return the top stack
value $i$.

\bytecode[switchcolumn]{
  b.label("tc_get_opcode_loop_end");
  // *S^p = get8(ARGUMENT + i);
  b.get("S^p");
  b.cst(b.label_address("ARGUMENT"));
  b.get("i");
  b.add();
  b.call("load_byte");
  b.store();
  // return opcode;
  b.retv();
}
\end{paracol}
\end{Paragraph}

With this we can now implement a function to perform a transition of the
parser's Finite State Machine. The following function takes a pointer $S^p$ to
the current state $S$ as parameter, as well as a token value $v$ and the
current value of $dst$. It performs the corresponding action, updates the value
at $S^p$ to the next state, and returns the new $dst$ value. It has 4 main
parts, corresponding to the 4 possible values of the current state, plus a
shared $5^{th}$ part:

\begin{Paragraph}
\begin{paracol}{2}
\rs{b.func("tc_parse_token", &["dst", "v", "S^p"], "dst'", &[])}

Get the value $S$ at address $S^p$.

\bytecode[switchcolumn]{
  // let S = *S^p;
  b.get("S^p");
  b.load();
  b.def("S");
}

Part 1. If $S$ (in the $7^{th}$ stack frame slot) is not 0, go to part 2.

\bytecode[switchcolumn]{
  // if S == OPCODE:
  b.get("S");
  b.cst_0();
  b.ifne("tc_parse_token_not_opcode");
}

Otherwise, compute $opcode(v)$ and store $S(v)$ at $S^p$ by calling
\verb!tc_get_opcode! on the least significant byte of $v$, $v \wedge 255$.
Write this opcode at $dst$ by calling \verb!tc_write_opcode!, and return the
result.

\bytecode[switchcolumn]{
  // return tc_write_opcode(dst, tc_get_opcode(v & 255, S^p));
  b.get("dst");
  b.get("v");
  b.cst8(255);
  b.and();
  b.get("S^p");
  b.call("tc_get_opcode");
  b.call("tc_write_opcode");
  b.retv();
}

Part 2. If $S$ is not 1, go to part 3.

\bytecode[switchcolumn]{
  b.label("tc_parse_token_not_opcode");
  // if S == DATA8:
  b.get("S");
  b.cst_1();
  b.ifne("tc_parse_token_not_data8");
}

Otherwise, store the byte $v$ at $dst$ and go to part 5 (to simplify we do
not check if $v$ actually fits in a byte).

\bytecode[switchcolumn]{
  // store_byte(dst, v);
  b.get("dst");
  b.get("v");
  b.call("store_byte");
  b.goto("tc_parse_token_end");
}

Part 3. If $S$ is not 2, go to part 4.

\bytecode[switchcolumn]{
  b.label("tc_parse_token_not_data8");
  // else if S == DATA16:
  b.get("S");
  b.cst8(2);
  b.ifne("tc_parse_token_not_data16");
}

Otherwise, store the half word $v$ at $dst$ and go to part 5 (to simplify we
do not check if $v$ actually fits in a half word).

\bytecode[switchcolumn]{
  // store_half(dst, v);
  b.get("dst");
  b.get("v");
  b.call("store_half");
  b.goto("tc_parse_token_end");
}

Part 4. $S$ is necessarily equal to 4. Store the word $v$ at $dst$ and continue
to part 5.

\bytecode[switchcolumn]{
  b.label("tc_parse_token_not_data16");
  // *dst = v;
  b.get("dst");
  b.get("v");
  b.store();
}

Part 5. Update the value at $S^p$ to 0, the next state after a transition from
state 1, 2, or 4.

\bytecode[switchcolumn]{
  b.label("tc_parse_token_end");
  // *S^p = OPCODE;
  b.get("S^p");
  b.cst_0();
  b.store();
}

Return the new $dst$ value, $dst+S$ (since $S$ is the number of bytes just
written).

\bytecode[switchcolumn]{
  // return dst + S;
  b.get("dst");
  b.get("S");
  b.add();
  b.retv();
}
\end{paracol}
\end{Paragraph}

We can finally implement the compiler's main function. It starts by
initializing $src$ to $\it{src\_buffer}+4$, $src\_end$ to
$src+\mathrm{mem32}[\it{src\_buffer}]$, $dst$ to $\it{dst\_buffer}+4$,
the token value $v$ to 0, and the Finite State Machine state $S$ to 0, in stack
frame slots 6, 7, 8, 9, and 10, respectively:

\bigskip \rs{b.func("tc_main", &["src_buffer", "dst_buffer"], "error", &[])}
\vspace{-0.9\baselineskip}
\begin{TwoColumns}
\bytecode{
  // let src = src_buffer + 4;
  b.get("src_buffer");
  b.cst8(4);
  b.add();
  b.def("src");
  // let end = src + *src_buffer;
  b.get("src");
  b.get("src_buffer");
  b.load();
  b.add();
  b.def("src_end");
  // let dst = dst_buffer + 4;
  b.get("dst_buffer");
  b.cst8(4);
  b.add();
  b.def("dst");
  // let v = 0;
  b.cst_0();
  b.def("v");
  // let S = OPCODE;
  b.cst_0();
  b.def("S");
}
\end{TwoColumns}

It continues with a loop which 1) skips spaces and returns 0 if $src\_end$ is
reached, 2) reads a token and performs the corresponding Finite State Machine
transition, 3) returns 1 if an invalid token was found:

\begin{Paragraph}
\begin{paracol}{2}
Step 1. Update $src$ to the result of \verb!tc_skip_spaces!($src$, $src\_end$).

\bytecode[switchcolumn]{
  b.label("tc_main_loop");
  // loop :
  // src = tc_skip_spaces(src, src_end);
  b.get("src");
  b.get("src_end");
  b.call("tc_skip_spaces");
  b.set("src");
}

Step 2. If $src<src\_end$, go to step 3.

\bytecode[switchcolumn]{
  // if src >= src_end :
  b.get("src");
  b.get("src_end");
  b.iflt("tc_main_ok");
}

Otherwise, \ie, if the end of the program is reached, set the value at
$\it{dst\_buffer}$ to the number of bytes written,
$dst-\it{dst\_buffer}-4$, and return 0 (meaning ``no error''). To simplify,
we do not check if $S$ is 0 (if not the program ends in the middle of an
instruction, which is an error).

\bytecode[switchcolumn]{
  // *dst_buffer = dst - dst_buffer - 4;
  b.get("dst_buffer");
  b.get("dst");
  b.get("dst_buffer");
  b.sub();
  b.cst8(4);
  b.sub();
  b.store();
  // return 0;
  b.cst_0();
  b.retv();
}

Step 3. Call the scanner to read a token and store its value in $v$. Update
$src$ to the result of \verb!tc_read_token!.

\bytecode[switchcolumn]{
  b.label("tc_main_ok");
  // src = tc_read_token(src, src_end, &v);
  b.get("src");
  b.get("src_end");
  b.ptr("v");
  b.call("tc_read_token");
  b.set("src");
}

Step 4. Perform the Finite State Machine transition corresponding to $v$.
Update $dst$ to the result of \verb!tc_parse_token!.

\bytecode[switchcolumn]{
  // dst = tc_parse_token(dst, v, &S);
  b.get("dst");
  b.get("v");
  b.ptr("S");
  b.call("tc_parse_token");
  b.set("dst");
}

Step 5. If $dst \ne 0$, go to step 6.

\bytecode[switchcolumn]{
  // if dst == 0 :
  b.get("dst");
  b.cst_0();
  b.ifne("tc_main_end_loop");
}

Otherwise, \ie, if an invalid token has been read, set the value at
$\it{dst\_buffer}$ to the location of the error,
$src-\it{src\_buffer}-4$, and return 1 (meaning ``error'').

\bytecode[switchcolumn]{
  // *dst_buffer = src - src_buffer - 4;
  b.get("dst_buffer");
  b.get("src");
  b.get("src_buffer");
  b.sub();
  b.cst8(4);
  b.sub();
  b.store();
  // return 1;
  b.cst_1();
  b.retv();
}

Step 6. Reinitialize $v$ to 0 for the next loop iteration, and go back to step
1.

\bytecode[switchcolumn]{
  b.label("tc_main_end_loop");
  // v = 0;
  b.cst_0();
  b.set("v");
  b.goto("tc_main_loop");
}
\end{paracol}
\end{Paragraph}

In summary the full code of our initial compiler is the following:

\rs{b.get_bytecode_listing(0..b.get_instruction_count() as usize, false)}

\rust{
  // generate command file to flash toyc0 with SAMBA
  let mut commands = Vec::new();
  commands.extend(b.boot_assistant_commands());
  commands.push(String::from("flash#"));
  commands.push(String::from("reset#"));
  write_lines("website/part3", "opcodes_compiler.txt", &commands)?;
}

\rust{
  // enter toyc0 code in RAM with memory editor
  let display = Rc::new(RefCell::new(TextDisplay::default()));
  context.set_display(display.clone());

  context.add_memory_region("toyc0", b.memory_region());
  context.micro_controller().borrow_mut().reset();
  context.run_until_get_char();
  let mut context1 = context.clone();

  const COMMAND_ADDRESS: u32 = 0x20080000;
  const BUFFER_RAM_ADDRESS: u32 = 0x20070000;
  context.type_ascii(&b.memory_editor_commands(BUFFER_RAM_ADDRESS));
}

To store it in flash memory we must enter it in RAM first, lets say at address
\rs{hex(BUFFER_RAM_ADDRESS)}, and then save it in flash. In the memory editor,
type ``w\rs{hex_word_low(BUFFER_RAM_ADDRESS)}''+Enter, and then store the
compiler size in bytes at this address by typing
``w\rs{hex_word_low(b.bytecode_size())}''+Enter. Continue by entering each word
of the compiler code, listed above, by typing its value followed by Enter.
Finally, save this code in flash memory (starting at
$page=\rs{dec(page_number(compiler_address))}$) by running the following
function:

\rust{
  // small function to store toyc0 in flash at 'compiler_address'
  let mut c = BytecodeAssembler::default();
  c.import_labels(context.memory_region("flash_driver"));
}
\begin{TwoColumns}
\rs{c.func("save", &[], "", &["nolink"])}\\
\bytecode{
  c.cst(BUFFER_RAM_ADDRESS);
  c.cst8(page_number(compiler_address).try_into().unwrap());
  c.call("buffer_flash");
  c.ret();
}
\end{TwoColumns}

For this enter the full code of the above function in an unused RAM region, for
instance starting at address \rs{hex(COMMAND_ADDRESS)}:

\rs{c.get_bytecode_listing(0..c.get_instruction_count() as usize, false)}

\rust{
  // enter above function in RAM and execute it.
  context.type_ascii(&c.memory_editor_commands(COMMAND_ADDRESS));
  context.type_ascii(&format!("W{:08X}\n", COMMAND_ADDRESS));
  context.type_ascii("R");

  let boot_mode_address = context.memory_region("foundations")
      .label_address("boot_mode_select_rom");
}

Then type ``w\rs{hex_word_low(COMMAND_ADDRESS)}'' followed by ``r'' to run it.
Alternatively, if you don't want to enter the full compiler code manually with
the memory editor, which is a bit tedious, you can ``cheat'' by saving it via
an external computer, as follows. First run the \verb!boot_mode_select_rom!
function by typing ``w\rs{hex_word_low(boot_mode_address)}''+Enter, followed by
``r''. Then reset the Arduino and, on the host computer, run the following
command to flash the compiler code and reset the Arduino again:

\rust{
  context1.type_ascii(&format!("W{:08X}\n", boot_mode_address));
  context1.type_ascii("R");
  context1.micro_controller().borrow_mut().reset();
  let mut flash_helper1 = FlashHelper::from_file(
      context1.micro_controller(), "website/", "part3/opcodes_compiler.txt")?;
}
\rs{host_log(&flash_helper1.read())}

\section{Command editor}

We can now write and compile our very first program in textual form. For this
we first need to enter it in memory with the text editor. This requires calling
the text editor, and then the compiler, with specific arguments. In turn, this
currently requires typing a few bytecode instructions {\em in binary form} with
the memory editor, as we did above to call {\tt buffer\_flash}. To avoid having
to do this in the next chapters, our first program is a {\em command editor}.
Its goal is to edit, compile and run small functions, called {\em commands},
such as the {\tt save} function above.

\subsection{User interface}

A task such as writing and compiling a program requires less than a dozen
distinct commands to edit the program, save it, compile it, save the compiled
code, etc. However, each command must usually be run several times (if the
compiler returns an error, the program must be edited, saved, and compiled
again). In order to avoid having to repeatedly type the same commands, the
command editor should be able to save up to 12 distinct commands in flash
memory. We number them from 1 to 12. It should then be able to load an existing
command, and to edit it if necessary. Finally, it should be able to compile and
run a command. To fulfill these requirements we define the command editor user
interface as follows:
\begin{itemize}
  \item typing a ``F$i$'' key between ``F1'' and ``F12'' included should load
  command number $i$ and display it. This command becomes the {\em current
  command}.

  \item typing ``e'' should run the text editor to edit the current command.
  Each command must be a function without argument, returning an integer value.

  \item typing ``s'' should save the current command in flash memory.

  \item typing ``r'' should compile the current command, run it, display its
  result, and wait until Enter is pressed (and not until any key press
  because releasing ``r'' can appear as a key press for commands using the
  flash memory driver -- see \cref{section:flash-memory-driver-impl}). If the
  compilation fails, the compiler result should be displayed instead.

  \item typing Escape should exit the command editor.
\end{itemize}

Finally, when launched, the command editor should load and display command
number 1. All commands are initially empty in flash memory.

\subsection{Implementation}\label{subsection:command-editor-implementation}

\rust{
  const PAGE_SIZE: u32 = 256;

  // Address of command editor code and source code in flash memory
  // (sizes include buffer header).
  const MAX_COMMAND_EDITOR_CODE_SIZE: u32 = 256;
  const MAX_COMMAND_EDITOR_SOURCE_SIZE: u32 = 1024;
  const COMMAND_SOURCE: u32 = 0xD0000;
  const NUM_COMMAND_EDITOR_COMMANDS: u32 = 12;
  const MAX_COMMAND_CODE_SIZE: u32 = PAGE_SIZE;
  const MAX_COMMAND_SOURCE_SIZE: u32 = PAGE_SIZE;
  let command_editor_code =
      next_page_address(context.memory_region("toyc0").end());
  let command_editor_source =
      COMMAND_SOURCE + NUM_COMMAND_EDITOR_COMMANDS * MAX_COMMAND_SOURCE_SIZE;

  // Address of current command source code and compiled code in RAM.
  const RAM_START: u32 = 0x20070000;
  const RAM_COMMAND_SOURCE: u32 = RAM_START;
  const RAM_COMMAND_CODE: u32 = RAM_COMMAND_SOURCE + MAX_COMMAND_SOURCE_SIZE;

  // Address of compiler code, backup code, and source code in flash
  // memory (sizes include buffer header).
  const MAX_COMPILER_CODE_KB: u32 = 12;
  const MAX_COMPILER_CODE_SIZE: u32 = MAX_COMPILER_CODE_KB * 1024;
  const MAX_COMPILER_SOURCE_KB: u32 = 48;
  const MAX_COMPILER_SOURCE_SIZE: u32 = MAX_COMPILER_SOURCE_KB * 1024;
  let compiler_code = command_editor_code + MAX_COMMAND_EDITOR_CODE_SIZE;
  // bytecode call instructions only support of 64 KB range after 0xC0000.
  assert!(compiler_code + MAX_COMPILER_CODE_SIZE < 0xC0000 + 65536);
  let compiler_code_page = page_number(compiler_code);
  let compiler_code_backup = 0xE0000;
  let compiler_code_backup_page = page_number(compiler_code_backup);
  let compiler_source = command_editor_source + MAX_COMMAND_EDITOR_SOURCE_SIZE;
  let compiler_source_page = page_number(compiler_source);

  // Address of compiler code and source code in RAM.
  const RAM_COMPILER_SOURCE: u32 = RAM_COMMAND_CODE + MAX_COMMAND_CODE_SIZE;
  const RAM_COMPILER_CODE: u32 = RAM_COMPILER_SOURCE + MAX_COMPILER_SOURCE_SIZE;

  let mut compiler_labels = HashMap::<String, Label>::new();
  compiler_labels.insert(
      String::from("main"),
      Label {
          offset: 0,
          description: String::default(),
      },
  );
  context.add_memory_region(
      "compiler_code",
      MemoryRegion::new(
          RegionKind::DataBuffer,
          compiler_code,
          MAX_COMPILER_CODE_SIZE,
          &compiler_labels,
          0,
          0,
          0,
          Vec::default(),
      ),
  );
  context.add_memory_region(
      "compiler_code_backup",
      MemoryRegion::new(
          RegionKind::DataBuffer,
          compiler_code_backup,
          MAX_COMPILER_CODE_SIZE,
          &HashMap::default(),
          0,
          0,
          0,
          Vec::default(),
      ),
  );
  context.add_memory_region(
      "compiler_source",
      MemoryRegion::new(
          RegionKind::DataBuffer,
          compiler_source,
          MAX_COMPILER_SOURCE_SIZE,
          &HashMap::default(),
          0,
          0,
          0,
          Vec::default(),
      ),
  );

  // generate source code of command editor program
  let mut b =
      BytecodeAssembler::create(RegionKind::DataBuffer, command_editor_code,
      true);
  b.import_labels(context.memory_region("graphics_card_driver"));
  b.import_labels(context.memory_region("keyboard_driver"));
  b.import_labels(context.memory_region("memory_editor"));
  b.import_labels(context.memory_region("flash_driver"));
  b.import_labels(context.memory_region("text_editor"));
  b.import_labels(context.memory_region("toyc0"));
}

We can now write the command editor source code. For this we assume that its
compiled code will eventually be stored in the next page after the opcodes
compiler, \ie, at address \rs{hex(command_editor_code)}%
=\hexa{C0000}+\rs{dec(command_editor_code-0xC0000)}.

To implement the above requirements we reserve 12 pages of flash memory, one
for each command, starting at address \rs{hex(COMMAND_SOURCE)}. This gives
$256-4=252$ bytes for the source code of each command, stored as a data buffer
(see \cref{subsection:data-buffer}). We can then write a function to load
command number $\it{command}$ (here numbered from 0 to 11) at address $\it{dst}$
(the right column shows source code; in particular, all numbers are in decimal
form):

\begin{Paragraph}
\begin{paracol}{2}
\rs{b.func("ced_load", &["command", "dst"], "", &[])}

Initialize $\it{dst}$ to an empty buffer.

\bytecode[switchcolumn]{
  // *dst = 0;
  b.new_line();
  b.get("dst");
  b.cst_0();
  b.store();
}

Compute the $\it{src}$ address of $\it{command}$. This is
$\rs{hex(COMMAND_SOURCE)}+256*\it{command}$.

\bytecode[switchcolumn]{
  // let src = (0xD0000 as *u32) + (command << 8);
  b.new_line();
  b.cst(COMMAND_SOURCE);
  b.get("command");
  b.cst8(8);
  b.lsl();
  b.add();
  b.def("src");
}

If the $\it{src}$ buffer size is greater than 252 this means that no command
has ever been stored here (each flash memory bit is initialized to 1). Then
return directly.

\bytecode[switchcolumn]{
  // if *src <= 252 { buffer_copy(src, dst); }
  b.new_line();
  b.get("src");
  b.load();
  b.cst8(252);
  b.ifgt("ced_load_end");
}

Otherwise copy the $\it{src}$ buffer to $\it{dst}$ and return.

\bytecode[switchcolumn]{
  b.get("src");
  b.get("dst");
  b.call("buffer_copy");
  b.new_line();
  b.label("ced_load_end");
  b.ret();
}
\end{paracol}
\end{Paragraph}

We continue with a function to display the command at $\it{src}$. For this we
simply reuse the {\tt ted\_draw} function of the text editor:

\begin{Paragraph}
\begin{paracol}{2}
\rs{b.func("ced_draw", &["src"], "", &[])}

Set the color to yellow, to make it easier to distinguish the command editor
and the text editor (which draws text in green).

\bytecode[switchcolumn]{
  // gpu_set_color(7, 7, 0);
  b.new_line();
  b.cst8(7);
  b.cst8(7);
  b.cst_0();
  b.call("gpu_set_color");
}

Compute the $\it{begin}$ address of the text, which is 4 bytes after $\it{src}$.

\bytecode[switchcolumn]{
  // let begin = src + 4;
  b.new_line();
  b.get("src");
  b.cst8(4);
  b.add();
  b.def("begin");
}

Compute the $\it{end}$ address of the text, which is $n$ bytes after
$\it{begin}$ (where $n$, the $\it{src}$ buffer size, is the value at address
$\it{src}$).

\bytecode[switchcolumn]{
  // let end = begin + *src;
  b.new_line();
  b.get("begin");
  b.get("src");
  b.load();
  b.add();
  b.def("end");
}

Draw the text with a zero $\it{gap}$ and a $\it{cursor}$ at the end (see
\cref{chapter:text-editor}).

\bytecode[switchcolumn]{
  // ted_draw(begin, end, 0, end);
  b.new_line();
  b.get("begin");
  b.get("end");
  b.cst_0();
  b.get("end");
  b.call("ted_draw");
  b.new_line();
  b.ret();
}
\end{paracol}
\end{Paragraph}

The next function compiles the source code at $\it{src}$, writes the compiled
code at $\it{dst}$, and runs it. It then displays the result and waits until
Enter is pressed.

\begin{Paragraph}
\begin{paracol}{2}
\rs{b.func("ced_run", &["src", "dst"], "", &[])}

Compile the code. The result, noted $\it{error}$, is pushed in the $6^{th}$
stack frame slot.

\bytecode[switchcolumn]{
  // let error = tc_main(src, dst);
  b.new_line();
  b.get("src");
  b.get("dst");
  b.call("tc_main");
  b.def("error");
}

If the compilation is successful (\ie, if $error=0$), run the compiled code
(which starts after the 4 bytes $\it{dst}$ header) and store its result in
$\it{error}$. Otherwise skip this step.

\bytecode[switchcolumn]{
  // if error == 0 { error = "calld" (dst + 4); }
  b.new_line();
  b.get("error");
  b.cst_0();
  b.ifne("ced_run_end");
  b.get("dst");
  b.cst8(4);
  b.add();
  b.calld();
  b.set("error");
}

Clear the screen, set the cursor to the top-left corner, draw $\it{error}$ in
hexadecimal, and wait until Enter is pressed.

\bytecode[switchcolumn]{
  // gpu_clear_screen();
  b.new_line();
  b.label("ced_run_end");
  b.call("gpu_clear_screen");
  // gpu_set_cursor(0, 0);
  b.new_line();
  b.cst_0();
  b.cst_0();
  b.call("gpu_set_cursor");
  // gpu_draw_hex_word(error);
  b.new_line();
  b.get("error");
  b.call("gpu_draw_hex_word");
  // while keyboard_get_char() != 10 {}
  b.new_line();
  b.label("ced_run_wait_enter");
  b.call("keyboard_get_char");
  b.cst8(10);
  b.ifne("ced_run_wait_enter");
  b.new_line();
  b.ret();
}
\end{paracol}
\end{Paragraph}

We can finally write the main command editor function. This function loops
until Escape is pressed, and performs the appropriate action for any other
typed key. It loads the current command in the 256 bytes region starting at
address \rs{hex(RAM_COMMAND_SOURCE)}, and compiles and runs it the next 256
bytes.

\begin{Paragraph}
\begin{paracol}{2}
\rs{b.func("command_editor", &[], "", &[])}

Initialize $\it{src}$ to \rs{hex(RAM_COMMAND_SOURCE)}.

\bytecode[switchcolumn]{
  // let src = 0x20070000 as *u32;
  b.new_line();
  b.cst(RAM_COMMAND_SOURCE);
  b.def("src");
}

Initialize $\it{command}$ to 0.

\bytecode[switchcolumn]{
  // let command = 0;
  b.new_line();
  b.cst_0();
  b.def("command");
}

Initialize $c$ to ``F1'' (see \cref{table:code_tables_choices}).

\bytecode[switchcolumn]{
  // let c = 0x80;
  b.new_line();
  b.cst8(0x80);
  b.def("c");
}

Step 1. If $c$ is not the Escape key go to step 2. Otherwise return.

\bytecode[switchcolumn]{
  // loop
  b.new_line();
  b.label("ced_loop");
  //  if c == 0x1B { return; }
  b.get("c");
  b.cst8(0x1B);
  b.ifne("ced_not_escape");
  b.ret();
}

Step 2. If $c$ is not between ``F1'' and ``F12'' included go to step 3.

\bytecode[switchcolumn]{
  //  if c >= 0x80 && c <= 0x8B
  b.new_line();
  b.label("ced_not_escape");
  b.get("c");
  b.cst8(0x80);
  b.iflt("ced_not_load");
  b.get("c");
  b.cst8(0x8B);
  b.ifgt("ced_not_load");
}

Otherwise set $\it{command}$ to $c-$``F1''.

\bytecode[switchcolumn]{
  // command = c - 0x80;
  b.new_line();
  b.get("c");
  b.cst8(128);
  b.sub();
  b.set("command");
}

Then load this new command and go to step 6 to display it.

\bytecode[switchcolumn]{
  // ced_load(command, src);
  b.new_line();
  b.get("command");
  b.get("src");
  b.call("ced_load");
  b.goto("ced_redraw");
}

Step 3. If $c$ is not equal to ``e'' go to step 4.

\bytecode[switchcolumn]{
  //  else if c == 'e' { text_editor(src, 0, 252); }
  b.new_line();
  b.label("ced_not_load");
  b.get("c");
  b.cst8(b'e');
  b.ifne("ced_not_edit");
}

Otherwise call the text editor to edit the current command (with a maximum text
length of 252 bytes). Then go to step 6 to display it.

\bytecode[switchcolumn]{
  b.get("src");
  b.cst_0();
  b.cst8(252);
  b.call("text_editor");
  b.goto("ced_redraw");
}

Step 4. If $c$ is not equal to ``e'' go to step 5.

\bytecode[switchcolumn]{
  //  else if c == 's' { buffer_flash(src, 256 + command); }
  b.new_line();
  b.label("ced_not_edit");
  b.get("c");
  b.cst8(b's');
  b.ifne("ced_not_save");
}

Otherwise save the current command at address
$\rs{hex(COMMAND_SOURCE)}+256*\it{command}$, which corresponds to page
$\rs{dec(page_number(COMMAND_SOURCE))}+\it{command}$.

\bytecode[switchcolumn]{
  b.get("src");
  b.cst(page_number(COMMAND_SOURCE));
  b.get("command");
  b.add();
  b.call("buffer_flash");
  b.goto("ced_redraw");
}

Step 5. If $c$ is not equal to ``r'' go to step 6.

\bytecode[switchcolumn]{
  //  else if c == 'r' { ced_run(src, src + 256); }
  b.new_line();
  b.label("ced_not_save");
  b.get("c");
  b.cst8(b'r');
  b.ifne("ced_redraw");
}

Otherwise compile and run the current command. The compiled code is written at
$\it{dst}=\it{src}+256$. Then continue to step 6.

\bytecode[switchcolumn]{
  b.get("src");
  b.get("src");
  b.cst(RAM_COMMAND_CODE - RAM_COMMAND_SOURCE);
  b.add();
  b.call("ced_run");
}

Step 6. Draw the current command, wait for a key to be pressed, store it in
$c$, and go back to step 1 to handle it.

\bytecode[switchcolumn]{
  b.label("ced_redraw");
  //  ced_draw(src);
  b.get("src");
  b.call("ced_draw");
  //  c = keyboard_wait_char();
  b.new_line();
  b.call("keyboard_wait_char");
  b.set("c");
  b.goto("ced_loop");
}
\end{paracol}
\end{Paragraph}

The command editor implementation is now complete, and is summarized below:

\rs{code(&b.get_toy0_source_code())}

\rust{
  let command_editor_source_code = b.get_toy0_source_code();
  assert!(command_editor_source_code.len() + 4 <
      MAX_COMMAND_EDITOR_SOURCE_SIZE as usize);

  let mut command_editor_labels = HashMap::new();
  command_editor_labels.insert(
      String::from("ram_command_source"),
      Label {
          offset: RAM_COMMAND_SOURCE - command_editor_source - 4,
          description: String::from("command source code in RAM"),
      },
  );
  command_editor_labels.insert(
      String::from("ram_compiler_source"),
      Label {
          offset: RAM_COMPILER_SOURCE - command_editor_source - 4,
          description: String::from("compiler source code in RAM"),
      },
  );
  command_editor_labels.insert(
      String::from("ram_compiler_code"),
      Label {
          offset: RAM_COMPILER_CODE - command_editor_source - 4,
          description: String::from("compiler code in RAM"),
      },
  );
  context.add_memory_region(
      "command_editor_commands",
      MemoryRegion::new(
          RegionKind::DataBuffer,
          COMMAND_SOURCE,
          NUM_COMMAND_EDITOR_COMMANDS * MAX_COMMAND_SOURCE_SIZE,
          &HashMap::new(),
          0,
          0,
          0,
          Vec::default(),
      ),
  );
  context.add_memory_region(
      "command_editor_source",
      MemoryRegion::new(
          RegionKind::DataBuffer,
          command_editor_source,
          command_editor_source_code.len() as u32 + 4,
          &command_editor_labels,
          0,
          0,
          0,
          Vec::default(),
      ),
  );

  context.add_memory_region("command_editor", b.memory_region());
}

\subsection{Compilation}

We now need to type this source code with the text editor, save it, compile it,
and store the compiled code. These 4 steps are explained below.

\subsubsection{Edit}

\rust{
  const MAX_SOURCE_LENGTH: u32 = 0x1000;
  const SOURCE_ADDRESS: u32 = 0x20070000;
  const CODE_ADDRESS: u32 = SOURCE_ADDRESS + MAX_SOURCE_LENGTH;

  const MAX_COMMAND_SIZE: u32 = 0x20;
  const EDIT_COMMAND_ADDRESS: u32 = 0x20080000;

  // Temporary program to run the text editor on SOURCE_ADDRESS.
  let mut c = BytecodeAssembler::default();
  c.import_labels(context.memory_region("text_editor"));
  c.func("temp", &[], "", &[]);
  c.cst(SOURCE_ADDRESS);
  c.cst_0();
  c.cst(MAX_SOURCE_LENGTH);
  c.call("text_editor");
  c.ret();
  assert!(c.bytecode_size() < MAX_COMMAND_SIZE);
}

Typing the source code requires launching the text editor first. For this, in
the memory editor, type ``w\rs{hex_word_low(EDIT_COMMAND_ADDRESS)}''+Enter,
followed by the code below (see \cref{section:text-editor-experiments}):

\rs{c.get_bytecode_listing(0..c.get_instruction_count() as usize, false)}

Then initialize an empty text buffer by typing
``w\rs{hex_word_low(SOURCE_ADDRESS)}''+Enter, followed by ``00000000''+Enter.
Run the text editor on this empty buffer by typing
``\rs{hex_word_low(EDIT_COMMAND_ADDRESS)}''+Enter, followed by ``r''. Finally,
type the command editor source code listed above, followed by Escape to return
in the memory editor.

Alternatively, if you don't want to type this source code, you can ``cheat'' by
saving it via an external computer, as follows. First run the
\verb!boot_mode_select_rom! function by typing
``w\rs{hex_word_low(boot_mode_address)}''+Enter, followed by ``r''. Then reset
the Arduino and, on the host computer, run the following command to flash the
source code and reset the Arduino again (you can then skip the ``Save'' step
below):

\rust{
  // Test the above instructions with a short text.
  context.type_ascii(&c.memory_editor_commands(EDIT_COMMAND_ADDRESS));
  // initialize empty buffer
  context.type_ascii(&format!("W{:08X}\n", SOURCE_ADDRESS));
  context.type_ascii("00000000\n");
  // run text editor
  context.type_ascii(&format!("W{:08X}\n", EDIT_COMMAND_ADDRESS));
  context.type_ascii("R");
  context.type_ascii("HELLO");
  assert_eq!(display.borrow().get_text(), "hello");
  context.type_keys(vec!["Escape"]);
  context.type_ascii(&format!("W{:08X}\n", SOURCE_ADDRESS));
  assert!(display.borrow().get_text().lines().next().unwrap()
      .ends_with("6F 6C6C6568 00000005 20070000"));

  // Then store the source code in RAM as if edited with the text editor.
  context.store_text(SOURCE_ADDRESS, command_editor_source_code.as_str());

  write_lines("website/part3", "command_editor.txt",
      &flash_helper_commands(command_editor_source_code.as_str(),
      command_editor_source))?;

  context1.run_until_get_char();
  context1.type_ascii(&format!("W{:08X}\n", boot_mode_address));
  context1.type_ascii("R");
  context1.micro_controller().borrow_mut().reset();
  flash_helper1 = FlashHelper::from_file(
      context1.micro_controller(), "website/", "part3/command_editor.txt")?;
}
\rs{host_log(flash_helper1.read().lines().next().unwrap())}

\subsubsection{Save}

\rust{
  const SAVE_SOURCE_COMMAND_ADDRESS: u32 =
      EDIT_COMMAND_ADDRESS + MAX_COMMAND_SIZE;

  // Temporary program to flash source code of command editor.
  let mut c = BytecodeAssembler::default();
  c.import_labels(context.memory_region("flash_driver"));
}

Before compiling this code we want to save it, in case something goes wrong. We
can save it after the 12 pages reserved for the commands, at address
\rs{hex(command_editor_source)}, which corresponds to page
\rs{dec_hex(page_number(command_editor_source))}. This can be done with the
following function:

\begin{TwoColumns}
\rs{c.func("save_source", &[], "", &["nolink"])}\\
\bytecode{
  c.cst(SOURCE_ADDRESS);
  c.cst(page_number(command_editor_source));
  c.call("buffer_flash");
  c.ret();
}
\end{TwoColumns}

Enter it in RAM after the ``edit'' function, at address
\rs{hex(SAVE_SOURCE_COMMAND_ADDRESS)}, by typing
``w\rs{hex_word_low(SAVE_SOURCE_COMMAND_ADDRESS)}''+Enter, followed by the full
code of this function, listed below. Then run it by typing
``w\rs{hex_word_low(SAVE_SOURCE_COMMAND_ADDRESS)}''+Enter, followed by ``r''.

\rs{c.get_bytecode_listing(0..c.get_instruction_count() as usize, false)}

\rust{
  assert!(c.bytecode_size() < MAX_COMMAND_SIZE);
  context.type_ascii(&c.memory_editor_commands(SAVE_SOURCE_COMMAND_ADDRESS));
  context.type_ascii(&format!("W{:08X}\n", SAVE_SOURCE_COMMAND_ADDRESS));
  context.type_ascii("R");

  // Check that the "cheat" above stores the correct values in flash.
  context.check_equal_buffer(&mut context1, command_editor_source);
}

\subsubsection{Compile}

\rust{
  const COMPILE_COMMAND_ADDRESS: u32 =
      SAVE_SOURCE_COMMAND_ADDRESS + MAX_COMMAND_SIZE;
  const COMPILE_RESULT_ADDRESS: u32 =
      COMPILE_COMMAND_ADDRESS + MAX_COMMAND_SIZE;

  // Temporary program to compile the source code.
  let mut c = BytecodeAssembler::default();
  c.import_labels(context.memory_region("toyc0"));
}

Compiling the code can be done with the following function, which writes
the compiled code at address \rs{hex(CODE_ADDRESS)} and the compiler's result
value at address \rs{hex(COMPILE_RESULT_ADDRESS)}:

\begin{TwoColumns}
\rs{c.func("compile_source", &[], "", &["nolink"])}\\
\bytecode{
  c.cst(COMPILE_RESULT_ADDRESS);
  c.cst(command_editor_source);
  c.cst(CODE_ADDRESS);
  c.call("tc_main");
  c.store();
  c.ret();
}
\end{TwoColumns}

Enter it in RAM after the ``save'' function, at address
\rs{hex(COMPILE_COMMAND_ADDRESS)}, by typing
``w\rs{hex_word_low(COMPILE_COMMAND_ADDRESS)}''+Enter, followed by the full
code of this function, listed below. Then run it by typing
``w\rs{hex_word_low(COMPILE_COMMAND_ADDRESS)}''+Enter, followed by ``r''.

\rs{c.get_bytecode_listing(0..c.get_instruction_count() as usize, false)}

\rust{
  assert!(c.bytecode_size() < MAX_COMMAND_SIZE);
  context.type_ascii(&c.memory_editor_commands(COMPILE_COMMAND_ADDRESS));
  context.type_ascii(&format!("W{:08X}\n", COMPILE_COMMAND_ADDRESS));
  context.type_ascii("R");
  assert!(display.borrow().get_text().lines().nth(6).unwrap()
      .ends_with("00000000 20080060"));

  // Check that the compiled code is equal to b's code.
  let words = b.bytecode_words();
  for i in 0..b.bytecode_size() / 4 {
    let word = context.micro_controller().borrow_mut().debug_get32(
        CODE_ADDRESS + 4 * (i + 1));
    assert_eq!(word, words[i as usize]);
  }
}

If all goes well the value at address \rs{hex(COMPILE_RESULT_ADDRESS)} should
be 0, because the compiler returns 0 if and only if the compilation is
successful. If this is not the case, run the ``edit'' function again, double
check the source code and fix any error found (you can also get the location of
the error at address \rs{hex(CODE_ADDRESS)}). Then save and compile the code
again. And repeat this until success.

\subsubsection{Store}

\rust{
  const SAVE_CODE_COMMAND_ADDRESS: u32 =
      COMPILE_RESULT_ADDRESS + MAX_COMMAND_SIZE;

  // - command to flash compiled command editor code
  let mut c = BytecodeAssembler::default();
  c.import_labels(context.memory_region("flash_driver"));
}

Once the compilation is successful, the compiled code can be stored in flash
memory. The following function stores it in the next page after the compiler
itself, \ie, at address \rs{hex(command_editor_code)}, which corresponds to
page \rs{dec_hex(page_number(command_editor_code))}:

\begin{TwoColumns}
\rs{c.func("save_code", &[], "", &["nolink"])}\\
\bytecode{
  c.cst(CODE_ADDRESS);
  c.cst8(page_number(command_editor_code).try_into().unwrap());
  c.call("buffer_flash");
  c.ret();
}
\end{TwoColumns}

Enter it in RAM after the ``compile'' function, at address
\rs{hex(SAVE_CODE_COMMAND_ADDRESS)}, by typing
``w\rs{hex_word_low(SAVE_CODE_COMMAND_ADDRESS)}''+Enter, followed by the full
code of this function, listed below. Then run it by typing
``w\rs{hex_word_low(SAVE_CODE_COMMAND_ADDRESS)}''+Enter, followed by ``r''.

\rs{c.get_bytecode_listing(0..c.get_instruction_count() as usize, false)}

\rust{
  context.type_ascii(&c.memory_editor_commands(SAVE_CODE_COMMAND_ADDRESS));
  context.type_ascii(&format!("W{:08X}\n", SAVE_CODE_COMMAND_ADDRESS));
  context.type_ascii("R");
}

\subsection{First commands}\label{subsection:first-commands}

\rust{
  let command_editor_main = context.memory_region("command_editor")
      .label_address("command_editor");
}

We can now try our command editor. Start it by typing
``w\rs{hex_word_low(command_editor_main)}''+Enter, followed by ``r'' (its main
function is at address \hexa{C0000}+%
\rs{dec(command_editor_main - 0xC0000)}=\rs{hex(command_editor_main)} -- see
\cref{subsection:command-editor-implementation}). The screen should now be
empty, because it displays command number 1, initially empty. Lets use this
command to show a welcome message when the command editor starts. Type ``e'' to
edit it, then type ``Welcome to the command editor." followed by Escape. At
this point the message you typed should be displayed in yellow. For now it is
only in RAM. Type ``s'' to save it in flash memory. We now want to define some
commands to create, load, edit, save, and compile a program.

\rust{
  context.type_ascii(&format!("W{:08X}\n", command_editor_main));
  context.type_ascii("R");
  assert!(display.borrow().get_text().is_empty());

  context.type_ascii("E");
  context.type_keys(vec!["Shift", "W", "~Shift"]);
  context.type_ascii("ELCOME TO THE COMMAND EDITOR.");
  context.type_keys(vec!["Escape"]);
  context.type_keys(vec!["S"]);
}
\bigskip

\paragraph*{New (F2)} initializes an empty text buffer at address
\rs{dec_hex(RAM_COMPILER_SOURCE)}, just after the memory region used by the
command editor (see \cref{fig:command-editor-memory-map}). Type ``F2'' followed
by ``e'' to edit it, then type its source code followed by Escape and ``s''
(the dummy data at the end describes the command):

\rust{
  context.type_keys(vec!["F2"]);
  let mut c = BytecodeAssembler::default();
  c.func("new_source_code", &[], "", &[]);
  c.new_line();
  c.cst(RAM_COMPILER_SOURCE);
  c.cst_0();
  c.store();
  c.new_line();
  c.cst_0();
  c.retv();
  let c_source = format!("{}\nd NEW_SOURCE_CODE", c.get_toy0_source_code());
  context.store_text(RAM_COMMAND_SOURCE, &c_source);
  context.type_keys(vec!["S"]);
}
\rs{code(&c_source)}

\paragraph*{Load (F3)} calls \hyperlink{buffer-copy}{buffer\_copy} to load a
program stored in flash memory at address \rs{dec_hex(compiler_source)}, just
after the command editor source code (see
\cref{fig:command-editor-memory-map}). Store its source code in command number
3:

\rust{
  context.type_keys(vec!["F3"]);
  let mut c = BytecodeAssembler::default();
  c.import_labels(context.memory_region("flash_driver"));
  c.func("load_source_code", &[], "", &[]);
  c.new_line();
  c.cst(compiler_source);
  c.cst(RAM_COMPILER_SOURCE);
  c.call("buffer_copy");
  c.new_line();
  c.cst_0();
  c.retv();
  let c_source = format!("{}\nd LOAD_SOURCE_CODE", c.get_toy0_source_code());
  context.store_text(RAM_COMMAND_SOURCE, &c_source);
  context.type_keys(vec!["S"]);
}
\rs{code(&c_source)}

\paragraph*{Edit (F4)} calls \hyperlink{text-editor}{text\_editor} to edit the
text buffer at address \rs{hex(RAM_COMPILER_SOURCE)}, with the word at address
\rs{dec_hex(RAM_COMPILER_CODE)} as initial offset, and a maximum length of
\rs{dec(MAX_COMPILER_SOURCE_KB)}~KB (including the 4 bytes header). The initial
offset corresponds to the header of a compiled code buffer (see
\cref{fig:command-editor-memory-map}) which, in case of a compilation error,
contains the error location. Hence, editing the source code after a compilation
error opens the text editor at the location of this error. Store the following
source code in command number 4:

\rust{
  context.type_keys(vec!["F4"]);
  let mut c = BytecodeAssembler::default();
  c.import_labels(context.memory_region("text_editor"));
  c.func("edit_source_code", &[], "", &[]);
  c.new_line();
  c.cst(RAM_COMPILER_SOURCE);
  c.cst(RAM_COMPILER_CODE); // 1st word contains offset of error if applicable
  c.load();
  c.cst(MAX_COMPILER_SOURCE_SIZE - 4);
  c.call("text_editor");
  c.new_line();
  c.cst_0();
  c.retv();
  let c_source = format!("{}\nd EDIT_SOURCE_CODE", c.get_toy0_source_code());
  context.store_text(RAM_COMMAND_SOURCE, &c_source);
  context.type_keys(vec!["S"]);
}
\rs{code(&c_source)}

\paragraph*{Save (F5)} calls \hyperlink{buffer-flash}{buffer\_flash} to save
the edited program in flash memory at address \rs{hex(compiler_source)}, which
corresponds to page \rs{dec(compiler_source_page)}. Store it in command number
5:

\rust{
  context.type_keys(vec!["F5"]);
  let mut c = BytecodeAssembler::default();
  c.import_labels(context.memory_region("flash_driver"));
  c.func("save_source_code", &[], "", &[]);
  c.new_line();
  c.cst(RAM_COMPILER_SOURCE);
  c.cst(compiler_source_page);
  c.call("buffer_flash");
  c.new_line();
  c.cst_0();
  c.retv();
  let c_source = format!("{}\nd SAVE_SOURCE_CODE", c.get_toy0_source_code());
  context.store_text(RAM_COMMAND_SOURCE, &c_source);
  context.type_keys(vec!["S"]);
}
\rs{code(&c_source)}

\paragraph*{Compile (F6)} calls \hyperlink{tc-main}{tc\_main} to compile the
source code at address \rs{hex(compiler_source)}, and to write the compiled
code at address \rs{dec_hex(RAM_COMPILER_CODE)} (just after the source code in
RAM, see \cref{fig:command-editor-memory-map}). It returns the compiler's
result, which is non-zero if a compilation error occurs. Store it in command
number 6:

\rust{
  context.type_keys(vec!["F6"]);
  let mut c = BytecodeAssembler::default();
  c.import_labels(context.memory_region("toyc0"));
  c.func("compile_source_code", &[], "", &[]);
  c.new_line();
  c.cst(compiler_source); // src_buffer
  c.cst(RAM_COMPILER_CODE); // dst_buffer
  c.call("tc_main");
  c.retv();
  let c_source = format!("{}\nd COMPILE_SOURCE_CODE",
  c.get_toy0_source_code());
  context.store_text(RAM_COMMAND_SOURCE, &c_source);
  context.type_keys(vec!["S"]);
}
\rs{code(&c_source)}

\paragraph*{Store (F7)} calls \hyperlink{buffer-flash}{buffer\_flash} to store
the compiled code in flash memory at address \rs{dec_hex(compiler_code)}, which
corresponds to page \rs{dec(compiler_code_page)} (after the command editor --
see \cref{fig:command-editor-memory-map}). Before that, this command backs up
the current compiled code by saving a copy of it at address
\rs{dec_hex(compiler_code_backup)}, which corresponds to page
\rs{dec(compiler_code_backup_page)}. Store it in command number 7:

\rust{
  context.type_keys(vec!["F7"]);
  let mut c = BytecodeAssembler::default();
  c.import_labels(context.memory_region("flash_driver"));
  c.func("save_compiled_code", &[], "", &[]);
  c.new_line();
  c.cst(compiler_code);
  c.cst(compiler_code_backup_page);
  c.call("buffer_flash"); // backup current compiled code
  c.new_line();
  c.cst(RAM_COMPILER_CODE);
  c.cst(compiler_code_page);
  c.call("buffer_flash"); // saved compiled code in RAM into flash
  c.new_line();
  c.cst_0();
  c.retv();
  let c_source = format!("{}\nd STORE_COMPILED_CODE", c.get_toy0_source_code());
  context.store_text(RAM_COMMAND_SOURCE, &c_source);
  context.type_keys(vec!["S"]);
}
\rs{code(&c_source)}

\paragraph*{Restore (F8)} calls \hyperlink{buffer-flash}{buffer\_flash} to
restore the backup created by the previous command, in case something goes
wrong. Store it in command number 8:

\rust{
  context.type_keys(vec!["F8"]);
  let mut c = BytecodeAssembler::default();
  c.import_labels(context.memory_region("flash_driver"));
  c.func("restore_backup_code", &[], "", &[]);
  c.new_line();
  c.cst(compiler_code_backup);
  c.cst(compiler_code_page);
  c.call("buffer_flash"); // backup current compiled code
  c.new_line();
  c.cst_0();
  c.retv();
  let c_source = format!("{}\nd RESTORE_BACKUP_CODE",
  c.get_toy0_source_code());
  context.store_text(RAM_COMMAND_SOURCE, &c_source);
  context.type_keys(vec!["S"]);
}
\rs{code(&c_source)}

\begin{Figure}
  \rs{define("mmapa", &hex(context.memory_region("toyc0").start))}
  \rs{define("mmapb", &hex(command_editor_code))}
  \rs{define("mmapc", &hex(compiler_code))}
  \rs{define("mmapd", &hex(COMMAND_SOURCE))}
  \rs{define("mmape", &hex(command_editor_source))}
  \rs{define("mmapf", &hex(compiler_source))}
  \rs{define("mmapg", &hex(compiler_code_backup))}
  \rs{define("mmaph", &hex(RAM_COMMAND_SOURCE))}
  \rs{define("mmapi", &hex(RAM_COMMAND_CODE))}
  \rs{define("mmapj", &hex(RAM_COMPILER_SOURCE))}
  \rs{define("mmapk", &hex(RAM_COMPILER_CODE))}
  \input{figures/chapter3/command-editor-memory-map.tex}

  \caption{The flash memory and RAM regions used by the command editor, and by
  the commands defined in \cref{subsection:first-commands}. White, blue and
  gray areas represent source code, bytecode and unused memory, respectively
  (not to scale).}\label{fig:command-editor-memory-map}
\end{Figure}

\subsection{Tests}

In order to test the above commands, type ``F2''+``r'' to create a new program,
and press Enter to return in the command editor. Then type ``F4''+``r'' to
edit this program, and type the following code, which contains an error on
purpose:

\rust{
  context.type_keys(vec!["F2"]);
  context.type_ascii("R");
  assert_eq!(display.borrow().get_text(), "00000000");
  context.type_ascii("\n");

  let c_source = "fn 0 cst_3 retv";

  context.type_keys(vec!["F4"]);
  context.type_ascii("R");
  context.type_ascii("FN 0 CST");
  context.type_keys(vec!["Shift", "-", "~Shift"]);
  context.type_ascii("3 RETV");
  assert_eq!(display.borrow().get_text(), c_source);
  context.type_keys(vec!["Escape"]);
  assert_eq!(display.borrow().get_text(), "00000000");
  context.type_ascii("\n");
}
\rs{code(c_source)}

\noindent Then type Escape to exit the text editor. The command's result, 0,
should be displayed. Press Enter to return in the command editor's main loop
(in the following we omit these ``press Enter'' instructions, for brevity).

Type ``F5''+``r'' to save this program, ``F2''+``r'' to create a new one, and
``F4''+``r'' to edit it. The screen should be empty. Type Escape to return
in the memory editor, then type ``F3''+``r'' to load the previously saved
program. Type ``F4''+``r'' to check that it is now loaded in RAM, and
Escape to return in the command editor.

\rust{
  context.type_keys(vec!["F5"]);
  context.type_ascii("R\n");
  context.type_keys(vec!["F2"]);
  context.type_ascii("R\n");
  context.type_keys(vec!["F4"]);
  context.type_ascii("R");
  assert!(display.borrow().get_text().is_empty());
  context.type_keys(vec!["Escape"]);
  context.type_ascii("\n");
  context.type_keys(vec!["F3"]);
  context.type_ascii("R\n");
  context.type_keys(vec!["F4"]);
  context.type_ascii("R");
  assert_eq!(display.borrow().get_text(), c_source);
  context.type_keys(vec!["Escape"]);
  context.type_ascii("\n");
}

Type ``F6''+``r'' to compile this program. The result should be 1, because {\tt
cst\_3} is an invalid opcode. Then type ``F4''+``r'' to edit the program. The
cursor should be just after this invalid opcode. Enter the correct code below:

\rust{
  context.type_keys(vec!["F6"]);
  context.type_ascii("R");
  assert_eq!(display.borrow().get_text(), "00000001");
  context.type_ascii("\n");

  context.type_keys(vec!["F4"]);
  context.type_ascii("R");
  let c_source = "fn 0 cst8 3 retv";
  context.type_keys(vec!["Backspace", "Backspace"]);
  context.type_ascii("8 3");
  assert_eq!(display.borrow().get_text(), c_source);
}
\rs{code(c_source)}

Finally, type Escape to exit the text editor, ``F5''+``r'' to save the
corrected code, and ``F6''+``r'' to compile it. The result should be 0 this
time.

\rust{
  context.type_keys(vec!["Escape"]);
  context.type_ascii("\n");
  context.type_keys(vec!["F5"]);
  context.type_ascii("R\n");
  context.type_keys(vec!["F6"]);
  context.type_ascii("R");
  assert_eq!(display.borrow().get_text(), "00000000");
  context.type_ascii("\n");
}

% This work is licensed under the Creative Commons Attribution NonCommercial
% ShareAlike 4.0 International License. To view a copy of the license, visit
% https://creativecommons.org/licenses/by-nc-sa/4.0/

\renewcommand{\rustfile}{chapter4}
\setcounter{rustid}{0}

\chapter[Control Circuits]{Control Circuits}\label{chapter:control-circuits}

Thanks to registers and memory circuits we can use an Arithmetic and Logic Unit
to perform computations without having to mentally memorize intermediate
results. Instead, as shown in the previous chapter, we can simply send a series
of pulse signals on the correct inputs, and in the correct order. But this
requires to memorize this procedure. And executing it manually is very slow and
error-prone, even if the circuit does each operation very quickly. To solve the
first issue a solution is to store some description of the desired procedure in
Random Access Memory. To solve the second one we can use new circuits to
execute this procedure for us, by sending the appropriate pulses. This chapter
explains how this can be done.

\section{Instructions}

A procedure such as the one presented in \cref{subsection:alu-and-ram-example}
could be described in an abstract way as ``read 3 numbers $a$, $b$, and $c$ in
input, compute $a+b-c$, and write the result in RAM at address $x$''. However,
representing such descriptions with one or more numbers which can be stored in
RAM is not easy. And figuring out which pulses to send to execute them would
also be quite complicated.

This procedure can also be described as a sequence of elementary actions:
``wait an input value'', ``send a pulse on selectInput'', ''send a pulse on
writeR0'', ``wait an input value'', ''send a pulse on writeR1'', etc. Each
action can easily be represented with a small number (\eg, 0 for ``wait an
input value'', 1 for ``send a pulse on selectInput'', etc). And each action is
easy to execute. However, such a description is hard to design and to
understand for humans (because its high level meaning is lost in the details).

A trade-off is to describe this procedure with more abstract actions, but not
too abstract either, called {\em instructions}. For instance, an instruction
could be ``wait an input value and store it in R0'', ``add the values in R0 and
R1 and store the result in R0'', or ``copy the value in R0 in RAM, at address
3''. As shown below, such instructions are not too complex to represent with a
number, called their {\em encoding} (to store them in memory). And they are
still quite simple to execute (each instruction only requires sending 2 or 3
pulses at most). Finally, a sequence of instructions is less hard to design and
to understand than the corresponding sequence of pulses (but still quite hard;
we address this problem in \cref{part:compiler}).

Simple procedures, also called {\em programs}, can be described with a sequence
of instructions, to be executed one after the other. For this we can store
their encoding one after the other in memory, \ie, at consecutive addresses.
Then, after the instruction at address $a$ is executed, the one at address
$a+1$ should execute\footnote{Assuming that each encoded instruction can fit in
the $n$ bits between two consecutive addresses.}.

\subsection{Jump instructions}

Some programs need to repeat the same sequence of instructions two or more
times. For instance, a ``calculator'' program needs to repeat forever the same
sequence (read two numbers in input, compute and output their sum, repeat). In
other words, after the last instruction of the sequence is executed, the
instruction at the next address should {\em not} be executed. Instead,
execution should restart at the first instruction of the sequence. This can be
described with a so called {\em jump instruction}. A ``jump to $a$''
instruction specifies that the next instruction to execute is the one at
address $a$.

In many cases a sequence of instructions must be repeated a precise number of
times. For instance, to compute $a*b$ with the circuit of
\cref{fig:alu-and-ram}, we can repeat $b$ times a sequence adding $a$ to R0
(initially set to 0). Then, after $a$ has been added to R0, there are two
cases: either we need to repeat the sequence again, or we need to continue with
the rest of the program (\eg, output the result $a*b$). This can be
described with a {\em conditional jump} instruction. Such an instruction either
jumps to a given address, or continues to the instruction at the next address,
depending on some condition (for instance, whether R0 is equal to 0 or not).

\section{A toy instruction set}\label{section:toy-insn-set}

To illustrate the above discussion we define in this section a concrete set of
instructions for a circuit such as the one in \cref{fig:alu-and-ram} (\ie, with
a RAM and two registers R0 and R1 as input of a very basic Arithmetic Unit).
These instructions are the following:
\begin{itemize}
  \item Memory:

  \begin{itemize}
    \item the Load instruction copies the value at a given address $a$ into the
    R0 register.

    \item the Store instruction copies the value in the R0 register at a
    given address $a$.
  \end{itemize}

  \item Arithmetic:

  \begin{itemize}
    \item the Add instruction adds the value at address $a$ to the value in the
    R0 register, and stores the result in R0.

    \item the Subtract instruction subtracts the value at address $a$ from the
    value in the R0 register, and stores the result in R0.
  \end{itemize}

  \item Jumps:

  \begin{itemize}
    \item the Jump instruction specifies that the next instruction to execute
    is the one at address $a$.

    \item the Jump If Zero instruction specifies that the next instruction to
    execute is the one at address $a$ if the value in R0 is equal to $0$.
    Otherwise execution continues with the instruction at the next address.

    \item the Jump If Carry instruction specifies that the next instruction to
    execute is the one at address $a$ if the last Add or Subtract instruction
    produced a non-zero carry bit. Otherwise execution continues with the
    instruction at the next address.
  \end{itemize}

  \item Input and output:

  \begin{itemize}
    \item the Input instruction waits for the user to press a button, and then
    copies the value on the input wires into the R0 register.

    \item the Output instruction displays the value in the R0 register, and
    then waits until the user presses a button.
  \end{itemize}
\end{itemize}

\subsection{Encoding}

The above {\em instruction set} contains 9 instructions. We can thus give them
numbers from 0 to 8, called {\em operation codes}, or {\em opcodes}. This
requires at least 4 bits to encode each instruction. But all instructions
except the last two have an associated address $a$, called an {\em operand}.
This operand must also be encoded as part of the instruction, which requires
more bits.

In the following we assume that the memory contains $2^5=32$ bytes, each with
their own address, and that R0, R1, and the Arithmetic Unit work on 8-bit
values. We then use 5 bits per address, and we encode each instruction in one
byte, as follows:

\bigskip \noindent
\rs{T8Instruction::Ldr(0).definition()}\\
\rs{T8Instruction::Str(0).definition()}\\
\rs{T8Instruction::Add(0).definition()}\\
\rs{T8Instruction::Sub(0).definition()}\\
\rs{T8Instruction::Jump(0).definition()}\\
\rs{T8Instruction::IfZ(0).definition()}\\
\rs{T8Instruction::IfC(0).definition()}\\
\rs{T8Instruction::In.definition()}\\
\rs{T8Instruction::Out.definition()}\\

The left column is the {\em instruction mnemonic}, an abbreviation of the
instruction name. The middle column is a symbolic description of the effect
each instruction. Here $\mathit{dst} \leftarrow \mathit{src}$ or $\mathit{src}
\rightarrow \mathit{dst}$ means a copy of the value in $\mathit{src}$ into
$\mathit{dst}$, and $\mathrm{mem8}[a]$ means the 8-bit value at address $a$.
Finally, the right column is the binary number corresponding to this
instruction, \ie, its encoding. For instance, the encoding of the LDR 7
instruction, which copies the byte at address $7=111_2$ into R0, is
$001_2$ followed by $7$ encoded in $5$ bits, $00111_2$, which gives
$00100111_2=39$.

\subsection{Example program}\label{subsection:adder-program}

With the above instruction set a ``calculator'' program adding numbers in an
endless loop can be implemented as follows:

\rust{
  let mut a = T8Program::default();
  a.input();
  a.str(6);
  a.input();
  a.add(6);
  a.output();
  a.jump(0);
  a.data(0, "the $a$ number");
}
\rs{a.get_listing()}

\rust{
  let outputs = T8Emulator::new().emulate(&a.get_machine_code(),
      &[7, 13, 17, 19], 2);
  assert_eq!(outputs, &[7 + 13, 17 + 19]);
}

\noindent where the left part gives the symbolic description of each
instruction, and the right part their encoding and their address (in gray).

The first two instructions read a number $a$ as input and store it at address
$6$. The next two instructions read a second number $b$, and add $a$ to it. The
last two instructions output the value in R0, which at this stage contains
$a+b$, and jump back to the first instruction to add two new numbers. The next
byte after these five instructions is the one used to store $a$.

\subsection{Notes}\label{subsection:int-overflow}

Adding two 8-bit numbers can give a 9-bit number. For instance,
$11111111_2=255$ plus $1$ gives the 9-bit number $100000000_2=256$. However,
the registers and the memory can only store 8-bit numbers, and the Arithmetic
Unit can only use 8-bit numbers as input. Hence, in practice, and unless a
program does something special with the carry bit (with the IFC instruction),
all additions are {\em modulo} $2^8=256$. This means that adding $a$ and $b$
does not give $a+b$ but the remainder of the division of $a+b$ by 256. It is
noted $(a+b) \mod 256$, where $x \mod m$ is defined as $x - \lfloor x / m
\rfloor * m$. For instance, adding $255$ and $1$ gives $0$\footnote{This {\em
modular arithmetic} is used in everyday life with hours. For example, 10 a.m
plus 5 hours is 3 p.m because $(10+5) \mod 12 = 3$.}.

Similarly, subtracting two numbers can give a negative result, but the
registers and the memory can only store nonnegative numbers. Hence, in
practice, and unless a program does something special with the carry bit, all
subtractions are modulo $256$ too. For instance, subtracting $1$ from $0$ gives
$255$ because $-1 \mod 256 = -1 - \lfloor -1 / 256 \rfloor * 256 = -1 -
(-1)*256 = 255$ (recall that $\lfloor y \rfloor$ means the integer part of $y$).

When $a+b$ differs from $(a+b) \mod 2^n$ we say that there is an (integer) {\em
overflow} (where $n$ is the Arithmetic Unit's ``bit width''). We say the same
when $a-b \ne (a-b) \mod 2^n$, $a*b \ne (a*b) \mod 2^n$, etc. With an
Arithmetic Unit such as the one in \cref{fig:alu}, there is an overflow if and
only if the carry output is $1$.

\section{Control circuits}

We now have a way to describe a sequence of instructions with some numbers
stored in memory. The next step, as described in the introduction of this
chapter, is to build a circuit to automatically {\em execute} these
instructions. Which means sending a corresponding sequence of pulse signals on
the registers, memory, and bus circuits. For instance, to execute an IN
instruction with the circuit in
\cref{fig:alu-and-ram}, one needs to:
\begin{itemize}
  \item connect the input wires to the bus by sending a pulse on
  ``selectInput''.

  \item wait for the user to press a button.

  \item store the input value in R0 by sending a pulse on ``writeR0''.
\end{itemize}

More generally, all instructions can be executed by sending appropriate pulses
1) on the correct wires, 2) in the correct order, and 3) at appropriate times
(signals must have time to propagate throughout the circuit between two
pulses). The first two items can be ensured with circuits of the following
form:

\begin{center}
  \input{figures/chapter4/sequencer1.tex}
\end{center}

When this circuit is powered on ``wire1'' and ``wire2'' change from 0 to 1 due
to the NOT gate. Changing $c$ from $0$ to $1$ (and back to $0$) makes the first
and second D flip-flops memorize 1. This resets ``wire1'' and ``wire2'' to 0,
and sets ``wire3'' to 1:

\begin{center}
  \input{figures/chapter4/sequencer2.tex}
\end{center}

Changing $c$ from $0$ to $1$ (and back to $0$) again makes the second and third
flip-flops memorize 0 and 1, respectively. This resets ``wire3'' to 0, and sets
``wire4'' to 1:

\begin{center}
  \input{figures/chapter4/sequencer3.tex}
\end{center}

Finally, changing $c$ from $0$ to $1$ (and back to $0$) one more time resets
``wire4'' to 0:

\begin{center}
  \input{figures/chapter4/sequencer4.tex}
\end{center}

In other words, with a series of pulses on the $c$ input, one gets two
simultaneous pulses on ``wire1'' and ``wire2'', followed by a pulse on
``wire3'' and then on ``wire4'':

\begin{center}
  \input{figures/chapter4/sequencer-signals.tex}
\end{center}

Each wire pulse starts and ends at the precise moment when $c$ switches from
$0$ to $1$. This shows that, with circuits like the one above, it is possible
to send pulses on specific wires, in a specific order. The only requirement is
the ability to send a series of pulses on a shared input $c$, which can be done
with a {\em clock}.

\subsection{Clock}

A clock is a circuit which generates a signal switching between $0$ and $1$ at
a constant frequency. A clock can be implemented in many ways. For instance,
one could use a pendulum acting on a switch. But this would not be very
practical, and can not produce high frequencies. Instead, a frequently used
method is to use the oscillations of a crystal. Crystals can oscillate one
million times per second or more (\ie, at $1$~MHz or more). In the following we
represent a clock with the symbol on the left:

\begin{center}
  \input{figures/chapter4/clock-symbol.tex}
  \input{figures/chapter4/clock-signal.tex}
\end{center}

A clock generates the signal shown above (right). A {\em period}, also called a
{\em clock cycle}, is the time between successive pulses. The clock frequency
is the number of pulses per second, \ie, the inverse of its period. Increasing
the frequency increases the number of instructions which are executed per
second. However, the frequency cannot be increased without limit. Indeed, there
must be enough time between two pulses for signals to propagate throughout the
circuit. For instance, computing an addition in an Arithmetic Unit takes some
time, because the input values have to propagate through all its logic gates,
up to the carry output. If a pulse is sent to write the sum in a register
before this delay, a wrong result will be stored.

\subsection{Control loop}

A circuit like the one above can generate a sequence of pulses to execute one
instruction. But each type of instruction needs a different sequence of pulses
to be executed. The solution is to use several circuits like this, one per type
of instruction. And to connect them to a binary decoder, so that the correct
subcircuit is used depending on the instruction opcode. For instance, if there
are only 4 different opcodes, we can use a circuit similar to the following:

\begin{center}
  \input{figures/chapter4/decode-and-execute.tex}
\end{center}

Depending on the two bits of the instruction opcode, $op_1op_0$, the above
circuit sends pulses on ``wire0-1'' to ``wire0-3'', or on ``wire1-1'' to
``wire1-3'', etc. Before this the instruction must be read in memory, so that
$op_1op_0$ contain the correct values. This can be done, as shown later, with a
so called FETCH circuit sending an appropriate sequence of pulses. Finally,
after the instruction has been executed, the next one must be fetched, decoded,
and executed. For this it suffice to connect the outputs of the EXECUTE
subcircuit back to the input of the FETCH circuit:

\begin{center}
  \input{figures/chapter4/fetch-decode-execute.tex}
\end{center}

In this way we get a pulse which loops forever in the FETCH, DECODE and EXECUTE
circuits, each time going through a specific EXECUTE subcircuit.

\section{A toy control unit}\label{section:toy-control-unit}

To illustrate the above discussions we design in this section a very basic {\em
control unit} for the circuit of \cref{fig:alu-and-ram-schema} (with an 8-bit
{\em architecture}, \ie, an 8-bit Arithmetic Unit, 8-bit registers, etc). As
its name implies, a control unit controls the rest of the circuit, called the
{\em processing unit} (\ie, the Arithmetic and Logic Unit, the registers, the
bus, etc). It does so by executing instructions stored in memory. We assume
here that these instructions are those defined in \cref{section:toy-insn-set}.

\begin{Figure}
  \input{figures/chapter4/T8-schema.tex}

  \caption{A basic control unit (yellow background) for the circuit in
  \cref{fig:alu-and-ram-schema} (white background), with the instruction set of
  \cref{section:toy-insn-set}.}\label{fig:T8-schema}
\end{Figure}

The core part of our example control unit is a control loop circuit with FETCH,
DECODE, and EXECUTE subcircuits, as presented above. To implement it we need
two new registers, in addition to R0 and R1 (see \cref{fig:T8-schema}):
\begin{itemize}
  \item the {\em Program Counter} (PC) register stores the address of the
  instruction being currently executed or, once it has been executed, the
  address of the next instruction to execute. Since addresses use only 5 bits,
  this register is a 5-bit register.

  \item the {\em Instruction Register} (IR) stores the encoding of the
  instruction being currently executed. This 8-bit register stores a copy of
  the original instruction in memory. This is necessary to have access to its
  value during its execution, which might require reading or writing values at
  other addresses in memory.
\end{itemize}

Once an instruction has been executed, the Program Counter value must be
incremented by one to execute the next instruction. Unless the last instruction
was a jump. In this case the Program Counter value must be replaced with the
operand of this jump instruction. To do this we include two more circuits in
our control unit (see \cref{fig:T8-schema}):
\begin{itemize}
  \item a 5-bit incrementer, which computes ``PC+1'', \ie, the value in the
  Program Counter register plus $1$. This is an adder circuit similar to the one
  in \cref{subsection:adder-circuit}, simplified for the case where one
  input is always $1$.

  \item a 5-bit {\em address bus}, to which we connect the Program Counter, the
  output of the above incrementer, and the 5 least significant bits of the
  Instruction Register (\ie, the address operand). This bus is also connected
  to the address decoder of the RAM and thus selects which address to read or
  write to.
\end{itemize}

Thanks to these components, we can increment the Program Counter by connecting
the output of the incrementer to the address bus, and by sending a pulse on
``writePC'' to store this value (see \cref{fig:T8-schema}). Likewise, we can
replace the Program Counter with the operand of a jump instruction by
connecting the Instruction Register to the address bus, and by sending a pulse
on ``writePC''.

\subsection{FETCH circuit}

With the above architecture, fetching an instruction can be done as follows:
\begin{itemize}
  \item send simultaneous pulses on ``selectPC'' and ``selectRAM'' to read the
  value in memory at the address stored in the Program Counter, and to get it
  on the data bus.

  \item send a pulse on ``writeIR'' to write this value in the Instruction
  Register.

  \item send a pulse on ``selectIR'' to prepare reading or writing a value at
  the address operand of the new instruction. Technically this step is part of
  the instruction's execution, but we include it in the FETCH circuit to
  avoid duplications (it is common to all instructions except IN and OUT).
\end{itemize}

\subsection{DECODE circuit}

Decoding an instruction can be done with a binary decoder with 3 inputs, namely
the 3 most significant bits of the Instruction Register. Plus a single
demultiplexer, controlled by the $4^{th}$ most significant bit, in order to
distinguish the IN and OUT instructions (the 3 most significant bits are
$111_2$ for both instructions).

\subsection{EXECUTE circuit}

The EXECUTE circuit has 9 subcircuits, one per type of instruction:

\medskip \paragraph{LDR} This subcircuit sends a pulse on ``writeR0'' to store
the value read from memory at the instruction's address operand (selected by
the last step of the FETCH circuit). It then increments the PC value with a
pulse on ``selectPC+1'', followed by one on ``writePC''.

\medskip \paragraph{STR} This subcircuit sends a pulse on ``selectR0'',
followed by a pulse on $w$ to store R0's value in memory, at the instruction's
address operand (selected by the last step of the FETCH circuit). It then
increments the PC value, as above.

\medskip \paragraph{ADD} This subcircuit sends a pulse on ``writeR1'' to store
the value at the instruction's address operand in R1. It then sends a pulse on
``selectALU'' to get the sum of the values in R0 and R1 on the bus, followed by
simultaneous pulses on ``writeR0'' and ``writeCarry'' to write it in R0 and
Carry. It then increments the PC value as above.

\medskip \paragraph{SUB} This subcircuit is almost the same as the ADD
subcircuit. It just sends an additional pulse on ``subtract'', at the same time
as the ``selectALU'' pulse. These pulses last until the one on ``writeR0''
starts. This ensures that the correct result, the difference of R0 and R1
values, is written in R0.

\medskip \paragraph{JMP} This subcircuit just sends a pulse on ``writePC'' to
replace the Program Counter with the instruction's address operand (selected by
the last step of the FETCH circuit).

\medskip \paragraph{IFZ} This subcircuit has two branches. The first, executed
if the value in R0 is $0$, is the same as the JMP subcircuit. The second,
executed if R0's value is not $0$, increments the PC value as for non-jump
instructions. The two branches are connected to a demultiplexer controlled by
the ``$\ne 0$'' signal (see \cref{fig:T8-schema}):

\begin{center}
  \input{figures/chapter4/jump-insn.tex}
\end{center}

\paragraph{IFC} This subcircuit is almost the same as the IFZ one,
except that its demultiplexer is controlled by the value of the Carry register.

\medskip \paragraph{IN} This subcircuit sends a pulse on ``selectInput'', and
then waits until a button is pressed. This can be done with a loop similar to
the control loop, with a demultiplexer to either wait, or to continue with the
next instruction:

\begin{center}
  \input{figures/chapter4/in-insn.tex}
\end{center}

\noindent In the latter case, this subcircuit sends a pulse on ``writeR0'', and
then increments the Program Counter as above.

\medskip \paragraph{OUT} This subcircuit sends a pulse on ``selectR0'' and then
waits until a button is pressed, with the same method as above. It then
increments the Program Counter's value.

% This work is licensed under the Creative Commons Attribution NonCommercial
% ShareAlike 4.0 International License. To view a copy of the license, visit
% https://creativecommons.org/licenses/by-nc-sa/4.0/

\renewcommand{\rustfile}{chapter5}
\setcounter{rustid}{0}

\rust{
  context.write_backup("website/backups", "shell.txt")?;
}

\chapter[Shell, Text Editor, and Compiler]{Shell, Text Editor,\\and
Compiler}\label{chapter:shell}

Our kernel can spawn an initial process which can now use all the computer
resources, thanks to the system calls added in the previous chapter. However,
we only have a useless test initial program for now. Moreover, in order to
store new programs on disk we need the command editor to edit, compile, and
save them. And we need to switch back and forth between the basic input output
system and the operating system to launch them. The goal of this chapter is to
solve these issues by making it possible to edit, compile, save, and launch
programs from the operating system alone.

For this we need a text editor and a compiler which can run as processes, and
which can read and write files (instead of data buffers). We already have a
text editor, but only in binary bytecode form. We thus need to rewrite it in
Toy, and to update it to work with files. Our compiler is already in Toy, but
also needs to be updated to work with files. Finally, to launch arbitrary
programs, we need something like the command editor, but more practical, called
a {\em shell}. This chapter provides these 3 programs. We test them at the
end with a ``Hello, Word!'' application.

\section{Shell}

\subsection{Requirements}

The shell must allow the user to interactively launch arbitrary programs. For
this the user must be able to specify which program to launch, and its
arguments. On disk programs and data are stored as files, and can thus be
specified with file names. Hence, for instance, ``{\tt toyc hello hello.toy}''
could be used to specify that the program stored in the file named ``{\tt
toyc}'' must be launched with the file names ``{\tt hello}'' and ``{\tt
hello.toy}'' as arguments. We thus require the following:
\begin{itemize}
  \item the shell should allow the user to type a line of text called a {\em
  command line}, or {\em command}, after a {\em prompt} ``{\tt >}''. To
  simplify, we do not require the possibility to insert or delete characters in
  the middle of a command, but only at the end.

  \item commands must be of the form ``{\em <program>} {\em <arguments>}''.
  Typing Enter should spawn the program stored in the file named {\em
  <program>}, with {\em <arguments>} as arguments (which can be empty). The
  launched program is then responsible to parse {\em <arguments>} in order to
  extract each individual argument.

  \item the shell must display on screen the output of the last executed
  command. Either an error message if the program could not be launched, or the
  text written to standard output by this program. It should then repeat the
  above steps.
\end{itemize}

To save memory we only require our shell to display the last executed command,
its output, and the currently edited command. Older commands are not displayed.
For instance, after a ``{\tt hello}'' command writing ``Hello, World!'' to
standard output, and while typing a new ``{\tt edit hello.toy}'' command, the
screen should display:
\begin{Code}
>hello
Hello, World!
>edit hello.to\_
\end{Code}

To simplify, for now, we do not require the possibility to save commands, as in
the command editor. Still, to avoid having to re-type recent commands, we
require the shell to keep in memory a {\em history} of the last $N$ executed
commands. It should then be possible, with the arrow up and down keys, to
replace the currently edited command with one from the above history.

\subsection{Data structures}

In order to meet the above requirements we use a buffer to store the currently
edited command, $N$ buffers to store the previous $N$ commands, and one buffer
to store the output of the last executed one (see
\cref{fig:shell-data-structures}).

After a command is run we must 1) add it to the history, and 2) remove the
oldest command from this history. To make this is easy to implement we link the
$N+1$ command buffers in a circular {\em doubly linked list}. This means that
each buffer has a link to a next one and to a previous one, and that these
links form two rings going in opposite directions (see the green and red links
in \cref{fig:shell-data-structures}). We also use a pointer to the currently
edited command. Then, after this command is run, it suffice to update this
pointer to the next command, and to clear this command, to automatically
achieve 1) and 2) above:

\begin{center}
\input{figures/chapter5/command-history.tex}
\end{center}

\noindent In fact this only requires links from one command to the next. We use
the opposite links to easily go back in the history, with the arrow up key.

In summary we use a {\tt Command} data structure, with two pointers to previous
and next commands, and a length indicating the number of characters of this
command (followed by the characters themselves -- see
\cref{fig:shell-data-structures}). We also use a {\tt Shell} data structure,
containing a pointer to one of the commands, as well as pointers to the output
of the last command: its beginning (inclusive), its end (exclusive), and the
limit of the underlying buffer (exclusive -- see
\cref{fig:shell-data-structures}).

\begin{Figure}
  \input{figures/chapter5/data-structures.tex}

  \caption{A shell with a history of two commands ($N=2$). The user is
  currently editing an ``edit hello.to'' command, of length 13. The previous
  command was ``hello'' and its output was ``Hello, World!''. The command
  before it was ``foo''.}\label{fig:shell-data-structures}
\end{Figure}

\subsection{Implementation}

We implement the shell by editing our test application, which already
contains some useful code, in order to avoid re-typing it. After the {\tt
entry} function, unchanged, we insert the {\tt load8}, {\tt store8}, {\tt
load16}, and {\tt store16} functions, copied from our native compiler (the last
two are not needed right now but will be useful later).

\rust{
  let command_editor_source = context.memory_region("command_editor_source");
  let ram_command_source =
      command_editor_source.label_address("ram_command_source");
  let ram_source = command_editor_source.label_address("ram_compiler_source");
  let kernel_source = context.memory_region("kernel_source").start;
  let kernel_code = context.memory_region("kernel_code").start;
  let application_source = context.memory_region("application_source").start;
  let compiler_source = context.memory_region("compiler_source").start;

  let display = Rc::new(RefCell::new(TextDisplay::default()));
  context.set_display(display.clone());
  context.micro_controller().borrow_mut().reset();
  context.run_until_get_char();

  // Start command editor.
  let command_editor_main = context
      .memory_region("command_editor")
      .label_address("command_editor");
  context.type_ascii(&format!("W{:08X}\n", command_editor_main));
  context.type_ascii("R");
  assert!(display
      .borrow()
      .get_text()
      .contains("Welcome to the command editor."));

  let mut t = Transpiler5::new_str(&context.get_text(application_source));
  t.add_unchanged("fn main(", "fn system_call(");
}

\toy{
fn load8(ptr: &u32) -> u32 [ /*LDRB_R0_R0_0*/30720; /*MOV_PC_LR*/18167; ]
fn load16(ptr: &u32) -> u32 [ /*LDRH_R0_R0_0*/34816; /*MOV_PC_LR*/18167; ]
fn store8(ptr: &u32, value: u32) [ /*STRB_R1_R0_0*/28673; /*MOV_PC_LR*/18167; ]
fn store16(ptr: &u32, value: u32) [ /*STRH_R1_R0_0*/32769; /*MOV_PC_LR*/18167; ]
}%toy

We continue with a function to read a token from a command, \ie, the program
name or an argument. This function uses a simplified version of
\cref{alg:scanner0}, with a pointer to $\mathit{src}$ instead of $\mathit{src}$
itself as parameter. It returns a pointer to the token's first character,
stores its length at address $\mathit{length}$ and updates $\mathit{src}$ to
the address of the first character after it. This allows reading a command with
repeated calls to this function with the same first two arguments.

\toy{
fn sh_read_token(src_p: &&u32, src_end: &u32, length: &u32) -> &u32 {
  let src = *src_p;
  while src < src_end && load8(src) == ' ' { src = src + 1; }
  if src >= src_end { return null; }
  let token = src;
  while src < src_end && load8(src) != ' ' { src = src + 1; }
  *length = src - token;
  *src_p = src;
  return token;
}
}%toy

\rust{
  t.add_unchanged("fn system_call(", "const OK: u32");
}

We also copy, after the existing {\tt OK} constant, the error codes from the
kernel (which are useful to return errors and analyze the result of commands):

\toy{
@const OK: u32 = 0;
const INVALID_ARGUMENT: u32 = 1;
const INVALID_STATE: u32 = 2;
const NOT_FOUND: u32 = 3;
const ALREADY_EXISTS: u32 = 4;
const OUT_OF_MEMORY: u32 = 5;
const INTERNAL_ERROR: u32 = 6;
}%toy

The shell needs to draw text on the screen, with double buffering to avoid
flickering (see \cref{subsection:double-buffering}). For this we implement the
following functions, after the existing system call functions (from {\tt
status} to {\tt reboot}). They are re-implementations of functions of the same
names in \cref{section:gpu-driver,section:ted-implementation}, in Toy and using
system calls (see these sections and the {\tt draw\_char} function from
\cref{section:streams-tests} for more details).

\rust{
  t.add_unchanged("fn status(", "fn wait_char(");
}

\toy{
fn gpu_set_register(id: u32, value: u32) {
  let buffer = (value & 255) << 16 | (32768 /*Select Register*/ | id);
  write(GPU, &buffer, 4);
}

fn gpu_set_double_buffer() {
  gpu_set_register(32 /*Display Configuration*/, 128);
  gpu_set_register(65 /*Memory Write Control 1*/, 1);
}

fn gpu_set_single_buffer() {
  gpu_set_register(65 /*Memory Write Control 1*/, 0);
  gpu_set_register(82 /*Layer Transparency 0*/, 0);
  gpu_set_register(32 /*Display Configuration*/, 0);
}

fn gpu_switch_buffer() {
  let buffer = 32768 /*Select Register*/ | 65 /*Memory Write Control 1*/;
  write(GPU, &buffer, 2);
  let layer = 0;
  read(GPU, &layer, 1);
  buffer = /*0 (Write Data) | */ 1 - layer;
  write(GPU, &buffer, 2);
  gpu_set_register(82 /*Layer Transparency 0*/, layer);
}

fn gpu_clear_screen() {
  gpu_set_register(142 /*Memory Clear Control*/, 192);
  let buffer = 0;
  loop {
    read(GPU, &buffer, 1);
    if buffer & 128 == 0 { return; }
  }
}

fn gpu_set_cursor(col: u32, row: u32) {
  gpu_set_register(42 /*Font Write Cursor H Position 0*/, col << 3);
  gpu_set_register(43 /*Font Write Cursor H Position 1*/, col >> 5);
  gpu_set_register(44 /*Font Write Cursor V Position 0*/, row << 4);
  gpu_set_register(45 /*Font Write Cursor V Position 1*/, row >> 4);
}

fn gpu_set_color(r: u32, g: u32, b: u32) {
  gpu_set_register(99 /*Foreground Color 0*/, r);
  gpu_set_register(100 /*Foreground Color 1*/, g);
  gpu_set_register(101 /*Foreground Color 2*/, b);
}

fn gpu_draw_char(c: u32) { gpu_set_register(2, c); }
}%toy

The shell also needs to allocate memory for its data structures, and to copy
commands from the history. For this we copy the following functions from the
Toy compiler and from the kernel (with a {\tt return null} instead of a {\tt
panic}):

\toy{
fn mem_allocate(size: u32, heap_p: &&u32, heap_limit: &u32) -> &u32 {
  let ptr = *heap_p;
  if size > heap_limit as u32 || ptr > heap_limit - size { return null; }
  *heap_p = ptr + size;
  return ptr;
}
fn mem_copy_non_overlapping(src: &u32, dst: &u32, size: u32) {
  let i = 0;
  while i < size {
    store8(dst + i, load8(src + i));
    i = i + 1;
  }
}
}%toy

We can now start the ``real'' shell implementation, beginning with a definition
of its data structures, and of their maximum sizes ({\tt NUM\_COMMANDS} is
equal to $N+1$; {\tt data} contains the first 4 characters of a command;
hence, {\tt \&c.data} points the first character of command {\tt c}):

\toy{
const NUM_COMMANDS: u32 = 4;
const MAX_COMMAND_LENGTH: u32 = 196;
const MAX_OUTPUT_SIZE: u32 = 512;

struct Command {
  previous: &Command,
  next: &Command,
  length: u32,
  data: u32
}
struct Shell {
  commands: &Command,
  output_begin: &u32,
  output_end: &u32,
  output_limit: &u32
}
}%toy

We continue with a function to copy the text of a command into another, and a
function to create a {\tt Shell} struct. The latter allocates all the required
memory, returns {\tt null} if this fails, and initializes the shell and its
commands otherwise. In particular, it initializes the {\tt previous} and {\tt
next} command links as illustrated in \cref{fig:shell-data-structures}. These
links stay unchanged after that.

\toy{
fn command_copy(src: &Command, dst: &Command) {
  dst.length = src.length;
  mem_copy_non_overlapping(&src.data, &dst.data, src.length);
}

fn sh_new(heap_p: &&u32, heap_limit: &u32) -> &Shell {
  let shell = mem_allocate(sizeof(Shell), heap_p, heap_limit) as &Shell;
  let command_size = sizeof(Command) - 4 + MAX_COMMAND_LENGTH;
  let command =
      mem_allocate(NUM_COMMANDS * command_size, heap_p, heap_limit) as &Command;
  let output = mem_allocate(MAX_OUTPUT_SIZE, heap_p, heap_limit);
  if shell == null || command == null || output == null { return null; }
  shell.commands = command;
  shell.output_begin = output;
  shell.output_end = output;
  shell.output_limit = output + MAX_OUTPUT_SIZE;
  let i = 0;
  let previous_command = command + (NUM_COMMANDS - 1) * command_size;
  while i < NUM_COMMANDS {
    previous_command.next = command;
    command.previous = previous_command;
    command.length = 0;
    previous_command = command;
    command = command + command_size;
    i = i + 1;
  }
  return shell;
}
}%toy

The next function appends up to {\tt length} characters from {\tt src} to the
shell's output buffer (depending on its remaining capacity):

\toy{
fn sh_print(self: &Shell, src: &u32, length: u32) {
  if length > self.output_limit - self.output_end {
    length = self.output_limit - self.output_end;
  }
  mem_copy_non_overlapping(src, self.output_end, length);
  self.output_end = self.output_end + length;
}
}%toy

We use it in the following function, which runs the command in $[\mathit{src},
\mathit{src\_end}[$. This function starts by extracting the program name, \ie,
the command's first token. If it is not empty it spawns this program, with the
rest of the command as arguments, and the shell's output buffer as standard
output buffer. It then appends to this buffer an error message if the program
could not be launched, if it ran out of memory, or if it crashed. Finally, it
appends a new line character if the output of the previous steps is not empty
and does not already ends with a new line:

\toy{
static CANT_FIND = ['C','a','n',''','t',' ','f','i','n','d',' '];
static CANT_LAUNCH = ['C','a','n',''','t',' ','l','a','u','n','c','h',' '];
static NOT_ENOUGH_MEMORY = ['O','u','t',' ','o','f',' ',\
'm','e','m','o','r','y'];
static CRASHED = [' ','c','r','a','s','h','e','d'];

const NEW_LINE: u32 = 10;

fn sh_run(self: &Shell, src: &u32, src_end: &u32) {
  let length = 0;
  let name = sh_read_token(&src, src_end, &length);
  if name == null { return; }
  let old_end = self.output_end;
  let result = spawn(self.output_limit, name, length, src, src_end - src, \
&self.output_end);
  if status(result) == NOT_FOUND {
    sh_print(self, CANT_FIND, 11);
    sh_print(self, name, length);
  } else if status(result) != OK {
    sh_print(self, CANT_LAUNCH, 13);
    sh_print(self, name, length);
  } else if result == OUT_OF_MEMORY {
    sh_print(self, NOT_ENOUGH_MEMORY, 13);
  } else if result == INTERNAL_ERROR {
    sh_print(self, name, length);
    sh_print(self, CRASHED, 8);
  }
  let new_line = NEW_LINE;
  if self.output_end > old_end && load8(self.output_end - 1) != NEW_LINE {
    sh_print(self, &new_line, 1);
  }
}
}%toy

The next 3 functions are used to draw commands and their output on screen. The
first one draws the characters in $[\mathit{src}, \mathit{src\_end}[$, starting
at column and row ($\mathit{col}$, {\tt *}$\mathit{row}$) on screen. It
automatically starts a new line when a ``new line'' character is found
or when the maximum line width is reached (100 characters). To simplify,
tabulations are not supported.

\toy{
fn sh_draw_string(src: &u32, src_end: &u32, col: u32, row: &u32) {
  let c = 0;
  while src < src_end {
    c = load8(src);
    src = src + 1;
    if c != NEW_LINE {
      gpu_draw_char(c);
      col = col + 1;
    }
    if c == NEW_LINE || col == 100 {
      col = 0;
      *row = *row + 1;
      gpu_set_cursor(col, *row);
    }
  }
}
}%toy

The second one draws the given command, after a ``{\tt >}'' prompt, and updates
{\tt *}$\mathit{row}$ to the next screen row:

\toy{
fn sh_draw_command(command: &Command, row: &u32) {
  gpu_draw_char('>');
  sh_draw_string(&command.data, &command.data + command.length, 1, row);
  *row = *row + 1;
}
}%toy

The third one draws the last executed command in green (if it is not empty),
its output in yellow (if it is not empty), and the currently edited command in
green:

\toy{
fn sh_draw(self: &Shell, current_command: &Command) {
  gpu_clear_screen();
  gpu_set_cursor(0, 0);
  gpu_set_color(0, 7, 0);
  let row = 0;
  if current_command.previous.length > 0 {
    sh_draw_command(current_command.previous, &row);
    gpu_set_cursor(0, row);
  }
  if self.output_end > self.output_begin {
    gpu_set_color(7, 7, 0);
    sh_draw_string(self.output_begin, self.output_end, 0, &row);
    gpu_set_cursor(0, row);
    gpu_set_color(0, 7, 0);
  }
  sh_draw_command(current_command, &row);
  gpu_switch_buffer();
}
}%toy

The next function handles characters typed on the keyboard in a loop, and
redraws the screen after each key typed (with the above function). It maintains
two pointers:
\begin{itemize}
  \item {\tt current} points to the currently edited command. It is moved to
  the next command after the current one is run.

  \item {\tt history} can point to any command. It is moved to the previous or
  next command with the arrow up and down keys, respectively. Selecting a new
  history command copies it into the current one.
\end{itemize}
\noindent The supported keys and their associated actions are the following:
\begin{itemize}
  \item the Escape key exits the shell. This can fail if the current process is
  the initial one (this is not always the case since the shell can also be
  spawned by a process). In this case we restore double buffering, which is
  disabled before calling {\tt exit}.

  \item the Enter key executes the current command, sets the next command as
  the current one, clears it, and resets the {\tt history} pointer to {\tt
  current}.

  \item the ArrowUp key moves the history command to the previous one, unless
  there is none (\ie, unless this would circle back to the current one).

  \item the ArrowDown key moves the history command to the next one, if there
  is one. If this goes back to the current command we clear it instead of
  copying it to itself.

  \item the Backspace key deletes the last character of the current command.
  Printable ASCII characters are appended to the current command, if it is not
  full.
\end{itemize}

\toy{
const BACKSPACE_KEY: u32 = 8;
const ENTER_KEY: u32 = 10;
const ESCAPE_KEY: u32 = 27;
const DELETE_KEY: u32 = 127;
const ARROW_UP_KEY: u32 = 245;
const ARROW_DOWN_KEY: u32 = 242;

fn sh_run_editor(self: &Shell) -> u32 {
  gpu_set_double_buffer();
  let current = self.commands;
  let history = current;
  let c = 0;
  loop {
    sh_draw(self, current);
    read(STANDARD_INPUT, &c, 1);
    if c == ESCAPE_KEY {
      gpu_set_single_buffer();
      exit(OK);
      gpu_set_double_buffer();
    } else if c == ENTER_KEY {
      self.output_end = self.output_begin;
      sh_run(self, &current.data, &current.data + current.length);
      gpu_set_double_buffer();
      current = current.next;
      current.length = 0;
      history = current;
    } else if c == ARROW_UP_KEY && history.previous != current {
      history = history.previous;
      command_copy(history, current);
    } else if c == ARROW_DOWN_KEY && history != current {
      history = history.next;
      if history == current {
        current.length = 0;
      } else {
        command_copy(history, current);
      }
    } else if c == BACKSPACE_KEY && current.length > 0 {
      current.length = current.length - 1;
    } else if c >= 32 && c < DELETE_KEY && current.length < MAX_COMMAND_LENGTH {
      store8(&current.data + current.length, c);
      current.length = current.length + 1;
    }
  }
}
}%toy

Finally, we delete the existing functions (from {\tt wait\_char} to {\tt
child}), and replace the main function with the following one. This function
simply creates and runs the shell if there are no arguments. Otherwise, for
testing purposes, it writes them to standard output.

\toy{
@fn main(args: &u32, args_end: &u32, heap: &u32, heap_limit: &u32) -> u32 {
  let shell = sh_new(&heap, heap_limit);
  if shell == null { return OUT_OF_MEMORY; }
  if args_end > args {
    write(STANDARD_OUTPUT, args, args_end - args);
    return OK;
  }
  return sh_run_editor(shell);
}
}%toy

\subsection{Compilation and tests}

Type ``F11''+``r'' in the command editor to load the current application source
code, and ``F4''+``r'' to edit it. Then update it as described above. For
reference, we also provide this new version in the {\tt shell\_v0.txt} file in
\toypcurl{sources.zip}. When done, type ``F12''+``r'' to save it and
``F9''+``r'' to compile it. If necessary, repeat these steps until the
compilation is successful. To copy the source code (at address
\rs{hex_dec(application_source)}) and the compiled code in files use F3 to load
the builder source code, and F4 to edit it. Then change its main function to
the following:

\rust{
  t.write_toy5("website/sources/shell_v0.txt")?;
  context.store_text(ram_source, &t.get_toy5());
  context.type_keys(vec!["F12"]);
  context.type_ascii("R\n");
  // F9 to compile it and save it
  context.type_keys(vec!["F9"]);
  context.type_ascii("R");
  assert_eq!(display.borrow().get_text(), "00000000");
  context.type_ascii("\n");

  let builder_source =
      context.get_text(context.memory_region("builder_source").start);
  let mut t = Transpiler5::new_str(&builder_source);
  t.add_unchanged("fn main() -> u32;", "static NAME = [");
  t.add_placeholder("#CODE", &format!("{}", kernel_code));
  t.add_placeholder("#SOURCE", &format!("{}", application_source));
}

\toy{
@static NAME = ['s','h','e','l','l'];
static SOURCE = ['s','h','e','l','l','.','t','o','y'];
@fn main() -> u32 {
@  const COMPILED_CODE: &u32 = #CODE;
  const SOURCE_CODE: &u32 = #SOURCE;
@  let result = buffer_write(COMPILED_CODE, NAME, 5);
  if result == OK { result = buffer_write(SOURCE_CODE, SOURCE, 9); }
@  if result == OK { flash_boot_loader_and_reset(); }
@  return result;
@}
}%toy

\noindent Finally, save the builder with F5, compile it with F6, and run it
with F7. If all goes well this should launch the kernel and the shell. To test
it, try the commands ``{\tt hello world}'' and ``{\tt shell hello world}''. The
former should fail because there is no program named ``{\tt hello}''. The
latter should print its arguments, \ie, ``{\tt hello world}''. You can also try
the ``{\tt shell}'' command, which should launch another shell. You can check
this with the arrow keys, which should not show the previous commands. Then
type Escape to return in the initial shell, and try the Escape, arrow up and
down keys again. Finally, reset the Arduino, which should restart with the
memory editor.

\rust{
  context.store_text(ram_source, &t.get_toy5());
  context.type_keys(vec!["F5"]);
  context.type_ascii("R\n");
  // F6 to compile it
  context.type_keys(vec!["F6"]);
  context.type_ascii("R");
  assert_eq!(display.borrow().get_text(), "00000000");
  context.type_ascii("\n");
  // F7 to run it (restarts with kernel)
  context.type_keys(vec!["F7"]);
  context.type_ascii("R");
  // TESTS OF SHELL
  context.run_until_get_char();
  assert_eq!(display.borrow().get_text(), ">");
  context.type_ascii("HELLO WORLD\n");
  assert_eq!(
      display.borrow().get_text(),
      ">hello world\nCan't find hello\n>"
  );
  context.type_ascii("SHELL HELLO\n");
  assert_eq!(display.borrow().get_text(), ">shell hello\n hello\n>");
}

\section{Text editor}

\subsection{Requirements}

The text editor should take as argument the name of the file to edit, and
should create it if it does not exist. It should also take an initial text
cursor position as an optional argument. The goal is to be able to edit a
source file directly at the location of an error indicated by the compiler.
Typing Escape should show a message asking the user whether to save the file or
not before exiting.

\subsection{Implementation}

We implement the text editor by editing the shell program, which already
contains some useful code, in order to avoid re-typing it. More precisely we
keep the beginning of its code, up to the {\tt mem\_allocate} function
(excluded), and delete everything else. We then re-implement the functions from
\cref{section:ted-implementation}, and the {\tt mem\_copy} function from
\cref{section:flash-memory-driver-impl} that they need (see these sections for
more details):

\rust{
  // Reset and go back in command editor.
  context.micro_controller().borrow_mut().reset();
  context.run_until_get_char();
  context.type_ascii(&format!("W{:08X}\n", command_editor_main));
  context.type_ascii("R");
  assert!(display
      .borrow()
      .get_text()
      .contains("Welcome to the command editor."));

  let mut t = Transpiler5::new_str(&context.get_text(application_source));
  t.add_unchanged("fn main(", "fn mem_allocate(");
}

\toy{
fn mem_copy(src: &u32, dst: &u32, n: u32) {
  let i = 0;
  if dst < src {
    while i + 4 <= n {
      *(dst + i) = *(src + i);
      i = i + 4;
    }
    while i < n {
      store8(dst + i, load8(src + i));
      i = i + 1;
    }
  } else {
    i = n;
    while i >= 4 {
      i = i - 4;
      *(dst + i) = *(src + i);
    }
    while i > 0 {
      i = i - 1;
      store8(dst + i, load8(src + i));
    }
  }
}

const ARROW_DOWN_KEY: u32 = 242;
const ARROW_LEFT_KEY: u32 = 235;
const ARROW_RIGHT_KEY: u32 = 244;
const ARROW_UP_KEY: u32 = 245;
const BACKSPACE_KEY: u32 = 8;
const DELETE_KEY: u32 = 127;
const ENTER_KEY: u32 = 10;
const ESCAPE_KEY: u32 = 27;
const PAGE_DOWN_KEY: u32 = 250;
const PAGE_UP_KEY: u32 = 253;
const TAB_KEY: u32 = 9;

fn ted_set_cursor(begin: &u32, cursor: &u32, gap: u32, new_cursor: &u32) \
-> &u32 {
  if new_cursor > cursor {
    mem_copy(cursor + gap, cursor, new_cursor - cursor);
  } else {
    mem_copy(new_cursor, new_cursor + gap, cursor - new_cursor);
  }
  return new_cursor;
}

fn ted_move_backward(begin: &u32, cursor: &u32, lines: u32, col: &u32, \
row: &u32) -> &u32 {
  let c = 0;
  let ptr = cursor;
  while ptr > begin {
    c = load8(ptr - 1);
    if c == ENTER_KEY {
      if *row == lines { break; }
      *row = *row + 1;
    } else if *row == 0 {
      if c == TAB_KEY {
        *col = *col + 2;
      } else {
        *col = *col + 1;
      }
    }
    ptr = ptr - 1;
  }
  return ptr;
}

fn ted_move_forward(cursor: &u32, gap: u32, end: &u32, lines: u32) -> &u32 {
  loop {
    if cursor == end - gap { return cursor; }
    if load8(cursor + gap) == ENTER_KEY {
      if lines == 1 { return cursor + 1; }
      lines = lines - 1;
    }
    cursor = cursor + 1;
  }
}

fn ted_handle_key(begin: &u32, cursor: &u32, gap: u32, end: &u32, c: u32) \
-> &u32 {
  let col = 0;
  let row = 0;
  if c == ARROW_LEFT_KEY && cursor > begin {
    return cursor - 1;
  } else if c == ARROW_RIGHT_KEY && cursor < end - gap {
    return cursor + 1;
  } else if c == ARROW_UP_KEY {
    return ted_move_backward(begin, cursor, 1, &col, &row);
  } else if c == ARROW_DOWN_KEY {
    return ted_move_forward(cursor, gap, end, 1);
  } else if c == PAGE_UP_KEY {
    return ted_move_backward(begin, cursor, 30, &col, &row);
  } else if c == PAGE_DOWN_KEY {
    return ted_move_forward(cursor, gap, end, 30);
  }
  return cursor;
}

fn ted_draw(begin: &u32, cursor: &u32, gap: u32, end: &u32) {
  gpu_clear_screen();
  gpu_set_cursor(0, 0);
  let r = 0;
  let c = 0;
  let col = 0;
  let row = 0;
  let ptr = ted_move_backward(begin, cursor, 15, &col, &row);
  if ptr == cursor { ptr = ptr + gap; }
  while ptr < end && r < 30 {
    c = load8(ptr);
    if c == ENTER_KEY {
      r = r + 1;
      gpu_set_cursor(0, r);
    } else if c == TAB_KEY {
      gpu_draw_char(' ');
      gpu_draw_char(' ');
    } else {
      gpu_draw_char(c);
    }
    ptr = ptr + 1;
    if ptr == cursor { ptr = ptr + gap; }
  }
  gpu_switch_buffer();
  gpu_set_cursor(col, row);
}

fn text_editor(buffer: &u32, offset: u32, max_length: u32) {
  let length = *buffer;
  if length > max_length { return; }
  let begin = buffer + 4;
  let cursor = begin + length;
  let end = begin + max_length;
  let gap = end - cursor;
  let c = 0;
  if offset > length { offset = length; }
  cursor = ted_set_cursor(begin, cursor, gap, begin + offset);
  gpu_set_color(0, 7, 0);
  gpu_set_double_buffer();
  ted_draw(begin, cursor, gap, end);
  loop {
    read(STANDARD_INPUT, &c, 1);
    if c == ESCAPE_KEY {
      *buffer = ted_set_cursor(begin, cursor, gap, end - gap) - begin;
      gpu_set_single_buffer();
      return;
    }
    if c == BACKSPACE_KEY {
      if cursor > begin {
        cursor = cursor - 1;
        gap = gap + 1;
      }
    } else if c < DELETE_KEY {
      if gap > 0 {
        store8(cursor, c);
        cursor = cursor + 1;
        gap = gap - 1;
      }
    } else {
      cursor = ted_set_cursor(begin, cursor, gap,
          ted_handle_key(begin, cursor, gap, end, c));
    }
    ted_draw(begin, cursor, gap, end);
  }
}
}%toy

The rest of the code is new and implements the above requirements. The
following function computes the value of the optional command line argument,
the initial cursor position. It stores the value of this argument, if present,
at address {\tt offset}. It returns an error if this argument is not a number,
and {\tt OK} otherwise. $[\mathit{args},\mathit{args\_end}[$ must be the rest
of the command line arguments, after the name of the file to edit.

\toy{
fn read_offset(args: &&u32, args_end: &u32, offset: &u32) -> u32 {
  let length = 0;
  let argument = sh_read_token(args, args_end, &length);
  if argument == null { return OK; }
  let i = 0;
  let c = 0;
  while i < length {
    c = load8(argument + i);
    if c < '0' || c > '9' { return INVALID_ARGUMENT; }
    *offset = (*offset) * 10 + (c - '0');
    i = i + 1;
  }
  return OK;
}
}%toy

The next function simply returns an error after writing a corresponding error
message to standard output. It is followed by some messages needed in {\tt
main}:

\toy{
fn write_error(src: &u32, length: u32, error: u32) -> u32 {
  write(STANDARD_OUTPUT, src, length);
  return error;
}

static USAGE = ['U','s','a','g','e',':',' ',
  'e','d','i','t',' ','f','i','l','e',' ',
  '[','o','f','f','s','e','t',']'];

static READ_ERROR = ['R','e','a','d',' ','e','r','r','o','r'];
static WRITE_ERROR = ['W','r','i','t','e',' ','e','r','r','o','r'];
static SAVE_PROMPT = ['S','a','v','e',' ','(','y','/','n',')','?'];
}%toy

The {\tt main} function starts by reading the command line arguments. If they
are invalid, it prints a message explaining their expected format and returns
an error. It also checks that the heap is large enough to edit a small text.

\toy{
@fn main(args: &u32, args_end: &u32, heap: &u32, heap_limit: &u32) -> u32 \{
  let length = 0;
  let name = sh_read_token(&args, args_end, &length);
  let offset = 0;
  if name == null || read_offset(&args, args_end, &offset) != OK {
    return write_error(USAGE, 25, INVALID_ARGUMENT);
  }
  if heap_limit < heap + 256 { return OUT_OF_MEMORY; }
}%toy

It continues by reading the file to edit, if it exists, into the data buffer
named {\tt buffer}, using all the available heap memory (minus 4 bytes for the
data buffer header). Otherwise it initializes an empty buffer.
\toy{
  let buffer = heap;
  let max_length = heap_limit - heap - 4;
  let stream = open(name, length, 'r');
  if status(stream) == OK {
    *buffer = read(stream, buffer + 4, max_length);
    close(stream);
    if status(*buffer) != OK {
      return write_error(READ_ERROR, 10, status(*buffer));
    }
    if *buffer == max_length { return OUT_OF_MEMORY; }
  } else {
    *buffer = 0;
  }
}%toy

The end of the function calls {\tt text\_editor} to edit this buffer. When it
returns, it displays a message asking the user whether the changes must be
saved or not, with {\tt ted\_draw}. If the user types ``y'' it saves the {\tt
buffer} content (without its 4 bytes header) into the edited file and returns
{\tt OK}, or an error if the file could not be saved.

\toy{
  text_editor(buffer, offset, max_length);
  ted_draw(SAVE_PROMPT, SAVE_PROMPT + 11, 0, SAVE_PROMPT + 11);
  let c = 0;
  read(STANDARD_INPUT, &c, 1);
  if c != 'y' { return OK; }
  stream = open(name, length, 'w');
  if status(stream) != OK {
    return write_error(WRITE_ERROR, 11, status(stream));
  }
  let n = write(stream, buffer + 4, *buffer);
  if status(n) != OK {
    return write_error(WRITE_ERROR, 11, status(n));
  }
  return OK;
\}
}%toy

\subsection{Compilation and tests}

Type ``F11''+``r'' in the command editor to load the current shell source code,
and ``F4''+``r'' to edit it. Then update it as described above. For reference,
we also provide this code in the {\tt edit\_v0.txt} file in
\toypcurl{sources.zip}. When done, type ``F12''+``r'' to save it and
``F9''+``r'' to compile it. If necessary, repeat these steps until the
compilation is successful. To copy the text editor source code and compiled
code in files use F3 to load the builder source code, and F4 to edit it. Then
change its main function to the following:

\rust{
  t.write_toy5("website/sources/edit_v0.txt")?;
  context.store_text(ram_source, &t.get_toy5());
  context.type_keys(vec!["F12"]);
  context.type_ascii("R\n");
  // F9 to compile it and save it
  context.type_keys(vec!["F9"]);
  context.type_ascii("R");
  assert_eq!(display.borrow().get_text(), "00000000");
  context.type_ascii("\n");

  let builder_source =
      context.get_text(context.memory_region("builder_source").start);
  let mut t = Transpiler5::new_str(&builder_source);
  t.add_unchanged("fn main() -> u32;", "static NAME = [");
  t.add_placeholder("#CODE", &format!("{}", kernel_code));
  t.add_placeholder("#SOURCE", &format!("{}", application_source));
}

\toy{
static NAME = ['e','d','i','t'];
static SOURCE = ['e','d','i','t','.','t','o','y'];
@fn main() -> u32 {
@  const COMPILED_CODE: &u32 = #CODE;
@  const SOURCE_CODE: &u32 = #SOURCE;
  let result = buffer_write(COMPILED_CODE, NAME, 4);
  if result == OK { result = buffer_write(SOURCE_CODE, SOURCE, 8); }
@  if result == OK { flash_boot_loader_and_reset(); }
@  return result;
@}
}%toy

\noindent Finally, save the builder with F5, compile it with F6, and run it
with F7. If all goes well this should launch the kernel and the shell. To test
the text editor, type ``{\tt edit test.txt}'' and Enter. Type some text and
then Escape and ``{\tt y}'' to save it. To check that this worked, type ``{\tt
edit test.txt 2}'': the text editor should display the text you just saved, and
the cursor should be under the third character. Finally, reset the Arduino,
which should restart with the memory editor.

\rust{
  context.store_text(ram_source, &t.get_toy5());
  context.type_keys(vec!["F5"]);
  context.type_ascii("R\n");
  // F6 to compile it
  context.type_keys(vec!["F6"]);
  context.type_ascii("R");
  assert_eq!(display.borrow().get_text(), "00000000");
  context.type_ascii("\n");
  // F7 to run it (restarts with kernel)
  context.type_keys(vec!["F7"]);
  context.type_ascii("R");

  // TESTS OF TEXT EDITOR
  context.run_until_get_char();
  assert_eq!(display.borrow().get_text(), ">");
  context.type_ascii("EDIT TEST.TXT\n");
  assert_eq!(display.borrow().get_text(), "");
  context.type_ascii("A TEST FILE");
  assert_eq!(display.borrow().get_text(), "a test file");
  context.type_keys(vec!["Escape"]);
  assert_eq!(display.borrow().get_text(), "Save (y/n)?");
  context.type_ascii("Y");
  assert_eq!(display.borrow().get_text(), ">edit test.txt\n>");
  context.type_ascii("EDIT TEST.TXT 11\n");
  assert_eq!(display.borrow().get_text(), "a test file");
  context.type_ascii(".");
  assert_eq!(display.borrow().get_text(), "a test file.");
}

\section{Compiler}

\subsection{Requirements}

The shell and the text editor start with the same source code which is in fact
useful for most programs (namely the {\tt entry} function, and the system call
functions). To avoid duplicating it in each program we require the possibility
to compile a program from several source files. We can then put the shared code
in a separate file, implemented only once, and compile it together with a
program specific file. The compiler should thus take as arguments the name of
the compiled program to generate, followed by one or more source file name(s).
In case of a compilation error it should write the error code, the error
location, and the name of the source file to standard output. The goal is to be
able to edit this file directly at the location of the error with the text
editor.

\subsection{Design}

We meet the above requirements with a simple method requiring very few changes
to our compiler, but which is not memory efficient (we improve it in
\cref{chapter:utilities}). More precisely, we load all the input files in RAM,
after the compiler's heap, which is itself only a part of the compiler
process's heap (compare \cref{fig:toyc6-memory-map} with
\cref{fig:toyc-memory-map,fig:process-ram-map}). And we compile each file with
{\tt tc\_parse\_program}, one by one, into the same $\it{dst\_buffer}$ that we
finally save into a file.

\begin{Figure}
  \input{figures/chapter5/toyc-memory-map.tex}

  \caption{The compiler process's heap (bottom), between its compiled code
  (red) and its stack (green), contains the generated code (yellow), the {\tt
  Compiler} struct (dark blue), the compiler's heap (shades of blue), and the
  input source code (white).}\label{fig:toyc6-memory-map}
\end{Figure}

\subsection{Implementation}\label{subsection:toyc-process}

We implement the above design by editing the current compiler code as
follows. We first replace the {\tt tc\_main} function declaration and the {\tt
main} function with the same {\tt entry} function as in the shell and the text
editor (we don't have a compiler able to compile several files yet, and thus
need to duplicate it for now):

\rust{
  // Reset and go back in command editor.
  context.micro_controller().borrow_mut().reset();
  context.run_until_get_char();
  context.type_ascii(&format!("W{:08X}\n", command_editor_main));
  context.type_ascii("R");
  assert!(display
      .borrow()
      .get_text()
      .contains("Welcome to the command editor."));

  let mut t = Transpiler5::new_str(&context.get_text(compiler_source));
}

\toy{
fn main(args: &u32, args_end: &u32, heap: &u32, heap_limit: &u32) -> u32;
fn exit(result: u32) -> u32;

fn entry(heap: &u32, heap_limit: &u32) {
  let args = heap + 4;
  let args_end = args + *heap;
  heap = (((args_end as u32 + 3) >> 2) << 2) as &u32;
  exit(main(args, args_end, heap, heap_limit));
}
}%toy

Then, after {\tt load8}, {\tt load16}, {\tt store8}, and {\tt store16}
(unchanged), and for the same reason, we copy the following functions from the
shell and the text editor:

\rust{
  t.add_unchanged("fn load8(", "fn panic_result(");
}

\toy{
fn sh_read_token(src_p: &&u32, src_end: &u32, length: &u32) -> &u32 {
  let src = *src_p;
  while src < src_end && load8(src) == ' ' { src = src + 1; }
  if src >= src_end { return null; }
  let token = src;
  while src < src_end && load8(src) != ' ' { src = src + 1; }
  *length = src - token;
  *src_p = src;
  return token;
}

const OK: u32 = 0;
const INVALID_ARGUMENT: u32 = 1;
const INVALID_STATE: u32 = 2;
const NOT_FOUND: u32 = 3;
const ALREADY_EXISTS: u32 = 4;
const OUT_OF_MEMORY: u32 = 5;
const INTERNAL_ERROR: u32 = 6;

fn status(result: u32) -> u32 { return result >> 24; }

fn system_call(id: u32, args: &u32) -> u32 [
  /*SVC*/ 57088;
  /*MOV_PC_LR*/ 18167;
]
fn exit(result: u32) -> u32 {
  return system_call(1, &result);
}
fn open(name: &u32, length: u32, mode: u32) -> u32 {
  return system_call(4, &name as &u32);
}
const STANDARD_OUTPUT: u32 = 1;
fn read(file_descriptor: u32, buffer: &u32, size: u32) -> u32 {
  return system_call(5, &file_descriptor);
}
fn write(file_descriptor: u32, buffer: &u32, size: u32) -> u32 {
  return system_call(6, &file_descriptor);
}
fn close(file_descriptor: u32) -> u32 {
  return system_call(7, &file_descriptor);
}
}%toy

We then keep all the existing code between {\tt panic\_result} and {\tt
tc\_parse\_fn} unchanged, but we remove the call to {\tt tc\_check\_symbols} in
{\tt tc\_parse\_program}. Otherwise a function declared in one file would need
to be implemented in this same file, which is too restrictive (recall that {\tt
tc\_check\_symbols} checks that all declared functions are effectively
implemented).

\rust{
  t.add_unchanged("fn panic_result(", "fn tc_parse_program(");
}

\toy{
@fn tc_parse_program(self: &Compiler) {
@  loop {
...@    if self.next_token == TC_FN {
...@      tc_parse_fn(self);
...@    } else if self.next_token == TC_STRUCT {
...@      tc_parse_struct(self);
...@    } else if self.next_token == TC_STATIC {
...@      tc_parse_static(self);
...@    } else if self.next_token == TC_CONST {
...@      tc_parse_const(self);
...@    } else {
@      if self.next_token != 0 { panic(23); }
      return;
@    }
@  }
@}
}%toy

The following code is new. It replaces the {\tt tc\_main} function and
implements the above requirements. We start with a function to write a decimal
number $x$ to standard output, which is needed to write an error code or its
location. This function writes $x$ divided by 10 by calling itself recursively,
followed by the remainder of this division. The next function is similar to the
one with the same name in the text editor:

\toy{
fn write_integer(x: u32) {
  let quotient = x / 10;
  x = x - 10 * quotient + '0';
  if quotient > 0 { write_integer(quotient); }
  write(STANDARD_OUTPUT, &x, 1);
}

fn write_error(src1: &u32, length1: u32, src2: &u32, length2: u32, \
error: u32) -> u32 {
  write(STANDARD_OUTPUT, src1, length1);
  write(STANDARD_OUTPUT, src2, length2);
  return error;
}
}%toy

The {\tt main} function starts by reading the command line arguments. If there
are less than two it prints a message explaining their expected format and
returns an error. It then increases the stack area, by decreasing the process's
heap limit, and checks that this heap is large enough to compile a small
program (see \cref{fig:toyc6-memory-map}).

\toy{
static USAGE = ['U','s','a','g','e',':',' ',
  't','o','y','c',' ','o','u','t','p','u','t',' ',
  'i','n','p','u','t','1',' ','i','n','p','u','t','2',' ',
  '.','.','.'];

static CANT_OPEN = ['C','a','n',''','t',' ','o','p','e','n',' '];
static CANT_READ = ['C','a','n',''','t',' ','r','e','a','d',' '];
static CANT_WRITE = ['C','a','n',''','t',' ','w','r','i','t','e',' '];
static ERROR = ['E','r','r','o','r',' '];
static AT = [' ','a','t',  ' '];
static IN = [' ','i','n',  ' '];

fn main(args: &u32, args_end: &u32, heap: &u32, heap_limit: &u32) -> u32 \{
  let out_length = 0;
  let in_length = 0;
  let out = sh_read_token(&args, args_end, &out_length);
  let in = sh_read_token(&args, args_end, &in_length);
  if out == null || in == null {
    write(STANDARD_OUTPUT, USAGE, 36);
    return INVALID_ARGUMENT;
  }
  const MAX_CODE_SIZE: u32 = 12288;
  const MAX_HEAP_SIZE: u32 = 18432;
  const MIN_SRC_SIZE: u32 = 256;
  heap_limit = heap_limit - 512;
  if heap_limit < heap + MAX_CODE_SIZE + sizeof(Compiler) + \
MAX_HEAP_SIZE + MIN_SRC_SIZE {
    return OUT_OF_MEMORY;
  }
}%toy

The main function continues by creating and initializing the {\tt compiler}
struct. Note that $\it{dst}$ is a multiple of 4, as required (see
\cref{section:toyc5-implementation}), thanks to the rounding done in the {\tt
entry} function. It also implements a ``panic handler'' writing the error code,
offset and source file name to standard output, as required ($\it{src}$ points
to the beginning of the source code loaded from the file whose name starts at
$\it{in}$).

\toy{
  let error = 0;
  let dst = heap;
  let compiler = (dst + MAX_CODE_SIZE) as &Compiler;
  compiler.dst = dst;
  compiler.dst_limit = compiler as &u32;
  compiler.heap = compiler.dst_limit + sizeof(Compiler);
  compiler.heap_limit = compiler.heap + MAX_HEAP_SIZE;
  compiler.symbols = null;
  let src = compiler.heap_limit;
  error = panic_result();
  if error != 0 {
    write(STANDARD_OUTPUT, ERROR, 6);
    write_integer(error);
    if in == null { return error; }
    write(STANDARD_OUTPUT, AT, 4);
    write_integer(compiler.src - src);
    return write_error(IN, 4, in, in_length, error);
  }
}%toy

This handler is followed by a loop which reads each input source file and
compiles it. Each file is loaded at $\it{src}$, which is then incremented by
the file size. If the file can be read successfully and if there is enough
memory to load its content, it is compiled with {\tt tc\_parse\_program}. This
requires initializing the scanner first, with {\tt tc\_read\_char} and {\tt
tc\_read\_token}, which in turn requires initializing the compiler's $\it{src}$
and $\it{src\_end}$ fields.

\toy{
  let stream = 0;
  let src_size = 0;
  while in != null {
    stream = open(in, in_length, 'r');
    if status(stream) != OK {
      return write_error(CANT_OPEN, 11, in, in_length, status(stream));
    }
    src_size = read(stream, src, heap_limit - src);
    close(stream);
    if status(src_size) != OK || src_size == heap_limit - src {
      return write_error(CANT_READ, 11, in, in_length, status(src_size));
    }
    compiler.src = src - 1;
    compiler.src_end = src + src_size;
    tc_read_char(compiler);
    tc_read_token(compiler);
    tc_parse_program(compiler);
    in = sh_read_token(&args, args_end, &in_length);
    src = src + src_size;
  }
}%toy

The {\tt main} function ends by checking that all declared functions, in all
files, are effectively implemented, and by writing the compiled code to disk:

\toy{
  tc_check_symbols(compiler.symbols, null);
  stream = open(out, out_length, 'w');
  if status(stream) != OK {
    return write_error(CANT_OPEN, 11, out, out_length, status(stream));
  }
  let n = write(stream, dst, compiler.dst - dst);
  if status(n) != OK {
    return write_error(CANT_WRITE, 12, out, out_length, status(n));
  }
  return OK;
\}
}%toy

\subsection{Compilation and tests}

To update the compiler as described above we first need some commands to load
and save its source code, at address \rs{hex_dec(compiler_source)} = page
\rs{dec(page_number(compiler_source))} (see \cref{fig:shell-memory-map}). For
this type ``F11''+``e'' in the command editor, update this command to:

\rust{
  let commands = context.memory_region("command_editor_commands").start;

  // Edit 'Load' (F11) command to load source code of compiler.
  context.type_keys(vec!["F11"]);
  let mut c = BytecodeAssembler::default();
  c.import_labels(context.memory_region("flash_driver"));
  c.func("load_compiler_source_code", &[], "", &[]);
  c.new_line();
  c.cst(compiler_source);
  c.cst(ram_source);
  c.call("buffer_copy");
  c.new_line();
  c.cst_0();
  c.retv();
  let original = context.get_text(commands + 10 * 256);
  let c_source = format!("{}\nd LOAD_COMPILER_SOURCE_CODE",
  c.get_toy0_source_code());
  context.store_text(ram_command_source, &c_source);
  context.type_keys(vec!["S"]);
}
\rs{code_changes(&c_source, &original, &[1, 3])}

\noindent and type ``s'' to save it. With the same method, update the F12
command to:

\rust{
  // Edit 'Save' (F12) command to save source code of compiler.
  context.type_keys(vec!["F12"]);
  context.type_ascii("S");
  let mut c = BytecodeAssembler::default();
  c.import_labels(context.memory_region("flash_driver"));
  c.func("save_compiler_source_code", &[], "", &[]);
  c.new_line();
  c.cst(ram_source);
  c.cst(page_number(compiler_source));
  c.call("buffer_flash");
  c.new_line();
  c.cst_0();
  c.retv();
  let original = context.get_text(commands + 11 * 256);
  let c_source = format!("{}\nd SAVE_COMPILER_SOURCE_CODE",
  c.get_toy0_source_code());
  context.store_text(ram_command_source, &c_source);
  context.type_keys(vec!["S"]);
}
\rs{code_changes(&c_source, &original, &[1, 3])}

Then use F11 and F4 to load and edit the current compiler source code, and
update it as described above. For reference, we also provide this code in the
{\tt toyc\_v0.txt} file in \toypcurl{sources.zip}. When done, use F12 and F9 to
save it and to compile it. If necessary, repeat these steps until the
compilation is successful. To copy the new compiler source code and compiled
code in files use F3 to load the builder source code, and F4 to edit it. Then
change its main function to the following:

\rust{
  t.write_toy5("website/sources/toyc_v0.txt")?;
  context.store_text(ram_source, &t.get_toy5());
  context.type_keys(vec!["F12"]);
  context.type_ascii("R\n");
  // F9 to compile it and save it
  context.type_keys(vec!["F9"]);
  context.type_ascii("R");
  assert_eq!(display.borrow().get_text(), "00000000");
  context.type_ascii("\n");

  let builder_source =
      context.get_text(context.memory_region("builder_source").start);
  let mut t = Transpiler5::new_str(&builder_source);
  t.add_unchanged("fn main() -> u32;", "static NAME = [");
  t.add_placeholder("#CODE", &format!("{}", kernel_code));
  t.add_placeholder("#SOURCE", &format!("{}", compiler_source));
}

\toy{
static NAME = ['t','o','y','c'];
static SOURCE = ['t','o','y','c','.','t','o','y'];
@fn main() -> u32 {
@  const COMPILED_CODE: &u32 = #CODE;
  const SOURCE_CODE: &u32 = #SOURCE;
@  let result = buffer_write(COMPILED_CODE, NAME, 4);
@  if result == OK { result = buffer_write(SOURCE_CODE, SOURCE, 8); }
@  if result == OK { flash_boot_loader_and_reset(); }
@  return result;
@}
}%toy

Finally, save the builder with F5, compile it with F6, and run it with F7. If
all goes well this should launch the kernel and the shell. To test our new
compiler, first create a ``{\tt hello.toy}'' file containing the following
program (\ie, type ``{\tt edit hello.toy}'' in the shell, enter this code, and
finally type Escape and ``{\tt y}'' to save it):

\rust{
  context.store_text(ram_source, &t.get_toy5());
  context.type_keys(vec!["F5"]);
  context.type_ascii("R\n");
  // F6 to compile it
  context.type_keys(vec!["F6"]);
  context.type_ascii("R");
  assert_eq!(display.borrow().get_text(), "00000000");
  context.type_ascii("\n");

  // F7 to run it (restarts with kernel)
  context.type_keys(vec!["F7"]);
  context.type_ascii("R");
  context.run_until_get_char();
  assert_eq!(display.borrow().get_text(), ">");

  context.type_ascii("EDIT HELLO.TOY\n");
  assert_eq!(display.borrow().get_text(), "");
  let mut t = Transpiler5::default();
}

\toy{
fn main();
fn entry(heap: &u32, heap_limit: &u32) { main(); }

fn system_call(id: u32, args: &u32) -> u32 [
  /*SVC*/ 57088;
  /*MOV_PC_LR*/ 18167;
]
fn exit(result: u32) -> u32 {
  return system_call(1, &result);
}
fn write(stream_id: u32, buffer: &u32, size: u32) -> u32 {
  return system_call(6, &stream_id);
}

static HELLO = ['H','e','l','l','o',',',' ','W','o','r','l','d','!'];
fn main() {
  write(1 /*standard output*/, HELLO, 13);
  exit(0);
}
}%toy

\rust{
  context.enter_text_editor_text(&t.get_toy5());
  context.type_keys(vec!["Escape"]);
  assert_eq!(display.borrow().get_text(), "Save (y/n)?");
  context.type_ascii("Y");
}

Then compile it with ``{\tt toyc hello hello.toy}''. If there is a compilation
error, edit the source code at the location indicated by the compiler. Once the
compilation is successful, run the compiled program with the ''{\tt hello}''
command line, which should print the ``Hello, World!'' message. You can then
reset the Arduino, which should restart with the memory editor.

\rust{
  // Step 2: compile it
  context.type_ascii("TOYC HELLO HELLO.TOY\n");
  assert_eq!(display.borrow().get_text(), ">toyc hello hello.toy\n>");
  // Step 3: run it
  context.type_ascii("HELLO\n");
  assert_eq!(display.borrow().get_text(), ">hello\nHello, World!\n>");
}

\section{Self hosting}

We now have everything we need to edit, compile and run new programs from our
operating system alone. Moreover, the file system contains the source code of
the shell, the text editor, and the compiler (and of the boot loader). We can
thus update them and recompile them from the operating alone too. In other
words, we no longer need our bytecode interpreter, the basic input output
system, the command editor, or anything else we built in the Flash1 memory
bank. In particular, we no longer need to restore the BIOS Vector Table in our
kernel to restart the basic input output system after a reset of the operating
system. To remove it, type ``F8''+``r'' in the command editor to load the
kernel source code, and ``F4''+``r'' to edit it. Then delete the {\tt
restore\_bios\_vector\_table} function, and remove the call to this function in
{\tt os\_init}. Use F10 to save these changes and F9 to compile them.

\rust{
  // Reset and go back in command editor.
  context.micro_controller().borrow_mut().reset();
  context.run_until_get_char();
  context.type_ascii(&format!("W{:08X}\n", command_editor_main));
  context.type_ascii("R");
  assert!(display
    .borrow()
    .get_text()
    .contains("Welcome to the command editor."));

  // Recompile kernel to remove 'restore_bios_vector_table'.
  let mut t = Transpiler5::new_str(&context.get_text(kernel_source));
  t.add_unchanged("fn os_init(", "fn restore_bios_vector_table(");
  t.add_unchanged("fn os_init(code: &u32, heap: &u32, stack: &u32) ",
      "\trestore_bios_vector_table();");
  t.add_unchanged("\tconst MAX_KERNEL_HEAP_SIZE:", "EOF");

  context.store_text(ram_source, &t.get_toy5());
  context.type_keys(vec!["F10"]);
  context.type_ascii("R\n");
  // F9 to compile it and save it
  context.type_keys(vec!["F9"]);
  context.type_ascii("R");
  assert_eq!(display.borrow().get_text(), "00000000");
  context.type_ascii("\n");
}

To copy this new kernel, and its source code (so that we can update it from the
operating system alone), use F3 to load the builder source code,
and F4 to edit it. Then change its main function, one last time, to the
following (the kernel source code is at address \rs{hex_dec(kernel_source)} --
see \cref{fig:shell-memory-map}):

\rust{
  let builder_source =
      context.get_text(context.memory_region("builder_source").start);
  let mut t = Transpiler5::new_str(&builder_source);
  t.add_unchanged("fn main() -> u32;", "static NAME = [");
  t.add_placeholder("#CODE", &format!("{}", kernel_code));
  t.add_placeholder("#SOURCE", &format!("{}", kernel_source));
}

\toy{
static NAME = ['t','o','y','s'];
static SOURCE = ['t','o','y','s','.','t','o','y'];
@fn main() -> u32 {
@  const COMPILED_CODE: &u32 = #CODE;
  const SOURCE_CODE: &u32 = #SOURCE;
@  let result = buffer_write(COMPILED_CODE, NAME, 4);
@  if result == OK { result = buffer_write(SOURCE_CODE, SOURCE, 8); }
@  if result == OK { flash_boot_loader_and_reset(); }
@  return result;
@}
}%toy

Finally, save the builder with F5, compile it with F6, and run it with F7. If
all goes well this should launch the kernel and the shell. Then reset the
Arduino. This should relaunch the kernel and the shell again.

\rust{
  context.store_text(ram_source, &t.get_toy5());
  context.type_keys(vec!["F5"]);
  context.type_ascii("R\n");
  // F6 to compile it
  context.type_keys(vec!["F6"]);
  context.type_ascii("R");
  assert_eq!(display.borrow().get_text(), "00000000");
  context.type_ascii("\n");
  // F7 to run it (restarts with kernel)
  context.type_keys(vec!["F7"]);
  context.type_ascii("R");
  assert_eq!(display.borrow().get_text(), ">");

  // Check reset starts kernel again.
  context.micro_controller().borrow_mut().reset();
  context.run_until_get_char();
  assert_eq!(display.borrow().get_text(), ">");
}

% This work is licensed under the Creative Commons Attribution NonCommercial
% ShareAlike 4.0 International License. To view a copy of the license, visit
% https://creativecommons.org/licenses/by-nc-sa/4.0/

\chapter*{Conclusion}
\addcontentsline{toc}{chapter}{Conclusion}

A microcontroller contains in a single chip a microprocessor, various memories,
and specialized circuits to communicate and interact with the external world.
In this part we connected an Arduino Due, based on an Atmel microcontroller, to
a Liquid Crystal Display (LCD), and to a keyboard. We used its boot program in
read-only memory, via an external computer, to store in its flash memory a
basic memory editor program. Like the Atmel's boot program, this editor allows
us to input programs and to run them. However, it does this by using the
keyboard and the LCD connected to the Arduino, \ie, without needing any
external computer. Our toy computer is thus fully autonomous. The rest of this
book illustrates this by progressively improving it, to make it more and more
usable, without using any external computer.

\subsubsection{Further readings}

This part barely scratched the surface of what microcontrollers can do. And it
only presented a very low level method to program them (because the goal of
this book it to program one from scratch). To know more about what
microcontrollers can do, how they work, and how to use them in more convenient
ways, you can read the following books:
\begin{itemize}
  \item ``Arduino Workshop, 2nd Edition: A Hands-on Introduction with 65
  Projects'' \cite{ArduinoWorkshop}. This book gives a very practical
  introduction to the Arduino microcontrollers. It explains how they can be
  programmed with the Arduino Integrated Development Environment, and shows how
  they can be used with many external components (including 7-segments
  displays, micro-SD cards, keypads, touchpads, motors, infrared sensors,
  GPS modules, external memories, etc).

  \item ``Embedded System Design with ARM Cortex-M Microcontrollers:
  Applications with C, C++ and MicroPython'' \cite{EmbeddedSystemDesign}. This
  book explains how microcontrollers work in general, and gives practical
  examples illustrating each aspect. For instance, it presents digital and
  analog signals, conversions between digital and analog signals, interrupts,
  clocks and timers, communication protocols, etc.

  \item ``Embedded Systems Fundamentals with Arm Cortex-M Based
  Microcontrollers: A Practical Approach'' \cite{EmbeddedSystemsFundamentals}.
  This book also gives ``theoretical'' and practical information about
  microcontrollers, with an emphasis on how to use them ``properly'' (\ie, to
  get efficient and responsive programs with low power requirements).
\end{itemize}

These books use programs written in textual form, which is much more
practical than hexadecimal numbers representing machine code or bytecode
instructions. Some of the above books introduce the {\em programming language}
that they use for this, at the same time as they present microcontrollers. But
none of them explains in detail how a program written in textual form can run
on a microprocessor, which can only execute machine code instructions. This
process is introduced in the next part, based on a toy programming language for
our toy computer.
