% This work is licensed under the Creative Commons Attribution NonCommercial
% ShareAlike 4.0 International License. To view a copy of the license, visit
% https://creativecommons.org/licenses/by-nc-sa/4.0/

\chapter[Logic Gates and Arithmetic Circuits]
  {Logic Gates and\\ Arithmetic Circuits}\label{chapter:logic-gates}

As explained in the previous chapter, computing arithmetic operations on binary
numbers boils down to the computation of simple logical operations such as
conjunctions and exclusive disjunctions. This chapter explains how these
operations can be implemented with electric circuits, and then how these
circuits can be combined to perform arithmetic operations.

\section{Transistors}

In order to implement a logical operation with an electric circuit we first
need a way to represent 0 and 1 with some electric states. One possibility is
to view a wire connected to the ground as 0, and a wire connected to the power
source (\eg, +5V) as 1. To implement a circuit for $\neg p$, for instance,
we can use an input wire for $p$, and an output wire for the result $\neg p$.
The circuit in the middle must then connect the output wire to the ground
(resp. power source) if the input wire is connected to the power source (resp.
ground). A simple switch can do this, provided it is controlled by the input
wire, instead of manually. In fact, as illustrated in the next sections, such
switches are sufficient to implement any logical operation.

An electric switch itself controlled by electricity connects or disconnects two
terminals, hereafter noted $a$ and $b$, depending on the voltage or current in
a third one, noted $c$. One method to do this is to use a {\em transistor}.
Another method is to use a {\em relay}. Transistors are much more efficient
than relays, and are used virtually everywhere. But relays are simpler to
understand and, for this reason, we use them in this chapter to explain how
logic gates work.

A relay can be built with an electromagnet controlling a metallic connector.
There are two types of relays connecting or disconnecting two terminals (see
\cref{fig:relays}):
\begin{itemize}
  \item in a {\em normally open} relay, the $a$ and $b$ terminals are
  disconnected when no current is flowing through the electromagnet. They are
  connected when the relay is {\em active}, \ie, when there is a current in the
  electromagnet.

  \item in a {\em normally closed} relay, the $a$ and $b$ terminals are
  connected when the relay is {\em inactive}, \ie, when there is no current in
  the electromagnet. They are disconnected when it is active.
\end{itemize}

In the following we represent relays with the symbols illustrated in
\cref{fig:relay-symbols}. We draw input terminals connected to the ground
(resp. power source) with a 0 (resp. 1) inside a black square. Similarly, we
use a 0 (resp. 1) inside a blue square for output terminals connected to the
ground (resp. power source). We represent those which are not connected to
anything with an X inside a red square. Finally, we draw wires connected to the
ground, to the power source, or to nothing in blue, green, and red,
respectively (see \cref{fig:relay-symbols}).

\begin{Figure}
  \input{figures/chapter2/relays.tex}

  \caption{The two types of relays used in this chapter. The electromagnet,
  when active (right), attracts a metallic piece. This connects the $a$ and $b$
  terminals of a normally open relay (top), and disconnects those of a normally
  closed one (bottom). When the electromagnet inactive, a spring moves the
  metallic piece away from it.}\label{fig:relays}
\end{Figure}

\begin{Figure}
  \input{figures/chapter2/relay-symbols.tex}

  \caption{The symbols and colors used for normally closed (left) and normally
  open (right) relays, as well as for wires and input (black) and output (blue)
  terminals connected to the ground, to the power source (up triangle), or to
  nothing (in red).}\label{fig:relay-symbols}
\end{Figure}

\section{Logic gates}

\subsection{NOT}

A {\em NOT gate} is a circuit implementing the logical not operation. This gate
can be built with two relays controlled by the same input\footnote{In practice,
with electromagnet relays, 0 can be represented with a terminal connected to
the ground or to nothing. Then a single normally closed relay is sufficient to
build a NOT gate \cite{MerciaRelay}. In this chapter we do as if it was not the
case. This leads to circuits which are much closer to those built with the most
common technology, namely Complementary Metal Oxide Semi-conductors (CMOS).}.
The first, normally closed, connects the output to the power source by default.
The second, normally open, connects the output to the ground when active (see
\cref{fig:not-gate}). Hence, when the input is 0 the first relay connects the
output to the power source, \ie, sets it to 1 (while the second does nothing).
Conversely, when the input is 1, the first relay has no effect but the second
connects the output to the ground, \ie, sets it to 0 (see \cref{fig:not-gate}).

\begin{Figure}
  \input{figures/chapter2/not-gate.tex}

  \caption{The two possible states of the NOT gate.}\label{fig:not-gate}
\end{Figure}

\subsection{AND and NAND}

\begin{Figure}
  \input{figures/chapter2/and-gate.tex} \\
  \medskip
  \input{figures/chapter2/nand-gate.tex}

  \caption{The four possible states of the AND (top) and NAND (bottom)
    gates.}\label{fig:and-gate}
\end{Figure}

The {\em AND gate} is a circuit implementing the logical and operation. This
circuit must connect the output to the power source when both inputs are 1.
This can be done with two normally open relays connected in series. Conversely,
this gate must connect the output to the ground when at least one input is 0.
This can be done with two normally closed relays connected in parallel (see
\cref{fig:and-gate}).

The {\em NAND gate} implements the negation of the logical and, \ie, it
computes $\neg (p \wedge q)$. It can be obtained by connecting a NOT gate to
the output of an AND gate. But a simpler method is to switch the power source
and the ground of the AND gate or, equivalently, the upper and lower halves of
this circuit\footnote{With the CMOS technology ``normally closed'' (resp.
``open'') transistors are only used in the upper (resp. lower) half of a gate.
Hence a CMOS AND gate is built with a NAND gate followed by a NOT.} (see
\cref{fig:and-gate}).

\subsection{OR and NOR}

The {\em OR gate} is a circuit implementing the logical or operation. This
gate must connect the output to the power source when at least one input is
1. This can be done with two normally open relays connected in parallel.
Conversely, this gate must connect the output to the ground when both inputs
are 0. This can be done with two normally closed relays connected in series
(see \cref{fig:or-gate}).

The {\em NOR gate} implements the negation of the logical or, \ie, it computes
$\neg (p \vee q)$. As the NAND gate, it can be obtained by switching the upper
and lower halves of the OR gate circuit (see \cref{fig:or-gate}).

\begin{Figure}
  \input{figures/chapter2/or-gate.tex} \\
  \medskip
  \input{figures/chapter2/nor-gate.tex}

  \caption{The four possible states of the OR (top) and NOR (bottom)
  gates.}\label{fig:or-gate}
\end{Figure}

\subsection{XOR}

The {\em XOR gate} implements the exclusive or operation. The result of $p
\oplus q$ is 1 when $p$ is 1 and $q$ is 0, or when $p$ is 0 and $q$ is 1. This
gate must thus connect the output to the power source if at least one these two
cases happens. This can be done with two sub circuits, one for each case,
connected in parallel. Each sub circuit must connect its output to the power
source when both inputs have a specific value. This can be done with two relays
connected in series: a normally open for $p$ or $q$, and a normally closed for
$\neg p$ or $\neg q$.

Conversely, the result of $p \oplus q$ is 0 when both ``$p$ is 0 or $q$ is 1''
and ``$p$ is 1 or $q$ is 0'' are true. The same reasoning as above leads to two
sub circuits connected in series, where each sub circuit uses two relays
connected in parallel. This lead to the final circuit shown in
\cref{fig:xor-gate}.

In the following, to simplify figures and to make it easier to distinguish each
logic gate, we represent them with their American National Standards Institute
(ANSI) symbols, shown in \cref{fig:gate-symbols}.

\begin{Figure}
  \input{figures/chapter2/xor-gate.tex}

  \caption{The four possible states of the XOR gate.}\label{fig:xor-gate}
\end{Figure}

\begin{Figure}
  \input{figures/chapter2/gate-symbols.tex}

  \caption{The ANSI symbols of the NOT, AND, NAND, OR, NOR and XOR logic
  gates.}\label{fig:gate-symbols}
\end{Figure}

\section{Multiplexers and demultiplexers}

Logic gates can be assembled to create more and more complex circuits. A simple
example is the {\em demultiplexer}, shown below, and represented with the
symbol on the right:

\begin{center}
  \input{figures/chapter2/demux.tex}
\end{center}

This circuit copies its input $i$ to the $o_s$ output, \ie, to $o_0$ if $s=0$
or to $o_1$ if $s=1$. It sets the other to 0. It can be viewed as a ``railroad
switch'' for signals. The {\em multiplexer}, shown below and represented with
the symbol on the right, does the opposite:

\begin{center}
  \input{figures/chapter2/mux.tex}
\end{center}

This circuit sets its output $o$ to the $i_s$ input, \ie, to $i_0$ if $s=0$, or
to $i_1$ if $s=1$.

\section{Arithmetic circuits}

\subsection{Adder}\label{subsection:adder-circuit}

\begin{Figure}
  \input{figures/chapter2/adder.tex}

  \caption{A circuit to add two 4-bit numbers (bottom) can be built with 4
    full-adder circuits (top right), each made of two half-adders (top left) and
    an OR gate. Here this circuit computes $0111_2+0011_2=1010_2$
    ($7+3=10$).}\label{fig:adder}
\end{Figure}

As shown in the previous chapter, the addition of two bits is simply their
exclusive disjunction, with a carry equal to their conjunction. In other words,
we can add two bits with an XOR gate, plus an AND gate for the carry. The
resulting circuit, called a {\em half-adder}, is illustrated in
\cref{fig:adder}.

As explained in \cref{subsection:binary-add}, adding two binary numbers $a$ and
$b$ must be done step by step, from right to left. At each step, one bit $a_i$
from $a$ must be added to one bit $b_i$ from $b$, and to the carry $c_{i-1}$
from the previous step. In other words, three bits must be added at each step,
but the above circuit can only add two. The solution is to connect two copies
of it: a first copy adds $a_i$ and $b_i$, and a second adds $c_{i-1}$ to the
result of the first. Each copy produces a new carry, but at most one of these
can be 1. Indeed, if $a_i+b_i$ gives a carry then the second stage necessarily
adds $c_{i-1}$ to $0$, which cannot give a carry. Hence the new carry $c_i$
resulting from $a_i+b_i+c_{i-1}$ can be computed with a disjunction of the
carries from the two half-adders. This leads to the {\em full adder} circuit
shown in \cref{fig:adder}.

Finally, to add two binary numbers with $n$ bits each, we simply need to
connect $n$ full-adder circuits, with the output carry $c_i$ of step $i$
connected to the input carry $c_{i-1}$ of step $i+1$ (see \cref{fig:adder}).

\subsection{Subtractor}

\begin{Figure}
  \input{figures/chapter2/subtractor.tex}

  \caption{A circuit to subtract two 4-bit numbers (bottom) can be built with 4
    full-subtractor circuits (top right), each made of two half-subtractors (top
    left) and an OR gate. Here this circuit computes $1010_2-0111_2=0011_2$
    ($10-7=3$).}\label{fig:subtractor}
\end{Figure}

Subtracting two binary numbers can be done with a very similar circuit. As
shown in the previous chapter, subtracting a bit $b$ from $a$ gives their
exclusive disjunction (as their addition), plus a carry equal to the
conjunction of $\neg a$ and $b$ (versus of $a$ and $b$ for an addition). In
other words, a circuit to subtract $b$ from $a$ can be obtained by adding a NOT
gate in a half-adder circuit. The result, called a {\em half-subtractor}, is
illustrated in \cref{fig:subtractor}.

Subtracting two binary numbers $a$ and $b$ must be done step by step, from
right to left. At each step, one bit $b_i$ from $b$, and the carry $c_{i-1}$
from the previous step, must be subtracted from a bit $a_i$ from $a$. In other
words, three bits must be subtracted at each step, but the above circuit can
only subtract two. The solution is to connect two copies of it: a first copy
subtracts $b_i$ from $a_i$, and a second subtracts $c_{i-1}$ from the result of
the first. Each copy produces a new carry, but at most one of these can be 1.
Indeed, if $a_i-b_i$ gives a carry then the second stage necessarily subtracts
$c_{i-1}$ from $1$, which cannot give a carry. Hence the new carry $c_i$
resulting from $a_i-b_i-c_{i-1}$ can be computed with a disjunction of the
carries from the two half-subtractors. This leads to the {\em full subtractor}
circuit shown in \cref{fig:subtractor}.

Finally, to subtract two binary numbers with $n$ bits each, we simply need to
connect $n$ full subtractor circuits, with the output carry $c_i$ of step $i$
connected to the input carry $c_{i-1}$ of step $i+1$ (see
\cref{fig:subtractor}).

\subsection{Arithmetic and Logic Unit}

\begin{Figure}
  \input{figures/chapter2/alu.tex}

  \caption{A simple Arithmetic Unit which can perform additions, subtractions,
    and comparisons of $4$-bit numbers.}\label{fig:alu}
\end{Figure}

As shown in \cref{subsection:binary-add}, multiplying two binary numbers $a$
and $b$ boils down to additions of left shifted copies of $a$, each multiplied
by a bit of $b$. Furthermore, $a_i * b_j$ gives the same result as $a_i \wedge
b_j$. Hence a circuit to multiply two $n$-bit binary numbers (yielding $2n$
bits) can be obtained with $n$ copies of an $n$-bit adder, plus $n^2$ AND gates
to compute the $a_i \wedge b_j$ terms.

Comparing two $n$-bit binary numbers is also easy to do. Indeed:
\begin{itemize}
  \item $a=b$ if and only if the $n$ least significant bits of $a-b$ are equal
  to 0.

  \item $a>b$ if and only if at least one of the $n$ least significant bits of
  $a-b$ is different from 0, and if there is no carry in the $n^{th}$ column
  (counting from 0).

  \item $a<b$ if and only if at least one of the $n$ least significant bits of
  $a-b$ is different from 0, and if there {\em is} a carry in the $n^{th}$
  column (counting from 0).
\end{itemize}
Hence a subtractor circuit, plus a another computing whether its output
(excluding the carry) is different from 0, is sufficient to compare two numbers.

Finally, circuits computing bitwise logical operations on $n$-bit numbers are
trivial to implement. Indeed, we just need $n$ copies of the corresponding
logic gate, each computing one bit of the result, independently of the others
(\ie, in parallel).

All these circuits can be put together into a larger circuit called an {\em
Arithmetic and Logic Unit}. Such a circuit accepts two binary numbers as input,
plus a third one specifying an operation to perform on them. It produces as
output the result of this operation, on the given numbers.

For instance, a very simple Arithmetic ``and Logic'' Unit which can only
perform additions, subtractions, and comparisons is shown in \cref{fig:alu}.
If its {\tt subtract} input is 1 it subtracts its two 4-bit inputs. Otherwise
it adds them. For this it uses a subtractor circuit where the NOT gates are
replaced with XOR gates, connected to the {\tt subtract} input. When this input
is 0, the XOR gates behave as a simple wire ($p \oplus 0 = p$), which gives an
adder circuit. When {\tt subtract} is 1 these gates behave as NOT gates ($p
\oplus 1 = \neg p$), yielding a subtractor. Finally, three OR gates compute
whether at least one bit of the output is 1. Together with the carry bit, this
can be used to compare the inputs, as explained above.

To conclude this chapter, it should be noted that a relay takes some time to
switch between its active and inactive states (because its moving metallic
piece cannot move instantly). This is the case for transistors too.
Consequently, the output of a logic gate does not change instantly when its
inputs change. And this is the same for all circuits. The more logic gates
there is between an input and an output, the longer it takes for an input
change to {\em propagate} to the output. These propagation delays must be taken
into account in some circuits, including some presented in the next chapters.
