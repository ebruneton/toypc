% This work is licensed under the Creative Commons Attribution NonCommercial
% ShareAlike 4.0 International License. To view a copy of the license, visit
% https://creativecommons.org/licenses/by-nc-sa/4.0/

\renewcommand{\rustfile}{chapter5}
\setcounter{rustid}{0}

\chapter{A Toy Microprocessor}\label{chapter:toy-microprocessor}

Thanks to control circuits we can now run a program stored in memory in an
automated way, which is much faster and safer than using a processing unit
manually. Together, a control unit and the processing unit it controls form a
{\em Central Processing Unit} (CPU), also called a {\em microprocessor}. This
chapter presents a very simple one, based on the design introduced in the
previous chapter. It also illustrates its capabilities with two example
programs.

\section{Implementation}

The circuit in \cref{fig:T8} is a microprocessor for the instruction set of
\cref{section:toy-insn-set}. It is based on the circuit of
\cref{fig:alu-and-ram} (extended to an 8-bit architecture), augmented with the
control unit designed in \cref{section:toy-control-unit}. Its physical layout
matches more or less the block diagram in \cref{fig:T8-schema}. This section
only presents the parts of this circuit which have not been explained in the
previous chapters.

\subsection{Input and output}

The input is made of 8 pins which are connected to the bus via tristate
buffers, plus a {\em Light Emitting Diode} (LED). When on, the LED indicates
that the microprocessor is waiting a value on the input pins. The user must set
them to 0 or 1 as desired, and press a button when done. Similarly, the output
is made of 8 pins connected to the bus, plus a LED. When on, the LED indicates
that a value is available on the output pins, and that the user must press a
button to resume the execution of the program.

The two LEDs are connected to the two loops inside the EXECUTE subcircuit (for
the IN and OUT instructions), which wait for the user to press a button. In
this way they are turned on when an IN or OUT instruction starts, and turned
off when the button is pressed.

\newlength{\fullpagefigureheight}
\setlength{\fullpagefigureheight}{\textheight-1.8\baselineskip}
\begin{Figure}[p]
  \begin{leftfullpage}
    \resizebox{!}{\fullpagefigureheight}{\input{figures/chapter5/T8-left.tex}}

    \caption{A toy microprocessor, implementing the instruction set of
    \cref{section:toy-insn-set},}\label{fig:T8}
  \end{leftfullpage}
\end{Figure}

\begin{Figure}[p]
  \begin{fullpage}
    \resizebox{!}{\fullpagefigureheight}{\input{figures/chapter5/T8-right.tex}}

    \caption*{and its 25 bytes Random Access Memory and 7 bytes Read-Only
    Memory.\hfill~}
  \end{fullpage}
\end{Figure}

The button is a push button, used for both IN and OUT instructions. Due to its
speed, the microprocessor might execute several instructions during the time
this button stays pressed. In particular, it might execute several IN or OUT
instructions. In this case the user would not have the time to enter a second
value, or to read the first output value. To avoid this, the push button is
connected to a circuit which generates a short pulse (lasting one clock cycle)
when it transitions from the ``open'' to the ``closed'' state. It is then
necessary to release and press again this button to generate a new pulse. The
pulse generator circuit is the following:

\begin{center}
  \input{figures/chapter5/push-button.tex}
\end{center}

\noindent The output of the D flip-flop is the push button's state at the
previous clock cycle. Hence the output of the AND gate is $1$ if and only if
the button is currently pressed, and was not pressed at the previous clock
cycle. In other words, it is $1$ only when the button transitions from the
``open'' to the ``closed'' state, as desired.

\subsection{Boot program}\label{subsection:toy-boot-program}

To run a program with our microprocessor we first need to store it in memory
(as we assumed earlier, all flip-flops are initially $0$, and thus all the
memory too). But we no longer have any manual control over the memory, since we
transferred it to the control unit. Hence the only way to store anything in
memory is to use a program reading values in input and writing them in memory.
But then we need to store this program in memory first! To solve this chicken
and egg problem, a solution is to store it in a Read-Only Memory (ROM), \ie, a
memory containing immutable values.

This is what the memory circuit in \cref{fig:T8} does: values at addresses 0 to
25 excluded (which is noted $[0, 25[\ = [0, 24]$) can be read and written, but
values at addresses $[25, 32[$ cannot be modified. On the other hand, the
latter do not need to be initialized first. They can thus contain a program
which is ready to be executed when the microprocessor {\em boots}, \ie, when it
is powered on. The ROM in \cref{fig:T8} contains the following {\em boot
program}:

\rust{
  let mut a = T8Program::default();
  {
    const ONE: u8 = 31;
    a.ldr(24);
    a.add(ONE);
    a.str(24);
    a.input();
    a.if_zero(0);
    a.jump(24);
    a.data(1, "the value $1$");
  }
}
\rs{a.get_listing_with_offset(25)}

Moreover, the Program Counter register uses NOT gates around its three most
significant bits, which give it the initial value $11100_2=28$. Hence, when it
boots, our microprocessor starts by executing the IN instruction at address
$28$. It thus waits a value $v_0$ in input. If this value is not $0$, the
instruction at address $30$ jumps to address $24$. Initially, the value at this
address, in RAM, is $0$. This corresponds to a STR $0$ instruction, which thus
stores $v_0$ at address $0$. The next 3 instructions, in $[25,27[$, add $1$ to
the value at address $24$. This value thus becomes $1$, which corresponds to
STR $1$. Hence, after a second value $v_1$ is read in input, and if it is not
$0$, the new instruction at address $24$ stores $v_1$ at address $1$. And so on
with the next input values: $v_2$ stored at address $2$, $v_3$ at address $3$,
etc. This loop ends when the input value is $0$. In this case the IFZ
instruction at address $29$ jumps to address $0$. The effect is to run the
program $v_0, v_1, \ldots v_n$ stored in memory by the previous steps, starting
at address $0$.

\section{Example programs}

The above microprocessor can run several ``useful'' programs. A first example
is the adder program of \cref{subsection:adder-program}. To run it, one first
need to enter its $6$ instructions with the boot program, followed by a $0$
value. After that the programs reads two values in input, outputs their sum,
and repeats these two steps forever.

\subsection{Multiplier}

Another example is a program to compute products. Since the Arithmetic Unit
cannot perform multiplications, nor shifts (see \cref{subsection:binary-mult}),
we need a program computing them with repeated additions. The general {\em
algorithm}, \ie, the specification of the main steps that this program must
follow, is the following:

\begin{algorithmic}[1]
\State read two numbers $a$ and $b$ in input, and initialize $c$ to $0$.
\State if $b=0$ go to step 4.
\State otherwise, subtract $1$ from $b$, add $a$ to $c$, and go back to step 2.
\State output $c$ and go back to step 1 to compute another product.
\end{algorithmic}

This algorithm executes step 3 $b$ times, and thus adds $a$ to $c$ $b$ times.
Hence, when $b=0$, $c$ contains $a * b$ and can be output. Assuming that $a$,
$b$, and $c$ are stored at addresses $A$, $B$, and $C$, respectively, and that
the values at addresses $\mathit{ZERO}$ and $\mathit{ONE}$ are $0$ and $1$,
this gives the following (abstract) instructions:
\begin{itemize}
  \item step 1: IN, STR A (store an input value in $a$), IN, STR B (store an
  input value in $b$), LDR ZERO, STR C (store $0$ in $c$).

  \item step 2: LDR B, IFZ ``step4''.

  \item step 3: SUB ONE, STR B (subtract $1$ from B, still in R0), LDR C, ADD
  A, STR C (add $a$ to $c$), JMP ``step2'' (go back to step 2).

  \item step 4: LDR C, OUT (output $c$), JMP 0 (go back to step 1).
\end{itemize}

Step 1 has $6$ instructions, and thus step 2 starts at address $6$. Step 2 and
3 have a total of $8$ instructions, and thus step 4 starts at address $6+8=14$.
We can thus replace IFZ ``step4'' with IFZ $14$, and JMP ``step2'' with JMP
$6$. We can also store $a$, $b$, $c$, $0$, and $1$ after the last instruction,
\ie, starting at address $17$ (since there are $17$ instructions). In the
following we use $\mathit{ONE}=17$, $\mathit{ZERO}=18$, $A=19$, $B=20$ and
$C=21$. This leads to the following {\em machine code}, \ie, a list of
instructions in binary form that the machine (the microprocessor) can directly
execute:

\rust{
  let mut a = T8Program::default();
  {
    const START: u8 = 0;
    const LOOP: u8 = 6;
    const END_LOOP: u8 = 14;
    const ONE: u8 = 17;
    const ZERO: u8 = 18;
    const A: u8 = 19;
    const B: u8 = 20;
    const C: u8 = 21;
    a.input();
    a.str(A);
    a.input();
    a.str(B);
    a.ldr(ZERO);
    a.str(C);
    a.ldr(B);
    a.if_zero(END_LOOP);
    a.sub(ONE);
    a.str(B);
    a.ldr(C);
    a.add(A);
    a.str(C);
    a.jump(LOOP);
    a.ldr(C);
    a.output();
    a.jump(START);
    a.data(1, "the value $1$");
    a.data(0, "the value $0$");
    a.data(0, "the $a$ number");
    a.data(0, "the $b$ number");
    a.data(0, "the $c$ number");
  }
}
\rs{a.get_listing()}

\rust{
  let outputs = T8Emulator::new().emulate(&a.get_machine_code(),
      &[13, 7, 19, 11], 2);
  assert_eq!(outputs, &[13 * 7, 19 * 11]);
}

To run it one fist need to enter its 17 instructions, plus the $\mathit{ONE}$
value, followed by a $0$, with the boot program.

\subsection{Prime numbers enumerator}

A third and last example is a program which outputs all the prime numbers less
than or equal to $255$ (the maximum value of an 8-bit number). For this the
general algorithm is the following:

\begin{algorithmic}[1]
\State add $1$ to $n$, supposed initially equal to $1$.
\State if $n$ is prime go to step 4.
\State go back to step 1.
\State output $n$ and go back to step 1 to find the next prime number.
\end{algorithmic}

This algorithm tests each number $n$ one by one, in increasing order, and
starting from $n=2$. Step 2 needs to check whether $n$ can be divided by some
number $f$ in $[2,n[$. For this, a simple method is to check all values of $f$
in decreasing order, from $n-1$ to $2$. Step 2 can then be replaced with the
following algorithm:

\begin{algorithmic}[1]
\State initialize $f$ to $n$.
\State subtract $1$ from $f$.
\State if $f=1$, $n$ is prime, stop (all $f$ values have been tested and no
divisor was found).
\State if $f$ does not divide $n$, go back to step 2.
\end{algorithmic}

Finally, to check whether $f$ divides $n$ or not, and since the Arithmetic Unit
cannot perform divisions, we can use an algorithm which repeatedly subtracts
$f$ from $n$. If this process ends with 0 then $f$ divides $n$,
otherwise it does not:

\begin{algorithmic}[1]
\State initialize $r$ to $n$.
\State subtract $f$ from $r$.
\State if $r=0$, $f$ divides $n$, $n$ is not prime, stop.
\State if $r<0$, $f$ does not divide $n$, stop.
\State go back to step 2 to continue the division of $n$ by $f$.
\end{algorithmic}

By putting together the above partial algorithms we get the following complete
algorithm:

\begin{algorithmic}[1]
\State add $1$ to $n$, supposed initially equal to $1$, and initialize $f$ to
$n$.
\State subtract $1$ from $f$.
\State if $f=1$ then $n$ is prime, go to step 9.
\State initialize $r$ to $n$.
\State subtract $f$ from $r$.
\State if $r=0$, $f$ divides $n$, $n$ is not prime. Go to step 1 to try the
next $n$.
\State if $r<0$, $f$ does not divide $n$, go back to step 2 to try the next $f$.
\State go back to step 5 to continue the division of $n$ by $f$.
\State output $n$ and go back to step 1 to find the next prime number.
\end{algorithmic}

Assuming that $n$ and $f$ are stored at addresses $N$ and $F$, respectively,
that the value at address $\mathit{ONE}$ is $1$, and by storing $r$ in R0, this
gives the following (abstract) instructions:

\begin{itemize}
  \item step 1: LDR N, ADD ONE, STR N (add $1$ to $n$), STR F (initialize $f$
  to $n$).

  \item step 2: LDR F, SUB ONE, STR F (subtract $1$ from $f$).

  \item step 3: SUB ONE, IFZ ``step 9'' (if $f-1=0$ go to step 9).

  \item step 4: LDR N (initialize $r$ to $n$).

  \item step 5: SUB F (subtract $f$ from $r$).

  \item step 6: IFZ 0 (if $r=0$, go back to step 1).

  \item step 7: IFC ``step 2'' (if $r<0$, go back to step 2).

  \item step 8: JMP ``step 5''

  \item step 9: LDR N, OUT, JMP 0 (output $n$ and go back to step 1).
\end{itemize}

Finally, with the same method as for the multiplier program, we can replace IFZ
``step 9'', IFC ``step 2'', JMP ``step5'' with IFZ 14, IFC 4, and JMP 10,
respectively. And we can store $n$, $f$ and $1$ after the last instruction,
\ie, starting at address $17$. By using $\mathit{ONE}=17$, $N=18$, and $F=19$
we get the following machine code:

\rust{
  let mut a = T8Program::default();
  {
    const B0: u8 = 0;
    const LOOP: u8 = 1;
    const B1: u8 = 4;
    const B2: u8 = 10;
    const END_LOOP: u8 = 14;
    const ONE: u8 = 17;
    const N: u8 = 18;
    const F: u8 = 19;
    a.ldr(N);
    a.add(ONE);
    a.str(N);
    a.str(F);
    a.ldr(F);
    a.sub(ONE);
    a.str(F);
    a.sub(ONE);
    a.if_zero(END_LOOP);
    a.ldr(N);
    a.sub(F);
    a.if_zero(B0);
    a.if_carry(B1);
    a.jump(B2);
    a.ldr(N);
    a.output();
    a.jump(LOOP);
    a.data(1, "the value $1$");
    a.data(1, "the value $n$");
    a.data(0, "the value $f$");
  }
}
\rs{a.get_listing()}

\rust{
  let outputs = T8Emulator::new().emulate(&a.get_machine_code(), &[], 25);
  assert_eq!(outputs, &[2, 3, 5, 7, 11, 13, 17, 19, 23, 29, 31, 37,
      41, 43, 47, 53, 59, 61, 67, 71, 73, 79, 83, 89, 97]);
}

To run it one fist need to enter its 17 instructions, plus the $\mathit{ONE}$
and initial $n$ values, followed by a $0$, with the boot program.
