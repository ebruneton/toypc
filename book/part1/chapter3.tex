% This work is licensed under the Creative Commons Attribution NonCommercial
% ShareAlike 4.0 International License. To view a copy of the license, visit
% https://creativecommons.org/licenses/by-nc-sa/4.0/

\chapter[Register and Memory Circuits]{Register and\\ Memory
Circuits}\label{chapter:memory-circuits}

The very basic Arithmetic Unit presented in the previous chapter can perform
several operations, but only one at a time. To do a sequence of operations one
has to memorize intermediate results, or to note them somewhere. For instance,
to add $a$, $b$, and $c$, one has to set the inputs to $a$ and $b$, memorize
the output $a+b$, and replace the inputs with $a+b$ and $c$ (changing the
inputs immediately changes the output, hence one cannot directly ``copy'' it to
an input). To avoid this mental or manual work, a solution is to use additional
circuits to memorize intermediate results. This chapter explains how this can
be done with logic gates.

\section{Memory cells}

\subsection{SR latch}

A circuit which can memorize a single bit must have some inputs to set the
value to memorize, and an output equal to the last memorized value, noted $Q$.
One possibility is to use one input to {\em set} the memorized value to 1,
noted $S$, and another to {\em reset} it to 0, noted $R$. Connecting $S$ to the
power source should change $Q$ to 1, but connecting it to the ground should
{\em not} change $Q$ to 0 (otherwise this circuit would have no ``memory'').
Likewise, setting $R$ to 1 should reset $Q$ to 0, but setting it to 0 should
not change $Q$. In particular, when both $S$ and $R$ are 0, $Q$ should keep its
memorized value, which can be 0 or 1.

The above requirements lead to a circuit whose output is not completely
determined by its current inputs, unlike all the circuits presented so far
(since $Q$ can be 0 or 1 when $S=R=0$). To achieve this a solution is to use a
``hidden'' input equal to the last value of $Q$, noted $Q_\mathrm{last}$. Then
$Q$ can be defined as a function of its inputs again ($Q=Q_\mathrm{last}$ if
$S=R=0$). By definition $Q_\mathrm{last}$ is the last output of the circuit,
which leads to a loop:

\begin{center}
  \input{figures/chapter3/sr-interface.tex}
\end{center}

When $S=R=0$, $Q$ should be equal to the $Q_\mathrm{last}$ input. For this the
circuit in the above box cannot simply connect $Q$ to $Q_\mathrm{last}$, since
an electric current cannot flow in a loop. A solution is to use two NOT gates
in series instead. Indeed, this leads to a loop which has two stable states,
$Q=0$ and $Q=1$:

\begin{center}
  \input{figures/chapter3/not-not-loop.tex}
\end{center}

To set $Q$ to 1 we need to force the output of the right NOT gate to 1 or,
equivalently, to force the output of the left NOT gate to 0. The latter can be
done by replacing the left NOT gate with a NOR gate connected to $S$:

\begin{center}
  \input{figures/chapter3/nor-not-loop.tex}
\end{center}

Indeed, this NOR gate behaves like a NOT gate when $S=0$ (since $\neg (x \vee
0) = \neg x$), but forces its output to 0 when $S=1$ (because $\neg (x \vee 1)
= 0$). Similarly, we can set $Q$ to 0 when $R=1$ by replacing the remaining NOT
gate with a NOR gate connected to $R$. This yields the following circuit,
called an {\em SR latch}:

\begin{center}
  \input{figures/chapter3/nor-nor-loop.tex}
\end{center}

If $S=R=1$ the two NOR gates force their output to 0. Switching from this state
to $S=R=0$ make them behave as NOT gates again, but starting with their input
and output equal to 0. This state is unstable: depending on which gate updates
its output to 1 first, the end result can be $Q=0$ or $Q=1$ (or, in theory, an
infinite oscillation between 0 and 1). For this reason $S$ and $R$ must not be
set to 1 at the same time. Note that the above unstable state also occurs when
the circuit is powered on. In the following we assume that the $R$ input of
each SR latch is briefly set to 1 when circuits are powered on, so that their
initial state is always 0.

\subsection{D latch}

Using set and reset inputs is only one possibility to change the memorized
value $Q$. Another is to set $Q$ to the current value of some ``data'' input,
noted $D$, when a ``copy'' input, noted $C$, is 1 (and to keep it unchanged
when $C=0$). In other words, $Q$ should be set to 1 when $D=1$ and $C=1$,
should be reset to 0 when $D=0$ and $C=1$, and should keep its value when
$C=0$. This is easy to do with an SR latch and 3 more gates to convert $D$ and
$C$ to appropriate values of $S$ and $R$:

\begin{center}
  \input{figures/chapter3/d-latch.tex}
\end{center}

This circuit is called a {\em D latch}. One advantage, compared to the SR
latch, is that it does not have forbidden inputs such as $S=R=1$. Indeed,
thanks to the gates in front of the SR latch, this case can never happen.

\subsection{D flip-flop}

As long as $C=1$, the output $Q$ of a D latch changes each time $D$ changes.
This is not practical to memorize the value of $D$ at a precise moment, unless
$C$ remains equal to 1 only for a very brief time (but long enough to allow the
SR latch to stabilize to a potentially new state). To solve this issue, a
possibility is to make sure that $D$ does not change while $C=1$. This can be
done with two D latches in series, using opposite values for $C$, as shown
below:

\begin{center}
  \input{figures/chapter3/d-flip-flop.tex}
\end{center}

\noindent Indeed:
\begin{itemize}
  \item when $C=1$ the output $D_1$ of the first D latch does not change when
  $D$ changes, because $C_0=0$. Hence the output of the second D latch does not
  change either, although its $C_1$ input is 1. In other words, the first D
  latch makes sure that $D_1$ does not change while $C_1=1$, as required.

  \item when $C=0$ the output $D_1$ of the first D latch changes each time $D$
  changes, because $C_0=1$. But then $C_1=0$ and thus the output $Q$ of the
  second D latch does not change.

  \item when $C$ changes from 1 to 0, the first D latch memorizes the value of D
  and its output $D_1$ changes to this potentially new value. But this takes
  some time, whereas $C_1$ changes immediately when $C$ changes. Hence, when
  $D_1$ changes, $C_1$ is already 0, and thus $Q$ does not change.

  \item when $C$ changes from 0 to 1, the first D latch keeps its state, \ie,
  $D_1$ remains equal to the current value of $D$ (this was the case since $C$
  was last set to 0). But $C_1$ also changes from 0 to 1. The second D latch
  thus memorizes $D$, and $Q$ changes to $D$.
\end{itemize}

In summary, this circuit\footnote{Other circuits can achieve the same effect,
with less gates and transistors (especially with relays \cite{MerciaRelay}).},
called a {\em D flip-flop}, memorizes the value of $D$ at the precise moment
when $C$ changes from 0 to 1, and keeps its state otherwise. In the following,
to simplify figures and to make it easier to distinguish memory cells, we
represent SR latches and D flip-flops as shown in
\cref{fig:memory-cells-symbols}.

\begin{Figure}
  \input{figures/chapter3/memory-cells-symbols.tex}

  \caption{SR latches and D flip-flops are represented with their currently
  memorized value inside a large black square. The D flip-flop symbol differs
  from the SR latch symbol with a small black triangle on its $C$
  input.}\label{fig:memory-cells-symbols}
\end{Figure}

\section{Memory circuits}

To memorize an intermediate result of the Arithmetic Unit we need to memorize
$n$ bits simultaneously. This is easy to do with $n$ flip-flops connected to
the same $C$ input:

\begin{center}
  \input{figures/chapter3/register.tex}
\end{center}

This circuit is called a {\em register}. It memorizes its input when $C$
changes from 0 to 1. Its output is the last memorized value.

To memorize several intermediate results we can connect the $n$ outputs of the
Arithmetic Unit to several $n$ bit registers. We can then choose in which
register to store this output by activating the $C$ input of only one of these
registers. For instance, the following circuit can store a 4-bit number in one
of 3 registers:

\begin{center}
  \input{figures/chapter3/registers.tex}
\end{center}

However, getting the value from one these registers is not very easy because
this circuit has too many output wires. To make it easier to use we can add one
more input per register, to optionally connect its outputs to $n$ shared
output wires. For this we can connect the $i^{th}$ flip flop of each register
to a shared output wire $o_i$ via a normally open relay:

\begin{center}
  \input{figures/chapter3/registers2.tex}
\end{center}

Together with additional inputs $G_j$, connected to the control inputs of all
the relays of the $j^{th}$ register, this allows connecting or disconnecting a
whole register to the shared output wires. Normally open relays used in this
way are called {\em tristate buffers} and are represented as shown in
\cref{fig:tristate}. Their name comes from the fact that their output can have 3
states: 0, 1, or ``disconnected''.

\begin{Figure}
  \input{figures/chapter3/tristate.tex}

  \caption{The four possible input combinations of a tristate buffer and the
  corresponding outputs (left). A tristate buffer is like a normally open relay
  (right).}\label{fig:tristate}
\end{Figure}

In the above circuit $G_0$ and $G_1$ must not be simultaneously set to 1.
Indeed, doing so could connect together the outputs of two flip flops with
different states, resulting in a short circuit. More generally, with more than
2 registers, at most one $G_j$ input must be set to 1 at a time. This can be
done with a {\em binary decoder}. A binary decoder with $k$ inputs $a_0$,
$a_1$, $\ldots$ $a_{k-1}$ has $2^k$ outputs $o_0$, $o_1$, $\ldots$ $o_{2^k-1}$.
It sets its output $o_j$ to its $i$ input, and all the others to $0$, where $j$
is the binary number $a_{k-1}\ldots a_1 a_0$. This circuit can be implemented
with several demultiplexers, as illustrated in \cref{fig:binary-decoder}.

\begin{Figure}
  \input{figures/chapter3/binary-decoder.tex}

  \caption{A binary decoder with 3 inputs. Here $a_2 a_1 a_0 = 110_2 = 6$,
  hence $o_6=i$.}\label{fig:binary-decoder}
\end{Figure}

\begin{Figure}
  \input{figures/chapter3/ram-8-4.tex}

  \caption{A Random Access Memory (RAM) storing eight 4-bit values. This
    circuit currently reads the value at address $a_2a_1a_0=011_2=3$, namely
    $1110_2$. Setting $w$ to 1 would write the input value $i_3i_2i_1i_0=1011_2$
    at address $3$.}\label{fig:ram-8-4}
\end{Figure}

We can connect it to $2^k$ registers as shown in \cref{fig:ram-8-4}. This new
circuit connects the outputs of the binary decoder to the $G_j$ inputs of the
registers and, via AND gates connected to a new $w$ input, to their $C_j$
inputs. This forms a {\em Random Access Memory} (RAM), called this way because
it allows {\em reading} and {\em writing} (\ie, to get and set) values in any
order. For instance, with the circuit in \cref{fig:ram-8-4}:
\begin{itemize}
  \item reading the value of the $j^{th}$ register can be done  by setting the
  $a_2 a_1 a_0$ inputs to the bits of $j$ in binary. The value is then obtained
  on the $o_3 o_2 o_1 o_0$ outputs. $j = a_2 a_1 a_0$ is called the {\em
  address} of this register.

  \item writing a value $v$ in the $j^{th}$ register can be done by setting
  $i_3 i_2 i_1 i_0$ to the bits of $v$ in binary, by setting the $a_2 a_1 a_0$
  inputs to the bits of $j$ in binary, and finally by changing $w$ from $0$ to
  $1$. The last step changes the $C$ inputs of all the flip flops of the
  $j^{th}$ register, making it memorize the shared inputs $i_3 i_2 i_1 i_0$.
\end{itemize}

This basic circuit uses one address per group of 4 bits. In practice, most
computers use one address per group of 8 bits, called a {\em byte}. They also
use much more than 3 bits per address. A 10-bit address can refer to
$2^{10}=1024$ bytes, called a {\em kilobyte} (KB). A 20-bit address can refer
to $1024$ kilobytes, called a {\em megabyte} (MB). And a 30-bit address can
refer to one {\em gigabyte} (GB -- $1024$ megabytes).

\section{Bus}

The above RAM circuit can store several intermediate results, but it has only
one input address and one output value. Hence it is not sufficient, for
instance, to directly add or subtract two intermediate results with the
Arithmetic Unit. To solve this we can use two separate registers as input of
the Arithmetic Unit, provided we have a way to copy values from the RAM to one
or the other of these registers.

A circuit which can copy values from one register to another, or from a
register to RAM or vice-versa, is called a {\em bus}. A 1-bit bus connecting
$n$ flip-flops is easy to build. We just need to connect the $D$ input of each
flip flop to a common wire, and to connect their $Q$ output to this same wire
via a tristate buffer, as in the RAM circuit:

\begin{center}
  \input{figures/chapter3/bus-1.tex}
\end{center}

The above circuit can copy the value from A to B by setting readA to 1 and then
by changing writeB from 0 to 1. The first step connects A's output to the bus
and thus to the D input of B. The second step memorizes this value in B.
Conversely, this circuit can copy the value from B to A by setting readB to 1
and then by changing writeA from 0 to 1. It easy to generalize to 3 or more
flip-flops. It can also be generalized to an $n$-bit bus, to copy values
between 2 or more $n$-bit registers, or the RAM. For instance, the circuit in
\cref{fig:bus-3} can copy values between two 3-bit registers. It is made of
three copies of the 1-bit bus, with shared ``read'' and ``write'' inputs.
Copying 3-bit values from A to B or vice versa can be done as with the 1-bit
bus.

To maintain a register connected in ``read mode'' to the bus we can memorize
the ``read'' inputs in SR latches. And, to make it easier to read another
register, we can connect the S input of each latch to the R input of all the
others (so that setting one resets the others -- as in the RAM, at most one
register must be connected to the bus at a time). For instance, the circuit in
\cref{fig:bus-controller} sets readA to 1, readB to 0, and readC to 0 when
selectA is 1, and keeps them in this state even if selectA is reset to 0.
Similarly, it sets readB (resp. readC) to 1, and resets the others to 0, when
selectB (resp. selectC) is 1. At most one ``select'' input must be set to 1 at
a time.

\begin{Figure}
  \input{figures/chapter3/bus-3.tex}

  \caption{A 3-bit bus (right) connected to two 3-bit registers
    (left).}\label{fig:bus-3}
\end{Figure}

\begin{Figure}
  \input{figures/chapter3/bus-controller.tex}

  \caption{A possible ``controller'' for a 3-bit bus, to maintain one of the 3
  registers connected to the bus.}\label{fig:bus-controller}
\end{Figure}

\section{Example}\label{subsection:alu-and-ram-example}

\begin{Figure}
  \input{figures/chapter3/alu-and-ram.tex}

  \caption{A 3-bit Arithmetic Unit (bottom) connected to 3 input bits (top
  right), a 4 values RAM (top left), and 2 registers R0 and R1 (middle), via a
  bus (right).}\label{fig:alu-and-ram}
\end{Figure}

We can now use the circuits presented in this chapter to memorize the
intermediate results of an Arithmetic and Logic Unit (ALU). The circuit in
\cref{fig:alu-and-ram} connects the ALU from \cref{fig:alu} (with 3 bits only
to simplify) to 3 input bits, a RAM, and two registers named R0 and R1, via a
bus, as schematized in \cref{fig:alu-and-ram-schema}.

Thanks to the bus, this circuit can copy values from any source (Input, RAM,
R0, or the ALU's output) to any sink (RAM, R0, or R1), which gives $4*3=12$
possibilities. For example, computing $a+b-c$ can be done as follows:
\begin{itemize}
  \item set the input to $a$ and copy it in R0. For this, first send a {\em
  pulse} on selectINPUT (\ie, set it to 1 for a short time and then reset it to
  0). Then send a pulse on writeR0 (this memorizes $a$ when writeR0 changes
  from 0 to 1).

  \item set the input to $b$ and copy it in R1 by sending a pulse on writeR1
  (there is no need to send a pulse on selectINPUT first since the bus
  controller keeps the last selected source connected).

  \item set and maintain ``subtract'' to 0, send a pulse on selectALU, and wait
  a short time until the ALU has computed $a+b$. Then store the result in R0 by
  sending a pulse on writeR0.

  \item set the input to $c$ and copy it in R1 by sending a pulse on
  selectINPUT, followed by a pulse on writeR1.

  \item set and maintain ``subtract'' to 1, send a pulse on selectALU, and wait
  a short time until the ALU has computed $a+b$ (in R0) minus $c$ (in R1). Then
  store the result in R0 by sending a pulse on writeR0.
\end{itemize}

\begin{Figure}
  \input{figures/chapter3/alu-and-ram-schema.tex}

  \caption{The block diagram corresponding to
  \cref{fig:alu-and-ram}.}\label{fig:alu-and-ram-schema}
\end{Figure}

At this stage we can use the is\_not\_zero and carry outputs, for instance, to
test if $a+b-c$ is equal to 0, or to compare $a+b$ and $c$. We can also store
$a+b-c$ in RAM for later use. For this it suffice to send a pulse on selectR0,
followed by a pulse on $w$, after having set $a_1a_0$ to the desired
destination address.
