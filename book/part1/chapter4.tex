% This work is licensed under the Creative Commons Attribution NonCommercial
% ShareAlike 4.0 International License. To view a copy of the license, visit
% https://creativecommons.org/licenses/by-nc-sa/4.0/

\renewcommand{\rustfile}{chapter4}
\setcounter{rustid}{0}

\chapter[Control Circuits]{Control Circuits}\label{chapter:control-circuits}

Thanks to registers and memory circuits we can use an Arithmetic and Logic Unit
to perform computations without having to mentally memorize intermediate
results. Instead, as shown in the previous chapter, we can simply send a series
of pulse signals on the correct inputs, and in the correct order. But this
requires to memorize this procedure. And executing it manually is very slow and
error-prone, even if the circuit does each operation very quickly. To solve the
first issue a solution is to store some description of the desired procedure in
Random Access Memory. To solve the second one we can use new circuits to
execute this procedure for us, by sending the appropriate pulses. This chapter
explains how this can be done.

\section{Instructions}

A procedure such as the one presented in \cref{subsection:alu-and-ram-example}
could be described in an abstract way as ``read 3 numbers $a$, $b$, and $c$ in
input, compute $a+b-c$, and write the result in RAM at address $x$''. However,
representing such descriptions with one or more numbers which can be stored in
RAM is not easy. And figuring out which pulses to send to execute them would
also be quite complicated.

This procedure can also be described as a sequence of elementary actions:
``wait an input value'', ``send a pulse on selectInput'', ''send a pulse on
writeR0'', ``wait an input value'', ''send a pulse on writeR1'', etc. Each
action can easily be represented with a small number (\eg, 0 for ``wait an
input value'', 1 for ``send a pulse on selectInput'', etc). And each action is
easy to execute. However, such a description is hard to design and to
understand for humans (because its high level meaning is lost in the details).

A trade-off is to describe this procedure with more abstract actions, but not
too abstract either, called {\em instructions}. For instance, an instruction
could be ``wait an input value and store it in R0'', ``add the values in R0 and
R1 and store the result in R0'', or ``copy the value in R0 in RAM, at address
3''. As shown below, such instructions are not too complex to represent with a
number, called their {\em encoding} (to store them in memory). And they are
still quite simple to execute (each instruction only requires sending 2 or 3
pulses at most). Finally, a sequence of instructions is less hard to design and
to understand than the corresponding sequence of pulses (but still quite hard;
we address this problem in \cref{part:compiler}).

Simple procedures, also called {\em programs}, can be described with a sequence
of instructions, to be executed one after the other. For this we can store
their encoding one after the other in memory, \ie, at consecutive addresses.
Then, after the instruction at address $a$ is executed, the one at address
$a+1$ should execute\footnote{Assuming that each encoded instruction can fit in
the $n$ bits between two consecutive addresses.}.

\subsection{Jump instructions}

Some programs need to repeat the same sequence of instructions two or more
times. For instance, a ``calculator'' program needs to repeat forever the same
sequence (read two numbers in input, compute and output their sum, repeat). In
other words, after the last instruction of the sequence is executed, the
instruction at the next address should {\em not} be executed. Instead,
execution should restart at the first instruction of the sequence. This can be
described with a so called {\em jump instruction}. A ``jump to $a$''
instruction specifies that the next instruction to execute is the one at
address $a$.

In many cases a sequence of instructions must be repeated a precise number of
times. For instance, to compute $a*b$ with the circuit of
\cref{fig:alu-and-ram}, we can repeat $b$ times a sequence adding $a$ to R0
(initially set to 0). Then, after $a$ has been added to R0, there are two
cases: either we need to repeat the sequence again, or we need to continue with
the rest of the program (\eg, output the result $a*b$). This can be
described with a {\em conditional jump} instruction. Such an instruction either
jumps to a given address, or continues to the instruction at the next address,
depending on some condition (for instance, whether R0 is equal to 0 or not).

\section{A toy instruction set}\label{section:toy-insn-set}

To illustrate the above discussion we define in this section a concrete set of
instructions for a circuit such as the one in \cref{fig:alu-and-ram} (\ie, with
a RAM and two registers R0 and R1 as input of a very basic Arithmetic Unit).
These instructions are the following:
\begin{itemize}
  \item Memory:

  \begin{itemize}
    \item the Load instruction copies the value at a given address $a$ into the
    R0 register.

    \item the Store instruction copies the value in the R0 register at a
    given address $a$.
  \end{itemize}

  \item Arithmetic:

  \begin{itemize}
    \item the Add instruction adds the value at address $a$ to the value in the
    R0 register, and stores the result in R0.

    \item the Subtract instruction subtracts the value at address $a$ from the
    value in the R0 register, and stores the result in R0.
  \end{itemize}

  \item Jumps:

  \begin{itemize}
    \item the Jump instruction specifies that the next instruction to execute
    is the one at address $a$.

    \item the Jump If Zero instruction specifies that the next instruction to
    execute is the one at address $a$ if the value in R0 is equal to $0$.
    Otherwise execution continues with the instruction at the next address.

    \item the Jump If Carry instruction specifies that the next instruction to
    execute is the one at address $a$ if the last Add or Subtract instruction
    produced a non-zero carry bit. Otherwise execution continues with the
    instruction at the next address.
  \end{itemize}

  \item Input and output:

  \begin{itemize}
    \item the Input instruction waits for the user to press a button, and then
    copies the value on the input wires into the R0 register.

    \item the Output instruction displays the value in the R0 register, and
    then waits until the user presses a button.
  \end{itemize}
\end{itemize}

\subsection{Encoding}

The above {\em instruction set} contains 9 instructions. We can thus give them
numbers from 0 to 8, called {\em operation codes}, or {\em opcodes}. This
requires at least 4 bits to encode each instruction. But all instructions
except the last two have an associated address $a$, called an {\em operand}.
This operand must also be encoded as part of the instruction, which requires
more bits.

In the following we assume that the memory contains $2^5=32$ bytes, each with
their own address, and that R0, R1, and the Arithmetic Unit work on 8-bit
values. We then use 5 bits per address, and we encode each instruction in one
byte, as follows:

\bigskip \noindent
\rs{T8Instruction::Ldr(0).definition()}\\
\rs{T8Instruction::Str(0).definition()}\\
\rs{T8Instruction::Add(0).definition()}\\
\rs{T8Instruction::Sub(0).definition()}\\
\rs{T8Instruction::Jump(0).definition()}\\
\rs{T8Instruction::IfZ(0).definition()}\\
\rs{T8Instruction::IfC(0).definition()}\\
\rs{T8Instruction::In.definition()}\\
\rs{T8Instruction::Out.definition()}\\

The left column is the {\em instruction mnemonic}, an abbreviation of the
instruction name. The middle column is a symbolic description of the effect
each instruction. Here $\mathit{dst} \leftarrow \mathit{src}$ or $\mathit{src}
\rightarrow \mathit{dst}$ means a copy of the value in $\mathit{src}$ into
$\mathit{dst}$, and $\mathrm{mem8}[a]$ means the 8-bit value at address $a$.
Finally, the right column is the binary number corresponding to this
instruction, \ie, its encoding. For instance, the encoding of the LDR 7
instruction, which copies the byte at address $7=111_2$ into R0, is
$001_2$ followed by $7$ encoded in $5$ bits, $00111_2$, which gives
$00100111_2=39$.

\subsection{Example program}\label{subsection:adder-program}

With the above instruction set a ``calculator'' program adding numbers in an
endless loop can be implemented as follows:

\rust{
  let mut a = T8Program::default();
  a.input();
  a.str(6);
  a.input();
  a.add(6);
  a.output();
  a.jump(0);
  a.data(0, "the $a$ number");
}
\rs{a.get_listing()}

\rust{
  let outputs = T8Emulator::new().emulate(&a.get_machine_code(),
      &[7, 13, 17, 19], 2);
  assert_eq!(outputs, &[7 + 13, 17 + 19]);
}

\noindent where the left part gives the symbolic description of each
instruction, and the right part their encoding and their address (in gray).

The first two instructions read a number $a$ as input and store it at address
$6$. The next two instructions read a second number $b$, and add $a$ to it. The
last two instructions output the value in R0, which at this stage contains
$a+b$, and jump back to the first instruction to add two new numbers. The next
byte after these five instructions is the one used to store $a$.

\subsection{Notes}\label{subsection:int-overflow}

Adding two 8-bit numbers can give a 9-bit number. For instance,
$11111111_2=255$ plus $1$ gives the 9-bit number $100000000_2=256$. However,
the registers and the memory can only store 8-bit numbers, and the Arithmetic
Unit can only use 8-bit numbers as input. Hence, in practice, and unless a
program does something special with the carry bit (with the IFC instruction),
all additions are {\em modulo} $2^8=256$. This means that adding $a$ and $b$
does not give $a+b$ but the remainder of the division of $a+b$ by 256. It is
noted $(a+b) \mod 256$, where $x \mod m$ is defined as $x - \lfloor x / m
\rfloor * m$. For instance, adding $255$ and $1$ gives $0$\footnote{This {\em
modular arithmetic} is used in everyday life with hours. For example, 10 a.m
plus 5 hours is 3 p.m because $(10+5) \mod 12 = 3$.}.

Similarly, subtracting two numbers can give a negative result, but the
registers and the memory can only store nonnegative numbers. Hence, in
practice, and unless a program does something special with the carry bit, all
subtractions are modulo $256$ too. For instance, subtracting $1$ from $0$ gives
$255$ because $-1 \mod 256 = -1 - \lfloor -1 / 256 \rfloor * 256 = -1 -
(-1)*256 = 255$ (recall that $\lfloor y \rfloor$ means the integer part of $y$).

When $a+b$ differs from $(a+b) \mod 2^n$ we say that there is an (integer) {\em
overflow} (where $n$ is the Arithmetic Unit's ``bit width''). We say the same
when $a-b \ne (a-b) \mod 2^n$, $a*b \ne (a*b) \mod 2^n$, etc. With an
Arithmetic Unit such as the one in \cref{fig:alu}, there is an overflow if and
only if the carry output is $1$.

\section{Control circuits}

We now have a way to describe a sequence of instructions with some numbers
stored in memory. The next step, as described in the introduction of this
chapter, is to build a circuit to automatically {\em execute} these
instructions. Which means sending a corresponding sequence of pulse signals on
the registers, memory, and bus circuits. For instance, to execute an IN
instruction with the circuit in
\cref{fig:alu-and-ram}, one needs to:
\begin{itemize}
  \item connect the input wires to the bus by sending a pulse on
  ``selectInput''.

  \item wait for the user to press a button.

  \item store the input value in R0 by sending a pulse on ``writeR0''.
\end{itemize}

More generally, all instructions can be executed by sending appropriate pulses
1) on the correct wires, 2) in the correct order, and 3) at appropriate times
(signals must have time to propagate throughout the circuit between two
pulses). The first two items can be ensured with circuits of the following
form:

\begin{center}
  \input{figures/chapter4/sequencer1.tex}
\end{center}

When this circuit is powered on ``wire1'' and ``wire2'' change from 0 to 1 due
to the NOT gate. Changing $c$ from $0$ to $1$ (and back to $0$) makes the first
and second D flip-flops memorize 1. This resets ``wire1'' and ``wire2'' to 0,
and sets ``wire3'' to 1:

\begin{center}
  \input{figures/chapter4/sequencer2.tex}
\end{center}

Changing $c$ from $0$ to $1$ (and back to $0$) again makes the second and third
flip-flops memorize 0 and 1, respectively. This resets ``wire3'' to 0, and sets
``wire4'' to 1:

\begin{center}
  \input{figures/chapter4/sequencer3.tex}
\end{center}

Finally, changing $c$ from $0$ to $1$ (and back to $0$) one more time resets
``wire4'' to 0:

\begin{center}
  \input{figures/chapter4/sequencer4.tex}
\end{center}

In other words, with a series of pulses on the $c$ input, one gets two
simultaneous pulses on ``wire1'' and ``wire2'', followed by a pulse on
``wire3'' and then on ``wire4'':

\begin{center}
  \input{figures/chapter4/sequencer-signals.tex}
\end{center}

Each wire pulse starts and ends at the precise moment when $c$ switches from
$0$ to $1$. This shows that, with circuits like the one above, it is possible
to send pulses on specific wires, in a specific order. The only requirement is
the ability to send a series of pulses on a shared input $c$, which can be done
with a {\em clock}.

\subsection{Clock}

A clock is a circuit which generates a signal switching between $0$ and $1$ at
a constant frequency. A clock can be implemented in many ways. For instance,
one could use a pendulum acting on a switch. But this would not be very
practical, and can not produce high frequencies. Instead, a frequently used
method is to use the oscillations of a crystal. Crystals can oscillate one
million times per second or more (\ie, at $1$~MHz or more). In the following we
represent a clock with the symbol on the left:

\begin{center}
  \input{figures/chapter4/clock-symbol.tex}
  \input{figures/chapter4/clock-signal.tex}
\end{center}

A clock generates the signal shown above (right). A {\em period}, also called a
{\em clock cycle}, is the time between successive pulses. The clock frequency
is the number of pulses per second, \ie, the inverse of its period. Increasing
the frequency increases the number of instructions which are executed per
second. However, the frequency cannot be increased without limit. Indeed, there
must be enough time between two pulses for signals to propagate throughout the
circuit. For instance, computing an addition in an Arithmetic Unit takes some
time, because the input values have to propagate through all its logic gates,
up to the carry output. If a pulse is sent to write the sum in a register
before this delay, a wrong result will be stored.

\subsection{Control loop}

A circuit like the one above can generate a sequence of pulses to execute one
instruction. But each type of instruction needs a different sequence of pulses
to be executed. The solution is to use several circuits like this, one per type
of instruction. And to connect them to a binary decoder, so that the correct
subcircuit is used depending on the instruction opcode. For instance, if there
are only 4 different opcodes, we can use a circuit similar to the following:

\begin{center}
  \input{figures/chapter4/decode-and-execute.tex}
\end{center}

Depending on the two bits of the instruction opcode, $op_1op_0$, the above
circuit sends pulses on ``wire0-1'' to ``wire0-3'', or on ``wire1-1'' to
``wire1-3'', etc. Before this the instruction must be read in memory, so that
$op_1op_0$ contain the correct values. This can be done, as shown later, with a
so called FETCH circuit sending an appropriate sequence of pulses. Finally,
after the instruction has been executed, the next one must be fetched, decoded,
and executed. For this it suffice to connect the outputs of the EXECUTE
subcircuit back to the input of the FETCH circuit:

\begin{center}
  \input{figures/chapter4/fetch-decode-execute.tex}
\end{center}

In this way we get a pulse which loops forever in the FETCH, DECODE and EXECUTE
circuits, each time going through a specific EXECUTE subcircuit.

\section{A toy control unit}\label{section:toy-control-unit}

To illustrate the above discussions we design in this section a very basic {\em
control unit} for the circuit of \cref{fig:alu-and-ram-schema} (with an 8-bit
{\em architecture}, \ie, an 8-bit Arithmetic Unit, 8-bit registers, etc). As
its name implies, a control unit controls the rest of the circuit, called the
{\em processing unit} (\ie, the Arithmetic and Logic Unit, the registers, the
bus, etc). It does so by executing instructions stored in memory. We assume
here that these instructions are those defined in \cref{section:toy-insn-set}.

\begin{Figure}
  \input{figures/chapter4/T8-schema.tex}

  \caption{A basic control unit (yellow background) for the circuit in
  \cref{fig:alu-and-ram-schema} (white background), with the instruction set of
  \cref{section:toy-insn-set}.}\label{fig:T8-schema}
\end{Figure}

The core part of our example control unit is a control loop circuit with FETCH,
DECODE, and EXECUTE subcircuits, as presented above. To implement it we need
two new registers, in addition to R0 and R1 (see \cref{fig:T8-schema}):
\begin{itemize}
  \item the {\em Program Counter} (PC) register stores the address of the
  instruction being currently executed or, once it has been executed, the
  address of the next instruction to execute. Since addresses use only 5 bits,
  this register is a 5-bit register.

  \item the {\em Instruction Register} (IR) stores the encoding of the
  instruction being currently executed. This 8-bit register stores a copy of
  the original instruction in memory. This is necessary to have access to its
  value during its execution, which might require reading or writing values at
  other addresses in memory.
\end{itemize}

Once an instruction has been executed, the Program Counter value must be
incremented by one to execute the next instruction. Unless the last instruction
was a jump. In this case the Program Counter value must be replaced with the
operand of this jump instruction. To do this we include two more circuits in
our control unit (see \cref{fig:T8-schema}):
\begin{itemize}
  \item a 5-bit incrementer, which computes ``PC+1'', \ie, the value in the
  Program Counter register plus $1$. This is an adder circuit similar to the one
  in \cref{subsection:adder-circuit}, simplified for the case where one
  input is always $1$.

  \item a 5-bit {\em address bus}, to which we connect the Program Counter, the
  output of the above incrementer, and the 5 least significant bits of the
  Instruction Register (\ie, the address operand). This bus is also connected
  to the address decoder of the RAM and thus selects which address to read or
  write to.
\end{itemize}

Thanks to these components, we can increment the Program Counter by connecting
the output of the incrementer to the address bus, and by sending a pulse on
``writePC'' to store this value (see \cref{fig:T8-schema}). Likewise, we can
replace the Program Counter with the operand of a jump instruction by
connecting the Instruction Register to the address bus, and by sending a pulse
on ``writePC''.

\subsection{FETCH circuit}

With the above architecture, fetching an instruction can be done as follows:
\begin{itemize}
  \item send simultaneous pulses on ``selectPC'' and ``selectRAM'' to read the
  value in memory at the address stored in the Program Counter, and to get it
  on the data bus.

  \item send a pulse on ``writeIR'' to write this value in the Instruction
  Register.

  \item send a pulse on ``selectIR'' to prepare reading or writing a value at
  the address operand of the new instruction. Technically this step is part of
  the instruction's execution, but we include it in the FETCH circuit to
  avoid duplications (it is common to all instructions except IN and OUT).
\end{itemize}

\subsection{DECODE circuit}

Decoding an instruction can be done with a binary decoder with 3 inputs, namely
the 3 most significant bits of the Instruction Register. Plus a single
demultiplexer, controlled by the $4^{th}$ most significant bit, in order to
distinguish the IN and OUT instructions (the 3 most significant bits are
$111_2$ for both instructions).

\subsection{EXECUTE circuit}

The EXECUTE circuit has 9 subcircuits, one per type of instruction:

\medskip \paragraph{LDR} This subcircuit sends a pulse on ``writeR0'' to store
the value read from memory at the instruction's address operand (selected by
the last step of the FETCH circuit). It then increments the PC value with a
pulse on ``selectPC+1'', followed by one on ``writePC''.

\medskip \paragraph{STR} This subcircuit sends a pulse on ``selectR0'',
followed by a pulse on $w$ to store R0's value in memory, at the instruction's
address operand (selected by the last step of the FETCH circuit). It then
increments the PC value, as above.

\medskip \paragraph{ADD} This subcircuit sends a pulse on ``writeR1'' to store
the value at the instruction's address operand in R1. It then sends a pulse on
``selectALU'' to get the sum of the values in R0 and R1 on the bus, followed by
simultaneous pulses on ``writeR0'' and ``writeCarry'' to write it in R0 and
Carry. It then increments the PC value as above.

\medskip \paragraph{SUB} This subcircuit is almost the same as the ADD
subcircuit. It just sends an additional pulse on ``subtract'', at the same time
as the ``selectALU'' pulse. These pulses last until the one on ``writeR0''
starts. This ensures that the correct result, the difference of R0 and R1
values, is written in R0.

\medskip \paragraph{JMP} This subcircuit just sends a pulse on ``writePC'' to
replace the Program Counter with the instruction's address operand (selected by
the last step of the FETCH circuit).

\medskip \paragraph{IFZ} This subcircuit has two branches. The first, executed
if the value in R0 is $0$, is the same as the JMP subcircuit. The second,
executed if R0's value is not $0$, increments the PC value as for non-jump
instructions. The two branches are connected to a demultiplexer controlled by
the ``$\ne 0$'' signal (see \cref{fig:T8-schema}):

\begin{center}
  \input{figures/chapter4/jump-insn.tex}
\end{center}

\paragraph{IFC} This subcircuit is almost the same as the IFZ one,
except that its demultiplexer is controlled by the value of the Carry register.

\medskip \paragraph{IN} This subcircuit sends a pulse on ``selectInput'', and
then waits until a button is pressed. This can be done with a loop similar to
the control loop, with a demultiplexer to either wait, or to continue with the
next instruction:

\begin{center}
  \input{figures/chapter4/in-insn.tex}
\end{center}

\noindent In the latter case, this subcircuit sends a pulse on ``writeR0'', and
then increments the Program Counter as above.

\medskip \paragraph{OUT} This subcircuit sends a pulse on ``selectR0'' and then
waits until a button is pressed, with the same method as above. It then
increments the Program Counter's value.
