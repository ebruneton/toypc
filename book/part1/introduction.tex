% This work is licensed under the Creative Commons Attribution NonCommercial
% ShareAlike 4.0 International License. To view a copy of the license, visit
% https://creativecommons.org/licenses/by-nc-sa/4.0/

\chapter*{Introduction}
\addcontentsline{toc}{chapter}{Introduction}

Before programming a toy computer from scratch it is useful to have some basic
ideas about how computers work. For this, one method is to build a toy computer
from scratch. It is possible to build one for real from individual
electromagnetic relays~\cite{MerciaRelay} or transistors~\cite{Megaprocessor}.
But this requires a lot of time and space, and a significant budget. Moreover,
the resulting computer would be too small to run or even store the toy programs
of this book. For this reason, this part presents how a toy computer {\em
could} be built, but does not give all the details necessary to physically
build one. It is organized as follows:
\begin{itemize}
  \item \cref{chapter:binary-numbers} briefly presents binary numbers, which
  are the basis of how computers work, and how to compute with them.

  \item \cref{chapter:logic-gates} explains how an electric circuit can compute
  additions and subtractions of binary numbers.

  \item \cref{chapter:memory-circuits} shows how loops in circuits can be used
  to memorize numbers, and in particular the (intermediate) results obtained
  with the above arithmetic circuits.

  \item \cref{chapter:control-circuits} shows how a circuit can control
  another, in order to make it perform a series of computations, specified by a
  {\em program}.

  \item \cref{chapter:toy-microprocessor} puts everything together to obtain a
  toy microprocessor, and shows how it can be programmed with a few examples.
\end{itemize}

\paragraph{Note} Most of the circuits presented in this part are also available
on CircuitVerse (\url{https://circuitverse.org/}), an online digital circuit
simulator. Thanks to it you can interact with the circuits presented in this
part, which helps getting a better and more practical understanding of how they
work. See the companion website of this book for more details (\toypcurl{}).
