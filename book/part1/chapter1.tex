% This work is licensed under the Creative Commons Attribution NonCommercial
% ShareAlike 4.0 International License. To view a copy of the license, visit
% https://creativecommons.org/licenses/by-nc-sa/4.0/

\chapter{Binary Numbers}\label{chapter:binary-numbers}

As its name implies, a computer performs computations, on numbers. A number is
an abstract concept which can be represented in many different concrete ways.
For example, the number of days in a week can be represented with ``seven'',
``7'', ``VII'', etc. Some representation methods, also called numeral systems,
are more practical than others to perform computations. For instance, doing
additions and multiplications is easier in the arabic numeral system than in
the roman one. In fact they are even easier to do in the so called {\em binary
numeral system}. Computers use it for this reason. In order to understand how
they work it is thus necessary to know first what binary numbers are, and how
to compute with them. This is the goal of this chapter.

\section{Binary numbers}

An arabic number such as 237 represents 2 times 100, plus 3 times 10, plus 7
times 1. In mathematical notation this gives
$$237 = 2*100+3*10+7*1 = 2*10^2+3*10^1+7*10^0$$
where $x^n$ denotes 1 if $n=0$, or $x * x * \ldots\,x$ ($n$ times) otherwise.
In other words, an arabic number is a sequence of digits between 0 and 9, where
the $i^{th}$ digit from the right (counting from 0) represents a quantity of
$10^i$.

A binary number is similar but uses two digits instead of ten, namely 0 and 1,
called {\em bits}. It is thus a sequence of bits, where the $i^{th}$ bit from
the right (counting from 0) represents a quantity of $2^i$. For example
$$101_2 = 1*2^2+0*2^1+1*2^0 = 1*4+0*2+1*1 = 5$$
where the subscript 2 indicates a binary number (to avoid confusions with
arabic numbers; $101_2 = 5 \ne 101$ = ``one hundred one''). Another example is
\begin{align*}
11101101_2 &= 1*2^7+1*2^6+1*2^5+0*2^4+1*2^3+1*2^2+0*2^1+1*2^0\\
  &=1*128+1*64+1*32+0*16+1*8+1*4+0*2+1*1\\
  &=237
\end{align*}

The leftmost bit of a binary number is called its {\em most significant} bit.
Conversely, the rightmost bit is called the {\em least significant}. The
$i^{th}$ bit from the right (counting from 0), is called bit number $i$, or
simply bit $i$.

Some numbers have a very simple binary representation and are frequently used.
For instance, $2^n$ is a one followed by $n$ zeros in binary, like $10^n$ in
arabic notation. Another example is $2^n-1$, which is simply $n$ ones (like
$10^n-1$ is $n$ nines in arabic). \cref{table:powers-of-two} gives some
examples of these numbers.

\begin{Table}
  \begin{tabular}{|r|r|r|r|r|}\hline
    \makecell{\thead[r]{n}} &
      \thead[r]{$2^n$} & \thead[r]{$2^n$} &
      \thead[r]{$2^n-1$} & \thead[r]{$2^n-1$} \\ \hline
    \makecell[r]{$0$} & $1$ & $1_2$ & $0$ & $0$ \\
    \makecell[r]{$1$} & $2$ & $10_2$ & $1$ & $1$ \\
    \makecell[r]{$2$} & $4$ & $100_2$ & $3$ & $11_2$ \\
    \makecell[r]{$3$} & $8$ & $1000_2$ & $7$ & $111_2$ \\
    \makecell[r]{$4$} & $16$ & $10000_2$ & $15$ & $1111_2$ \\
    \makecell[r]{$5$} & $32$ & $100000_2$ & $31$ & $11111_2$ \\
    \makecell[r]{$6$} & $64$ & $1000000_2$ & $63$ & $111111_2$ \\
    \makecell[r]{$7$} & $128$ & $10000000_2$ & $127$ & $1111111_2$ \\
    \makecell[r]{$8$} & $256$ & $100000000_2$ & $255$ & $11111111_2$ \\
    \makecell[r]{$16$} & $65536$ & $10000000000000000_2$
      & $65535$ & $1111111111111111_2$ \\ \hline
  \end{tabular}
  \caption{Some frequently used powers of 2, in arabic and binary
  notation.}\label{table:powers-of-two}
\end{Table}

\section{Arithmetic operations}

\subsection{Addition}\label{subsection:binary-add}

Adding two binary numbers can be done as with arabic numbers. Namely one column
at a time, from right to left. For instance, adding $1101010_2$ and $101110_2$
can be done as follows:

\newlength{\defaulttabcolsep}
\setlength{\defaulttabcolsep}{\tabcolsep}
\newcommand*{\carry}[2]{\color{red0} \overset{#1}{\textcolor{black}{#2}}}
\newcommand*{\borrow}[2]{\color{red0} \underset{#1}{\textcolor{black}{#2}}}

\setlength{\tabcolsep}{1pt}
\begin{center}
\begin{tabular}{rrrrrrrrr}
  & $\carry{1}{\phantom{0}}$ & $\carry{1}{1}$ & $1$ & $\carry{1}{0}$ &
                 $\carry{1}{1}$ & $\carry{1}{0}$ & $1$ & $0$ \\
$+$ &              &     & $1$ & $0$ & $1$ & $1$ & $1$ & $0$ \\
\hline \makecell{~}
             & $1$ & $0$ & $0$ & $1$ & $1$ & $0$ & $0$ & $0$
\end{tabular}
\hspace{16mm}
\begin{tabular}{rrrr}
    & $1$ & $\carry{1}{0}$ & $6$ \\
$+$ &                & $4$ & $6$ \\
\hline \makecell{~}
               & $1$ & $5$ & $2$
\end{tabular}
\end{center}
\setlength{\tabcolsep}{\defaulttabcolsep}

\noindent Starting from the right, we add $0$ and $0$, which gives $0$. We then
add $1$ and $1$, which gives $2 = 10_2$. Since this is more than one bit, we
put the least significant one, here $0$, in the current column, and we {\em
carry} the most significant one, here $1$, in the column on the left (shown in
red). This is similar to the addition of the equivalent arabic numbers, shown
on the right, where $6 + 6$ gives $12$, leading to a carry of $1$.

We continue by adding $0$ and $1$, plus the carry from the previous column,
which gives $2$ again. In the next step we add $1$ and $1$, plus the carry from
the previous column, which gives $3 = 11_2$. We thus put $1$ at the bottom of
this column, and carry $1$ in the next one. And so on for the remaining columns.

Although the overall process is the same for binary and arabic numbers, adding
binary numbers is much easier, as stated above. Indeed, there are only $2*2=4$
possible cases when adding two bits, but $10*10=100$ cases when adding two
decimal digits. These four cases are summarized in \cref{table:addition-table}.

\begin{Table}
  \begin{tabular}{r|r|r|r|} \cline{2-4} result bit:
    & \makecell{\thead[r]{a}} & \thead[r]{b} & \thead[r]{a+b} \\ \cline{2-4}
    & \makecell{0} & 0 & 0 \\
    & \makecell{0} & 1 & 1 \\
    & \makecell{1} & 0 & 1 \\
    & \makecell{1} & 1 & 0 \\ \cline{2-4}
  \end{tabular}
  \hspace{8mm}
  \begin{tabular}{r|r|r|r|} \cline{2-4} carry bit:
    & \makecell{\thead[r]{a}} & \thead[r]{b} & \thead[r]{a+b} \\ \cline{2-4}
    & \makecell{0} & 0 & 0 \\
    & \makecell{0} & 1 & 0 \\
    & \makecell{1} & 0 & 0 \\
    & \makecell{1} & 1 & 1 \\ \cline{2-4}
  \end{tabular}
  \caption{The binary addition tables.}\label{table:addition-table}
\end{Table}

\subsection{Subtraction}

Similarly, subtracting two binary numbers can be done as with arabic numbers.
For instance, subtracting $101110_2$ from $1101010_2$ can be done as follows:

\setlength{\tabcolsep}{1pt}
\begin{center}
\begin{tabular}{rrrrrrrr}
             & $1$ & $1$ & $0$ & $1$ & $0$ & $1$ & $0$ \\
$-$ & $\borrow{1}{\phantom{0}}$ & $\borrow{1}{1}$ &
   $\borrow{1}{0}$ & $\borrow{1}{1}$ & $1$ & $1$ & $0$ \\
\hline \makecell{~}
             & $0$ & $1$ & $1$ & $1$ & $1$ & $0$ & $0$
\end{tabular}
\hspace{16mm}
\begin{tabular}{rrrr}
    &                       $1$ & $0$ & $6$ \\
$-$ & $\borrow{1}{\phantom{0}}$ & $4$ & $6$ \\
\hline \makecell{~}
                          & $0$ & $6$ & $0$
\end{tabular}
\end{center}
\setlength{\tabcolsep}{\defaulttabcolsep}

Starting from the right, we subtract $0$ from $0$, and then $1$ from $1$, which
gives $0$ in both cases. In the next step, since we cannot subtract $1$ from
$0$, we subtract it from $10_2=2$ instead, which gives $1$. We thus put a $1$
in the current column, and a carry of $1$ in the subtrahend on the left column
(shown in red). This is similar to the subtraction of the equivalent arabic
numbers, shown on the right, where $0 - 4$ is replaced with $10 - 4$, yielding
the result $6$ and the carry $1$.

We continue by subtracting $1$, plus the carry from the previous column (\ie, a
total of $2$), from $1$. Since this is not possible we subtract them from
$11_2=3$ instead, which gives the result $1$ and the carry $1$. And so on for
the remaining columns.

As with additions, there are only four possible cases when subtracting two
bits, which is much simpler than the hundred possible cases for decimal
digits. These four cases are summarized in \cref{table:subtraction-table}.

\begin{Table}
  \begin{tabular}{r|r|r|r|} \cline{2-4} result bit:
    & \makecell{\thead[r]{a}} & \thead[r]{b} & \thead[r]{a-b} \\ \cline{2-4}
    & \makecell{0} & 0 & 0 \\
    & \makecell{0} & 1 & 1 \\
    & \makecell{1} & 0 & 1 \\
    & \makecell{1} & 1 & 0 \\ \cline{2-4}
  \end{tabular}
  \hspace{8mm}
  \begin{tabular}{r|r|r|r|} \cline{2-4} carry bit:
    & \makecell{\thead[r]{a}} & \thead[r]{b} & \thead[r]{a-b} \\ \cline{2-4}
    & \makecell{0} & 0 & 0 \\
    & \makecell{0} & 1 & 1 \\
    & \makecell{1} & 0 & 0 \\
    & \makecell{1} & 1 & 0 \\ \cline{2-4}
  \end{tabular}
  \caption{The binary subtraction tables.}\label{table:subtraction-table}
\end{Table}

\subsection{Multiplication}\label{subsection:binary-mult}

Multiplying two binary numbers can also be done as with arabic numbers. Namely
by multiplying the first by each bit / digit of the second. And by adding the
results, each shifted by one bit / digit to the left from the previous one. For
instance, multiplying $1101010_2$ by $101110_2$ can be done as follows:

\setlength{\tabcolsep}{1pt}
\begin{center}
  \begin{tabular}{rrrrrrrrrrrrrr}
                        & & & & & & & $1$ & $1$ & $0$ & $1$ & $0$ & $1$ & $0$ \\
                        & $*$ & & & & & & & $1$ & $0$ & $1$ & $1$ & $1$ & $0$ \\
\cline{2-14} \makecell{~}
                        & & & & & & & $0$ & $0$ & $0$ & $0$ & $0$ & $0$ & $0$ \\
                    & & & & & & $1$ & $1$ & $0$ & $1$ & $0$ & $1$ & $0$ & \\
                & & & & & $1$ & $1$ & $0$ & $1$ & $0$ & $1$ & $0$ & & \\
            & & & & $1$ & $1$ & $0$ & $1$ & $0$ & $1$ & $0$ & & & \\
        & & & $0$ & $0$ & $0$ & $0$ & $0$ & $0$ & $0$ & & & & \\
    & & $1$ & $1$ & $0$ & $1$ & $0$ & $1$ & $0$ & & & & & \\
\cline{2-14} \makecell{~}
& $1$ & $0$ & $0$ & $1$ & $1$ & $0$ & $0$ & $0$ & $0$ & $1$ & $1$ & $0$ & $0$
  \end{tabular}
  \hspace{16mm}
  \begin{tabular}{rrrrr}
 &     & $1$ & $0$ & $6$ \\
 & $*$ &     & $4$ & $6$ \\
\cline{2-5} \makecell{~}
 &     & $6$ & $3$ & $6$ \\
 & $4$ & $2$ & $4$ &     \\
\cline{2-5} \makecell{~}
 & $4$ & $8$ & $7$ & $6$
  \end{tabular}
\end{center}
\setlength{\tabcolsep}{\defaulttabcolsep}

Here again, although the process is the same, multiplying binary numbers is
much easier than arabic numbers. Indeed, multiplying the first number by each
bit of the second boils down to multiplications by $0$ or $1$, which are
trivial. By contrast, multiplying an arabic number by a decimal digit requires
using a multiplication table with 100 possible cases. It also involves carries.

Some multiplications are even easier to do than with the general method
described above. In particular, multiplying $x$ by $2^n$ can be done by simply
shifting $x$ by $n$ bits to the left, \ie, by adding $n$ zeros on the right.
For instance, $1101010_2=106$ multiplied by $2^3=8$ is simply
$1101010\mathbf{000}_2=848$. This is similar to multiplications by $10^n$ in
arabic notation (for example, $46$ times $10^3=1000$ is $46\mathbf{000}$).
Shifting a binary number $x$ by $n$ bits to the left is noted $x \ll n$.

The opposite operation, shifting $x$ by $n$ bits to the right, \ie, dropping
the $n$ least significant bits, is noted $x \gg n$. It corresponds to dividing
$x$ by $2^n$. For instance, shifting $1101010_2=106$ by 3 bits to the right
gives $1101_2=13=\lfloor 106 / 2^3 \rfloor$\footnote{The $\lfloor x \rfloor$
notation designates the integer part of $x$. For instance, $106 / 8 = 13.25$
and $\lfloor 13.25 \rfloor = 13$.}. This is similar to dropping the $n$ least
significant digits of an arabic number, which divide it by $10^n$ (for example,
$4876$ shifted to the right by 2 digits is $48=\lfloor 4876 / 100 \rfloor$).
Dividing arbitrary binary numbers can be done as with arabic numbers, but is
not presented here.

\subsection{Conversions}

Computers do all their computations with binary numbers because, as shown
above, this is much easier to do than with arabic numbers. However, humans
prefer to specify inputs with arabic numbers, and to get results in arabic too.
This requires converting arabic numbers to binary ones, and vice versa.

One method to convert an arabic number to binary is to convert each digit from
left to right, and to multiply the result by $10$ before adding the next digit.
For instance, to convert $46$ to binary, we start by converting $4$, which
gives $100_2$. We multiply this by $10=8+2$, which can be done by shifting
$100_2$ by $3$ bits and by $1$ bit to the left, and by adding the results:
$100\mathbf{000}_2+100\mathbf{0}_2=101000_2$. Finally we convert $6$, which
gives $110_2$ and we add this to the previous result, yielding $101110_2$. This
method is well suited for computers since it only involves computations on
binary numbers (plus a small conversion table for each digit from $0$ to $9$).

Another method consists in dividing the arabic number by 2 repeatedly. The
remainders give the bits of the equivalent binary number, from right to left.
For instance, dividing $46$ by $2$ repeatedly gives $23$ (remainder $0$), $11$
(remainder $1$), $5$ (remainder $1$), $2$ (remainder $1$), $1$ (remainder $0$),
and $0$ (remainder $1$). The corresponding binary number is thus $101110_2$.
Since this method involves divisions on arabic numbers, it is more adapted for
humans than for computers.

Similarly, one method to convert a binary number to arabic is to ``convert''
each bit from left to right, and to multiply the result by $2$ before adding
the next bit. For instance, converting $101110_2$ gives successively $1$,
$1*2+0=2$, $2*2+1=5$, $5*2+1=11$, $11*2+1=23$, and $23*2+0=46$. Since this
method involves multiplications of arabic numbers, it is more adapted for
humans. But it can also be used on computers, if necessary.

Another method to convert a binary number is to divide it by 10 repeatedly. The
remainders, converted to arabic, give the digits of the equivalent arabic
number, from right to left. It is well suited for computers since it only
involves computations on binary numbers (plus a small conversion table for each
binary number from $0$ to $1001_2=9$).

\section{Logical operations}

Binary numbers can also be used to perform {\em logical operations}, unlike
arabic numbers. A logical operation computes whether some {\em proposition} is
true of false, depending on the status of one or more other propositions. A
proposition is a statement which is either true or false.

Consider for example a keyboard. A proposition might be ``the E key is
currently pressed'', ``the left Shift key is currently released'', or ``the e
letter is currently pressed''. They are either true or false, depending on the
current state of the keyboard. These propositions, noted
$\mathrm{KeyPressed}(k)$, $\mathrm{KeyReleased}(k)$, and
$\mathrm{LetterPressed}(l)$, are not completely independent. Some can be
computed from the others. For example, we can compute $\mathrm{KeyReleased}(k)$
as the opposite of $\mathrm{KeyPressed}(k)$. This logical operation is
the {\em negation}, also called {\em not}, and is noted $\neg$:
\begin{equation*}
  \mathrm{KeyReleased}(k) = \neg \mathrm{KeyPressed}(k)
\end{equation*}

We can also compute whether the proposition ``a Shift key is pressed'' is true
from the above propositions. Indeed, this is the case if at least one of the
two Shift keys is pressed. This logical operation is the {\em disjunction},
also called {\em or}, and is noted $\vee$:
\begin{equation*}
  \mathrm{ShiftPressed} = \mathrm{KeyPressed}(\mathrm{LeftShift}) \vee
    \mathrm{KeyPressed}(\mathrm{RightShift})
\end{equation*}

The keyboard is in ``uppercase mode'' if a Shift key is currently pressed, or
if caps are locked, but not both (a Shift key reverses the effect of
CapsLock). This logical operation is the {\em exclusive disjunction},
also called {\em exclusive or}, and is noted $\oplus$:
\begin{equation*}
  \mathrm{UppercaseMode} = \mathrm{ShiftPressed} \oplus \mathrm{CapsLocked}
\end{equation*}

As a last example, we can also compute whether
$\mathrm{LetterPressed}(\mathrm{E})$ is true from
$\mathrm{KeyPressed}(\mathrm{E})$ and $\mathrm{UppercaseMode}$. Indeed, this is
the case if both are true. This logical operation is the {\em conjunction},
also called {\em and}, and is noted $\wedge$:
\begin{align*}
  \mathrm{LetterPressed}(\mathrm{E}) &=
    \mathrm{KeyPressed}(\mathrm{E}) \wedge \mathrm{UppercaseMode} \\
  \mathrm{LetterPressed}(\mathrm{e}) &=
    \mathrm{KeyPressed}(\mathrm{E}) \wedge \neg \mathrm{UppercaseMode}
\end{align*}

The above logical operations do not depend on the meaning of the propositions,
but only on whether they are true or false. And their result is either true or
false. For instance, $\neg$ true is false, $p \wedge q$ is true if and only if
both $p$ and $q$ are true, $p \vee q$ is true if at least one of $p$ and $q$ is
true, etc. By representing true with 1 and false with 0, they can be seen as
operations on individual bits. This gives, for example, $\neg 1 = 0$, $1 \wedge
0 = 0$, $1 \wedge 1 = 1$, etc. By doing this for all possible cases we get the
{\em truth table} of each operation, represented in
\cref{table:logical-tables}. Note that the truth tables of $\oplus$ and
$\wedge$ are identical to those giving the result and carry bit of $a+b$,
respectively (see \cref{table:addition-table}). The result bit of $a-b$ is also
equal to $a \oplus b$, and the carry bit is $b \wedge \neg a$ (see
\cref{table:subtraction-table}). Hence, it suffice to know how to implement
these logical operations with electric circuits, or other technologies, in
order to be able to implement arithmetic circuits.

\begin{Table}
  \begin{tabular}[t]{|c|c|} \hline
    \makecell{\thead[c]{p}} & \thead[c]{$\neg p$} \\ \hline
    \makecell{0} & 1 \\
    \makecell{1} & 0 \\ \hline
  \end{tabular}
  \hspace{8mm}
  \begin{tabular}[t]{|c|c|c|} \hline
    \makecell{\thead[c]{p}} & \thead[c]{q} & \thead[c]{$p \wedge q$} \\ \hline
    \makecell{0} & 0 & 0 \\
    \makecell{0} & 1 & 0 \\
    \makecell{1} & 0 & 0 \\
    \makecell{1} & 1 & 1 \\ \hline
  \end{tabular}
  \hspace{8mm}
  \begin{tabular}[t]{|c|c|c|} \hline
    \makecell{\thead[c]{p}} & \thead[c]{q} & \thead[c]{$p \vee q$} \\ \hline
    \makecell{0} & 0 & 0 \\
    \makecell{0} & 1 & 1 \\
    \makecell{1} & 0 & 1 \\
    \makecell{1} & 1 & 1 \\ \hline
  \end{tabular}
  \hspace{8mm}
  \begin{tabular}[t]{|c|c|c|} \hline
    \makecell{\thead[c]{p}} & \thead[c]{q} & \thead[c]{$p \oplus q$} \\ \hline
    \makecell{0} & 0 & 0 \\
    \makecell{0} & 1 & 1 \\
    \makecell{1} & 0 & 1 \\
    \makecell{1} & 1 & 0 \\ \hline
  \end{tabular}
  \caption{The truth tables of not ($\neg$), and ($\wedge$), or ($\vee$), and
  exclusive or ($\oplus$).}\label{table:logical-tables}
\end{Table}

We can then generalize these logical operations from individual bits to whole
binary numbers. By definition, a {\em bitwise} logical operation on two binary
numbers is done by applying it on each bit separately, column by column. Thus,
for instance:

\setlength{\tabcolsep}{1pt}
\begin{center}
  \begin{tabular}{rrrrr}
             & $1$ & $1$ & $0$ & $0$ \\
    $\wedge$ & $1$ & $0$ & $1$ & $0$ \\
    \hline \makecell{~}
             & $1$ & $0$ & $0$ & $0$
  \end{tabular}
  \hspace{16mm}
  \begin{tabular}{rrrrr}
             & $1$ & $1$ & $0$ & $0$ \\
      $\vee$ & $1$ & $0$ & $1$ & $0$ \\
    \hline \makecell{~}
             & $1$ & $1$ & $1$ & $0$
  \end{tabular}
  \hspace{16mm}
  \begin{tabular}{rrrrr}
             & $1$ & $1$ & $0$ & $0$ \\
    $\oplus$ & $1$ & $0$ & $1$ & $0$ \\
    \hline \makecell{~}
             & $0$ & $1$ & $1$ & $0$
  \end{tabular}
\end{center}
\setlength{\tabcolsep}{\defaulttabcolsep}

This can be used to perform several logical operations in parallel (since there
is no carry each column can be computed independently of the others, possibly
at the same time). For instance, we can represent the current state of a 100
keys keyboard with a 100 bits binary number $S$, using one bit per key. We can
then do the following operations, which are commonly used in many similar
contexts:
\begin{itemize}
  \item to check whether at least one letter key is pressed, we can compute $S
  \wedge L$, where $L$ is the binary number whose $i^{th}$ bit is $1$ if and
  only if the $i^{th}$ key is a letter. If the result is $0$ no letter key is
  pressed, otherwise at least one is pressed.

  \item if a new set of keys if pressed, we can compute the representation of
  the new keyboard state with $S' = S \vee P$, where $P$ represents the newly
  pressed keys. For instance, if the $0^{th}$ and $3^{rd}$ keys are currently
  pressed, and if the user presses the $0^{th}$ and $2^{nd}$ keys\footnote{A
  pressed key can be ``pressed'' again due to autorepeat.}, we get $S=1001_2$,
  $P=101_2$ and $S'=1101_2$. This correctly represents the fact that the
  $0^{th}$, $2^{nd}$, and $3^{rd}$ keys are now pressed.

  \item if a new set of keys if released, we can compute the representation of
  the new keyboard state with $S' = S \wedge \neg R$, where $R$ represents the
  newly released keys. For instance, if the $0^{th}$ and $3^{rd}$ keys are
  currently pressed, and if the user releases the third, we get $S=1001_2$,
  $R=1000_2$ and $S'=1$. This correctly represents the fact that only the
  $0^{th}$ key remains pressed.
\end{itemize}

\section{Hexadecimal numbers}

Binary numbers are very practical to perform computations, but are not very
compact. Arabic numbers are much more compact (a given number has about $3.3$
less digits than bits on average), but converting between binary and arabic is
not so easy. To solve these issues {\em hexadecimal} numbers are commonly used.

Hexadecimal numbers are like arabic numbers, but use 16 digits instead of 10.
They are called hex digits and are noted $0$, $1$, $2$, $3$, $4$, $5$, $6$,
$7$, $8$, $9$, $A$ ($=10$), $B$ ($=11$), $C$ ($=12$), $D$ ($=13$), $E$ ($=14$),
and $F$ ($=15$). An hexadecimal number is thus a sequence of hex digits, where
the $i^{th}$ hex digit from the right (counting from 0) represents a quantity
of $16^i$. For instance
$$ED_{16} = E_{16}*16^1+D_{16}*16^0 = 14*16+13 = 237$$
where the subscript 16 indicates an hexadecimal number (to avoid confusions with
words or arabic numbers; $10_{16} = 16 \ne 10$ = ``ten'').

\begin{Table}
  \begin{tabular}{|r|r|r|r|r|r|r|r|} \hline
    \makecell{\thead[r]{binary}} & \thead[r]{hex} &
    \makecell{\thead[r]{binary}} & \thead[r]{hex} &
    \makecell{\thead[r]{binary}} & \thead[r]{hex} &
    \makecell{\thead[r]{binary}} & \thead[r]{hex} \\ \hline
    \makecell[r]{0000} & 0 & 0100 & 4 & 1000 & 8 & 1100 & C \\
    \makecell[r]{0001} & 1 & 0101 & 5 & 1001 & 9 & 1101 & D \\
    \makecell[r]{0010} & 2 & 0110 & 6 & 1010 & A & 1110 & E \\
    \makecell[r]{0011} & 3 & 0111 & 7 & 1011 & B & 1111 & F \\ \hline
  \end{tabular}
  \caption{Conversion between binary and
  hexadecimal.}\label{table:hex-binary-conversion}
\end{Table}

Each hex digit can be represented with up to 4 bits, and each group of 4 bits
can be represented with an hex digit, as shown in
\cref{table:hex-binary-conversion}. It is thus very easy to convert a binary
number to hexadecimal: simply convert each group of 4 bits independently, with
\cref{table:hex-binary-conversion}. For instance, to convert $11101101_2$, we
convert $1110_2$ ($E_{16}$), $1101_2$ ($D_{16}$), and concatenate the results,
yielding $ED_{16}$. Conversely, to convert $ED_{16}$ to binary we simply
concatenate the conversions of $E_{16}$ ($1110_2$) and $D_{16}$ ($1101_2$),
yielding $11101101_2$.

Hexadecimal numbers are thus compact (a given number has about 4 times less hex
digits and bits) and easy to convert to and from binary, which solves the above
issues. On the other hand, doing arithmetic computations with them is harder
than with arabic numbers (this involves tables with $16*16=256$ entries). But
this is not necessary since we can convert them to binary, do computations in
binary, and convert the result back to hexadecimal.

