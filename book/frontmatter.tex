% This work is licensed under the Creative Commons Attribution NonCommercial
% ShareAlike 4.0 International License. To view a copy of the license, visit
% https://creativecommons.org/licenses/by-nc-sa/4.0/

% Title page -------------------------------------------------------------------

\begin{titlepage}

\begin{flushleft}
\sffamily\fontsize{34}{36}\bfseries\selectfont
Programming\\
a toy computer\\
from scratch

\vspace{12cm}
\fontsize{16}{18}\selectfont
A practical introduction to computer systems\\
\vspace{1cm}
Eric Bruneton
\end{flushleft}
\end{titlepage}

% Copyright page ---------------------------------------------------------------

\thispagestyle{empty}

\vspace*{\fill}

\begin{flushleft}
\textcopyright\ 2024 \, Eric Bruneton

\medskip

\ccbyncsa\ This book is available under the
\href{https://creativecommons.org/licenses/by-nc-sa/4.0/}{Creative Commons
BY-NC-SA 4.0 License}. The programs it contains are also available separately,
under the \href{https://www.gnu.org/licenses/gpl-3.0.en.html}{GNU General
Public License v3}.

\medskip

\textbf{Version} \\ This book was built from commit
\gitcommithash[shortHash=false]\ on \gitcommitdate[formatDate] in the source
code repository. The latest version can be downloaded at \toypcurl{}.

\medskip

\textbf{Source code} \\
The source code of this book, in \href{https://www.latex-project.org/}{\LaTeX}\
and \href{https://www.rust-lang.org/}{Rust}, is available at \\
\url{https://github.com/ebruneton/toypc}. \\
The programs it describes are also available separately, at \\
\toypcurl{}.

\medskip

\textbf{Feedback} \\
Please report errors or potential improvements at \\
\url{https://github.com/ebruneton/toypc/issues}.
\end{flushleft}

\clearpage

% Table of contents ------------------------------------------------------------

\setcounter{tocdepth}{1}
\tableofcontents

\chapter*{Introduction}
\addcontentsline{toc}{chapter}{Introduction}

Billions of people are using computers or smartphones, which are computers
before being phones. One doesn't need to understand how computers work to use
them, but if you want to know, this book might help you.

Popular science books about this topic intentionally leave out many details. On
the other hand, textbooks emphasize theoretical aspects and focus on a narrow
topic. This book is different. Its goal is to introduce how computer hardware
and common programming languages and operating systems work, via a practical
example which can be understood down to the smallest detail.

For this it proposes you to assemble and program your own toy computer. And to
make sure not to omit any details, it explains how you can do this from scratch,
without using any existing programming tool. It is organized in four parts:
\begin{itemize}
  \item the first part briefly presents the main basic ideas used to design
  microprocessors, which are the core component of a computer. This is
  necessary to understand the main concepts used in the next parts. It ends
  with the presentation of a virtual, toy microprocessor, which can be
  simulated online, and of a few programs using it.

  \item the second part explains how the components of your toy computer work,
  how they can be programmed, and how to assemble them. Based on this, it then
  describes how to build a basic system allowing programs to use the computer's
  keyboard and screen. Finally, it presents how an initial program can read and
  execute other programs, based on this input and output system.

  \item the third part explains how a computer can be programmed in a language
  which can be ``easily'' understood by humans, unlike the 0s and 1s used by
  its microprocessor. For this it describes how to progressively build a toy
  language, and a program which can translate it into 0s and 1s that the
  microprocessor can execute. In order to give you an idea of what common
  programming languages look like, this toy language is an extremely simplified
  version of real and popular ones.

  \item the fourth part explains how users can easily store files and launch
  applications on their computer, thanks to a (set of) program(s) called an
  operating system. For this it describes how to progressively build a toy
  operating system for your toy computer. For the same reason as above, this
  system is an extremely simplified version of real, frequently used ones.
\end{itemize}

\subsubsection{Target audience}

This book is designed for people looking for a practical and fully detailed
example introducing how microprocessors, programming languages and operating
systems work. It does not explain the theories and principles behind this. If
you want to learn them, you should read computer science textbooks instead
(some references are provided at the end of each part). Conversely, if you only
want to understand the general ideas, it is better to read popular science
books instead.

\subsubsection{How to read this book}

You can read this book without actually assembling or programming a toy
computer, just to understand how this could be done. In this case you can skip
the tutorial-like sections, which describe concrete steps to follow (plug this
wire here, type this on the keyboard, press this button, etc). This is the
case, in particular, of the ``Experiments'' and ``Compilation and tests''
sections.

Alternatively, you can read this book while following the instructions on an
emulator. In this case you do not need to assemble a toy computer, nor to buy
the necessary components for this (described in \cref{appendix:bom}). Instead,
simply use the emulator provided at \toypcurl{emulator.html}. For this you need
a desktop or laptop computer (tablets and smartphones are not really usable for
this task).

Finally, you can buy the components, assemble them, and follow the instructions
for real. This is more costly but probably more fun than the previous methods.
This method also requires a desktop or laptop computer, with a USB port and
capable of running python3. You can also use all three methods: start with the
first one, then re-read the book with the second method and optionally with the
third, if you feel that you need to (typing a program is slower than reading it
-- this can trigger some questions, and finding the answers yourself can give
you a better understanding).

Note also that, if you are stuck or simply want to skip some steps, you can
follow the instructions of any chapter without doing those of the previous ones
(once the computer is physically assembled, if you choose this method). Hence,
for instance, you can skip the instructions of part 2, do those of part 3 on
the emulator, and those of part 4 for real. You can also already have a look at
the final programs and operating system obtained at the end of this book, on
the emulator, by opening the following link:
\toypcurl{emulator.html?script=backups/final.txt}. See the companion website of
this book for more details (\toypcurl{}).
